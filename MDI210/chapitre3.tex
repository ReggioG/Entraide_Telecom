Problème : résoudre $Ax = b$ sachant $A$ inversible.

\subsection{Méthode de Gauss (pour une matrice quelconque)}

	Par combinaisons linéaires successives entre lignes de $A$ on se ramène à $(MA)x = Mb$ où $MA$ est triangulaire supérieure.
	On résout ensuite cette équation par une méthode de remontée.
	
	\begin{algorithm}[h]
	\caption{\textcolor{RoyalBlue}{Étape d'élimination}}
	\Entree{Matrice $A$.}
	$A' \lar A$ \;
	\Tq{$\dim(A') > 1$}
	{
		$\texttt{pivot} \lar a'_{i,1} \neq 0$ (qui existe car $A'$ est inversible) \;
		\Si{$i \neq 1$}
		{
			$L_i \leftrightarrow L_1$ \;
		}
		\PourTous{$i > 1$ tel que $a_{i,1} \neq 0$}
		{
			$L_i \lar L_i - \frac{a_{i,1}}{a_{1,1}} L_1$ \;
			\tcp{on annule tous les termes de la colonne du pivot situés sous la diagonale}
		}
		\tcp{notre matrice $A'$ n'a alors que des 0 sous le premier terme, qui est non nul}
		\tcp{on met alors de côté la première ligne et la première colonne de $A'$}
		$A' \lar (a'_{i,j})_{2 \leq i,j}$
	}
	On replace les lignes et colonnes supprimées au fur et à mesure pour obtenir une matrice triangulaire ($MA$).
	\end{algorithm}

	Un pivot trop petit en valeur absolue peut causer des erreurs d'arrondi du fait de la division par le pivot.
	D'où deux stratégies :
	\begin{itemize}
	\item[\textbullet] \textit{pivot partiel} : on prend le terme de plus grande valeur absolue de la colonne courante, sur ou en dessous de la diagonale,
	\item[\textbullet] \textit{pivot total} : on choisit le terme de plus grande valeur absolue de la matrice résiduelle, i.e. si on est à l'étape $n - k + 1$, la matrice constituée des $k$ dernières lignes et colonnes (plus couteux).
	\end{itemize}
	
	La complexité est en $O \left( \frac{2n^3}{3} \right)$ sans choix du pivot.

\subsection{Méthode de Gauss-Jordan (variante de la précédente)}

	Dans la phase d'éliminations on élimine également les terme au-dessus de la diagonale.
	On obtient ainsi une matrice diagonale, ce qui permet de calculer efficacement l'inverse.

\subsection{Factorisation LU}

	\begin{thm}
		Soit $A = (a_{i,j}) \in \GL_n(\K)$ telle que $\forall k \in \iniff{1}{n},
			\begin{pmatrix}
				a_{11} & \ldots & a_{1k} \\
				\vdots &        & \vdots \\
				a_{k1} & \ldots & a_{kk} \end{pmatrix}
			\in \GL_k(\K)$.
		Alors $A$ admet une factorisation sous la forme $A = LU$ avec $L$ triangulaire inférieure et $U$ triangulaire supérieure.
		De plus on peut choisir $\forall i \in \iniff{1}{n}, (L)_{ii} = 1$ et la décomposition est alors unique.
	\end{thm}
	
	Cela signifie que, dans l'algorithme de Gauss, les pivots successifs peuvent toujours être pris sur la diagonale.
	Si la factorisation échoue ou peut toujours permuter des lignes de $A$ pour arriver à une matrice $A'$ qui admet une factorisation LU.
	
	Cette factorisation est utile lorsque plusieurs systèmes linéaires sont à résoudre : on résout un système sur $L$ puis un système sur $U$.

\subsection{Méthode de Cholesky (matrices symétriques définies positives)}

	\begin{thm}
		Soit $A \in \Sym^{++}_n(\K)$.
		Il existe une matrice triangulaire $B$ vérifiant $A = B \transp{B}$.
		De plus on peut imposer que tous les éléments diagonaux de $B$ soient tous strictement positifs et la factorisation $A = B \transp{B}$ est alors unique.
	\end{thm}
	
	En pratique, on calcule $B$ colonne par colonne à partir de $\forall 1 \leq i \leq j \leq n, a_{ij} = \sum_{k = 1}^i b_{ik} b_{jk} = a_{ji}$.
	
	\begin{rem}
		Le déterminant de la matrice peut alors se calculer facilement : $\det(A) = (b_{11} b_{22} \cdots b_{nn})^2$.
	\end{rem}
	
	Un système $Ax = b$ devient alors $B \transp{B} x = b$.
	Pour résoudre le système on résout $By = b$ puis $\transp{B}x = y$.
	La complexité de la factorisation suivie de la résolution est en $O \left( \frac{n^3}{3} \right)$.