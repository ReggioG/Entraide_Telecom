La forme standard d'un problème d'optimisation est la suivante :
\begin{itemize}
	\item[\textbullet] on maximise une forme linéaire $z$ de $n$ variables $x_1, \ldots, x_n$,
	\item[\textbullet] avec $m$ contraintes linéaires : $\forall i \in \iniff{1}{m}, \sum_{j = 1}^n a_{ij}x_j \leq b$,
	\item[\textbullet] et $n$ contraintes de positivité : $\forall j \in \iniff{1}{n}, x_j \geq 0$.
\end{itemize}

Ces équations déterminent un \textbf{polyèdre des contraintes} qui est convexe.
Les $n$-uplets $(x_1,\ldots,x_n)$ qui satisfont ces contraintes s'appellent \textbf{solutions réalisables} du problème.
Ce sont les points intérieurs (au sens large) du polyèdre des contraintes.
La solution optimale est celle qui maximise $z$.

S'il n'existe aucune solution réalisable, le problème des dit infaisable.
S'il existe des solutions réalisables mais que $z$ n'y admet pas de maximum fini alors le problème est dit non-borné.

\begin{thm}
	Soit un problème de programmation linéaire dont le polyèdre des contraintes est non vide et dont la fonction à maximiser est majorée sur ce polyèdre.
	Alors le problème admet un maximum fini atteint en au moins un sommet du polyèdre est contraintes.
\end{thm}

Pour trouver le maximum on passe alors itérativement d'un sommet à un sommet adjacent de façon à augmenter la valeur de $z$.

À partir des données initiales on construit un \emph{dictionnaire} de variables : 
%%%% [...]

\begin{thm}[Théorème de Bland]
	Il ne peut y avoir cyclage lorsque, à toute itération effectuée à partir d'un dictionnaire dégénéré, on choisit les variables entrante et sortante comme celles de plus petit indice parmi les candidats possibles.
\end{thm}
