La forme standard d'un problème d'optimisation est la suivante :
\begin{itemize}
	\item[\textbullet] on maximise une forme linéaire $z = \sum_{j = 1}^n c_j x_j$, la \textbf{fonction objectif},
	\item[\textbullet] avec $m$ contraintes linéaires : $\forall i \in \iniff{1}{m}, \sum_{j = 1}^n a_{ij}x_j \leq b_i$,
	\item[\textbullet] et $n$ contraintes de positivité : $\forall j \in \iniff{1}{n}, x_j \geq 0$.
\end{itemize}

Ces équations déterminent un \textbf{polyèdre des contraintes} qui est convexe.
Les $n$-uplets $(x_1,\ldots,x_n)$ qui satisfont ces contraintes s'appellent \textbf{solutions réalisables} du problème.
Ce sont les points intérieurs (au sens large) du polyèdre des contraintes.
La \textbf{solution optimale} est celle qui maximise $z$.

S'il n'existe aucune solution réalisable, le problème est dit infaisable.
S'il existe des solutions réalisables mais que $z$ n'y admet pas de maximum fini alors le problème est dit non-borné.

\begin{thm}
	Soit un problème de programmation linéaire dont le polyèdre des contraintes est non vide et dont la fonction à maximiser est majorée sur ce polyèdre.
	Alors le problème admet un maximum fini atteint en au moins un sommet du polyèdre est contraintes.
\end{thm}

Pour trouver le maximum on passe alors itérativement d'un sommet à un sommet adjacent de façon à augmenter la valeur de $z$.

À partir des données initiales on construit un \emph{dictionnaire} de variables $x_1, \ldots, x_{n + m}$ où $x_1, \ldots, x_n$ sont appelées \emph{varaibles de décision}, \emph{varaibles de choix} ou encore \emph{varaibles principales} ou \emph{initiales} et $x_{n + 1}, \ldots, x_{n + m}$ (aussi positives) s'appellent \emph{variables d'écart} telles que $x_{n + i} = b_i - \sum_{j = 1}^n a_{ij} x_j$.

Un dictionnaire est un système d'équations linéaires liant $x_1, \ldots, x_{n + m}$ et satisfaisant :
\begin{itemize}
	\item[\textbullet] les équations expriment $z$ et $m$ des $n + m$ variables (\textbf{variables de base}) en fonction des $n$ autres variables (\textbf{variables hors-base}),
	\item[\textbullet] équivalence avec le dictionnaire définissant les variables d'écart et la fonction objectif.
\end{itemize}

Pour une base fixée on obtient une \textbf{solution basique} en mettant à $0$ toutes les variables hors base.
Une solution est dite \textbf{dégénérée} lorsque des variables principales sont nulles.

\begin{algorithm}[h]
	\caption{\textcolor{RoyalBlue}{Itération de l'algorithme du simplexe}}
	\Entree{Un dictionnaire : $I \uplus J = \iniff{1}{n + m}, z = z^* + \sum_{j \in J} c_j x_j, \forall i \in I, x_i = b_i + \sum_{j \in J} a_{ij} x_j, b_i \geq 0$}
	\Si{$\forall j \in J, c_j \leq 0$}
	{
		Fin de l'algorithme, on retourne $z^*$ qui est une solution optimale.
	}
	\Sinon
	{
		$j_0 \lar \argmax_j \{ c_j, j \in J \}$ (choix autre que le maximum possible tant que supérieur à $0$) \;
		\tcp{$x_{j_0}$ est la variable qui va rentrer en base}
		$i_0 \lar \argmin_i \left\{ \frac{- b_i}{a_{i,j_0}}, i \in I, \mid a_{i,j_0} < 0 \right\}$ \;
		\tcp{$x_{i_0}$ est la variable sortante}
		On extrait $x_{j_0}$ de l'expression courante de $x_{i_0}$ \;
		On remplace $x_{j_0}$ par sa nouvelle expression dans $z$ et dans l'expression des autres $x_i$ \;
		On réitère avec le nouveau dictionnaire construit \;
	}
\end{algorithm}

\begin{thm}[Théorème de Bland]
	Il ne peut y avoir cyclage lorsque, à toute itération effectuée à partir d'un dictionnaire dégénéré, on choisit les variables entrante et sortante comme celles de plus petit indice parmi les candidats possibles.
\end{thm}

On obtient le \textbf{problème auxiliaire} en ajoutant une variable $x_0$ positive telle que $\forall i \in \iniff{1}{m}, \sum_{j = 1}^n a_{ij}x_j - x_0 \leq b_i$.
Ce problème toujours réalisable en prenant $x_0$ assez grand et s'il n'était pas réalisable initialement (i.e. avec $x_0 = 0$) on peut chercher le $x_0$ minimum.

\textbf{Méthode à deux phases} : pour un dictionnaire non réalisable a priori on introduit $x_0$ et une fonction cible $w = -x_0$. On fait rentrer $x_0$ en base et on itère pour essayer de trouver un dictionnaire réalisable.
