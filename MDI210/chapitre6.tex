S'il existe $m$ réels $y_i \geq 0$ tels que $\forall j \in \iniff{1}{n}, \sum_{i = 1}^m a_{i,j} y_i \geq c_j$ alors on a, pour toute solution réalisable de $(P)$ : $z \leq \sum_{i = 1}^m b_i y_i$.
Le \textbf{problème dual} $(D)$ du problème primal $(P)$ s'écrit : minimiser $\sum_{i = 1}^m b_i y_i$ avec les contraintes $\forall j \in \iniff{1}{n}, \sum_{i = 1}^m a_{i,j} y_i \geq c_j$ et $\forall i \in \iniff{1}{m}, y_i \geq 0$.

\begin{rem}
	Le problème dual de $(D)$ est $(P)$.
\end{rem}

\begin{pop}
	Soit $(x_1^*, \ldots, x_n^*)$ une solution réalisable de $(P)$ et $(y_1^*, \ldots, y_m^*)$ une solution réalisable de $(D)$.
	On a $\sum_{j = 1}^n c_j x_j^* \leq \sum_{i = 1}^m b_i y_i^*$.
	En cas d'égalité les deux solutions sont optimales pour leurs problèmes respectifs.
\end{pop}

\begin{cor}
	Si $(P)$ admet une solution réalisable et est non bornée alors $(D)$ est infaisable.
\end{cor}

\begin{thm}[de la dualité]
	Si $(P)$ a une solution optimale $(x_1^*, \ldots, x_n^*)$ alors $(D)$ a une solution optimale $(y_1^*, \ldots, y_m^*)$ et $\sum_{j = 1}^n c_j x_j^* = \sum_{i = 1}^m b_i y_i^*$.
\end{thm}

\begin{pop}
	Si $(P)$ admet une solution primale telle que, dans son dernier dictionnaire obtenu par la méthode du simplexe, $z = z^* + \sum_{k = 1}^{n + m} d_k x_k$, alors une solution optimale de $(D)$ est donnée par $\forall i, y_i^* = - d_{n + i}$.
\end{pop}

\begin{thm}[des écarts complémentaires]
	Une solution $(x_1^*, \ldots, x_n^*)$ de $(P)$ est optimale si et seulement s'il existe $(y_1^*, \ldots, y_m^*)$ une solution de $(D)$ vérifiant :
	\begin{itemize}
		\item[\textbullet] $\forall i \in \iniff{1}{m}, \left( \sum_{j = 1}^n a_{ij} x_j^* < b \right) \implies \left( y_i^* = 0 \right)$
		\item[\textbullet] $\forall i \in \iniff{1}{n}, \left( x_j^* > 0 \right) \implies \left( \sum_{i = 1}^m a_{ij} y_i^* = c_j \right)$
	\end{itemize}
	De plus $(y_1^*, \ldots, y_m^*)$ constitue une solution optimale de $(D)$.
\end{thm}

\subsection{Signification économique du dual :}

	\begin{itemize}
		\item[\textbullet] $b_i$ est la quantité totale de la ressource $i$,
		\item[\textbullet] $a_{ij}$ est la quantité de la ressource $i$ consommée par la fabrication d'une unité de produit $j$,
		\item[\textbullet] $x_j$ est la quantité fabriquée de produit $j$,
		\item[\textbullet] $c_j$ est la valeur unitaire du produit $j$,
		\item[\textbullet] $y_i$ est le prix implicite, la valeur unitaire de la ressource $i$.
	\end{itemize}

	\begin{thm}
		Supposons que la base optimale de $(P)$ est non dégénérée.
		Pour des variations $\delta b_i$ des $b_i$ on considère le problème $(P_\delta)$ avec les contraintes linéaires $\forall i \in \iniff{1}{m}, \sum_{j = 1}^n a_{ij} x_j \leq b_i + \delta b_i$.
		On suppose que les $\delta b_i$ sont suffisamment faibles pour que la base optimale de $(P)$ soit encore réalisable pour $(P_\delta)$.
		La variation de $z$ vaut alors $\sum_{i = 1}^m \delta b_i yi^*$ où $(y_1^*, \ldots, y_m^*)$ est solution optimale de $(D)$.
	\end{thm}

L'utilisation du problème dual permet de résoudre un problème où la solution nulle n'est pas réalisable mais où $\forall j, c_j \leq 0$.
Un tel problème est dit \textbf{dual-réalisable}.
