\usepackage{amsthm} %ou \usepackage{ntheorem}
\usepackage{amsmath}
\usepackage{amssymb}
\usepackage{mathrsfs}
\usepackage{amsfonts}

\definecolor{vert}{rgb}{0,0.6,0}


% Une macro pour obtenir de grandes fractions dans les formules en ligne.
\def\frc#1#2{\displaystyle{#1\over#2}}

% Une macro pour les vecteurs qui donne de meilleurs résultats que \overrightarrow.
\def\vect#1{%
	\vbox{\lineskip=-.04em\baselineskip=0pt
	\halign{##\cr
	\leaders\hbox{$\scriptstyle{-}$\kern-.4em}\hfil$\scriptstyle{\rightarrow}$\cr
	$#1$\cr}}}

% Majuscules d'anglaise.
\DeclareSymbolFont{rsfscript}{U}{rsfs}{m}{n}
\DeclareSymbolFontAlphabet{\mathrsfs}{rsfscript}
\newcommand\scr{\mathrsfs}


% Des macros pour les notations usuelles.
\newcommand{\ensemblenombre}[1]{\mathbf{#1}}
\newcommand{\N}{\ensemblenombre{N}}
\newcommand{\Z}{\ensemblenombre{Z}}
\newcommand{\Q}{\ensemblenombre{Q}}
\newcommand{\R}{\ensemblenombre{R}}
\newcommand{\C}{\ensemblenombre{C}}
\newcommand{\K}{\ensemblenombre{K}}
\newcommand{\U}{\ensemblenombre{U}} % Groupe du cercle unité complexe
\newcommand{\T}{\ensemblenombre{T}} % Tore [0;2π[
\newcommand{\mes}{\ensemblenombre{M}} % Espace de mesures
\newcommand\M{\mathfrak{M}}
\newcommand\E{\mathcal{E}}
\newcommand\parties{\mathcal{P}}
\newcommand\GL{\mathcal{GL}}
\newcommand\Sym{\mathcal{S}}
\newcommand\aSym{\mathcal{A}}
\newcommand\proba{\mathbf{P}}
\newcommand\esp{\mathbf{E}}
\newcommand\Orth{\mathcal{O}}
\newcommand\cont{\mathcal{C}}
\newcommand\li{[\![}
\newcommand\ri{]\!]}
\newcommand{\diff}{\mathop{}\mathopen{}\mathrm{d}}
\newcommand{\abs}[1]{\left\lvert#1\right\rvert}
\newcommand{\norme}[1]{\left\lVert#1\right\rVert}
\newcommand{\norm}[1]{\left\lVert#1\right\rVert}
\newcommand{\transp}[1]{{\vphantom{#1}}^{\mathit t}{#1}}
\newcommand{\scal}[2]{\left\langle #1 \mid #2 \right\rangle}
\newcommand{\compl}[1]{{#1}^{\mathcal{C}}} % symbole du complémentaire en exposant
\newcommand\indep{\protect\mathpalette{\protect\independenT}{\perp}} % symbole d'indépendance en probas
\def\independenT#1#2{\mathrel{\rlap{$#1#2$}\mkern3mu{#1#2}}}
\newcommand\rel{\mathcal{R}} % Pour les relations binaires
\newcommand\rar{\rightarrow}
\newcommand\lar{\leftarrow}

% Notation d'ensembles en algèbre
\newcommand\Hom{\mathrm{Hom}}
\newcommand\End{\mathrm{End}} % Endomorphismes
\newcommand\Isom{\mathrm{Isom}} % Isométries
\newcommand\Aut{\mathrm{Aut}} % Automorphismes
\newcommand\Int{\mathrm{Int}}


%
% Intervalles
\newcommand{\intervalle}[4]{\mathopen{#1}#2\mathclose{}\mathpunct{};#3\mathclose{#4}}
\newcommand{\inff}[2]{\intervalle{[}{#1}{#2}{]}}
\newcommand{\inof}[2]{\intervalle{]}{#1}{#2}{]}}
\newcommand{\info}[2]{\intervalle{[}{#1}{#2}{[}}
\newcommand{\inoo}[2]{\intervalle{]}{#1}{#2}{[}}
\newcommand{\iniff}[2]{\intervalle{[\![}{#1}{#2}{]\!]}}

% Secondes définitions d'intervalles, taille ajustable
\newcommand{\genericinterval}[4]
{
	\mathopen{}\mathclose{\left#1#2\mathclose{}\mathpunct{};#3\right#4}
}
\newcommand{\intff}[2]{\genericinterval{[}{#1}{#2}{]}}
\newcommand{\intoo}[2]{\genericinterval{]}{#1}{#2}{[}}
\newcommand{\intof}[2]{\genericinterval{]}{#1}{#2}{]}}
\newcommand{\intfo}[2]{\genericinterval{[}{#1}{#2}{[}}
\newcommand{\intiff}[2]{\genericinterval{[\![}{#1}{#2}{]\!]}}


%
% Opérateurs mathématiques
%
\DeclareMathOperator{\card}{Card}
\DeclareMathOperator{\tr}{Tr}
\DeclareMathOperator{\Ker}{Ker}
\DeclareMathOperator{\Vect}{Vect}
\DeclareMathOperator{\Span}{Span} % Version anglaise de Vect
\DeclareMathOperator{\indic}{\mathbf{1}}
\DeclareMathOperator{\argth}{argth}
\DeclareMathOperator{\Id}{Id}
\DeclareMathOperator{\Gram}{Gram}
\DeclareMathOperator{\diag}{diag}
\DeclareMathOperator{\Mat}{Mat}
\DeclareMathOperator{\Sp}{Sp}
\DeclareMathOperator{\im}{Im}
\DeclareMathOperator{\Cov}{Cov}
\DeclareMathOperator{\Var}{Var}
\DeclareMathOperator{\Supp}{Supp}
\DeclareMathOperator{\sinC}{sinC}
\DeclareMathOperator{\sgn}{sgn}
\DeclareMathOperator{\argmin}{arg\, min}
\DeclareMathOperator{\cond}{cond} % Fonction conditionnement en optimisation linéaire
\DeclareMathOperator{\proj}{proj}
\DeclareMathOperator{\Bias}{Bias}
\DeclareMathOperator{\MSE}{MSE}
\DeclareMathOperator{\pgcd}{pgcd}
\DeclareMathOperator{\ppcm}{ppcm}


% Pour des symboles « inférieur ou égal », « supérieur ou égal », « ensemble vide » et « parallèles » conformes aux usages français.
\DeclareSymbolFont{AmsA}{U}{msa}{m}{n}
\SetSymbolFont{AmsA}{bold}{U}{msa}{b}{n}
\DeclareMathSymbol\leq\mathrel{AmsA}{"36}
\DeclareMathSymbol\geq\mathrel{AmsA}{"3E}
\DeclareSymbolFont{AmsB}{U}{msb}{m}{n}
\SetSymbolFont{AmsB}{bold}{U}{msb}{b}{n}
\DeclareMathSymbol\emptyset\mathord{AmsB}{"3F}
\def\parallel{\mathrel{/\!/}}
