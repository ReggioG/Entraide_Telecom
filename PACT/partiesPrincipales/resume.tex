\section{Résumé du sujet choisi en français}

	Notre projet consiste en l’élaboration d’une application exclusivement utilisable sur ordinateur qui propose à l’utilisateur une liste de lecture musicale adaptée à ses goûts mais aussi à son humeur du moment.
	Nous l’avons baptisé FeelList.
		
	Le fonctionnement de l’application est assez simple.
	De prime abord l’ordinateur prend une photo de l’utilisateur afin d’en extraire des traits du visage caractéristiques d’une certaine humeur.
	Après la détermination de l’humeur, l’application propose à l’utilisateur une liste de lecture composée de morceaux extraits de sa bibliothèque personnelle mais aussi d’une bibliothèque par défaut que nous aurons rendue publique sur un cloud.
	En plus de faire gagner du temps à l’utilisateur - qui ne sait pas choisir parmi les nombreuses pistes audio de sa bibliothèque - FeelList a aussi la qualité de faire découvrir de nouveaux morceaux à ce dernier.

	Cependant une question majeure se pose : quelle type de musique écouter dans telle ou telle situation ?
	En effet le choix d’un morceau adapté à une humeur est éminemment individuel et subjectif.
	C’est pour cela que nous permettons à l’utilisateur de choisir.
	Lors de la configuration initiale, l’application propose à l’utilisateur d’indiquer dans quelle humeur il souhaiterait écouter tel ou tel morceau extrait de sa bibliothèque personnelle.
	Le détail de cet étiquetage est présenté plus loin dans ce rapport.

	Ce projet s’inscrit parfaitement dans le thème « du geste au son ».
	En effet, à partir d’une photo nette de l’utilisateur, l’application propose une traduction musicale de certaines modalités du visage.
	Autrement dit, elle passe du geste au son.

\section{English summary}

	Our project is to build a computer application that suggests a musical playlist adapted to the user’s tastes and to their mood.
	We decided to name it FeelList.
		
	The way it goes is quite simple.
	The application starts off by taking a picture of the user in order to extract from it relevant signs of a prominent mood.
	Once FeelList has managed to determine the user’s mood it delivers a musical playlist containing tunes from his personal library but also some from a public library that would have been put on a cloud beforehand.
	In a nutshell, FeelList isn’t only a huge timesaver for people that don’t know what to listen to, it also helps users discover new authors and albums.
		
	However not everyone agrees on the type of music you want to listen to once you’re in a certain mood.
	The choice of listening to a particular type of music when you’re in a special mood is unmistakably personal and subjective.
	FeelList therefore offers such a choice.
	The first time you download the app, you have to fill in a grid in which you indicate in which mood you would like to listen a large choice of tunes from your own library.

	Our project fits in perfectly with the yearly theme : « turn movements into sounds » as it evidently associates musical playlists with face motions that transcript moods and emotions.