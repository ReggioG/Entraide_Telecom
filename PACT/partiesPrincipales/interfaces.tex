\begin{itemize}
	\item \textbf{Interface "caractérisation des visages" - "classification automatique des images"\\}
		Le bloc de classification pourra demander au bloc de caractérisation de lui fournir une matrice $n \times p$ d'informations caractérisant le visage de la personne, où $n$ est le nombre d'images exploitables dans une vidéo de 3s prise par la webcam, et $p$ est le nombre de caractérisant le visage.
	\item \textbf{Interface "classification automatique des images" - "intersection émotion répertoire"\\}
		Le bloc d'intersection pourra demander au bloc de caractérisation l'humeur de l'utilisateur.
		Celle-ci sera sous forme de chaîne de caractères.
	\item \textbf{Interface "classification automatique des images" - "intersection émotion - goûts musicaux - bibliothèque"\\}
		Le bloc d'intersection pourra demander au bloc de caractérisation l'humeur de l'utilisateur.
		Celle-ci sera sous forme de chaîne de caractères.
	\item \textbf{Interface "caractérisation audio des morceaux" - "classification automatique des morceaux"\\}
		Le bloc de classification automatique des morceaux pourra demander au bloc de caractérisation 3 matrices de taille $n \times p$ où $n$ est le nombre de morceaux (dans une bibliothèque convenue par le bloc de caractérisation) et $p$ est le nombre de paramètres (par exemple, le tempo serait l'un deux).
	\item \textbf{Interface "classification automatique des morceaux" - "intersection émotion répertoire"\\}
		Le bloc d'intersection pourra demander au bloc de classification de \textit{tagger} les musiques d'une bibliothèque donnée avec l'humeur adaptée au morceau.
	\item \textbf{Interface "classification automatique des morceaux" - "intersection émotion - goûts muicaux - bibliothèque"\\}
		Le bloc d'intersection pourra demander au bloc de classification de \textit{tagger} les musiques d'une bibliothèque donnée avec l'humeur adaptée au morceau.
	\item \textbf{Interface "apposition et affichage" - "intersection émotion répertoire"\\}
		Le bloc apposition et affichage peut demander au bloc d'intersection une playlist  de morceaux du répertoire de l'utilisateur à jouer.
	\item \textbf{Interface "apposition et affichage" - "intersection émotion répertoire"\\}
		Le bloc apposition et affichage peut demander au bloc d'intersection une playlist  de morceaux de la bibliothèque par défaut à jouer.
\end{itemize}