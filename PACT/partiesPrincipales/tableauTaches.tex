\begin{tabular}{llm{1.5cm}m{1.5cm}}
	Tâche & Description                                                              & Démontrée au PAN3    & Intégrée au PAN3    \\
	\hline \hline
	T1    & \multicolumn{3}{c}{\cellcolor{LightSteelBlue} Prendre en compte les besoins utilisateur (module Focus Group)}         \\
	T1.1  & Réaliser un guide de discussion                                          &                      & \cs                 \\
	T1.2  & Contacter les participants à la session                                  &                      & \cs                 \\
	T1.3  & Tenir la réunion et filmer, noter, animer, observer                      &                      & \cs                 \\
	T1.4  & Rédiger un rapport sur la réunion                                        &                      & \cs                 \\
	\hline
	T2    & \multicolumn{3}{c}{\cellcolor{LightSteelBlue} Mener une étude de l'impact vie privée de l'application (module Vie privée)}     \\
	T2.1  & Réaliser la première partie du rapport (contextualisation et pertinence) &                      & \cs                 \\
	T2.2  & Réaliser la deuxième partie du rapport sur l'étude des risques           &                      & \cs                 \\
	\hline
	T3    & \multicolumn{3}{c}{\cellcolor{LightSteelBlue} Classifier les images et morceaux par émotions (module classification)} \\
	T3.1  & Réaliser un pseudo-code du classifieur                                   &                      & \cs                 \\
	T3.2  & Implémenter ce code en Java                                              & \cs                  &                     \\
	T3.3  & Valider le classifieur sur des données test                              & \cs                  &                     \\
	T3.4  & Évaluer son efficacité sur nos échantillons (morceaux et images)         & \cs                  &                     \\
	\hline
	T4    & \multicolumn{3}{c}{\cellcolor{LightSteelBlue} Modifier l'interface graphique préexistante (module intégration)}       \\
	T4.1  & Étudier l'interface graphique de aTunes                                  &                      & \cs                 \\
	T4.2  & Réaliser les éléments graphiques (logo, smileys)                         &                      & \cs                 \\
	T4.3  & Intégrer nos créations à l'interface                                     &                      & \cs                 \\
	\hline
	T5    & \multicolumn{3}{c}{\cellcolor{LightSteelBlue} Associer une playlist à une émotion (module recommandation)}            \\
	T5.1  & Définir les paramètres sur lesquels se base la sélection (protocole)     &                      & \cs                 \\
	T5.2  & Coder l'algorithme de constitution de playlist (système fonctionnel)     & \cs                  &                     \\
	\hline
	T6    & \multicolumn{3}{c}{\cellcolor{LightSteelBlue} Création des bases de données (module intégration)}                     \\
	T6.1  & Index des musiques                                                       &                      & \cs                 \\
	T6.2  & Base de données des émotions                                             &                      & \cs                 \\
	T6.3  & Émulation d'une base de musiques en ligne                                & \cs                  &                     \\
	\hline
	T7    & \multicolumn{3}{c}{\cellcolor{LightSteelBlue} Assemblage des différents éléments (module intégration)}                \\
	T7.1  & Étude préalable du code de aTunes                                        &                      & \cs                 \\
	T7.2  & Remplacement de ce qui est modifié par rapport au code d'origine         &                      & \cs                 \\
	T7.3  & Ajout des codes des autres modules pour les fonctionnalités spécifiques  & \cs                  &                     \\
	\hline
	T8    & \multicolumn{3}{c}{\cellcolor{LightSteelBlue} Extraire les données des images (module traitement image)}              \\
	T8.1  & Redressement géométrique d'une image                                     &                      & \cs                 \\
	T8.1  & Transformation RVB/HSV                                                   &                      & \cs                 \\
	T8.1  & Repérage et analyse des composantes connexes                             &                      & \cs                 \\
	T8.1  & Lecture/écriture dans un fichier                                         &                      & \cs                 \\
	T8.1  & Comparaison de deux visages                                              & \cs                  &                     \\
	\hline
	T9    & \multicolumn{3}{c}{\cellcolor{LightSteelBlue} Analyse de caractéristiques audio (module audio MFCC)}                  \\
	T9.1  & Codage et tests de comparaison des MFCC                                  &                      & \cs                 \\
	T9.2  & Intégration avec le module classification                                & \cs                  &
\end{tabular}