\setlength{\fboxsep}{3.4mm}

\begin{wrapfigure}{r}{0.35\textwidth}
	\fbox{
	\begin{minipage}[c]{5cm}
		\emph{Matériel :}
		\begin{itemize}
			\item Ordinateur : PC ou Mac.
			\item Webcam : intégrée à l'ordinateur ou clipsable et reliée à l'ordinateur.
		\end{itemize}
	\end{minipage}
	}
\end{wrapfigure}

Pierre adore écouter de la musique, si bien qu’il possède quelques centaines de musiques dans son répertoire de morceaux sur son ordinateur.
Cependant, il lui arrive parfois de ne pas savoir quel morceau écouter, du fait de cette trop grande diversité.
Un jour, alors qu’il a invité son ami Paul chez lui, ce dernier est surpris de voir Pierre passer de longues minutes à choisir le morceau qu’ils vont écouter, puis s'irrite de le voir systématiquement interrompre chaque morceau qui a l'air prometteur au bout de quelques secondes pour en proposer un autre.
Paul conseille donc à Pierre de télécharger l’application FeelList pour lui faire gagner du temps dans sa recherche de la musique la plus adéquate.

Cette application suit un principe simple : à partir de très courtes vidéos du visage de l’utilisateur, l'application identifie son expression et l'associe à un état ou une émotion (joie, tristesse, agacement, fatigue...), puis lui propose une liste de morceaux en adéquation avec son humeur et ses goûts musicaux.
Elle a pour vocation d'être à la fois une aide au choix personnalisée et un moyen de découverte musicale pour des utilisateurs assez statiques positionnés devant la webcam.
C'est exactement ce qu'il lui faut !

Pierre décide alors de télécharger l’application et de l’essayer tout de suite.\\

\begin{wrapfigure}{l}{0.42\textwidth}
	\fbox{
	\begin{minipage}[c]{6cm}
		\emph{Installation et initialisation de FeelList sur PC/Mac :}
		\begin{itemize}
			\item L'utilisateur crée un profil avec son nom d’utilisateur, une photo neutre de son visage.
			\item Il coche, parmi une liste générique, les styles de musique qu’il préfère (rock, rap, RnB,...).
		\end{itemize}
	\end{minipage}
	}
\end{wrapfigure}

Il se place devant la webcam de son ordinateur.
L’application prend à intervalle régulier de très courtes vidéos et détecte son expression (joyeuse car son ami est chez lui).
Puis, elle se connecte via internet à la banque son par défaut FeelList qui se trouve sur le \textit{cloud}.
Les morceaux de cette banque par défaut ont été arbitrairement répartis par catégories d' « émotions »  et classées par styles.
L'application sélectionne et propose alors une liste de musiques joyeuses correspondant à l'état d'esprit de Pierre et respectant les critères de style musical qu'il a choisis lors de la création de son profil.
Pierre est enchanté de ces propositions, d'autant plus qu'elles lui permettent de découvrir de nouveaux morceaux !\\

\begin{wrapfigure}{r}{0.55\textwidth}
	\fbox{
	\begin{minipage}[c]{7.6cm}
		\emph{Intégration de morceaux de son propre répertoire musical aux propositions de FeelList :}
		\begin{itemize}
			\item L'utilisateur se connecte sur son profil.
			\item Il importe son répertoire musical.
			\item Il classe tous ou une partie de ses morceaux en les taggant : à côté de chaque titre se trouvent différents smileys cliquables (parmi lesquelles « \textit{Angry} », « \textit{Tired} », « \textit{Happy} »...).
		\end{itemize}
	\end{minipage}
	}
\end{wrapfigure}

Dans la semaine, alors que Pierre travaille sur son ordinateur, il décide de réutiliser l'application pour travailler en musique.
Cependant, il voudrait pouvoir écouter aussi les morceaux figurant dans son propre répertoire en plus de ceux de la banque son par défaut.
À son agréable surprise, il découvre que c'est possible.

Par exemple, il appose le smiley « \textit{Angry} » aux morceaux qu'il aime écouter lorsqu'il est agacé, qu'ils soient tranquilles pour l'apaiser ou au contraire forts et expressifs pour l'aider à évacuer sa colère, en fonction de ce qu'il préfère écouter selon son humeur.
Ensuite, tout au long de sa soirée de travail, l'application prend de courtes vidéos du visage de Pierre à intervalles réguliers pour cerner son humeur, et lui propose d'écouter un panel de morceaux issus pour certains de son propre répertoire, les autres venant de la banque par défaut.
Les propositions évoluent avec ses changements prolongés d'humeur.

Cependant, si Pierre n'étiquette aucun morceau ou pas assez, FeelList ne proposera que des morceaux issus de la banque par défaut.
De plus, si Pierre n'est pas connecté à internet, il n'aura accès qu'au « mode hors ligne » de l'application, c'est-à-dire que les morceaux proposés ne seront issus que de son propre répertoire.
S'il n'a ni étiqueté de morceaux, ni de connexion internet, l'application ne fonctionne pas en mode détection des émotions et il sera obligé de lancer des musiques “à la main” comme sur un lecteur classique.\\

\begin{wrapfigure}{l}{0.42\textwidth}
	\fbox{
	\begin{minipage}[c]{6cm}
		\emph{Option film :}
		\begin{itemize}
			\item Ils clipsent une petite caméra au-dessus de l'écran plat de Jacques.
			\item Ils la relient à l'ordinateur.
			\item Jacques se connecte sur son profil (qu'il vient de créer).
			\item Il clique sur « Option film » et entre sa durée.
			\item Il augmente la luminosité dans la pièce si FeelList lui affiche un message signalant qu'elle ne peut pas procéder à la détection du visage et l'analyse des émotions dans la pénombre.
		\end{itemize}
	\end{minipage}
	}
\end{wrapfigure}

Deux mois plus tard Pierre et Paul sont invités à aller voir un film chez Jacques, un autre de leurs amis, à l'occasion de l'anniversaire de ce dernier.
Ils projettent ensuite de passer la fin de la soirée ensemble, et en musique. Pierre et Paul parlent alors de FeelList à Jacques qui semble tout de suite intéressée par le concept.

L'application détecte le visage de Jacques et se focalise sur uniquement sur celui-ci en ignorant Pierre et Paul, puisque c'est le profil de Jacques qui est utilisé.
Elle est programmée pour se mettre en route et débuter la prise de photos à intervalles réguliers et leur analyse 10 minutes avant la fin du film.
Cependant, ils interrompent le film à plusieurs reprises car Jacques reçoit de nombreux coups de fils pour son anniversaire : FeelList en prend compte, car elle détecte quand l’utilisateur n’est plus dans son champ de vision et décale sa mise en route de la durée des absences de Jacques.

Les trois amis sont très émus par la fin du film, et l'application leur propose une série de morceaux, issus à la fois du répertoire de Jacques et de la banque son par défaut, en accord avec l'expression relativement attristée de Jacques et ses goûts musicaux.
Pierre, Paul et Jacques passent alors une très bonne soirée, satisfaits d’avoir pu garder l’atmosphère du film bien qu’il soit fini, les propositions évoluant ensuite avec l'enthousiasme croissant de Jacques qui reste assis devant la caméra à bavarder joyeusement avec ses amis.