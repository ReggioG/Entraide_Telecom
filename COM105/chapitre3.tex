\subsection{Canal de propagation}

	\begin{hyp}
		Le bruit $b(t)$ est i.i.d. gaussien de moyenne nulle, de fonction d'autocorrélation $r_{bb}(\tau) := \esp(b(t + \tau) b(t))$ et satisfait $r_{bb}(\tau) = \frac{N_0}{2}$.
	\end{hyp}
	
	\begin{pop}
		Soit $x(t)$ le signal émis et $y(t)$ le signal reçu.
		Le canal multi-trajets conduit à $y(t) = c_p(t) \star x(t) + b(t)$.
	\end{pop}
	
	Lorsque $c_p(t) = \delta(t)$ le canal est appelé \textbf{canal gaussien}, car seul le bruit gaussien vient perturber la transmission.
	Dans ce cas $y(t) = x(t) + b(t)$.
	C'est notamment vrai dans les cas suivants :
	\begin{itemize}
	\item[\textbullet] Faisceaux hertziens : entre antenne fixes avec une visibilité directe entre elles $\rar$ antenne émettrice directive orienté vers l'antenne de réception $\rar$ ni dispersion, ni écho.
	\item[\textbullet] Liaisons satellitaires : en première approximation l'onde ne subit pas d'obstacle entre l'émission par le satellite et la réception par une antenne parabolique.
	\item[\textbullet] Réseaux câblés co-axiaux : produisent très peu de multitrajets.
	\end{itemize}
	
	Lorsque $c_p(t) \neq \delta(t)$, le canal est appelé \textbf{canal sélectif en fréquence}, car alors $Y(f) = C_p(f) X(f) + B(f)$ (en prenant les TF) $\rar$ $C_p$ n'est plus constante et donc agit différemment selon les fréquences.
		

\subsection{Description de l'émetteur}

	\begin{defn}
	Transmission d'un signal provenant d'un code correcteur d'erreur, mathématiquement on a :
	$$x(t)=\sum_{n=0}^{N-1} s_{n}g(t-nT_{s})$$
	avec $g(t)$ le filtre d'émission, $\{s_{n}\}_{n}$ la suite de symboles s'exprimant en fonction des données et $T_{s}$ le temps-symbole.
	\end{defn}
	
	\begin{defn}
	On note \textbf{$\mathcal{M}$} l'ensemble des valeurs possibles pour chaque symbole $s_{n}$ et \textbf{$M = card(\mathcal{M})$} le \textbf{le nombre de valeurs possibles pour chaque symbole}.
	
	Une fois \textbf{$M$} fixé, on appelle \textbf{constellation} la manière dont sont répartis les \textbf{$M$} valeurs possibles des symboles sur l'axe des réels. Voici les deux constellations les plus utilisées :
	\begin{itemize}
	\item $M$-OOK : $\{0;A;2A;...;(M-2)A;(M-1)A\}$, les valeurs sont espacées de $A$.
	\item $M$-PAM : $\{-(M-1)A;-(M-3)A;...;-A;A;...;(M-3)A;(M-1)A\}$, les valeurs sont espacées de $2A$.
	\end{itemize}
	\end{defn}
	
	\begin{defn}
	On note \textbf{$\mathcal{M}=\{s^{(m)}\}_{m=0,\cdots,M-1}$}. Ainsi, à l'instant $n$, $s_{n}$ prend une des valeurs $s^{(m)}$.
	Comme la suite des symboles est i.i.d, on définit la moyenne symbole \textbf{$m_{s}$}, la variance symbole \textbf{$\sigma^{2}_{s}$} et \textbf{l'énergie symbole $E_{s}$}. Ces variables ne dépendent pas de $n$; on les définit ainsi :
	$$m_{s}=\frac{1}{M}\sum_{m=0}^{M-1}s^{(m)} \qquad
	\sigma^{2}_{s}=\frac{1}{M}\sum_{m=0}^{M-1}(s^{(m)}-m_{s})^2 \qquad
	E_{s}=\frac{1}{M}\sum_{m=0}^{M-1}s^{(m)^2}=m_s^2+\sigma_s^2$$
	\end{defn}	
	
	\begin{pop}
		Résultats à connaître (pour aller plus vite) :
		$\begin{array}{c|ccc}
		           & m_s                & \sigma_s^2               & E_s \\ \hline
		\text{OOK} & \frac{A(M - 1)}{2} & \frac{A^2 (M^2 - 1)}{12} & \frac{A^2(2M^2 - 3M + 1)}{6} \\
		\text{PAM} & 0                  & \frac{A^2 (M^2 - 1)}{3}  & \frac{A^{2}(M^{2}-1)}{3}
		\end{array}$
	\end{pop}
	
	\begin{pop}
		L'énergie consommée pour émettre un bit d'information (\textbf{énergie bit}) s'écrit :
		$$E_b = \frac{1}{\log_{2}(M)} \left( E_{s}E_{g}+m_{s} \sum_{n\neq 0} h_{n} \right)$$
		Avec $E_g=\int g(t)^2dt$ l'énergie du filtre d'émission et $h_n=h(nT_s)$ où $h(t)=g(-t)\star g(t)$.
		Sauf indication contraire, le filtre choisi amènera toujours à $E_g=1$ et $m_s=0$ \textbf{ou} $h_n=0$ pour tout $n\neq 0$. Ainsi, on retiendra :
		$$E_b = \frac{E_s}{\log_2(M)}$$
	\end{pop}

\subsection{Description du récepteur}

	\begin{pop}
		La suite optimale au sens de la probabilité d'erreur est la suivante :
		$$z_{n} = z(nT_{s}) \quad \text{avec} \quad z(t)=g(-t)*y(t)\ .$$
	\end{pop}
	
	\begin{pop}
	Dans le contexte d'un canal gaussien on a $z_{n} = h_{n} \star s_{n} + w_{n}$.
	\end{pop}
	
	\begin{defn}[Filtre de Nyquist, Filtre en racine de Nyquist]
	
		\begin{itemize}
		
		\item Un filtre de réponse impulsionnelle $l(t)$ est dit de Nyquist, si et seulement si :
		$$l_{n} = l(nT_{s}) =
		\left\{ \begin{array}{rcr}
			\neq 0 & \text{pour} & n = 0 \\
			0      & \text{pour} & n \neq 0 \\
		\end{array} \right.$$
		\item Un filtre est dit en racine de Nyquist si et seulement si le filtre $l(-t)\star l(t)$ est un filtre de Nyquist. Autrement dit, le filtre convolué à son filtre adapté est de Nyquist.
		\end{itemize}
	\end{defn}
	
	\begin{pop}[Canal gaussien à temps discret]
		Une fois la contrainte de filtre de Nyquist vérifiée par $h(t)$, l'équation $z_{n} = h_{n} \star s_{n} + w_{n}$ se simplifie en :
		$$z_{n} = s_{n} + w_{n}$$
		Avec $w_n$ un bruit blanc gaussien de variance $\frac{N_0}{2}$.
	\end{pop}
	
	\begin{pop}
		La largeur de bande, notée $B$, de tout filtre de Nyquist ou en racine de Nyquist vérifie $B\geq \dfrac{1}{T_{s}}$.
	\end{pop}
	
\subsection{Détecteur optimal}

	\begin{pop}
		Soit $\{s_n\}_n$ une suite de symboles et le canal gaussien à temps discret donnée plus haut ($z_{n} = s_{n} + w_{n}$). Alors le détecteur optimal obtient le symbole $\hat{s}_n$ de la manière suivante :
		$$\hat{s}_n =
		\left\{ \begin{array}{ccl}
			s^{(0)} & \text{si} & z_n\in ]-\infty ,t^{(0)}] \\
			s^{(m)} & \text{si} & z_n\in ]t^{(m-1)} ,t^{(m)}]\;\text{pour}\; m\in \{1,\cdots, M-2\} \\
                        s^{(M-1)} & \text{si} & z_n\in ]t^{(M-2)} ,+\infty[ \\
		\end{array}\right .$$
		Avec, pour $m\in \{0,\cdots,M-2\}$, les seuils suivants : $t^{(m)} = \dfrac{s^{(m)}+s^{(m+1)}}{2}$.
	\end{pop}

\subsection{Performances}
	
	\begin{pop}
		Si l'étiquetage permet d'avoir seulement un bit de différent entre deux symboles adjacents, alors on a cette relation :
		$$P_b=\frac{1}{\log_2(M)}P_e$$
		Avec $P_b$ et $P_e$ les probabilités d'erreur bit et symbole respectives.
		(NdR : En réalité, le symbole $=$ dans l'équation plus haut est un $\approx$. Cependant, pour les applications en COM105, on a bien un $=$. Ces propos ont été affirmés par Philippe CIBLAT lors du dernier amphi de cours)
	\end{pop}	
	
	\begin{pop}
		Si l'hypothèse sur l'émetteur est vérifiée, la constellation 2-PAM admet les performances suivantes :
		$$P_b = P_e = Q \left( \sqrt{2 \frac{E_b}{N_0}} \right) \quad \text{avec} \quad Q := x \mapsto \frac{1}{\sqrt{2\pi}} \int_x^{+\infty} e^{-u^2/2} \diff u\ .$$
	\end{pop}
	
	%$$P_\text{dB} = 10 \log_{10} (P)$$	