\subsection{Canal de propagation}

	\begin{hyp}
		Le bruit $b(t)$ est i.i.d. gaussien de moyenne nulle, de fonction d'autocorrélation $r_{bb}(\tau) := \esp(b(t + \tau) b(t))$ et satisfait $r_{bb}(\tau) = \frac{N_0}{2}$.
	\end{hyp}
	
	\begin{pop}
		Soit $x(t)$ le signal émis et $y(t)$ le signal reçu.
		Le canal multi-trajets conduit à $y(t) = c_p(t) \star x(t) + b(t)$.
	\end{pop}
	
	Lorsque $c_p(t) = \delta(t)$ le canal est appelé \textbf{canal gaussien}, car seul le bruit gaussien vient perturber la transmission.
	Dans ce cas $y(t) = x(t) + b(t)$.
	C'est notamment vrai dans les cas suivants :
	\begin{itemize}
	\item[\textbullet] Faisceaux hertziens : entre antenne fixes avec une visibilité directe entre elles $\rar$ antenne émettrice directive orienté vers l'antenne de réception $\rar$ ni dispersion, ni écho.
	\item[\textbullet] Liaisons satellitaires : en première approximation l'onde ne subit pas d'obstacle entre l'émission par le satellite et la réception par une antenne parabolique.
	\item[\textbullet] Réseaux câblés co-axiaux : produisent très peu de multitrajets.
	\end{itemize}
	
	Lorsque $c_p(t) \neq \delta(t)$, le canal est appelé \textbf{canal sélectif en fréquence}, car alors $Y(f) = C_p(f) X(f) + B(f)$ (en prenant les TF) $\rar$ $C_p$ n'est plus constante et donc agit différemment selon les fréquences.
		

\subsection{Description de l'émetteur}

	\begin{defn}
	\textit{Transmission d'un signal provenant d'un code correcteur d'erreur, mathématiquement on a : }
	$$x(t)=\sum_{n=0}^{N-1} s_{n}g(t-nT_{s})$$
	avec $g(t)$ la porte de longuer $T_{s}$, la suite de symbole avec : $s_{n}=Ac_{n}$
	\end{defn}
	
	\begin{pop}
	Si on veut augmenter le débit $D$, on peut augmenter le nombre de symbole possibles afin d'augmenter le nombre de bits par symbole : $T_{b}= \frac{T_{b}}{\log_{2}(M)}$, on peut donc chercher à augmenter $M$ pour améliorer le débit.
	\end{pop}
	
	\begin{pop}
		Une fois la valeur de $M$ choisie, il faut disposer les valeurs de $M$ pour former une \textbf{constellation}.
		\begin{itemize}
		\item M-OOK : Disposition de manière régulière.
		\item M-PAM : Disposition régulière également mais on s'autorise des phase (dans le cadre des valeurs réelles, on peut donc avoir des amplitude négatives). 
		\end{itemize}
	\end{pop}

	\begin{pop}
		Résultats à connaître (pour aller plus vite) :
		\begin{itemize}
		\item OOK : $m_{s}$ = $A(M-1)/2$;     $\sigma_{s}^{2}$=$A^{2}(M^{2}-1)/12$;     $E_{s}=A^{2}(2M^{2}-3M+1)/6$
		\item PAM : $m_{s}$ = 0;    $\sigma_{s}^{2}$=$A^{2}(M^{2}-1)/3$;     $E_{s}=A^{2}(M^{2}-1)/3$;
		\end{itemize}
		\end{pop}
		\begin{pop}
		\textit{L'énergie consommée pour émettre un bit d'information s'écrit : }
		$$E_{b}=\frac{1}{\log_{2}(M)} \left( E_{s}E_{g}+m_{s} \sum_{n\neq 0} h_{n} \right)$$
		Avec $E_{g}$ l'énergie du filtre d'émission.
	\end{pop}

\subsection{Description du récepteur}

	\begin{pop}
		La suite optimale au sens de la probabilité d'erreur est la suivante :
		$$z_{n} = z(nT_{s}) \quad \text{avec} \quad z(t)=g(-t)*y(t)\ .$$
	\end{pop}
	
	\begin{pop}
	\textit{Dans le contexte d'un canal gaussien on a : }
	$$z_{n} = h_{n} \star s_{n} + w_{n}$$
	\end{pop}
	
	\begin{defn}[Filtre de Nyquist]
		Un filtre de réponse impulsionnelle $l(t)$ est dit de Nyquist, si et seulement si :
		$$l_{n} = l(nT_{s}) =
		\left\{ \begin{array}{rcr}
			\neq 0 & \text{pour} & n = 0 \\
			0      & \text{pour} & n \neq 0 \\
		\end{array} \right.$$
	\end{defn}
	
	\begin{pop}
	\textit{La largeur de bande notée B, de tout filtre de Nyquist ou en racine de Nyquist vérifie} $$B\geq \dfrac{1}{T_{s}}$$
	\end{pop}