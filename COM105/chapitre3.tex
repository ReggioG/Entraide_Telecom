\subsection{Canal de propagation}

	\begin{hyp}
		Le bruit $b(t)$ est i.i.d. gaussien de moyenne nulle, de fonction d'autocorrélation $r_{bb}(\tau) := \esp(b(t + \tau) b(t))$ et satisfait $r_{bb}(\tau) = \frac{N_0}{2}$.
	\end{hyp}
	
	\begin{pop}
		Soit $x(t)$ le signal émis et $y(t)$ le signal reçu.
		Le canal multi-trajets conduit à $y(t) = c_p(t) \star y(t) + b(t)$.
	\end{pop}
	
	Lorsque $c_p(t) = \delta(t)$ le canal est appelé \textbf{canal gaussien}, car seul le bruit gaussien vient perturber la transmission.
	Dans ce cas $y(t) = x(t) + b(t)$.
	C'est notamment vrai dans les cas suivants :
	\begin{itemize}
	\item[\textbullet] Faisceaux hertziens : entre antenne fixes avec une visibilité directe entre elles $\rar$ antenne émettrice directive orienté vers l'antenne de réception $\rar$ ni dispersion, ni écho.
	\item[\textbullet] Liaisons satellitaires : en première approximation l'onde ne subit pas d'obstacle entre l'émission par le satellite et la réception par une antenne parabolique.
	\item[\textbullet] Réseaux câblés co-axiaux : produisent très peu de multitrajets.
	\end{itemize}
	
	Lorsque $c_p(t) \neq \delta(t)$, le canal est appelé \textbf{cana sélectif en fréquence}, car alors $Y(f) = C_p(f) X(f) + B(f)$ (en prenant les TF) $\rar$ $C_p$ n'est plus constante et donc agit différemment selon les fréquences.
		

\subsection{Description de l'émetteur}