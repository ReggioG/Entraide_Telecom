\subsection{Régularisation}

	Il est utile qu'une (sous-)martingale soit la plus régulière possible.
	
	\begin{thm}[Régularisation]
		Soit $X = (X_t)$ une sous-martingale pour la filtration standard $\mathcal{F} = (\mathcal{F}_t)$.
		Si $t \mapsto \esp X_t$ est continue à gauche, alors $(X)$ admet une modification, càdlàg qui est une $(\mathcal{F}_t)$-sous-martingale.
		En particulier, toute martingale admet une modification.
	\end{thm}
	
	Dans toute la suite du cours, les sous-martingales sur $\mathbf{T} = \R_+$ seront supposées càdlàg et la filtration standard.


\subsection{Théorème d'arrêt}

	\begin{lem}
		Soit $X$ une v.a intégrable.
		Soit $\zeta$ une famille de tribus de $\mathcal{F}$.
		Alors la famille $\{ \esp[X \mid \mathcal{G}], \mathcal{G} \in \zeta \}$ est uniformément intégrable.
	\end{lem}
	
	\begin{lem}
		Soit $X$ et $Y$ deux v.a intégrables par une tribu $\mathcal{G}$.
		Si $\forall A \in \mathcal{G}, \esp [\indic_1 X] \geq \esp [\indic_1  Y]$ alors $X \geq Y$ p.s.
	\end{lem}
	
	\begin{thm}[Théorème d'arrêt 1]
		Soit $X = (X_t)$ une martingale et soit $\vartheta$ et $\varsigma$ deux temps d'arrêt tels que $\vartheta \leq \varsigma \leq K$ où $K$ est constante.
		Alors $X_{\varsigma}$ et $X_{\vartheta}$ sont dans $\mathcal{L}^1$ et $\esp[ X_{\varsigma} \mid \mathcal{F}_{\vartheta}] = X_{\vartheta}$ p.s.
	\end{thm}
	
	Ce théorème se généralise facilement sur une martingale.
	
	\begin{thm}[Théorème d'arrêt 2]
		Soit $X = (X_t)$ une  martingale telle que $X_t = \esp [Z \mid \mathcal{F}_t]$ p.s, où $Z \in \mathcal{L}^1$.
		Si $\vartheta \leq \varsigma$ sont deux temps d'arrêt, alors $X_{\varsigma}, X_{\vartheta} \in \mathcal{L}^1$ et $\esp [X_{\varsigma} \mid \mathcal{F}_{\vartheta}] = X_{\vartheta}$ p.s.
	\end{thm}
	
	\begin{thm}
		Si $X$ est une $\mathcal{F}_t$-martingale et $\varsigma$ est un temps d'arrêt, alors le processus arrêté $X^{\varsigma} = (X_{t \wedge \varsigma})_{t \in \varsigma}$ est une martingale.
	\end{thm}


\subsection{Convergences, inégalités maximales}

	\begin{thm}
		Soit $X$ une sous-martingale telle que $\sup_t \esp X_t^+ < \infty$.
		Alors $X_t$ converge p.s vers une v.a $X_\infty \in \mathcal{L}^1$.
	\end{thm}
	
	\begin{cor}
		Toute sous-martingale positive $X$ converge p.s vers une v.a $X_\infty \geq 0$.
	\end{cor}
	
	\begin{thm}[Inégalités maximales]
		Soit $X$ une sous-martingale.
		Alors $\forall a > 0, \forall t \geq 0, \proba \left[ \sup_{s \in \intff{0}{1}} X_s > a \right] \leq \frac{\esp \abs{X_t}}{a}$.
		Si $X$ est une martingale ou une sous-martingale positive et si $\forall t \geq 0, X_t \in \mathcal{L}^p$ avec $p > 1$, alors $\forall a > 0, \forall t \geq 0$,
		$$\norme{\sup_{s \in \intff{0}{1}} \abs{X_s}}_p \leq \frac{p}{p - 1} \norme{X_t}_p
			\qquad \text{et} \qquad
			\norme{\sup_{t \in \R_+} \abs{X_t}}_p \leq \frac{p}{p - 1} \norme{X_t}_p$$
		où $\norme{Z}_p := \left( \esp \left[ \abs{Z}^p \right] \right)^{1/p}$.
	\end{thm}
	
	\begin{thm}
		Soit $X$ une martingale bornée dans $\mathcal{L}^p$ où $p > 1$, i.e $\sup_{t \in \mathbf{T}} \esp \left[ \abs{X_t}^p \right] < \infty$.
		Alors $X$ converge p.s et dans $\mathcal{L}^p$.
	\end{thm}
	
	\begin{thm}
		Soit $X$ une martingale.
		Alors les trois assertions suivantes sont équivalentes :
		\begin{enumerate}[(i)]
			\item La famille $(X_t)_{t \in \R_+}$ est uniformément intégrable.
			\item $X_t$ converge dans $\mathcal{L}^1$ pour $t \longrightarrow \infty$.
			\item $\exists Z \in \mathcal{L}^1, X_t = \esp \left[ Z \mid \mathcal{F}_t \right]$ p.s.
		\end{enumerate}
		Par ailleurs, pour tout temps d'arrêt $\varsigma$, $X_{\varsigma} = \esp [Z \mid \mathcal{F}_{\varsigma}]$ où $Z$ est la v.a décrite en \textit{(iii)}.
	\end{thm}


\subsection{Martingales de carré intégrable}

	\begin{defn}
		Une martingale $(X_t)$ est dite de carré intégrable si $\forall t \geq 0, \esp \left[ X_t^2 \right] < \infty$.
	\end{defn}
	
	Par extension directe du cas discret, nous avons :
	\begin{itemize}
		\item[\textbullet] $\forall s \in \intfo{0}{t}, \esp \left[ (X_t - X_s)^2 \mid \mathcal{F}_s \right] = \esp \left[ X_t^2 - X_s^2 \mid \mathcal{F}_s \right]$,
		\item[\textbullet] $X_t$ est à accroissements orthogonaux : $\forall 0 \leq u < v \leq s < t, \esp [(X_t - X_s)(X_v - X_u)] = 0$.
		\item[\textbullet] Pour toute subdivision $0 = t_0 \leq t_1 < \ldots < t_n = t$, $\esp \left[ (X_t - X_0)^2 \right] = \sum_{i = 1}^n \esp \left[ (X_{t_i} - X_{t_{i - 1}})^2 \right]$.
	\end{itemize}
	
	On s'intéresse dans toute la suite à l'espace
	$$\mathbf{H}_c^2 = \left\{ X \mid X\ \text{est une martingale continue}, X_0 = 0, \sup_t \esp X_t^2 < \infty \right\}\ .$$
	Plus exactement, $\mathbf{H}_c^2$ est l'ensemble des classes d'équivalence à l'indistinguabilité près.
	
	\begin{thm}
		Soit $X \in \mathbf{H}_c^2$.
		Alors :
		\begin{enumerate}[(i)]
			\item $X_t \underset{t \to \infty}{\longrightarrow} X_{\infty}$ p.s et dans $\mathcal{L}^2$.
			\item Soit $(X^n)$ une suité d'éléments de $\mathbf{H}_c^2$ telle que $X_\infty^n \underset{t \to \infty}{\longrightarrow} Z$ dans $\mathcal{L}^2$.
				Alors $\exists X \in \mathbf{H}_c^2$ telle que $Z = X_\infty$ p.s et $\forall t, X_t^n \longrightarrow X_t$ dans $\mathcal{L}^2$.
			\item L'espace $\mathbf{H}_c^2$ est un Hilbert muni du produit scalaire $\scal{X}{Y} = \esp [X_\infty Y_\infty]$.
		\end{enumerate}
	\end{thm}


\begin{ex}[Temps d'atteinte d'un niveau]
	Soit $a, b \geq 0$ et $X_t := B_t - bt$ où $B$ est un MB.
	On note $\varsigma_a := \inf \{ t \mid X_t = a \}$ et $T_a := \inf \{ t \mid B_t = a \}$.
	\begin{enumerate}[1)]
		\item \textit{Posons $\forall t \geq 0, \forall u \in R, M_t^u := \esp \left( u B_t - \frac{u^2 t}{2} \right)$.
			Montrer que $(M_t^u)_{t \in \R_+}$ est une martingale.
			Quelle est son espérance ?}
			
			On a vu que $X$ sur $\R$ est un MB ssi $\forall \theta \in \R, M_t^\theta := \exp (i \theta X_t - \frac{\theta^2 t}{2})$ est une martingale.
		\item \textit{En choisissant convenablement $u$, calculer $\esp[e^{-\lambda \varsigma_a} \indic_{\varsigma_a < \infty}]$, $\lambda \geq 0$.
			Indication : appliquer le théorème d'arrêt à $Z_a \wedge u$ et $0$.}
			
		\item \textit{En déduire $\proba[\varsigma_a < \infty]$.
			Qu'obtient-on en prenant $b = 0$ ?}
			
		\item \textit{Posons $S_t = \sup \{ B_u \mid u \in \intff{0}{t} \}$.
			Montrer que $T_a \overset{\mathcal{L}}{=} a^2 T_1$ et $T_1 \overset{\mathcal{L}}{=} \frac{1}{S_1^2}$.}
	\end{enumerate}
\end{ex}