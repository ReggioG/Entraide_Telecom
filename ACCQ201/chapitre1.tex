\subsection{Relations d'équivalence et structures quotient}

	\begin{defn}
		Une \textbf{relation} $\rel$ sur un ensemble $E$ est la donnée d'une partie $\rel \subset E \times E$.
		On écrit $x \rel y$ si $(x,y) \in \rel$.
	\end{defn}

	\begin{defn}
		Une relation $\rel$ sur $E$ est dite :
		\begin{itemize}
			\item[\textbullet] réflexive si $\forall x \in E, x \rel x$,
			\item[\textbullet] symétrique si $\forall x, y \in E, \left( x \rel y \right) \implies \left( y \rel x \right)$,
			\item[\textbullet] transitive si $\forall x, y, z \in E, \left( x \rel y \text{ et } y \rel z \right) \implies x \rel z$
		\end{itemize}
		et on dit que c'est une relation d'équivalence si ces trois conditions sont vérifiées.
		Dans ce cas on note aussi $x \sim_\rel y$ ou encore $x \equiv y \mod \rel$.
	\end{defn}

	\begin{defn}
	Soit $\sim$ une relation d'équivalence sur $E$ et $A \subset E$.
	On dit que $A$ est une classe d'équivalence pour la relation $\sim$ si $A$ est non vide, $\forall x,y \in A, x \sim y$ et $\forall x \in A, \forall y \not\in A, x \not\sim y$.
	\end{defn}

	\begin{note}
		On note $E / \sim$ l'ensemble des classes d'équivalences pour $\sim$, appelé \textbf{ensemble quotient} de $E$ par $\sim$.
	\end{note}

	\begin{pop}
		Les classes d'équivalence forment une partition de $E$.
	\end{pop}

	\begin{cor}
		On a $\abs{E} = \sum_{A \in E/\sim} \abs{A}$.
	\end{cor}

	On dispose de la projection canonique de $E$ sur $E/\sim$, $\pi \colon x \mapsto \bar{x} = \{ y \in E \mid x \sim y \}$, où $x$ est un représentant de $\bar{x}$.
	On dit que $\mathcal{S} \subset E$ est un \textbf{système de représentants} si $\pi_{|\mathcal{S}} \colon \mathcal{S} \overset{\sim}{\to} E/\sim$.

	\begin{thm}
		Soit $f \colon E \to F$ une application.
		On a équivalence entre les deux assertions suivantes :
		\begin{enumerate}
			\item $f$ est compatible à $\sim$, i.e. $\forall x,y \in E, x \sim y \implies f(x) = f(y)$
			\item $\exists g \colon E/\sim \to F, f = g \circ \pi$.
		\end{enumerate}
		Si ces conditions sont vérifiées, cette application $g$ est unique.
		On dit qu'elle est l'application déduite de $f$ par passage au quotient par $\sim$.
	\end{thm}

	\begin{defn}
		Un \textbf{groupe} $(G,*,e)$ est la donnée de $G$ non vide, $*$ une loi de composition interne et $e \in G$ tels que $*$ est associative, $e$ est neutre et tout élement est inversible.
		On dit que ce groupe est \textbf{abélien} si $*$ est commutative.
	\end{defn}

	\begin{defn}
		Un sous-ensemble $H$ du groupe $G$ est appelé \textbf{sous groupe} si : $e \in H$, il est stable par inversion et par composition.
	\end{defn}

	\begin{pop}
		Une intersection quelconque de sous-groupes de $G$ est encore un sous-groupe de $G$.
	\end{pop}

	\begin{pop}
		Soit $G$ un groupe et $S \subset G$.
		Notons $\langle S \rangle \subset G$ l'intersection de tous les sous-groupes de $G$ qui contiennent $S$.
		Alors $\langle S \rangle$ est un sous-groupe de $G$ contenant $S$ et c'est le plus petit d'entre eux.
		On l'appelle \textbf{sous-groupe} engendré par $S$ dans $G$.
	\end{pop}

	\begin{pop}
		On a aussi $S = \{ s_1^{m_1} * \cdots * s_r^{m_r} \mid r \in \N, s_i \in S, m_i \in \Z \}$.
	\end{pop}

	\begin{pop}
		Soit $H$ un sous-groupe de $G$.
		On définit $x \sim y \iff x^{-1}y \in H$.
		Alors $\sim$ est une relation d'équivalence et $G/\sim$ est noté $G/H$ $(x \mod H) = xH$.
	\end{pop}

	\begin{defn}
		Soit $H$ un sous-groupe de $G$.
		On définit l'\textbf{indice} de $H$ dans $G$ par $[G : H] = \abs{G/H}$ (éventuellement infini).
	\end{defn}

	\begin{pop}
		Soit $G$ un groupe fini.
		Alors tout sous-groupe $H$ de $G$ est fini, d'indice fini et on a $\abs{G} = \abs{H} \cdot [G : H]$.
	\end{pop}


\subsection{Action d'un groupe sur un ensemble}

	\begin{defn}
		Une action de $(G,*,e)$ sur un ensemble $X$ est la donnée d'une application
		$\begin{array}{ccc} G \times X & \to & X \\ (g,x) & \mapsto & g \cdot x \end{array}$
		telle que $\forall x \in X, e \cdot x = x$ et $\forall g, g' \in G, \forall x \in X, g \cdot (g' \cdot x) = (g * g') \cdot x$.
	\end{defn}

	\begin{defn}
		Soit $G$ agissant sur $X$.
		Pour tout élément $x \in X$, on définit son stabilisateur $G_x = \{ g \in G \mid g \cdot x = x \}$ et son orbite $O_x = G \cdot x = \{ g \cdot x \mid g \in G \}$.
	\end{defn}

	\begin{pop}
		Le stabilisateur $G_x$ est un sous-groupe de $G$.
	\end{pop}

	\begin{pop}
		La relation $\sim$ définie par $(x \sim x') \iff (\exists g \in G, x' = g \cdot x)$ est une relation d'équivalence.
		Les orbites de l'action sont alors précisément le classes d'équivalences pour $\sim$.
	\end{pop}

	\begin{ex}
		Des actions classiques de $H$ sur $G$, avec $H$ sous-groupe de $G$, sont :
		\begin{enumerate}
			\item translation à gauche, $(h,x) \mapsto x h^{-1}$,
			\item translation à droite, $(h,x) \mapsto hx$,
			\item conjugaison, $(h,x) \mapsto h x h^{-1}$.
		\end{enumerate}
	\end{ex}

	\begin{pop}
		Soit $G$ agissant sur $X$.
		Alors $\forall x \in X$, $g \mapsto g \cdot x$ induit par passage au quotient une bijection $G/G_x \overset{\sim}{\to} O_x$ et en particulier $\abs{O_x} = [G : G_x]$.
	\end{pop}

	\begin{thm}[Formule de Burnside]
		Soit $G$ un groupe fini agissant sur $X$ fini.
		Alors
		$$\abs{X/G} = \frac{1}{\abs{G}} \sum_{g \in G} \abs{\{ x \in X \mid g \cdot x = x \} }$$
		autrement dit le nombre d'orbites de l'action est égal à l'espérance du nombre de points fixes d'un élément aléatoire de $G$.
	\end{thm}


\subsection{Morphismes}

	\begin{defn}
		Un \textbf{(homo)morphisme} de $(G,*,e)$ dans $(G',*',e')$ est une application $f \colon G \to G'$ telle que : $f(e) = e'$, $\forall x \in G, f(x^{-1}) = f(x)^{-1}$, $\forall x,y \in G, f(x * y) = f(x) *' f(y)$.
		Pour $G = G'$ c'est un \textbf{endomorphisme}, si $f$ est bijective c'est un \textbf{isomorphisme} et pour les deux à la fois c'est un \textbf{automorphisme}.
	\end{defn}

	\begin{pop}
		Les images directes et réciproques de sous-groupes par un morphisme de groupes sont des sous-groupes.
	\end{pop}

	\begin{defn}
		Soit $f \colon G \to G'$ un morphisme de groupes.
		Alors
		\begin{enumerate}
			\item $\Ker(f) = f^{-1}(e')$ est un sous-groupe de $G$ appelé \textbf{noyau} de $f$,
			\item $\im(f) = f(G)$ est un sous-groupe de $G'$ appelé \textbf{image} de $f$.
		\end{enumerate}
	\end{defn}

	\begin{pop}
		Un morphisme $f \colon G \to G'$ est injectif si et seulement si $\Ker(f) = \{ e \}$.
	\end{pop}


\subsection{Groupe quotient d'un groupe abélien par un sous-groupe}

	Soit $H$ s-g de $(G,+,0)$ abélien.

	\begin{lem}
		On munit $G/H$ d'une structure de groupe abélien avec la loi $+$ telle que $\bar{a} + \bar{b} = (a + H) + (b + H) = (a + b) + H = \overline{a + b}$.
		L'ensemble quotient muni de cette loi est appelé \textbf{groupe quotient} de $G$ par $H$.
	\end{lem}

	%\begin{rem}
		%Lorsque $G$ n'est pas abélien il faut que $H$ soit distingué (ou normal).
	%\end{rem}

	\begin{pop}
		La projection canonique $\pi \colon G \to G/H$ est un morphisme de groupes, surjectif, de noyau $H$.
	\end{pop}

	\begin{thm}
		Les sous-groupes de $G/H$ sont en bijection avec les sous-groupes de $G$ contenant $H$.
	\end{thm}


\subsection{Morphisme défini par passage au quotient}

	\begin{thm}[de factorisation]
		Soient $f \colon G \to G'$ un morphisme de groupes abéliens, $H$ un s-g de $G$ et $\pi \colon G \to G/H$ la projection canonique.
		Alors
		$$\left( H \subset \Ker(f) \right) \iff \left( \exists ! g \colon G/H \to G', f = g \circ \pi \right)$$
		et on dit que $g$ se déduit de $f$ par passage au quotient par $H$.
		De plus, on a alors $\im(g) = \im(f)$ et $\Ker(g) = \Ker(f)/H$.
	\end{thm}

	\begin{cor}
		Si $f \colon G \to G'$ est un morphisme de groupes abéliens, $f$ induit par passage au quotient un isomorphisme
		$$G/\Ker(f) \overset{\sim}{\to} \im(f)\ .$$
		En particulier, si $G$ est fini on a $\abs{G} = \abs{\Ker(f)} \cdot \abs{\im(f)}$.
	\end{cor}


\subsection{Sous-groupes monogènes, ordre d'un élément}

	\begin{defn}
		Si  $x \in G$, on définit l'\textbf{ordre} de $x$, noté $\omega_G(x)$ comme le plus petit $n \in \N^*$ tel que $x^n = e$ s'il existe, et $+\infty$ sinon.
	\end{defn}

	\begin{lem}
		Soit $f \colon G \to G'$ un morphisme de groupes injectif.
		Alors $\forall x \in G, \omega(f(x)) = \omega(x)$.
	\end{lem}

	\begin{lem}
		Pour tout diviseur $d$ de $\omega(x)$ on a $\omega(x^d) = \frac{w(x)}{d}$.
	\end{lem}

	\begin{lem}
		Soit $G$ un groupe.
		Pour tout $x \in G$, il existe un unique morphisme de groupes $f_x \colon \Z \to G$ envoyant $1$ sur $x$, c'est $k \mapsto x^k$ et son image est $\langle x \rangle$.
	\end{lem}

	\begin{lem}
		Tout s-g non nul $H$ de $(\Z,+,0)$ est de la forme $H = N\Z$ où $N = [\Z : H] = \min(H \cap \N^*)$.
	\end{lem}

	\begin{pop}
		Soit $G$ un groupe et $x \in G$.
		\begin{enumerate}
			\item Si $\omega(x) = \infty$, $f_x \colon \Z \overset{\sim}{\to} \langle x \rangle$.
			\item Si $\omega(x) < \infty$, $\Ker(f_x) = \omega(x) \Z$, $f_x$ induit par passage au quotient un isomorphisme entre $\Z / \omega(x)\Z$ et $\langle x \rangle$ et $\langle x \rangle = \{ e, x, x^2, \ldots, x^{\omega(x) - 1} \}$.
		\end{enumerate}
		Dans les deux cas $\abs{\langle x \rangle} = \omega(x)$.
	\end{pop}

	\begin{cor}
		Soit $x$ d'ordre fini.
		Alors $\forall n \in \Z, \left( x^n = e \right) \iff \left( \omega(x) \mid n \right)$.
	\end{cor}

	\begin{thm}[\textbf{Lagrange}]
		Soit $G$ un groupe fini.
		Alors, pour tout $x \in G$, $x^{\abs{G}} = e$ et $\omega(x) \vert G$.
	\end{thm}


\subsection{Groupes cycliques}

	\begin{pop}
		Soit $(G,*,e)$ un groupe fini, $N = \abs{G}$ son ordre et $g_0 \in G$.
		Les assertions suivantes sont équivalentes :
		\begin{enumerate}[(i)]
			\item $g_0$ est d'ordre $N$,
			\item $G = \langle g_0 \rangle$,
			\item il existe un isomorphisme $\varphi \colon \Z / N\Z \overset{\sim}{\to} G$ qui envoie $\bar{1}$ sur $g_0$,
			\item tout élément $g \in G$ peut s'écrire $g = g_0^n$ pour un certain $n \in \N$.
		\end{enumerate}
	\end{pop}

	\begin{defn}
		Un groupe fini vérifiant les assertions précédentes est appelé \textbf{groupe cyclique}.
		On dit que $g_0$ est un \textbf{générateur} de $G$.
	\end{defn}

	\begin{pop}
		Soit $N \in \N^*$.
		\begin{enumerate}
			\item Si $d \mid n$, $d\Z / N\Z = \left\{ \bar{0}, \bar{d}, \ldots, \overline{\left( \frac{N}{d} - 1 \right)d} \right\}$ est un sous groupe de $\Z / N\Z$.
				Inversement, tout s-g de $\Z / N\Z$ est de cette forme pour un entier $d$ divisant $N$ uniquement déterminé, mettant en bijection les s-g de $\Z / N\Z$ et les diviseurs de $N$.
			\item Si $d_1$ et $d_2$ divisent $N$, on a $( d_2 \Z / N\Z \subset d_1 \Z / N\Z ) \iff (d_1 \mid d_2)$.
			\item Le sous-groupe $d\Z / N\Z$ est cyclique, d'ordre $\frac{N}{d}$ et d'indice $[\Z/N\Z : d\Z/N\Z] = d$.
			\item Les éléments de $d\Z / N\Z$ sont les éléments de $\Z / N\Z$ (additif) dont l'ordre divise $\frac{N}{d}$.
			\item Soit $x \in \Z$.
				On a $\langle \bar{x} \rangle = d\Z / N\Z$ où $d = \pgcd(x,N)$.
				Alors $x$ est d'ordre $\frac{N}{d}$ (additif).
				En particulier, $\bar{x}$ est générateur de $\Z / N\Z$ si et seulement si $x$ est premier avec $N$.
		\end{enumerate}
	\end{pop}

	\begin{cor}[\textbf{Bézout}]
		Soit $a, b \in \Z$ non tous les deux nuls.
		Alors $\exists m,n \in \Z, ma + nb = \pgcd(a,b)$.
	\end{cor}

	\begin{cor}
		Soit $(G,*,e)$ groupe cyclique d'ordre $N$.
		Alors pour tout $y \in G$ et $d \mid N$ on a $(\exists x \in G, y = x^d) \iff \left( y^{\frac{N}{d}} = e \right)$.
		Pour $k \in \N^*$ on a $(\exists x \in G, y = x^d) \iff \left( y^{\frac{N}{\pgcd(N,k)}} = e \right)$.
		En particulier, si $k$ est premier avec $N$, tout élément de $G$ est une puissance $k$-ième.
	\end{cor}

	