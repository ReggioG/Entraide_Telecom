\subsection{Relations d'équivalence et structueres quotient}

	\begin{defn}
		Une \textbf{relation} $\rel$ sur un ensemble $E$ est la donnée d'une partie $\rel \subset E \times E$.
		On écrit $x \rel y$ si $(x,y) \in \rel$.
	\end{defn}

	\begin{defn}
		Une relation $\rel$ sur $E$ est dite :
		\begin{itemize}
			\item[\textbullet] réflexive si $\forall x \in E, x \rel x$,
			\item[\textbullet] symétrique si $\forall x, y \in E, \left( x \rel y \right) \implies \left( y \rel x \right)$,
			\item[\textbullet] transitive si $\forall x, y, z \in E, \left( x \rel y \text{ et } y \rel z \right) \implies x \rel z$
		\end{itemize}
		et on dit que c'est une relation d'équivalence si ces trois conditions sont vérifiées.
		Dans ce cas on note aussi $x \sim_\rel y$ ou encore $x \equiv y \mod \rel$.
	\end{defn}
	
	\begin{defn}
	Soit $\sim$ une relation d'équivalence 
	\end{defn}
