\paragraph{Modalités d'une démission :} elle suppose une volonté claire et sans équivoque de la part du salarié de rompre son contrat de travail.
On ne peut interdire à un employé de démissionner, même pendant une période donnée.

Rappel : dans le cas d'un CDD on ne peut pas démissionner sans raison valable (le fait de trouver un CDI autre part en constituant une).
Pour un CDI oui.

Il n’existe pas de formalisme a priori en CDI pour démissionner, pas de motif à donner.
Il faut tout de même se renseigner auprès de la convention collective (CC) sur les possibles formalismes qu’elle a instauré (mais la jurisprudence ne sanctionne pas les cas de non respect de ces formalismes).

Une grande souplesse est donc accordée au salarié pour quitter facilement et rapidement l'entreprise alors que le formalisme est beaucoup plus contraignant du côté de l'employeur.

\paragraph{}
L’employeur ne peut refuser une démission.
La démission ne se présume pas.
Elle doit être explicitement déclarée par le salarié.

Pour un employeur, mieux vaut avoir une preuve écrite de démission.
Mais même si la démission a été formalisée par écrit, le salarié peut se retracter dans un délai court en expliquant les raisons et le contexte.
Par exemple un employeur peut avoir exercé des pressions, et amène le salarié à poser sa démission mais celui-ci change d'avis ensuite.

\paragraph{}
Une absence prolongée ne peut supposer une démission.
En cas d’absence prolongée, il faut d’abord faire une mise en demeure (AR) auprès de l’employé pour lui demander de justifier son absence.
S’il ne répond pas ou ne donne pas de raison valable, on peut licencier pour faute grave.

\paragraph{Cas de faute de la part de l'employeur.}
Dans un cas de manquement fautif de l'employeur à ses devoir, l'employé peut quitter l’entreprise sans donner sa démission (car perte de préavis etc), c'est un \textbf{autolicenciement}.
Un tel motif peut être : harcèlement, discrimination, non-paiement des salaires, modification du contrat de travail sans son accord, non-respect du droit au repos hebdomadaire ou encore tabagisme passif.

Il faut faire en AR une lettre de démission avec la liste de ses griefs.
Si l’employé fait parvenir sa lettre au conseil des prud’hommes, la jurisprudence existante va permettre au salarié à conserver ses droits avec une procédure d’autolicenciement.

La prise d'acte est différente d'une démission ou d'un licenciement.
Elle est faite pour libérer un salarié d’une situation bloqué dans un emploi dangereux, désagréable ou ne respectant pas ses droits.

À charge au tribunal des prud’hommes de qualifier sous un mois cette prise d’acte.
Soit elle est requalifiée en licenciement pour cause et sérieuse et le salarié peut toucher des indemnités de licenciement classiques et être indemnisé par Pôle emploi.
Soit la prise d’acte est requalifiée en démission.
Dans ce second cas, le salarié n’a droit à rien : ni indemnités de départ, ni indemnités chômage.
Il peut même être redevable de l’indemnité correspondant au préavis qu’il n’a pas exécuté.
On ne peut revenir sur une décision du tribunal.

A part dans le pénal, les enregistrements effectués à l’insu de la cible ou les vols de mail non publics ou ne nous étant pas destinés ne sont pas recevables dans une affaire.

Il n'y a pas de préavis ni de rétractation.
Inutile pour l’employeur de licencier pour faute grave, c’est la prise d’acte qui sera prise en compte.

\paragraph{}
Le salarié respecter une durée de préavis fixé par la CC.
Le code du travail ne prévoit rien pour la durée du préavis de démission de type autolicenciement.
Si ni la CC ni le contrat de travail ne fonctionnent, il faut se fier aux cas d’usage concernant les préavis « classiques » de ce type d’emploi.
On donne la démission à effet non immédiat (quelquees semaines / mois en général).
L’employeur peut dispenser l’employé d’effectuer son préavis mais doit tout de même le payer pendant le préavis.

Si des congés étaient prévus dans le préavis, cela le reporte d'autant.

Pendant le préavis, si l’employé a un accident du travail ou une maladie professionnel, tout est gelé.
On va attendre les résultats médicaux de l’employé pour voir si besoin de licenciement pour inaptitude.
Pour un arrêt maladie plus bénin, aucun changement dans la durée du préavis n'a lieu et les indemnités touchées sont celles liées aux arrêts maladie.

Dans le cas où on démissionne pour suivre le conjoint par exemple, les droits allocation chômage sont conservés.
Ce peut être aussi en cas de démission pour soin à un enfant handicapé ou de démagement dû à un conjoint violent, dépôt de plainte à l'appui.
Idem si l’entreprise déménage et que vous ne voulez pas déménager : allocations conservées.

Si un employé quitte son entreprise pour aller chez la concurrence malgré avoir signé une clause de non concurrence, le premier employeur peut demander la fin du nouveau contrat de travail.
Il demandera aussi des dommages et intérêts et peut demander à ce que l’employé rembourse la période de non concurrence qu’il lui avait versée.
Il faut le prouver avec un dépôt de plainte (liste des motifs non exhaustive ici).

\paragraph{}
Un autre mode de rupture particulier est le départ à la retraite.
Avant 70 ans le départ à la retraite ne peut être prononcé par l’employeur. 

\paragraph{}
Dernier mode de rupture : la procédure en résiliation judiciaire du contrat de travail, quand le salarié ne veut pas démissionner mais conserver ses droits.
C’est le conseil des prud’homme qui va mettre fin au contrat de travail. Si le salarié apporte les éléments qui justifient la rupture du contrat (salaires impayés ou en retard, harcèlement etc...).
Pendant la durée de la procédure, on doit continuer à travailler.

Si les manquements ne sont pas jugés suffisants, la procédure poursuit.
La demande de procédure en résiliation judicaire est prioritaire sur un licenciement de la part de l’employeur qui aurait pris connaissance de la procédure et qui voudrait y échapper.
Si les manquements ne sont pas assez caractérisés, le conseil des prud’homme va ensuite examiner le licenciement de l’employeur.
C’est l’ordre chronologique qui compte.
Seul le salarié peut demander cette procédure.