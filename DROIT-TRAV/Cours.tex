\documentclass[a4paper,11pt]{article}

\usepackage[utf8]{inputenc}
\usepackage[T1]{fontenc}
\usepackage{lmodern}
\usepackage{amsmath}
\usepackage{amssymb}
\usepackage{amsthm}
\usepackage{textcomp}
\usepackage{wasysym}
\usepackage[top=2.2cm, bottom=2cm, left=2.2cm, right=2.2cm]{geometry}
\usepackage{graphicx}
\usepackage[svgnames,table]{xcolor}
\usepackage{tikzsymbols}
\usepackage{longtable}
\usepackage{listings}
\usepackage{eurosym}
\usepackage{float}
\usepackage{enumerate}
\usepackage{wrapfig}
\usepackage[francais]{babel}

\theoremstyle{remark}
\newtheorem*{rem}{\textcolor{Teal}{Remarque}}

\renewcommand*\rmdefault{ppl}

\usepackage[colorlinks=true,linkcolor=purple]{hyperref}


\begin{document}

\renewcommand{\contentsname}{Sommaire}
\setcounter{tocdepth}{1}

%%%%%%%%%%%%%%%%%
% Page de titre %
%%%%%%%%%%%%%%%%%


\begin{titlepage}
	\newcommand{\HRule}{\rule{\linewidth}{0.5mm}}

	\center

	\textsc{\LARGE Télécom ParisTech}\\[1.2cm]

	\HRule \\[0.8cm]
	{ \huge \bfseries Cours de droit du travail}\\[0.4cm]
	\HRule \\[1.4cm]
	
	\tableofcontents


\end{titlepage}

\section{Introduction}
	\subsection{Historique}

	Les premiers usages étaient pour les militaires.
	Les exemples les plus connus sont :
	\begin{itemize}
		\item[\textbullet] chiffrement par décalage, ou de Jules César,
		\item[\textbullet] chiffrement par substitution, on applique une fonction de permutation sur les lettres,
		\item[\textbullet] chiffrement par permutation : on divise le texte en groupes de lettres de même taille et à chaque groupe on applique une permutation entre les lettres.
	\end{itemize}
	
	La plupart des systèmes actuels peuvent se voir comme combinaisons de ces transformations élémentaires.
	
	\paragraph{Principe de Kerckhoff :} quasiment toutes les caractéristiques du système doivent être publiques, mais le chiffrement dépend juste d'un paramètre secret, la clé.
	
	Intérêt :
	\begin{itemize}
		\item[\textbullet] déploiement à grande échelle,
		\item[\textbullet] ne compromet pas toute la sécurité du système si une seule clé s'ébruite,
		\item[\textbullet] permet d'analyser la sécurité par des tiers.\\
	\end{itemize}
	
	Peut-on prouver qu'un système est sûr, mesurer sa sécurité ?
	Deux approches possibles :
	\begin{itemize}
		\item[\textbullet] sécurité informationnelle \textrightarrow\ théorie de l'information (Shannon).\\
			1948 : \textit{A mathematical theory of communications}\\
			1949 : \textit{Communication theory of secrecy systems}
		\item[\textbullet] sécurité computationnelle \textrightarrow\ théorie de la complexité.
	\end{itemize}


\subsection{Sécurité informationnelle}

	\begin{center}
	\begin{tikzpicture}
		\tikzstyle{block} = [rectangle, draw=blue, thick, fill=blue!20, text width=5.3em, text centered, rounded corners, minimum height=2em]
		\node [block] (A) at (0,0) {Alice};
		\node (X) at (4,0) {$M$};
		\node [block] (B) at (8,0) {Bob};
		\node [block] (E) at (4,-1.5) {Eve};
		\draw      (A) to (X);
		\draw [->] (X) to (B);
		\draw [->] (X) to (E);
	\end{tikzpicture}
	\end{center}
	
	Modèle : A veut envoyer un message clair $M \in \mathcal{M}$. $M$ est vu comme une variable aléatoire de loi donnée.
	
	\begin{defn}
		\textbf{Fonction de chiffrement} : $E \colon \mathcal{M} \times \mathcal{K} \to \mathcal{C}$ où $\mathcal{K}$ est l'espace des clés et $\mathcal{C}$ est l'espace des chiffrés.
	\end{defn}
	
	Au préalable Alice et Bob se sont mis d'accord sur une clé $K \in \mathcal{K}$.
	$K$ est également vue comme une v.a de loi donnée.\\
	
	Alice envoie à Bob le chiffré $C = E_K(M)$.
	Eve voit donc passer $C$.
	Quelle information peut-elle en déduire sur $M$ ?
	

\subsection{Les systèmes parfaits}
	
	\begin{defn}
		Dans un \textbf{système parfait au sens de Shannon}, la connaissance de $C$ n'apporte aucune info sur $M$.
	\end{defn}
	
	Deux variantes :
	\begin{itemize}
		\item[\textbullet] \textit{en moyenne} : $H(M \mid C) = H(M)$,
		\item[\textbullet] \textit{dans tous les cas} : égalité des lois, $\forall M, \forall C, p(M \mid C) = p(M)$.
	\end{itemize}
	
	\begin{ex}
		On a les données suivantes :
		$$\mathcal{M} = \{ \text{oui}, \text{non} \}, \quad p(\text{oui}) = \frac{3}{4}, p(\text{non}) = \frac{1}{4}, \quad
		\mathcal{K} = \{ K_1, K_2, K_3 \}, \quad p(K_1) = p(K_2) = p(K_3) = \frac{1}{3}, \quad
		\mathcal{C} = \{ x, y, z \}$$
		
		et le chiffrage donné par
		$\begin{array}{l|ll}
		    & \text{oui} & \text{non} \\
		K_1 & $x$        & $y$ \\
		K_2 & $y$        & $x$ \\
		K_3 & $z$        & $y$ \\
		\end{array}$.
		
		Eve voit $C$ et en déduit que Alice a envoyé $M$ avec probabilité $p(M \mid C)$.
		Par formule de Bayes il vient $p(\text{oui} \mid x) = \frac{ p(x \mid \text{oui}) p(\text{oui}) }{p(x)}$, avec $p(x) = p(x \mid \text{oui}) p(\text{oui}) + p(x \mid \text{non}) p(\text{non})$ et idem avec $y$ et $z$ ou $M = \text{oui}$.
		
		On trouve $p(x) = \frac{1}{3}$, $p(y) = \frac{5}{12}$ et $p(z) = \frac{1}{4}$.
		
		Supposons $C = x$.
		Alors $p(\text{oui} \mid x) = \frac{3}{4}$ et $p(\text{non} \mid x) = \frac{1}{4}$ donc Eve n'a rien appris de plus que ce qu'elle connaissait déjà.
		
		Supposons $C = y$.
		Alors $p(\text{oui} \mid y) = \frac{3}{5}$ et $p(\text{non} \mid y) = \frac{2}{5}$.
		L'incertitude est plus grande, en un sens Eve a appris quelque chose de nouveau par rapport à ce qu'elle estimait précedemment.
		
		Supposons $C = z$.
		Alors $p(\text{oui} \mid z) = 1$ et $p(\text{non} \mid z) = 0$.
		Donc le clair se déduit du chiffré.
		
		En conclusion ce système n'est pas parfait.
	\end{ex}
	
	\begin{ex}[Un système parfait : le one-time pad, ou masque jetable]
		On prend $\mathcal{M} = \{ 0, 1 \}^n$, $\mathcal{K} = \{ 0, 1 \}^n$ avec une distribution uniforme des clés et $\mathcal{C} = \{ 0, 1 \}^n$.
		On prend $E_K(M) := M \oplus K$.
		Alors $\forall M, \forall C, p(M \mid C) = p(M)$.
		Cette méthode est lourde : clé de même longueur que le message.
	\end{ex}
	
	\begin{thm}
		Dans un système parfait on a nécessairement $\card(\mathcal{K}) \geq \card(\mathcal{C})$.
	\end{thm}
	
	\begin{proof}
		Pour un $M$ donné on a
		$$\forall C, p(C \mid M) = \frac{p(C,M)}{p(M)} = \frac{p(C \mid M)}{p(M \mid C)} = p(C) > 0$$
		donc il existe une clé $K$ telle que $C = E_K(M)$, d'où le résultat.
	\end{proof}
	
	\begin{rem}
		On a toujours $\card(C) \geq \card(M)$ pour pouvoir opérer un décodage (fonction $E$ injective).
		Le cas le plus économique en sytème parfait est donc $\card(\mathcal{K}) = \card(\mathcal{C}) = \card(\mathcal{M})$ et alors $\forall M, \forall C, \exists ! K, C = E_K(M)$.
	\end{rem}
	
	On peut ensuite en déduire que la distribution de $K$ doit être uniforme \textrightarrow\ c'est le one-time pad.


\subsection{Distance d'unicité}

	Scénario : réutilisation d'une même clé pour chiffrer plusieurs messages.
	
	\begin{ex}[Chiffrement par substitution]
		On prend $\mathcal{M} = \mathcal{C} = \{ \mathsf{A},\mathsf{B},\mathsf{C},\ldots,\mathsf{Z} \}$ et $\mathcal{K}$ l'ensemble des permutations de $\mathcal{M}$.
		La réutilisation d'une clé permet d'étendre $E \colon \mathcal{M} \times \mathcal{K} \to \mathcal{C}$ en $E \colon \mathcal{M}^n \times \mathcal{K} \to \mathcal{C}^n$.
		
		Ce système s'attaque par analyse des fréquences dès lors qu'on sait que le clair est restreint à un certain sous-ensemble de $\mathcal{M}^n$, e.g texte en français qui n'est pas une suite de lettres “très aléatoire”.
	\end{ex}
	
	Question : à partir de quelle longueur de chiffré peut-on retrouver la permutation clé ?
	
	Prenons un alphabet $\mathcal{A}$.
	Alors $\mathcal{M} = \mathcal{A}^n$ avec une certaine redondance (non uniformité dans $\mathcal{A}$).
	
	\begin{defn}
		\textbf{Entropie du langage} : $h = \lim_{n \to \infty} \frac{1}{n} H(M)$.
	\end{defn}
	
	\begin{defn}
		\textbf{Redondance du langage} : $r = 1 - \frac{h}{\log_2(\card(\mathcal{A}))}$.
	\end{defn}
	
	Intuitivement un texte de $n$ symboles dans le langage peut se compresser en $nh$ bits ($n \to \infty$).
	
	En longueur $n$, parmi les $\card(A)^n$ suites de $n$ symboles possibles, il y en a $\simeq 2^{nh}$ qui sont dans le langage.
	Les chiffrés doivent sembler aléatoires, on peut espérer en avoir $\card(A)^n$.
	
	À chaque chiffré correspond $\displaystyle \frac{2^{nh + \log_2(\# \mathcal{K})}}{(\# \mathcal{A})^n}$ couples $(M,K)$ possibles.
	
	Une recherche exhaustive donne un déchiffrement unique dès lors que $\frac{2^{nh + \log_2(\# \mathcal{K})}}{(\# \mathcal{A})^n} \leq 1$, où $\frac{2^{nh + \log_2(\# \mathcal{K})}}{(\# \mathcal{A})^n} = 2^{n(h - \log_2 \# \mathcal{A}) + \log_2 \# \mathcal{K}} = 2^{-rn \log_2 \# \mathcal{A} + \log_2 \# \mathcal{K}}$.
	
	\begin{defn}
		\textbf{Distance critique} : $n = \frac{\log_2 \# \mathcal{K}}{r \log_2 \# \mathcal{A}}$.
	\end{defn}
	
	\begin{ex}
		Pour une substitution simple, en français, $\# \mathcal{K} = 26!$, donc $\log_2 \# \mathcal{K} \simeq 88$, $r \in \inff{0{,}75}{0{,}8}$ et $\# \mathcal{A} = 26$ donc $\log_2 \# \mathcal{A} \simeq 4{,}7$.
		Alors $n \simeq 25$.
		En pratique, à la main, on peut casser le chiffrement pour $n$ jusqu'à 200 ou 250.
	\end{ex}
	
	Autre exemple avec sécurité informationnelle : partage de secret à seuil.
	
	\begin{ex}[Le système de Shamir]
		Un “distributeur” dispose d'un secret $S \in \F_q$ et veut donner une part de ce secret à $n$ participants de sorte que :
		\begin{itemize}
			\item[\textbullet] $t$ participants qui se concertent peuvent reconstruire le secret avec leurs parts,
			\item[\textbullet] $t - 1$ participants n'ont aucune info sur $S$.
		\end{itemize}
		
		On supposera $q > n$.
		Le distributeur choisit $t - 1$ éléments $a_1,\ldots,a_{t-1} \in \F_q$ uniformément indépendants.
		Il prend le polynôme $P(X) = S = a_1 X + a_2 X^2 + \ldots + a_{t - 1} X^{t - 1}$.
		
		Étant choisis $x_1,\ldots,x_n \in \F_q^\times$ distincts, on calcule $y_i = P(x_i)$.
		Au participant n°$i$ est envoyé le couple $(x_i,y_i)$.
		L'assignation des $x_i$ peut être publique, la part de secret est contenue dans $y_i$.
		
		Cela vérifie bien les propriétés voulues :
		\begin{itemize}
			\item[\textbullet] Si $t$ participants se concertent $P$ est déterminé de façon unique par interpolation de Lagrange : $P(X) = \sum_{i = 1}^t y_i \prod_{j \neq i} \frac{X - x_i}{x_i - x_j}$ et alors $S = P(0)$.
			\item[\textbullet] Si $t - 1$ participants se concertent ils ne peuvent en déduire aucune information car toutes les valeurs de $S$ sont compatibles par interpolation de Lagrange.
		\end{itemize}
	\end{ex}


\subsection{Sécurité computationnelle}

	En cryptographie symétrique, A et B possèdent une clé secrète commune.
	En cryptographie asymétrique on veut communiquer de façon sécurisée sans avoir eu la possibilité au préalable de se mettre d'accord sur un secret commun.
	
	Historique : 1974, Merkle, projet de fin d'études \textit{Énigmes}.
	
	Bob prépare un grand nombre $N$ de messages clairs qui disent « la clé n°$i$ est $k_i$ ».
	Il les chiffre de façon pas très sûre, cassable en temps $T$.
	Il publie ces chiffres (dans le désordre).
	
	Alice choisit une devinette, elle la résout et apprend un couple $(i,k_i)$.
	Elle dit à Bob « je connais la clé n°$i$ ».
	Eve ne sait pas quelle devinette correspond à la clé $i$, elle doit les résoudre toutes \textrightarrow\ en moyenne temps $\frac{N}{2} T$.
	
	\begin{ex}[\textbf{Diffie-Hellman} (1976)]
		Système d'échange de clé (key agreement).
		
		\begin{table}\centering
		\begin{tabular}{aca}
			Alice & \textit{Public} & Bob \\
			\hline
			 & $\F_q$ & \\
			 & $\alpha \in \F_q^\times$ & \\
			$r \in \Z$ aléatoire & & $s \in \Z$ aléatoire \\
			$\beta = \alpha^r$ & $\overset{\beta}{\longrightarrow} $ & \\
			 & $\overset{\gamma}{\longleftarrow}$ & $\gamma = \alpha^s$ \\
			$\gamma^r = \alpha^{rs}$ & & $\beta^s = \alpha^{rs}$
		\end{tabular}
		\end{table}
		
		À la fin $\alpha^{rs}$ est leur secret commun.
		Eve voit passer $q,\alpha,\beta,\gamma$.
		Elle doit en déduire un certain $\delta$ tel que $\exists r,s, \delta = \alpha^{rs}, \beta = \alpha^r, \gamma^s$.
		C'est le \textit{problème de Diffie-Hellman}.
		Cela ressemble (mais n'est pas équivalent) au problème du log discret : connaissant $q,\alpha,\beta$, trouver $r$ tel que $\beta = \alpha^r$.
	\end{ex}
	
	\begin{ex}[\textbf{RSA}, 1978]
		Méthode de chiffrement et de signature.
		
		Bob choisit $p$ et $q$ premiers secrets.
		Il calcule $N = pq$, alors $\varphi(N) = (p - 1)(q - 1)$.
		Il choisit $e \in \N$ premier à $\varphi(N)$ et $d$ son inverse, i.e $ed = 1 \mod{\varphi(N)}$.
		
		La clé publique de Bob est $(N,e)$, où $N$ est appelé \textit{module RSA} et $e$ est l'\textit{exposant de chiffrement}.
		L'entier $d$ sera appelé \textit{exposant de déchiffrement} et constitue la clé secrète.
		
		Alice veut envoyer un message clair à Bob encodé comme un élément $m \in (\Z / N\Z)^\times$.
		Elle calcule $c = m^e$, le chiffré, et l'envoie à Bob.
		Bob déchiffre $c^d = m^{ed} = m$.
		
		Eve peut vouloir :
		\begin{enumerate}[1 -]
			\item retrouver le secret primitif de Bob, i.e $p$ et $q$, ensuite elle trouve $\varphi(N)$ puis $d = e^{-1}$ par Euclide,
			\item retrouver juste $d$ à partir de $(N,e)$,
			\item être capable de retrouver un $m$ correspondant à un $c$,
			\item retrouver un bit d'information sur $m$ à partir de $c$ où $m$ est vu comme un entier dans $\iniff{1}{N - 1}$.
		\end{enumerate}
		
		Remarque : en théorie de la complexité on appelle “faciles” les problèmes résolubles en temps polynomial en la taille des entrées.
		On appelle difficile les problèmes résolubles en tant exponentiel \textrightarrow\ sous classe NP.
		
		En crypto on veut typiquement que le déchiffrement soit NP : facile si on connaît la clé et difficile si on ne la connaît pas.
		
		Dans le cas de RSA :
		\begin{enumerate}[1 \textrightarrow\ ]
			\item problème de factorisation donné $N$, trouver ses facteurs premiers $p$ et $q$.
				Le problème est réputé NP, mais ni P ni NP-complet.
			\item on peut montrer que ça équivaut à 1.
			\item extraction de racine $e$\up{ième} dans $\Z / N\Z$, $m = \sqrt[e]{c} \mod{N}$.
				Si $e = 2$ on a vu que c'était équivalent à la factorisation de $N$.
				Mais lorsque $e$ est impair (souvent $e = 3$), cela pourrait être plus facile quee la factorisation mais estimé non polynomial.
			\item équivalent à 3.
				Plus précisément on a une réduction polynomiale de l'un à l'autre.
				Soit $A$ une “boîte noire” qui nous dit si $m < \frac{N}{2}$ ou non à partir du chiffré $c$.
				Or on a $2^e c = (2m)^e$ le chiffré correspondant au clair $2m \mod{N}$.
				Si $m < \frac{N}{2}$, $2m \mod{N} = 2m$ pair et si $m > \frac{N}{2}, 2m \mod{N} = 2m - N$ impair.
				Ensuite, par dichotomie, en $\log_2(N)$ appels à la boîte $A$ on a complètement localisé $m$.
		\end{enumerate}
	\end{ex}
	
	\begin{ex}
		Public : $\F_q$ et $\alpha \in \F_q^\times$.
		Clé publique de Bob : $\beta \in \F_q^\times$.
		Clé secrète : $r \in \Z$ tel que $\beta = \alpha^r$.
		
		Alice veut envoyer le message en clair $m \in \F_q^\times$.
		Elle choisit $s \in \Z$ aléatoire et calcule $c_1 = \alpha^s$ et $c_2 = \beta^s m$.
		Elle envoie $(c_1,c_2)$ à Bob qui déchiffre par $c_2 c_1^{-r} = \beta^s m \alpha^{-rs} = m$.\\
		
		\noindent Eve peut vouloir :
		\begin{enumerate}[1.]
			\item retrouver la clé secrète à partir de la clé publique \textrightarrow\ problème du logarithme discret,
			\item retrouver $m$ à partir de $(c_1,c_2)$ où sont connus $F_q, \alpha, \beta = \alpha^r, c_1 = \alpha^s, c_2 = m \alpha^{rs}$.
				Alors retrouver $m$ revient à retrouver $\alpha^{rs}$.
				C'est le problème de DH.
		\end{enumerate}
		
		DH et El Gamal utilisent la structure de groupe de $\F_q$.
		La sécurité repose essentiellement sur la difficulté du log discret dans ce groupe.
		De même que la factorisation on ne sait pas le résoudre en temps polynomial, mais on sait faire mieux que exponentiel.
		
		Formule générale : $L_{\varepsilon,c}(t) = \exp \left( c \cdot t^\varepsilon \log(t)^{1 - \varepsilon} \right)$.
		Si $\varepsilon = 0$ on est dans le cas polynomial : $L_{\varepsilon,c}(t) = t^c$.
		Si $\varepsilon = 1$ on est dans le cas exponentiel : $L_{\varepsilon,c}(t) = e^{ct}$.
		
		On sait résoudre les problèmes de factorisation et de log discret en $\varepsilon = \frac{1}{3}$.
		En pratique, avec les puissances de calcul actuelles, cela veut dire que l'on doit prendre des clés de taille environ 2000 bits.
	\end{ex}

\section{Administration du travail}
	\subsection{Généralités}

	Soit $(\Omega,\mathcal{F},\proba)$ un espace de probabilités.
	Soit $d \in \N^*$, $E = \R^d$ et $\mathcal{E} = \mathcal{B}(E)$.
	
	On note $\mu \colon B \mapsto \proba(X^{-1}(B))$ la loi de probabilité de $X$.
	
	Soit $\mathbf{T}$ un “ensemble d'indices” qui représente le temps.
	En général $\mathbf{T} = \R_+$.
	
	\begin{defn}
		Un processus à valeurs dans $(E,\mathcal{E})$ indexé par $\mathbf{T}$ est une famille de v.a $X = (X_t)_{t \in \mathbf{T}}$ à valeurs dans $(E,\mathcal{E})$.
		Pour tout $\omega \in \Omega$, l'application $t \mapsto X_t(\omega)$ est appelé \textbf{trajectoire} de $X$.
	\end{defn}
	
	La famille $X$ peut-être vue comme une application $\Omega \to E^{\mathbf{T}}$ de toutes les trajectoires possibles.
	Il faut donc définir une tribu sur $E^{\mathbf{T}}$ et caractériser la mesure.
	
	Soit $t \in \mathbf{T}$, on pose $\mathcal{G}_t := \sigma(\xi_t)$ la tribu sur $E^{\mathbf{T}}$ engendrée par la projection $\xi_t \colon \begin{array}{lcr} E^{\mathbf{T}} & \to & E \\ x & \mapsto & x(t) \end{array}$.
	Cette tribu est donc constituée des ensembles $\{ x \in E^{\mathbf{T}} \mid x(t) \in H \}$ où $H$ parcourt $\mathcal{E}$.
	
	\begin{defn}
		La \textbf{tribu de Kolmogorov} est la tribu $\mathcal{G}$ engendrée par la famille $\{ \mathcal{G}_t \}_{t \in \mathbf{T}}$.
	\end{defn}
	
	D'une manière équivalente, $\mathcal{G}$ est la plus petite tribu rendant mesurables toutes les applications $\xi_t$ où $t$ parcourt $\mathbf{T}$.
	Avec cette construction $X \colon \Omega \to E^{\mathbf{T}}$ est $\mathcal{F}/\mathcal{G}$-mesurable de loi $\mu$ l'image de $\proba$ par $X$.
	
	Étant donné une loi de probabilité $\mu$ sur $\left( E^{\mathbf{T}}, \mathcal{G} \right)$, il est facile de construire un processus de loi $\mu$ : il suffit de prendre $(\Omega, \mathcal{F}, \proba) = \left( E^{\mathbf{T}}, \mathcal{G}, \mu \right)$ et de poser $X(\omega) = \omega$.
	
	Ce processus est appelé \textbf{processus canonique}.
	
	\begin{defn}[\textbf{Lois fini-dimensionnelles}]
		Soit $\mathcal{J}$ l'ensemble des parties finies de $\mathbf{T}$ et $I = \{ t_1,\ldots,t_n \} \in \mathcal{J}$ où $t_1 < t_2 < \cdots < t_n$.
		Soit $\mu_I$ la loi du vecteur $(X_{t_1},\ldots,X_{t_n})$.
		En notant $\mathcal{G}_I := \sigma(\xi_I)$ la sous-tribu de $\mathcal{G}$ engendrée par $\xi_I \colon \begin{array}{lcr} E^{\mathbf{T}} & \to & E^I \\ x & \mapsto & (x(t_1),\ldots,x(t_n)) \end{array}$, la loi $\mu_I$ peut être définie sur $(E^I, \mathcal{G}_I)$ comme étant l'image de $\mu$ par $\xi_I$.
	\end{defn}
	
	\begin{rem}
		$\mathcal{G}_I$ est la collection des ensembles $\{ x \in E^{\mathbf{T}} \mid (x(t_1),\ldots,x(t_n)) \in H \}$ où $H \in \xi^{\otimes I}$ est la tribu produit sur $E^I$.
		Donc $\mathcal{G}_I$ peut être identifiée à $\mathcal{E}^{\otimes I}$ et on peut caractériser $\mu_I$ par $\forall H_1,\ldots,H_n \in \mathcal{E}, \mu_I(H_1 \times \cdots \times H_n) = \proba(X_{t_1} \in H_1, \ldots, X_{t_n} \in H_n)$.
	\end{rem}
	
	\begin{defn}
		La famille des lois fini-dimensionnelles de $X$ est la famille des $\mu_I$ où $I$ parcourt $\mathcal{J}$.
	\end{defn}
	
	\begin{pop}
		Si deux lois $\mu$ et $\nu$ sur $\left( E^{\mathbf{T}}, \mathcal{G} \right)$ possèdent les mêmes lois fini-dimensionnelles alors elles sont égales.
	\end{pop}
	
	\begin{proof}
		$\mathcal{G}$ est engendré par l'algèbre $\bigcup_{I \in \mathcal{J}} \mathcal{G}_I$.
		Comme $\mu$ et $\nu$ coïncident sur cette algèbre elles coïncident sur $\mathcal{G}$.
	\end{proof}
	
	\begin{pop}
		Les lois fini-dimensionnelles satisfont la \textbf{condition de compatibilité} suivante : pour tout $I = \{ t_1,\ldots,t_n \}$ avec $t_1 < \cdots < t_n$, pour $p \in \iniff{1}{n}$ et $J = \{ t_1,\ldots,t_{p - 1},t_{p + 1},\ldots,t_n \} \subset I$, pour toutes les familles  $(H_i)$ de $\mathcal{E}$, on a $\mu_I(H_1 \times \cdots H_{p - 1} \times E \times H_{p + 1} \times \cdots \times H_n) = \mu_J(H_1 \times \cdots H_n)$.
	\end{pop}
	
	\begin{thm}[\textbf{Kolmogorov}]
		Soit $(\mu_I)_{I \in \mathcal{J}}$ une famille de lois sur $\left( E^I, \mathcal{E}^{\otimes I} \right)_{I \in \mathcal{J}}$.
		Si elle vérifie les conditions de compatibilité, $(\mu_I)_{I \in \mathcal{J}}$ est la famille de lois fini-dimensionnelles d'une unique mesure de probabilités $\mu$ sur $\left( E^{\mathbf{T}}, \mathcal{G} \right)$.
	\end{thm}
	
	{\Large \noindent\lightning} Ici $E = \R^d$. Cela ne marche pas pour tous types de $E$.
	
	\begin{ex}
		Prenons $E = \R$.
		Soit $\nu$ une mesure sur $\R$.
		Supposons $\mu_I = \otimes^n \nu$, avec $n = \card(I)$.
		Alors il existe un processus aléatoire tel que ...
		TODO
	\end{ex}
	
	\begin{defn}
		Soit $X$ et $X'$ deux processus définis sur le même espace de probabilités.
		\begin{itemize}
			\item[\textbullet] On dit que $X'$ est une \textbf{modification} de $X$ si $\forall t \in \mathbf{T}, \proba(X_t = X_t') = 1$.
			\item[\textbullet] On dit que $X$ et $X'$ sont \textbf{indistinguables} si $\proba(\forall t \in \mathbf{T}, X_t = X_t') = 1$ en admettant que $\{ \forall t \in \mathbf{T}, X_t = X_t' \} \in \mathcal{F}$.
		\end{itemize}
	\end{defn}
	
	\begin{ex}
		Soit $\Omega = \mathbf{T} = \inff{0}{1}$, $\mathcal{F} = \mathcal{B}(\inff{0}{1})$, $\proba$ la mesure de Lebesgue sur $\iniff{0}{1}$ et $\forall t \in \mathbf{T}, X_t(\omega) = \delta_{t,\omega} = \indic_{ \{ t \} }(\omega)$ et $\forall t, X_t'(\omega) = 0$.
		Alors $\forall t \in \mathbf{T}, \proba( \omega \mid X_t(\omega) \neq X_t'(\omega) ) = \proba( \{ t \} ) = 0$ mais $\proba( \omega \mid \exists t \in \mathbf{T}, X_t(\omega) \neq X_t'(\omega) ) = \proba(\inff{0}{1}) = 1$.
	\end{ex}
	
	Question : peut-on trouver une condition sur $\mu$ qui rende le processsus $X$ continu, au moins avec la probabilité 1, i.e. “presque toutes les trajectoires sont continues”, si cela a un sens ?
	Non, comme le montr l'exemple précédent.
	En effet les lois fini-dimensionnelles de $X$ et $X'$ sont identiques.
	Donc $X$ et $X'$ ont la même loi $\mu$.
	
	Cet exemple montre que l'ensemble des processus continus n'est pas mesurable par la tribu de Kolmogorov.
	En effet, si $\cont(\inff{0}{1})$ était mesurable, on aurait $\mu \left( \cont(\inff{0}{1}) \right) = 1$ car $\mu$ est la loi de $X' \in \cont(\inff{0}{1})$.
	En même temps $\mu \left( \cont(\inff{0}{1}) \right) = 0$ car $\mu$ est la loi de $X$.


\subsection{Le mouvement brownien}

	\begin{defn}
		Un processus aléatoire est dit \textbf{gaussien} si toutes ses lois fini-dimensionnelles sont gaussiennes.
	\end{defn}
	
	\begin{defn}
		Un \textbf{mouvement brownien au sens large (MBL)} est un processus scalaire gaussien $X$ sur $\mathbf{T} = \R_+$ tel que $\forall t \in \mathbf{T}, \esp X_t = 0$ et $\forall t,s \in \mathbf{T}, \esp[X_t X_s] = \min(t,s)$.
	\end{defn}
	
	\begin{pop}
		Le MBL existe.
	\end{pop}
	
	\begin{proof}
		Il nous faudra prouver que les conditions de compatibilité sont satisfaites.
		Pour tout $I = \{ t_1,\ldots,t_n \}, t_1 < \cdots < t_n$ il nous suffira de prouver que $\mu_I$ est une loi de probabilité.
		Ainsi $\mu_J$ pour tout $J \subset I$ sera la marginale correspondante de $\mu_I$.
		Cela revient à prouver que $\Gamma := (t_i \wedge t_j)_{1 \leq i,j \leq n}$ est une matrice de covariance, i.e une matrice semi-définie positive.
		En effet, avec $t_0 := 0$, $\forall x = \begin{pmatrix} x_1 \\ \vdots \\ x_n \end{pmatrix} \in \R^n$,
		$$\transp{x} \Gamma_I x = \sum_{i,j = 1}^n x_i x_j (t_i \wedge t_j)
		                        = \sum_{i,j = 1}^n x_i x_j \sum_{l = 1}^{i \wedge j} (t_l - t_{l - 1})
		                        = \sum_{l = 1}^n (t_l - t_{l - 1}) \left( \sum_{i = l}^n x_i \right)^2
		                        \geq 0$$
	\end{proof}
	
	\begin{defn}
		Soit $\sigma(X_t)$ la sous-tribu de $\mathcal{F}$ engendrée par la v.a $\xi_t \circ X$.
		La tribu engendrée par $\{ \sigma(X_s) \}_{0 \leq s \leq t}$, noté $\sigma(X_s, 0 \leq s \leq t)$ représente le \textbf{passé} de $X$ antérieur à $t$.
	\end{defn}
	
	\begin{pop}
		Un processus $X$ est un MBL si et seulement si il satisfait les conditions suivantes :
		\begin{enumerate}[(i)]
			\item Il est à accroissement indépendants, i.e $\forall s,t \geq 0, X_{t+s} - X_t$ est indépendant de $\sigma(X_u, 0 \leq u \leq t)$.
			\item Il est gaussien centré et $\forall t \geq 0, \esp[X_t^2] = t$.
		\end{enumerate}
		Par ailleurs les accroissements d'un MBL satisfont $\forall s,t \geq 0, X_{t+s} - X_t \overset{\mathcal{L}}{=} X_s - X_0 \overset{\mathcal{L}}{=} X_s \sim \mathcal{N}(0,s)$.
	\end{pop}
	
	\begin{proof}
		Si $X$ est un MBL, il suffit de prouver le premier point.
		Comme la loi de $X$ est caractérisée par les lois fini-dimensionnelles, il suffit de prouver $\forall t_0,\ldots,t_{n + 1}$ tel que $0 = t_0 < t_1 < \cdots < t_n = t < t_{n + 1} = t + 1$, la v.a $X_{t_{n + 1}} - X_{t_n}$ et le vecteur $(X_{t_0},\ldots,X_{t_n})$ sont indépendants comme $(X_{t_0},\ldots,X_{t_{n + 1}})$ est gaussien.
		
		Le vecteur $(X_{t_0},\ldots,X_{t_n},X_{t_{n + 1}} - X_{t_n})$ l'est par transformation linéaire, et il suffit de prouver la décorrélation $\forall i \in \iniff{0}{1}, \esp \left[ (X_{t_{n + 1}} - X_{t_n}) X_{t_i} \right] = 0$.
		C'est immédiat : $\esp \left[ X_{t_{n + 1}} X_{t_i} \right] - \esp \left[ X_{t_n} X_{t_i} \right] = t_{n + 1} \wedge t_i - t_n \wedge t_i = t_i - t_i = 0$.
		
		Réciproquement, si les deux points sont satisfaits, il suffit de prouver que $\esp \left[ X_{t + s} X_t \right] = t$.
		En effet $\esp \left[ X_{t + s} X_t \right] = \esp \left[ (X_{t + s} - X_t) X_t \right] + \esp \left[ X_t^2 \right] = \esp \left[ X_{t + s} - X_t \right] \esp \left[ X_t \right] + \esp \left[ X_t^2 \right] = \esp \left[ X_t^2 \right] = t$.
		
		Enfin on sait que $X_{t + s} - X_t$ est gaussienne et il est facile de vérifier qu'elle est centrée et de variance $s$.
	\end{proof}
	
	\begin{thm}[Kolmogorov]
		Soit $\mathbf{T}$ un intervalle de $\R$ et $(X_t)_{t \in \mathbf{T}}$ un processus à valeurs dans $E^{\mathbf{T}}$.
		Supposons $\exists \alpha,\beta \in \R_+^*, \exists C > 0, \forall s,t \in \mathbf{T}, \esp \left[ \norme{X_t - X_s}^\beta \right] \leq C \abs{t - s}^{1 + \alpha}$.
		Alors $X$ admet une modification $\tilde{X} = \left( \tilde{X}_t \right)_{t \in \mathbf{T}}$ dont toutes les trajectoires $t \mapsto \tilde{X}_t(\omega)$ sont continues.
	\end{thm}
	
	\begin{defn}
		Un \textbf{mouvement brownien (MB)} ou processus de Wiener est un MBL dont toutes les trajectoires sont continues et nulles en $t = 0$.
	\end{defn}
	
	\begin{pop}
		Le MB existe.
	\end{pop}
	
	\begin{proof}
		Soit $X$ un MBL.
		$\esp \left[ (X_s - X_t)^4 \right] = (t - s)^2 \esp \left[ U^2 \right]$ où $U \sim \mathcal{N}(0,1)$ et on applique le théorème de Kolmogorov.
	\end{proof}
	
	TODO\\
	
	\begin{pop}
		Soit $B$ un MB.
		Alors $\limsup_{t \to \infty} \frac{B_t}{\sqrt{t}} \overset{\text{p.s.}}{=} +\infty$, $\liminf_{t \to \infty} \frac{B_t}{\sqrt{t}} \overset{\text{p.s.}}{=} -\infty$, $\lim_{t \to \infty} \frac{B_t}{t} = 0$, $\limsup_{t \searrow 0} \frac{B_t}{\sqrt{t}} \overset{\text{p.s.}}{=} +\infty$ et $\liminf_{t \searrow 0} \frac{B_t}{\sqrt{t}} \overset{\text{p.s.}}{=} -\infty$.
		De plus le processus donné par $Z_t = t B_{1/t}$ est un MBL.
	\end{pop}
	
	\begin{rem}
		On peut prouver des résultats plus fins, comme $\limsup_{t \to \infty} \frac{B_t}{\sqrt{t \log \log t}} \overset{\text{p.s.}}{=} 1$ ou $\liminf_{t \to \infty} \frac{B_t}{\sqrt{t \log \log t}} \overset{\text{p.s.}}{=} -1$.
	\end{rem}
	
	\begin{proof}
		Soit $R := \limsup_{t \to \infty} \frac{B_t}{\sqrt{t}}$.
		On a $\forall s > 0, R = \limsup_{t \to \infty} \frac{B_{t + s}}{\sqrt{t + s}} = \limsup_{t \to \infty} \frac{B_{t + s}}{\sqrt{t}} = \limsup_{t \to \infty} \frac{B_{t + s} - B_s}{\sqrt{t}}$.
		Par conséquent $R$ est indépendante de $\sigma(B_u, u \leq s)$ pour tout $s$.
		Donc $R$ est indépendante de la tribu $\sigma(B)$ engendrée par $B$.
		Comme $R$ est $\sigma(B)$-mesurable, $R$ est indépendant d'elle même : $\forall H \in \mathcal{B}(\R), \proba(R \in H) = P(R \in H)^2$.
		Donc $P(R \in H)$ vaut $0$ ou $1$.
		Donc $R = a$ avec proba $1$ où $a \in \intff{-\infty}{+\infty}$.
		
		Supposons $a < \infty$.
		Soit $b > a$ quelconque.
		Comme $R = a$ on peut vérifier que $\proba \left( \frac{B_t}{\sqrt{t}} > b \right) \underset{t \to \infty}{\longrightarrow} 0$.
		Mais par ailleurs $\frac{B_t}{\sqrt{t}} \sim \mathcal{N}(0,1)$, d'où une contradiction.
		Par conséquent $R = \infty$.
		La 2\up{e} et la 3\up{e} convergences se démontrent de la même façon.
		
		Pour prouver la 3\up{e} convergence et le résultat sur $Z_t = t B_{1/t}$.
		Nous avons que $Z_t$ est une gaussienne centrée.
		On peut prouver facilement que $\esp Z_t Z_s = s \wedge t$.
		$Z_t$ est continue sur $\intoo{0}{\infty}$ car $B_t$ est continue.
		Alors $\lim_{t \searrow 0} Z_t = \lim_{t \searrow 0} t B_{1/t} = \lim_{u \to \infty} \frac{B_u}{u} \overset{\text{p.s.}}{=} 0$.
		Donc $Z_t$ est un MBL dont presque toutes les trajectoires sont continues sur $\intfo{0}{\infty}$.
		Nous avons alors $\limsup_{t \searrow 0} \frac{B_t}{\sqrt{t}} = \limsup_{t \to \infty} \frac{Z_t}{\sqrt{t}} = +\infty$.
	\end{proof}
	
	\begin{cor}
		Avec probabilité 1 on a :
		\begin{enumerate}[(i)]
			\item Le MB passe une infinité de fois par chaque point de $\R$.
			\item Le MB n'est dérivable ni à droite pour tout $t \in \R_+$, ni à gauche pour tout $t \in \R_+^*$.
		\end{enumerate}
	\end{cor}
	
	\begin{proof}
		Pour \textit{(i)}, utiliser les convergences de la proposition précédente conjointement avec la continuité du MB.
		
		Pour \textit{(ii)}, prenons $t > 0$.
		Pour $s > 0$ nous avons $\frac{B_{t+s} - B_t}{s} = \frac{1}{\sqrt{s}} \cdot \frac{B_{t+s} - B_t}{\sqrt{s}}$, mais $Z_s := B_{t + s} - B_t$ est un MB.
		Comme $\limsup_{s \searrow 0} \frac{Z_s}{\sqrt{s}} \overset{\text{p.s.}}{=} +\infty$ on a le résultat.
	\end{proof}


\subsection{Mesurabilité du MB}

	On peut considérer un processus $X \colon \Omega \to E^{\mathbf{T}}$ où $\mathbf{T} = \R_+$ comme une application $\Omega \times \mathbf{T} \to E$ qui, à chaque couple $(\omega,t) \in \Omega \times \mathbf{T}$, associe $X_t(\omega)$.
	Si on adopte ce point de vue, on est amené à considérer la mesurabilité de $X$ par rapport à la tribu-produit $\mathcal{F} \otimes \mathcal{B}(\mathbf{T})$.
	
	\begin{defn}
		Un processus $X = (X_t, t \in \mathbf{T})$ à valeurs dans $E$ est dit \textbf{mesurable} si l'application $(\omega,t) \mapsto X_t(\omega)$ est mesurable de $(\Omega \times \mathbf{T}, \mathcal{F} \otimes \mathcal{B}(\mathbf{T})$ dans $(E,\mathcal{E})$.
	\end{defn}
	
	En présence de mesurabilité, les trajectoires $t \mapsto X_t(\omega)$ à $\omega$ fixé sont mesurables pour la tribu $\mathcal{B}(\mathbf{T})$.
	En particulier, le bruit blanc n'est pas mesurable en ce sens (bien qu'il soit mesurable au sens de Kolmogorov) car ses trajectoires sont trop irrégulières si $\nu$ n'est pas un Dirac.
	Aucune trajectoire de ce processus n'est borélienne.
	
	Quand le processus $X$ est mesurable, l'intégrale $\int_a^b \varphi(X_t(\omega)) \diff t$ a un sens pour toute fonction mesurable $\varphi$, et par Fubini $\esp \left[ \int_a^b \varphi(X_t(\omega)) \diff t \right] = \int_a^b \esp \varphi(X_t(\omega)) \diff t$ si $\int_a^b \esp \abs{\varphi(X_t(\omega))} \diff t < \infty$.
	
	\begin{note}
		Si $\forall \omega \in \Omega, t \mapsto X_t(\omega)$ est continue à droite (resp. à gauche), on dit que le processus est continu à droite (resp. à gauche).
	\end{note}
	
	\begin{pop}
		Si un processus $X$ est continu à gauche ou à droite, il est mesurable (par rapport à la tribu produit).
	\end{pop}
	
	\begin{proof}
		Supposons $X$ continu à gauche.
		Pour tout $n \in \N$, soit $X_n(t) := X \left( \frac{\lfloor nt \rfloor}{n} \right)$.
		Alors on peut vérifier que $X_n(t) \underset{n \to \infty}{\longrightarrow} X(t)$.
		Or $X_n(t)$ est toujours mesurable.
		Donc $X$ l'est par passage à la limite.
	\end{proof}
	
	\begin{cor}
		Le MB est mesurable.
	\end{cor}

	Notre but est maintenant de construire une intégrale de type $\int_0^t \varphi(s) \diff B_s$ où $B$ est un MB et où $\varphi$ est une fonction déterministe qui appartient à une classe appropriée.


\subsection{Rappels sur les fonctions à variations finies}

	Soit $\cont_0(\R_+)$ (resp. $\cont_0^+(\R_+)$) l'ensemble des fonctions continues (resp. continues croissantes) issues de zéro.
	Soit $\pi_t = \{ t_0,\ldots,t_n \}, 0 = t_0 < t_1 < \ldots < t_n = t$ une subdivision finie de l'intervalle $\intff{0}{t}$.
	
	\begin{defn}
		La \textbf{variation approchée} d'une fonction $f \in \cont_0(\R_+)$ sur la subdivision $\pi_t$ est $V_1(f,\pi_t,t) := \sum_{i = 1}^n \abs{f(t_i) - f(t_{i - 1})}$.
		La fonction $f$ est dit à \textbf{variation finie} si $\forall t, V_1(f,t) := \sup_{\pi_t} V_1(f,\pi_t,t)$ est finie.
	\end{defn}
	
	\begin{pop}
		Si $f \in \cont_0(\R_+)$ est à variations finies, alors elle s'écrit d'une manière unique $f = f_+ - f_-$ où :
		\begin{enumerate}[(i)]
			\item $f_+ \in \cont_0^+(\R_+)$, $f_- \in \cont_0^+(\R_+)$,
			\item $\forall f_+',f_-' \in \cont_0^+(\R_+)$ telles que $f = f_+' - f_-'$ on a $f_+' - f_+ \in \cont_0^+(\R_+)$ et $f_-' - f_- \in \cont_0^+(\R_+)$.
		\end{enumerate}
	\end{pop}
	
	Nous savons que si $g \in \cont_0^+(\R_+)$ alors la fonction d'ensemble $\mu(\intof{a}{b}) := g(b) - g(a)$ pour tout $a \leq b$ est une mesure (de Radon) positive sur $\R_+$.
	Soit $\diff f_+$ et $\diff f_-$ les mesures associées à $f_+$ et $f_-$ de cette façon.
	Pour toute fonction borélienne $\varphi$ sur $\R_+$ qui satisfait $\int \abs{\varphi} \diff f_+ < \infty$ et $\int \abs{\varphi} \diff f_- < \infty$, nous écrirons $\int \varphi \diff f := \int \varphi \diff f_+ - \int \varphi \diff f_- = \int \varphi (\diff f_+ - \diff f_-)$.
	C'est l'intégrale de Lebesgue-Stieltjes par rapport à une fonction à variation finie.


\subsection{Variation quadratique d'un MB}

	\begin{defn}
		La \textbf{variation quadratique approchée} d'une fonction $f \in \cont_0(\R_+)$ sur la subdivision $\pi_t$ est $V_2(f,\pi_t,t) := \sum_{i = 1}^n (f(t_i) - f(t_{i-1}))^2$.
	\end{defn}
	
	\begin{pop}
		Si $f$ est à variation finie alors $V_2(f,\pi_t,t) \underset{\abs{\pi_t} \to 0}{\longrightarrow} 0$ où $\abs{\pi_t} := \max_i \abs{t_i - t_{i-1}}$.
	\end{pop}
	
	\begin{proof}
		Comme $f$ est continue sur $\intff{0}{t}$, elle est uniformément continue, i.e $\forall \varepsilon > 0, \exists \eta > 0, \forall t_1,t_2 \in \intff{0}{t}, \abs{t_1 - t_2} < \eta \implies \abs{f(t_1) - f(t_2)} < \varepsilon$.
		Par conséquent, si $\abs{\pi_t} < \eta$,
		$$V_2(f,\pi_t,t) \leq \varepsilon \sum_{i = 1}^n \abs{f(t_i) - f(t_{i-1})} = \varepsilon V_1(f,\pi_t,t) \leq \varepsilon V_1(f,t)\ .$$
		Comme $\varepsilon$ est quelconque, on a le résultat.
	\end{proof}
	
	\begin{thm}
		Sur tout intervalle $\intff{0}{1}$ où $t  0$, presque toutes les trajectoires d'un MB sont à variation infinie.
	\end{thm}
	
	\begin{proof}
		On montre $\forall t > 0, V_2(B,\pi_t,t) \underset{\abs{\pi_t} \to 0}{\overset{\mathcal{L}^2}{\longrightarrow}} t\quad  (*)$.
		En effet, soit $Y_n := \sum_{i = 1}^n \left( B_{t_i} - B_{t_{i-1}} \right)^2$.
		En écrivant $B_{t_i} - B_{t_{i-1}} = \sqrt{t_i - t_{i-1}} Z_i$ où $Z_i \sim \mathcal{N}(0,1)$ et où les $Z_i$ sont indépendantes, on a $\esp Y_n = t$ et
		$$\Var(Y_n) = \sum_{i = 1}^n \Var \left( \left( B_{t_i} - B_{t_{i-1}} \right)^2 \right)
			= \Var \left( Z_1^2 \right) \sum_{i = 1}^n (t_i - t_{i - 1})^2
			\leq \Var \left( Z_1^2 \right) \abs{\pi_t} \sum_{i = 1}^n (t_i - t_{i - 1})
			= \Var \left( Z_1^2 \right) \abs{\pi_t} t$$
		qui tend vers $0$ avec $\abs{\pi_t}$, d'où $(*)$.
		
		Considérons une suite de subdivisions $\pi_t^n$ telle que $\abs{\pi_t^n} \underset{n \to \infty}{\longrightarrow} 0$.
		Les v.a $Y_n$ associées tendent dans $\mathcal{L}^2$, donc en probabilité, vers $t$.
		Par conséquent il existe une sous-suite $\left( \pi_t^{\varphi(n)} \right)$ telle que $V_2(B,\pi_t^{\varphi(n)},t) \underset{n \to \infty}{\overset{\text{p.s.}}{\longrightarrow}} t > 0$.
		La proposition précédente nous dit que sur cet ensemble de proba 1, $B_t$ n'est pas à variation finie.
	\end{proof}

	Conclusion : on ne peut pas utiliser la théorie de Lebesgue pour construire des intégrales du type $\int_0^t \varphi(s) \diff B_s$.


\subsection{L'intégrale de Wiener}

	L'intégrale de Wiener est définie sur l'espace de Hillbert $L^2(\R_+)$ des fonctions de carré intégrable par rapport à la mesure de Lebesgue sur $\R_+$.
	C'est une isométrie entre cet espace et l'espace de Hilbert $\mathcal{L}^2$ des variables aléatoires qui ont un 2\up{nd} moment fini.
	Rappelons que ces deux espaces sont munis des normes respectives $\norme{\varphi}_{L^2(\R_+)} = \left( \int_{\R_+} \varphi(s)^2 \diff s \right)^{\frac{1}{2}}$ et $\norme{X}_{\mathcal{L}^2} = \left( \esp \left[ X^2 \right] \right)^{\frac{1}{2}}$.
	
	\begin{thm}[\textbf{Intégrale de Wiener}]
		Soit $B$ un MB.
		Il existe un opérateur linéaire isométrique $I \colon L^2(\R_+) \to \mathcal{L}^2$, unique à une classe d'équivalence près pour l'égalité presque partout, et qui satisfait $I(\indic_{\intof{s}{t}}) = B_t - B_s$ pour tous $0 \leq s \leq t$.
		Par ailleurs $\esp[I(\varphi)] = 0$ pour tout $\varphi \in L^2(\R_+)$.
		Nous écrivons $I(\varphi) = \int_{\R_+} \varphi(t) \diff B_t$.
	\end{thm}
	
	\begin{rem}
		Dire que $I$ est une isométrie revient à dire $\forall \varphi \in L^2(\R_+), \esp \left[ I(\varphi)^2 \right] = \int \varphi^2(x) \diff x$.
	\end{rem}
	
	\begin{proof}
		Dans un premier temps, nous construisons $I$ sur l'ensemble $\mathcal{E}$ des fonctions en escalier, i.e de la forme $\varphi = \sum_{i = 1}^n a_i \indic_{\intof{t_{i-1}}{t_i}}$ où $0 \leq t_0 < t_1 < \ldots < t_n$.
		Par linéarité $I(\varphi) = \sum_{i = 1}^n a_i \left( B_{t_{i-1}} - B_{t_i} \right)$.
		Aussi nous avons sur $\mathcal{E}$,
		$$\norme{I(\varphi)}_{\mathcal{L}^2}^2
			= \esp \left[ \left( \sum_{i = 1}^n a_i (B_{t_{i-1}} - B_{t_i}) \right)^2 \right]
			= \sum_{i = 1}^n a_i^2 (t_i - t_{i-1})
			= \norme{\varphi}_{L^2(\R_+)}^2\ .$$
		Comme $\mathcal{E}$ est dense dans $L^2(\R_+)$, l'opérateur $I$ se prolonge d'une manière unique en une isométrie sur $L^2(\R_+)$.
		Il reste à prouver que $\esp[I(\varphi)] = 0$ pour tout $\varphi \in L^2(\R_+)$.
		Le résultat est évident sur $\mathcal{E}$.
		
		Soit $(\varphi_n)_n$ une suite de $\mathcal{E}$ qui tend vers $\varphi$ dans $L^2(\R_+)$.
		On a
		$$\abs{\esp[I(\varphi)]}
			= \abs{\esp[I(\varphi - \varphi_n)]}
			\leq \esp \abs{I(\varphi - \varphi_n)}
			\leq \norme{I(\varphi - \varphi_n)}_{\mathcal{L}^2}
			= \norme{\varphi - \varphi_n}_{L^2(\R_+)}$$
		en utilisant $\esp \abs{X} \leq \left( \esp[X^2] \right)^{1/2}$ et $I$ est une isométrie.
		
		Comme $\abs{\varphi - \varphi_n}_{L^2(\R_+)} \longrightarrow 0$ nous avons le résultat.
	\end{proof}
	
	\begin{pop}
		On a $\forall \varphi \in L^2(\R_+), I(\varphi) \sim \mathcal{N} \left( 0, \norme{\varphi}_{L^2(\R_+)}^2 \right)$.
	\end{pop}
	
	\begin{proof}
		On sait que $\esp[I(\varphi)] = 0$ et $\esp \left[ I(\varphi)^2 \right] = \norme{\varphi}^2_{L^2(\R_+)}$.
		Reste à établir la gaussianité.
		Pour ceci il suffit d'approximer $\varphi$ par une suite de fonctions dans $\mathcal{E}$ (dont les intégrales de Wiener sont par construction des gaussiennes) et de passer à la limite en utilisant un résultat du chapitre sur la loi gaussienne.
	\end{proof}

\section{La juridiction du travail}
	\noindent Le contexte est comme Monte-Carlo avec une variable observée en plus : $((X_1,Z_1), \ldots, (X_n,Z_n))$ i.i.d dans $S \times \R$, $X_1 \sim \mu$ et $\esp[Z_1]$ est connu.
Soit $\varphi \colon S \to \R$ tel que $\esp \abs{\varphi(X_1)} < \infty$, on cherche $I_\mu = \esp[\varphi(X_1)]$.

On peut se ramener à $\esp Z_1 = 0$, et on pose $\hat{I}_n^{(cv)} = \frac{1}{n} \sum_{i = 1}^n \left( \varphi(X_i) - Z_i \right)$.

\begin{pop}
	Si $\esp \abs{\varphi(X_1)} < \infty$ et $\esp \abs{Z_1} < \infty$, $\hat{I}_n^{(cv)}$ est sans biais et fortement consistant.
	Si de plus $\esp \left[ \abs{\varphi(X_1)}^2 \right] < \infty$ et $\esp \left[ \abs{Z_1}^2 \right] < \infty$ alors :
	\begin{itemize}
		\item[\textbullet] $\Var \left( \hat{I}_n^{(cv)} \right) = \frac{1}{n} \Var(\varphi(X_1) - Z_1)$ et $\hat{I}_n^{(cv)}$ est asymptotiquement normal avec variance $\sigma^2 = \Var(\varphi(X_1) - Z_1)$, i.e $\sqrt{n} \left( \hat{I}_n^{(cv)} - I \right) \overset{\mathcal{L}}{\longrightarrow} \mathcal{N}(0,\sigma^2)$,
		\item[\textbullet] un estimateur consistant de $\sigma^2$ est $\hat{\sigma}^2 = \frac{1}{n - 1} \sum_{i = 1}^n \left( (\varphi(X_i) - Z_i) - \hat{I}_n^{(cv)} \right)^2$.
	\end{itemize}
\end{pop}

%~ Pour un tel estimateur, le but est de réduire  ce risque $L^2$ au maximum en choisissant bien $Z_1$.
%~ Pour un tel estimateur, on obtient facilement les mêmes résultats que pour Monte-Carlo : forte consistance, normalité asymptotique et estimation consistante de la variance.

\begin{rem}
	Cela comprend Monte-Carlo : $Z_1 = 0$, et les variables antithétiques : $Z_1 = \frac{1}{2}(\varphi(X_1) - (\varphi \circ L)(X_1))$.
\end{rem}

\begin{rem}
	VC est plus performante que MC si $\Var(\varphi(X_1) - Z_1) \leq \Var(\varphi(X_1))$.
\end{rem}

Pour prévenir d'une mauvaise variable de contrôle, on définit l'estimateur $\frac{1}{n} \sum_{i = 1}^n (\varphi(X_i) - \beta Z_i)$, à utiliser si $\Var(\varphi(X_1) - \beta Z_1) \leq \Var(\varphi(X_1))$.
C'est vérifié avec $\beta^* = \argmin_\beta \Var \left( \varphi(X_1) - \beta Y_1 \right) = \esp[\varphi(X_1)Z_1]/\esp[Z_1^2]$.

%~ Soit $f_1,\ldots,f_m$ une collection de fonctions dont on connaît les intégrales.
%~ Supposons $\forall L \in \iniff{1}{m}, \int f_L \diff \lambda = 0$.
%~ Alors VC donne $\frac{1}{n} \sum_{i = 1}^n \left[ \varphi(u_i) - \sum_{j = 1}^m \beta_j f_j(u_i) \right]$.

%~ \begin{ex}
	%~ $(f_L)$ polynômes, $(f_L)$ base de Fourier ou $(f_L)$ indicatrices.
%~ \end{ex}


\subsection{Propriétés asymptotiques : $Z_1 \in \R^m$}

	On estime $I = \esp[\varphi(X_1)]$ par $\hat{I}_n^{(cv)}(\beta) = \frac{1}{n} \sum_{i = 1}^n \left( \varphi(X_i) - \transp{\beta}Z_i \right)$ où $\beta \in \R^m$, avec encore $\esp Z_1 = 0$.

	La valeur théorique pour minimiser la variance, si $\esp \left[ Z_1 \transp{Z_1} \right]$ est inversible, est $\beta^* = \esp \left[ Z_1 \transp{Z_1} \right]^{-1} \esp[Z_1 \varphi(X_1)]$.
	En pratique on l'estime.
	Notons $Z_{n,m} = \transp{\begin{pmatrix} Z_1 & \cdots & Z_n \end{pmatrix}} \in \R^{n \times m}$, et $\mathbf{\varphi}_i = \transp{\left(\varphi(X_1), \ldots, \varphi(X_n) \right)} \in \R^n$.
	
	Alors $\hat{\beta}_n = (\transp{Z_{n,m}} Z_{n,m})^{+} \transp{Z_{n,m}} \varphi_n = \left( \frac{1}{n} \sum Z_i \transp{Z_i} \right)^{-1} \frac{1}{n} \sum Z_i \varphi(X_i)$ (en notant $A^+$ l'inverse généralisé de $A$).
	
	\begin{pop}
		Supposons $\esp \abs{\varphi(X_1)} < \infty$, $\forall k \in \iniff{1}{m}, \esp \abs{\varphi(X_1) Z_{k,1}} < \infty$ et $\esp[Z_1 \transp{Z_1}]$ existe et est inversible.
		Alors $\hat{I}_n^{(cv)}(\hat{\beta}_n) \overset{\text{p.s.}}{\longrightarrow} I$ (fortement consistant).
		Si de plus $\esp \abs{\varphi(X_1)}^2 < \infty$, alors $\sqrt{n} \left( \hat{I}_n^{(cv)}(\hat{\beta}_n) - I \right) \overset{\mathcal{L}}{\longrightarrow} \mathcal{N}(0,\sigma_m^2)$ avec $\sigma_m^2 = \Var \left( \varphi(X_1) - \transp{{\beta^*}} Z_1 \right)$.
	\end{pop}
	
	\begin{rem}
		L'estimation de $\hat{\beta}_n$ n'a pas d'effet en l'asymptotique (c'est comme si on connnaissait $\beta^*$).
	\end{rem}
	
	\begin{rem}
		D'autres estimateurs de $\beta^*$ peuvent être légitimes sous condition d'inversibilité et de consistance.
		%~ Lorsque $\esp[Z_1 \transp{Z_1}]$ est connu, $\hat{\beta} = \esp[Z_1 \transp{Z_1}]^{-1} \frac{1}{n} \sum_{i = 1}^n \left( Z_i (\varphi(X_i) - \bar{\varphi}) \right)$.
	\end{rem}
	
	\begin{rem}
		On a $\forall m \geq 0, \sigma_{m + 1} \leq \sigma_m$ et $\sigma_0^2$ correspond à la variance de Monte-Carlo.
		Un estimateur de la variance de $\hat{I}_n^{(cv)}(\hat{\beta}_n)$ est donné par $\hat{\sigma}^2 = \frac{1}{n} \sum_{i = 1}^n \left( \varphi(X_i) - \transp{\hat \beta_n} Z_i - \hat{I}_n^{(cv)}(\hat{\beta}_n) \right)^2$.
	\end{rem}
	
	\begin{pop}
		Supposons $\esp \abs{\varphi(X_1)}^2 < \infty$, $\forall k \in \iniff{1}{m}, \esp \abs{\varphi(X_1) Z_{k,1}} < \infty$ et $\esp[Z_1 \transp{Z_1}]$ est inversible.
		Alors $\hat{\sigma}^2 \underset{n \to \infty}{\longrightarrow} \sigma_m^2$.
	\end{pop}


\subsection{Complexité du calcul}

	Règles du temps de calcul : générer $X_1$, générer $Z_{1,k}$ pour un $k$ et évaluer $\varphi(X_1)$ comptent chacun pour une opération élémentaire.
	
	\begin{tabular}{ll}
	Méthode & Nombre d'opérations élémentaires \\
	\hline
	Monte Carlo & $O(n)$ \\
	Calcul de $\hat{\beta}_n$ & $O(m^2 n + m^3)$ \\
	Variables de contrôle (avec $\hat{\beta}_n$ donné) & $O(mn)$
	\end{tabular}
	
	Donc la méthode avec variable de contrôle est mieux que Monte Carlo lorsque $\frac{\sigma_m}{\sigma_0} \leq \frac{1}{m}$.

\section{Formalisme d'un contrat}



\end{document}
