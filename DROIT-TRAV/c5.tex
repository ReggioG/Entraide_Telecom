\subsection{Motifs d'un CDD.}
	La conclusion d’un contrat à durée déterminée (CDD) n’est possible que pour l’exécution d’une tâche précise et temporaire et seulement dans les cas énumérés par la loi.
	On peut embaucher sous contrat à durée déterminée dans les cas suivants :
	\begin{itemize}
		\item[\textbullet] Remplacement d’un salarié absent (maladie, congé maternité, accident du travail, congés). Il n'est pas possible de remplacer un gréviste ou plusieurs personnes à la fois.
		\item[\textbullet] Remplacement d’un salarié passé provisoirement à temps partiel (congé parental d’éducation, temps partiel pour création ou reprise d’entreprise...).
		\item[\textbullet] Attente de la prise de fonction d’un nouveau salarié : on peut remplacer en CDD une personne qui a quitté l'entreprise ou changé de poste, dans l’attente de l’entrée en fonction de son remplaçant embauché en CDI.
		\item[\textbullet] Attente de la suppression définitive du poste du salarié ayant quitté définitivement l’entreprise.
		\item[\textbullet] Accroissement temporaire de l’activité de l’entreprise.
		\item[\textbullet] Emplois à caractère saisonnier : dont les tâches sont appelées à se répéter chaque année selon une périodicité à peu près fixe, en fonction du rythme des saisons ou des modes de vie collectifs. Il peut comporter une clause de reconduction pour la saison suivante.
		\item[\textbullet] Recrutement d’ingénieurs et de cadres, prévu selon un accord de branche étendu ou, à défaut, un accord d’entreprise.
		\item[\textbullet] Contrats à durée déterminée « d’usage », pour certains emplois par nature temporaires, dans des secteurs d'activité définis.
		\item[\textbullet] Cas particuliers : travaux d'urgence, mesures de sauvetages... Mais (sous réserve de dérogations) l est interdit d’employer un salarié en CDD pour effectuer des travaux dangereux.
		\item[\textbullet] Le CDD « senior ».
		\item[\textbullet] Le CDD « joueur professionnel » pour salarié de jeu vidéo compétitif.
	\end{itemize}

\subsection{Autres contrats à durée limitée}
	Les stages sont à part, il ne s'agit pas d'un contrat de travail.
	Au-delà de deux mois une gratification (indeminité) du stagiaire est obligatoire, avec un minimum de 3,6 \euro par heure.
	L'ancienneté n'y est pas comptabilisée.
	
	Le contrat d'apprentissage lui est apparenté à un CDD lorsqu'il est à durée limité mais peut aussi être un CDI.
	S'il donne suite à un CDI, la date d'ancienneté considérée est celle du contrat d'apprentissage.

	Les dispositions de l'intérim sont très proches, c'est un Contrat de Travail Temporaire.
	Il présente une relation triangulaire : un contrat de mise à disposition est conclu entre l'entreprise de travail temporaire (ETT) et l'entreprise utilisatrice, et un contrat de mission est conclu entre l'entreprise de travail temporaire et le salarié.
	L'employeur du salarié reste donc l'ETT, aucun contrat ne lie l'entreprise utilisatrice à l'intérimaire.
	Cela n'empêche pas que le salarié peut demander une requalification en CDI dans l'entreprise utilisatrice et poursuivre aussi bien son employeur (l'agence d'intérim) que l'entreprise utilisatrice selon l'infraction dénoncée.
	
	L'oubli du renouvellement d'un contrat de mission d'un intérimaire requalifie le contrat en CDD avec l'entreprise utilisatrice si l'on va aux prud'hommes.
	L'entreprise peut alors se retourner contre l'ETT.


\subsection{Modalités du CDD}
	La durée maximale du CDD est de 18 mois, renouvelable deux fois.

	Il contient les mêmes mentions que le CDI ainsi que la durée exacte de date à date et le motif de recours.
	Dans le cas d'un remplacement on indique généralement le nom du salarié remplacé et ses fonctions.
	
	S’il n’y a pas de contrat, ça devient un CDI (dans le cas des intérimaires, le salarié ne devient pas employé de l'entreprise d’intérim, mais directement de l'entreprise utilisatrice).

	\paragraph{}
	Dans le cas d’un remplacement pour arrêt maladie ou accident travail, on connait la date début mais pas forcément celle de fin.
	On emploie alors avec un contrat de travail à terme imprécis, comme la date de retour du salarié remplacé indiqué.
	La rémunération n’est pas forcément la même que celle du salarié remplacé.
	Mais le salarié CDD peut toucher absolument tous les avantages que touchait l’ancien employé qui était en CDI, à voir selon la convention d'entreprise.
	
	Il n'y a pas d’obligation pour l’employeur d’engager le remplaçant en cas de démission du salarié absent.
	De manière générale embaucher en CDI n'est pas une obligation après un CDD.

	À l’issue du CDD de remplacement est prévu une indemnité de précarité de l’ordre de 10\% de la rémunération totale durant toute la durée du contrat, heures sup comprise (sauf accord collectifs ou accord de branche avec un pourcentage inférieur).
	Cette indemnité est annulée si l’employé remplaçant est embauché à l’issu du CDD en CDI.
	Si l’on refuse le CDI, on perd également l’indemnité.
	L’abandon de poste supprime aussi l’indemnité.

	\paragraph{}
	Si l’on rompt le CDD pour un CDI, on peut démissionner, mais c'est le seul motif valable.
	Sinon, en cas de démission formelle, l’employeur peut demander au conseil des prud’hommes de condamner le salarié de payer les salaires restants avant la fin du contrat.
	En cas de faute grave, l’employeur peut virer le salarié sans préjudice en retour.