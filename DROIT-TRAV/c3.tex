La juridiction du travail, créée en 1907, est le 2\up{nd} acteur public.
En cas de conflit entre employé et employeur, le conseil des prud’hommes spécialisé dans le droit du travail privé est titulaire.
C'est la seule juridiction spécialisée en droit du travail.

Il y a 258 conseils des prud'hommes aujourd'hui, contre 271 en 2008 (suppressions par la loi Dati).

\subsection{Composition du conseil des prud’hommes}

	Il y a cinq sections selon le domaine d'activité, plus une formation de référé (procédure d'urgence) :
	\begin{itemize}
		\item[\textbullet] agriculture,
		\item[\textbullet] commerce et service,
		\item[\textbullet] industrie,
		\item[\textbullet] activités diverses,
		\item[\textbullet] encadrement (pour tous les cadres de France).
	\end{itemize}
	
	Dans ces sections ceux qui jugent ne sont pas des professionnels, ce sont des juges élus ayant reçu une courte formation (pas de magistrat).
	C'est une juridiction paritaire (autant d'employeurs que d'employés) et élective.
	
	Les élections qui avaient lieu jusqu'ici tous les cinq ans ont été abrogées en raison de leur coût et de leur faible taux de participation.
	À compter de 2017, ils seront nommés conjointement par le ministre de la justice et le ministre chargé du travail, en fonction de la représentativité de leur organisations syndicales et patronales pour une durée de 4 ans.

	\paragraph{}
	Plusieurs audiences prud’hommales existent.
	Au sein de chaque section on a :
	\begin{itemize}
		\item[\textbullet] \textbf{Bureau de conciliation et d'orientation (BCO)} : 1 président et 1 vice-président, donc 1 employeur et 1 employé.
			C'est une première étape visant à une conciliation pour éviter le bureau de jugement.
		\item[\textbullet] \textbf{Bureau de jugement} : 2 présidents et 2 vice-présidents, donc 2 employeurs et 2 employés.
		\item[\textbullet] \textbf{Formation de référé} : un conseiller salarié et un conseiller employeur.
		\item[\textbullet] \textbf{Audience de départage} : tribunal d'instance.
	\end{itemize}
	
	Les élus sont démarqués par des médailles, pas des robes de magistrat.
	
	En cas d'égalité entre les conseillers des bureaux (president vs vice-président), toutes les voix sont égales : pas d’avantage pour le président. On fait alors appel à un juge professionnel dit « départiteur » : audience de départage au tribunal d’instance (autre juridiction).


\subsection{Caractéristiques du BCO}

	\begin{itemize}
		\item[\textbullet] Audience à huit clôt avec 4 conseillers.
		\item[\textbullet] Indemnités fixées par des barêmes indicatifs.
		\item[\textbullet] La BCO permet la conciliation de 10\% des litiges, 90\% finissent en jugement.
		\item[\textbullet] La BCO donne lieu à un PV de conciliation.
			Il peut être partiel et donc laisser une partie du dossier à la charge bureau de jugement.
		\item[\textbullet] On peut ordonner à l’employeur de fournir des documents confidentiels de l’entreprise si nécessaire pour sa défense, pour éviter de devoir voler les documents.
			Même si l’on vole, on peut être blanchi en prouvant que les documents volés étaient nécessaires, mais problème de condamnation pénale du salarié sinon.
		\item[\textbullet] La comparution est obligatoire mais depuis la loi El Komri on peut se faire représenter (autre personne, avocat).
		\item[\textbullet] On peut faire un appel à un témoin / expert ou visiter l'entreprise.
		\item[\textbullet] On peut demander une formation restreinte (juste 2 conseillers), on peut même demander directement un juge de départage (procédure plus rapide).
		\item[\textbullet] La BCO est une phase obligatoire sauf exception :
			\begin{itemize}
				\item[\textopenbullet] Redressement, liquidation judiciaire.
				\item[\textopenbullet] CDD transformé en CDI.
				\item[\textopenbullet] Prise d’acte de la rupture du contrat de travail (Cf \ref{autoli}).
			\end{itemize}
	\end{itemize}

	Les contentieux liés aux heures supplémentaires sont les plus durs à régler.
	Consigner les informations par mail est alors important, même en l'absence de réponse.
	Dans des cas comme le télé-travail se pose le problème du calcul des heures : caméras de surveillances mais problèmes avec la CNIL, géolocalisation des véhicules.
	L'employeurs a accès aux outils de travail : ordinateur, téléphone portable.
	
	Depuis la loi El Komri, le salarié a un droit de déconnexion, mais il est difficile à mettre en pratique.


\subsection{Caractéristiques du bureau de jugement (BJ)}

	\begin{itemize}
		\item[\textbullet] Procédure orale, c'est une audience publique avec présence d'un avocat obligatoire et présence personnelle ou représentation.
		\item[\textbullet] Principe du respect du contradictoire : toutes les pièces doivent être communiquées entre les parties \textrightarrow\ pas de surprise, les dossiers sont connus.
		\item[\textbullet] Après le débat : mise en délibéré, puis examen du dossier et décision ou nouvelle audience avec juge départiteur.
		\item[\textbullet] On peut court-circuiter la procédure en inférant la procédure des référés dans un licenciement mal effectué. On accélère donc l’acquisition des documents ou solde tout compte (1 mois).
	\end{itemize}

	La durée totale de la procédure est de un an.
	
	Dans un cas de problème de remise de papier de rupture de contrat (pas de remise en question de celui-ci) on saisit la formation des référés.


\subsection{Formalités du conseil des prud'hommes}

	\begin{itemize}
		\item[\textbullet] Gratuit et pas besoin d’avocat.
		\item[\textbullet] Formulaire de 11 pages à remplir pour la demande.
	\end{itemize}
	
	Règle générale en justice : les documents les plus valables sont ceux qui sont écrits. Ainsi on peut consigner par mail ce qui s’est passé dans l’entreprise.
	Les prud'hommes n'agissent que sur les litiges individuels (pas de manif, licenciement groupé etc..).
	
	Vocabulaire : le demandeur est l'attaquant et le défendeur est l'attaqué.
	
	\paragraph{Comment choisir son conseil des prud'hommes ?}
	On choisit le conseil attaché à la zone (région / département) du défendeur (lieu de travail).
	Si le lieu de travail n’est pas fixe, on prend le conseil du domicile du salarié.
	Dans certains cas, d’autres conseils peuvent être choisis comme celui lié au siège social de l'entreprise.


\subsection{Cinq juridictions qui interviennent dans le droit du travail}

	\begin{tabular}{m{4cm}|m{11cm}}
	\textit{Tribunal} & \textit{Exemples de cas type} \\ \\

	Tribunal administratif & Pour les salariés protégés, ils ne peuvent être licencié sans accord de l’inspecteur du travail au cours de leur mandat. Cela est plus contraignant donc mal vu par les entreprises. Idem pour la modification d'un contrat. \\ \\

	Tribunal d’instance & Pour les litiges en dessous de 10 000 \euro, principalement immobilier. Droit de la consommation, impayé de loye élection, ou élection de représentant du personnel pour le travail. \\ \\

	Tribunal de grande instance & Pour les litiges au dessus de 10 000 \euro. Il traite les cas d'accords d'entreprise contestés, d'annulation d'un réglement intérieur et de manière générale tout ce qui relève du collectif. \\ \\

	Tribunal du commerce & Procédure collective, en cas de problème économique pour l'entreprise : redressement judiciaire, liquidation. \\ \\

	Tribunal pénal (de police ou correctionnel) & Accident mortel au travail ou infraction pénale dans le droit du travail : accident avec manquement aux règles de sécurité.
	\end{tabular}\\
	
	Si un des partis n’est pas satisfait du jugement, la cours d’appel est saisie (magistrat professionnel).
	Si encore pas satisfait par le nouveau jugement, c'est la cours de cassation.
	Cette dernière ne réexamine pas l'affaire mais vérifie si la loi a bien été appliquée : pas de vice de procédure, de violation d'une liberté, d'un article...
	
	Le délai est alors de 5 à 8 ans.
