Le droit social inclut le droit du travail et le droit de la sécurité sociale.
Le droit du travail régit la relation entre employeur privé et salarié.
Il ne concerne pas le domaine public (géré par le droit de la fonction publique).

\paragraph{Interférences entre droits :}
	\begin{itemize}
		\item[\textbullet] Salarié hybride dit « protégé », du style syndicat, CE, délégué du personnel : dépend du droit de la fonction publique \textrightarrow\ licenciement lié au tribunal administratif.
		\item[\textbullet] Lors d’accident mortel sur un chantier : droit pénal, le patron est responsable pénalement. Un moyen d'y échapper est de déléguer sécurité et hygiène à un cadre : il faut savoir ce à quoi on s'engage.
	\end{itemize}

\paragraph{Partenaires sociaux :}
	Des accords d'entreprise ou de branche sont négociés entre la direction et les représentants du personnel.
	Ces accords ont pour buts d'améliorer les conditons de travail.
	Les deux principaux types de partenaires sont qui interviennent comme médiateurs dans les négociations sont :
	\begin{itemize}
		\item[\textbullet] les organisations patronales (syndicats) : MEDEF, UPA, CGPME,
		\item[\textbullet] les organisations syndicales représentatives pour les employés : CGT, CFDT, FO, CGCCFE (cadres), UNSA.
	\end{itemize}
	En cas de modification importante du droit social, le législateur doit négocier avec ces organisations.

	Le salarié élu est estimé plus exposé car il présente des revendications collectives.
	Le nombre d'heures est plus grand car il s'occupe de son rôle de délégué (de 10 à 20 heures par mois, défini selon la législation en vigueur).