\paragraph{Temps de travail effectif :} temps durant lequel le salarié est à la disposition de l’employeur, où il se conforme à ses directives sans pouvoir vaquer à ses occupations personnelles.

Par exemple le dejeuner, le transport entre le lieu de travail et le domicile ou les pauses ne sont pas comptabilisés comme du temps de travail.

Dans le cas de déplacements à l'étranger ou entre les sites, soit la CC a prévu des mesures pour les longs et courts trajets, soit le temps passé à l’extérieur pour l'entreprise compte en heures supplémentaires.

Le temps d'habillage / déshabillage (uniforme, tenue de sécurité) est considéré comme temps de travail.
Il existe sinon des primes d'habillage / déshabillage.

\paragraph{Astreinte :} période de temps ou l’employé n’est pas à la disposition permanente et immédiate de l’employeur mais a l’obligation de demeurer chez lui ou proche de chez lui afin d’etre en mesure d’intervenir pour accomplir un travail au service de l’entreprise.
Le temps passé durant la mission est bien du temps de travail effectif.
Le reste du temps de la journée est considéré en dehors du temps de travail.
En contrepartie du temps d’astreinte, une compensation financière est versé ou des avantages.

\paragraph{}
Légalement, la durée de travail effectif d’un salarié est fixé à 35h par semaine.
Au-delà on est en heures sup.
La durée hebdomadaire maximale est de 48h.
La durée mininmale de repos quotidien entre deux journées de travail est de 10h.

Il est impossible de travailler plus de 6h consécutive.
Le temps de pause mininmum est de 20 min.
On peut travailler 6 jours par semaine maximum.
Le travail le dimanche se fait jusqu’à 13h.

On peut travailler un jour férié (11 dans l'année) mais pas un jour chômé (exemple : fête du travail) sauf exception pour certains secteurs.
En cas d'arrêt de travail pendant des congés payés, ces congés payés sont reportés.

Sur la fiche de paye, il y a une ligne annotée « Salaire de base » avec une colonne « Taux » (combien je gagne brut par heure), puis « Nombre d'heures » (pendant le mois) puis le total \texttt{Taux $\times$ NbHeures} (toujours brut).

\paragraph{Paiement des heures supplémentaires.}
Un employé ne peut pas refuser d'effectuer des heures supplémentaires, mais la loi a l'avantage d'imposer une norme commune quant au traitement de ces heures.

Les heures sup sont majorés normalement à 25\%.
Certaines d’entre elles le sont à 50\% à partir de 8 heures sup.
Le plafond d’heures sup annuel est de 180 heures sup/an maximum (peut varier selon la CC, 220h maximum).
Si le salarié dépasser ce maximum, les nouvelles heures sup sont majorés de 100\% (50\% pour une entreprise de moins de 20 salariés).
Ces “super” heures sup donnent droit à une journée de repos compensateur.

Le salarié peut contester le compte d’heures sup figurant sur la fiche de paye.
Il lui faut fournir les preuves et constituer un tableau récapitulatif du compte heures qu’il estime correct.

\paragraph{}
Quand l’employeur ne sais exactement combien de temps ses salarié vont devoir travailler est établie une convention individuelle de forfait : rémunération forfaitaire en nombre de jours ou heures sur l’année. Cela permet aussi d’eviter les heures sup et est bien adapté aux cadres.

Le forfait annuel individuel maximum est de 217 jours ou 1607 heures.

\paragraph{Travail de nuit :} de 21h jusqu'à 7h du matin.
Pour être travailleur de nuit il faut effectuer des heures de travail au moins deux fois par semaines dans cette plage horaire.

