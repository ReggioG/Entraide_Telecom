\subsection{Établissement d'un arrêt maladie}
	Pour un arrêt de travail pour maladie il faut un document fourni par le médecin.
	Deux volet de l'arrêt de travail doivent être transmis à la caisse primaire d'assurance maladie (CPAM) et un troisième à l'employeur.
	La remise doit se faire sous deux jours ouvrables, ou autre pour l'employeur selon la convention collective.
	La déclaration de la maladie justifie l'absence au travail, et permet de percevoir des indemnités de la Sécurité sociale et de l'employeur.
	La raison médicale de l’absence n’est pas communiquée à l’employeur (secret médical).

	Durant l'arrêt de travail, le salarié doit respecter les obligations suivantes :
	\begin{itemize}
		\item[\textbullet] suivre les prescriptions de son médecin,
		\item[\textbullet] se soumettre aux contrôles médicaux organisés par l'employeur et la CPAM,
		\item[\textbullet] respecter l'interdiction de sortie ou les heures de sorties autorisées,
		\item[\textbullet] s'abstenir de toute activité, sauf autorisation du médecin.
	\end{itemize}

	Un contrôle peut être exercé par la Sécurité sociale et l'employeur, avec des pénalités financières en cas de non respect des obligations.

	En cas de doute de la véracité de l’arrêt maladie, l’employeur peut demander une contre visite médicale pour vérifier l’arrêt maladie.
	On parle d'arrêt de complaisance.
	Le salarié ne peut pas refuser la consultation.
	Une contre visite ne vaut que pour un arrêt et peut être facilement contre-carrée.
	Si le médecin traitant re-délivre un arrêt, il faut effectuer une 2\up{nd} contre visite.
	Une rupture de contrat peut alors être envisagée par l'employeur.

\subsection{Licenciement pour maladie.}
	Aucun salarié ne peut être licencié en raison de son état de santé, ce qui serait considéré comme discriminatoire.
	Cependant, la maladie du salarié peut avoir des conséquences sur sa capacité à reprendre un emploi ou sur la bonne marche de l'entreprise, qui peuvent justifier un licenciement.

	Le licenciement d'un salarié déclaré inapte, par le médecin du travail, à reprendre son emploi est possible :
	\begin{itemize}
		\item[\textbullet] si son employeur est dans l’incapacité de lui proposer un nouvel emploi adapté à ses capacités,
		\item[\textbullet] ou si le salarié refuse le(s) poste(s) proposé(s) correspondant aux préconisations du médecin du travail.
	\end{itemize}

	Une perturbation au sein de l'entreprise peut être évoquée dans les conditions suivantes pour licencier un salarié :
	\begin{itemize}
		\item[\textbullet] son absence prolongée ou ses absences répétées perturbent le fonctionnement de l’entreprise,
		\item[\textbullet] l'employeur se trouve dans la nécessité de pourvoir à son remplacement définitif,
		\item[\textbullet] et l'origine des absences du salarié n'est pas liée à un manquement de l'employeur a son obligation de sécurité (par exemple, absences liées à un harcèlement moral).
	\end{itemize}

	Les motifs du licenciement sont appréciés au cas par cas par le juge.
	La convention collective applicable peut prévoir que le salarié absent pour maladie ne peut être licencié qu'au terme d’un certain délai.

	Que ce soit pour inaptitude ou perturbation, l'employeur doit respecter la procédure de licenciement pour motif personnel.

	\textrightarrow\ L'arrêt maladie ne protège en rien de la rupture du contrat.

\subsection{Retour du salarié.}
	Au retour d’un arrêt de travail supérieur à 1 mois, il faut revenir voir le médecin du travail sous 8 jours pour définir l’aptitude au travail (visite de reprise).
	C'est à l'employeur d'organiser le rendez-vous.
	Le salarié ne peut pas reprendre son poste avant la visite.
	Le médecin du travail peut déclarer une aptitude complète, partielle (restriction médicale sur des postes de l'entreprise) ou nulle.

	Une deuxième visite au bout de 15 jours est possible si pas d’accord du médecin : période d'étude de poste pour chercher un nouveau poste à occuper, non payée.
	La recherche durant cette première période incombe à l'employeur.
	
	Durant la deuxième visite, le médecin peut dire si l’employé est inapte à tous postes.
	Dans ce cas, on peut licencier sous 1 mois le salarié, appelé mois de reclassement (non payé).
	Financièrement il est plus intéressant de se remettre en arrêt maladie.
	
	Une tentative de reclassement peut être sanctionnée par les prud'hommes.
	C'est le cas si l'on prouve que la recherche de la part de l'employeur n'a pas été suffisante : un mail “lettre circulaire” n'est pas assez.
	Il y a obligation de moyen mais pas de résultat.