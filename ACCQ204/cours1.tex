\subsection{Définition des codes en blocs}

	Principe : $\underset{m}{k\ \text{bits}} \longrightarrow \underset{c \in \mathcal{C}}{n\ \text{bits}}$ avec $n > k$.
	On a $\dim(\mathcal{C}) = k$.
	
	\begin{defn}
		Rendement : $R = \frac{k}{n}$.
	\end{defn}
	
	\begin{note}
		$C(k,n,d_{\min})$.
	\end{note}
	
	\begin{defn}
		Capacité de détection : $t \leq d_{\min} - 1$. Capacité de correction : $t \leq \lfloor \frac{d_{\min} - 1}{2} \rfloor$.
	\end{defn}
	
\subsection{Les codes linéaires en blocs}

	$\mathcal{C}$ est un sev de $\GF(2)^n$.
	Alors $d_{\min} = \min_{c \neq 0} w_H(c)$ (poids de Hamming).

	\begin{defn}
		Matrice génératrice : $G = [ I_{k \times k} \mid P_{k \times (n - k)} ] \in \M_{k \times n}$ sous forme systématique telle que $c = m \cdot G = [ m \mid n - k\ \text{bits de parité}]$.
	\end{defn}

	\begin{defn}
		Matrice de parité : $H \in \M_{(n - k) \times n}$ la matrice génératrice de $\mathcal{C}^\perp$, donc $\forall c \in \mathcal{C}, c \cdot \transp{H} = 0$.
		Sous forme systématique : $H = [ -\transp{P} \mid I_{n - k}]$.
	\end{defn}

	\begin{defn}
		Vecteur syndrôme : $s = r \transp{H}$ avec $r$ le mot reçu.
		Alors $s = 0$ ssi $R$ est un mot de code.
	\end{defn}

	\begin{thm}[Borne de singleton]
		$d_{\min} \leq n - k + 1$, d'où la correction d'erreur $2t \leq d_{\min} - 1 \leq n - k$.
	\end{thm}

	On effectue alors un décodage par maximum likelyhood : si les éléments de l'alphabet de départ sont équiprobables, on cherche $\max p(y \mid x)$.


\subsection{Transformations}

	\begin{itemize}
		\item[\textbullet] Extension : rajouter des bits de parité.
		\item[\textbullet] Allongement : rajouter des bits d'info.
		\item[\textbullet] Perforation : supprimer des bits de parité.
		\item[\textbullet] Raccourcissement : supprimer des bits d'info.
		\item[\textbullet] Augmentation : ajouter des bits d'info sans modifier la longueur.
		\item[\textbullet] Expurgation : supprimer des bits d'info sans modifier la longueur.
	\end{itemize}
