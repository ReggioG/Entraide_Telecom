\subsection{Définitions et propriétés}

	\begin{defn}
		Une mesure de proba $\mu$ sur $(\R^d, \mathcal{B}(\R^d)))$ est caractérisée par sa \textbf{fonction de répartition}
		$$F_\mu \colon (x_1,\ldots,x_d) \mapsto \mu \left( \prod_i \intof{-\infty}{x_i} \right)\ .$$
	\end{defn}
	
	\begin{pop}
		\begin{itemize}
			\item[\textbullet] $F_\mu$ croissante.
			\item[\textbullet] $F_\mu(-\infty) = 0$, $F_\mu(+\infty) = 1$
			\item[\textbullet] $F_\mu$ est continue à droite et $\mu({x_0}) = F_\mu(x_0) - F_\mu(x_0^-)$
		\end{itemize}
	\end{pop}
	
	\begin{thm}
		Deux mesures distintes ne peuvent pas avoir la même fonction de répartition.
	\end{thm}
	
	Soit $X \colon \Omega \to \R^d$ une v.a.
	On note $P_X = \proba \circ X^{-1}$ la loi de $X$.
	$P_X$ est une mesure de proba sur $\R^d$.
	On note $F_X$ sa fonction de répartition.
	Pour $d = 1$, $F_X(x) = \proba(X \leq x)$.
	
	\begin{defn}
		Soit $(\mu_n)_n, \mu$ des mesures de proba sur $\R^d$.
		On dit que $\mu_n$ \textbf{converge faiblement (ou étroitement)} vers $\mu$ si $\lim_n F_{\mu_n}(x) = F_\mu(x)$ en tout $x$ point de continuité de $F_\mu$.
		On note $\mu_n \Rightarrow \mu$.
	\end{defn}

	\begin{defn}
		$(X_n)_n, X$ v.a. sur $\R^d$.
		On dit que $X_n$ \textbf{converge en loi} vers $X$ (noté $X_n \overset{\mathcal{L}}{\longrightarrow} X$) si $P_{X_n} \Rightarrow P_X$.
	\end{defn}

	\begin{pop}
		Si $X_n \overset{\proba}{\longrightarrow} X$ alors $X_n \overset{\mathcal{L}}{\longrightarrow} X$.
	\end{pop}

	\begin{thm}[de représentation de \textbf{Skorohod}]
		Soit $(\mu_n)_n, \mu$ des mesures de proba sur $\R^d$ telles que $\mu_n \Rightarrow \mu$.
		Il existe un espace de proba et des v.a. $(Y_n), Y$ sur cet espace à valeurs dans $(\R^d,\mathcal{B}(\R^d))$ telles que $Y \sim \mu$, $\forall n, Y_n \sim \mu_n$ et $Y$ est limite simple des $Y_n$, i.e. $\forall \omega, Y_n(\omega) \longrightarrow Y(\omega)$.
	\end{thm}

	\begin{thm}[de continuité]
		Soit $h \colon \R^d \to \R^p$ mesurable et continue sur $C \in \mathcal{B}(\R^d)$ tel que $\proba(X \in C) = 1$.
		Si $X_n \overset{\mathcal{L}}{\longrightarrow} X$ alors $h \circ X_n \overset{\mathcal{L}}{\longrightarrow} h \circ X$.
	\end{thm}
	
	\begin{thm}
		Les affirmations suivantes sont équivalentes :
		\begin{enumerate}[(i)]
			\item $\mu_n \Rightarrow \mu$,
			\item pour toute fonction $f$ continue, $\mu_n(f) \longrightarrow \mu(f)$,
			\item pour toute fonction $f$ lipschitzienne, $\mu_n(f) \longrightarrow \mu(f)$,
			\item pour tout $A \in \mathcal{B}(\R^d)$ tel que $\mu(\delta A) = 0$ (frontière de $A$), $\mu_n(A) \longrightarrow \mu(A)$.
		\end{enumerate}
	\end{thm}
	
	\begin{thm}[de \textbf{Portmanteau}]
		On a équivalence entre :
		\begin{enumerate}[(i)]
			\item $X_n \overset{\mathcal{L}}{\longrightarrow} X$,
			\item $\forall f \colon \R^d \to \R$ continue bornée, $\esp(f(X_n)) \longrightarrow \esp(f(X))$,
			\item $\forall A \subset \R^d$ tel que $\proba(X \in \delta A) = 0$, on a $\proba(X_n \in A) \longrightarrow \proba(X \in A)$ où $\delta A = \Bar{A} \setminus \mathring{A}$,
			\item $\forall t \in \R^d, \lim_n \phi_{X_n}(t) = \phi_X(t)$.
		\end{enumerate}
	\end{thm}
	
	\begin{thm}
		Soit $m \in \N^*$ et $h \colon \R^d \to \R^p$ mesurable et continue sur $C \in \mathcal{B}(\R^d)$ tel que $\proba(X \in C) = 1$.
		Si $\mu_n \Rightarrow \mu$ alors $\mu_n h^{-1} \Rightarrow \mu h^{-1}$.
	\end{thm}
	
	\begin{thm}[Procédé de Cramer-Wold]
		Soit $X_n, X$ des v.a. sur $\R^d$.
		On a $X_n \overset{\mathcal{L}}{\longrightarrow} X \iff \forall t, \scal{t}{X_n} \overset{\mathcal{L}}{\longrightarrow} \scal{t}{X}$.
	\end{thm}


\subsection{Mesures tendues}

	\begin{lem}[d'Helly]
		Soit $(F_n)_n$ une suite de fonctions de répartition.
		Il existe une sous-suite $(\varphi(n))_n$ et $F \colon \R \to \intff{0}{1}$ croissante, continue à droite, telle que $\lim_n F_{\varphi(n)}(x) = F(x)$ en tout $x$ point de continuité de $F$.
	\end{lem}
	
	On ajoute une condition pour que la limite vérifie $\lim_{x \to -\infty} F(x) = 0$ et $\lim_{x \to +\infty} F(x) = 1$.

	\begin{defn}
		$(\mu_n)_n$ est dite \textbf{tendue} si $\forall \varepsilon > 0, \exists \mathcal{K} \ \text{compact}, \forall n, \mu_n(\mathcal{K}) \geq 1 - \varepsilon$.
	\end{defn}

	Dans le cas $d = 1$ on peut prendre $\mathcal{K} = \intff{-K}{K}$.
	
	\begin{defn}
		$(X_n)_n$ est \textbf{tendue} si $\forall \varepsilon > 0, \exists \mathcal{K} \ \text{compact}, \forall n, \proba(X_n \in \mathcal{K}) \geq 1 - \varepsilon$.
	\end{defn}
	
	\begin{pop}
		\begin{enumerate}[(i)]
			\item Toute famille finie de mesures de probabilité est tendue.
			\item Une suite de mesures qui converge faiblement est tendue.
		\end{enumerate}
	\end{pop}

	\begin{thm}[de Prokhorov]
		Soit une famille $\mathcal{M}$ de mesures de probabilité.
		Alors $\mathcal{M}$ est tendue si et seulement si elle est relativement séquentiellement compacte pour la topologie de la convergence faible, i.e. de toute suite $(\mu_n)_n$ de $\mathcal{M}$ on peut extraire une sous suite $(\mu_{\varphi(n)})_n$ telle que $\mu_{\varphi(n)} \Rightarrow \mu$ avec $\mu$ une mesure de probabilité.
	\end{thm}

	\begin{pop}
		Soit $(\mu_n)_n$ tendue.
		Si toute sous-suite faiblement convergente de $(\mu_n)_n$ converge vers $\mu$, alors $\mu_n \Rightarrow \mu$.
	\end{pop}


\subsection{Fonction caractéristique}

	\begin{defn}
		La \textbf{fonction caractéristique} d'une mesure de proba $\mu$ sur $\R^d$ est
		$\phi_\mu \colon \begin{array}{rcl}
		\R^d & \to & \C \\
		t & \mapsto & \int e^{i \scal{t}{x}} \diff \mu(x)
		\end{array}$.
	\end{defn}
	
	\begin{thm}
		$\varphi_\mu = \varphi_\nu \implies \mu = \nu$.
	\end{thm}
	
	\begin{pop}
		Soit $\mu$ une mesure sur $\R$ et $p \in \N$ tel que $\int \abs{x}^p \diff \mu(x) < \infty$.
		Alors $\phi_\mu$ est $p$ fois continument dérivable et $\forall t \in \R$, sa $p$\up{e} dérivées satisfait $\forall t \in \R, \phi_\mu^{(p)} = \int i^p x^p e^{itx} \diff \mu(x)$.
	\end{pop}

	\begin{ex}
		$\forall t \in \R, \phi_{\normale(0,1)}(t) = e^{-t^2/2}$.
	\end{ex}

	\begin{pop}
		Pour $Y = AX + b$ on a $\forall t, \phi_Y(t) = e^{i \scal{t}{b}} \phi_X(A^* t)$.
	\end{pop}

	\begin{pop}
		$\phi_\mu$ est continue en zéro.
	\end{pop}

	\begin{thm}[de Lévy]
		Soit $(\mu_n)_n$, $\mu$ des mesures de probabilité sur $\R^d$.
		$\mu_n \Rightarrow \mu$ ssi $\forall t \in \R^d, \phi_{\mu_n}(t) \longrightarrow \phi_\mu(t)$.
	\end{thm}


\subsection{Théorème centrale limite}

	\begin{note}
		$\normale(m,\sigma^2)$ désigne la loi de densite $\rho(x) = \frac{1}{\sqrt{2\pi} \sigma} e^{-\frac{(x - m)^2}{2 \sigma^2}}$, et si $\sigma^2 = 0$ c'est la loi $\delta_m$.
	\end{note}

	Pour $X$ gaussien, sa fonction caractéristique vérifie $\phi_X(t) = e^{i\scal{t}{m}} e^{-\frac{1}{2} \scal{t}{\Sigma t}}$ où $m = \esp(X)$ et $\Sigma = \Cov(X)$.
	
	\begin{thm}[central limite]
		Soit $(X_n)_n$ iid tel que $\esp(\norme{X_1}^2) < \infty$.
		Alors
		$$\frac{1}{\sqrt{n}} \sum_{i = 1}^n (X_i - \esp(X_1)) \Rightarrow \normale(0,\Cov(X_1))\ .$$
	\end{thm}

	\begin{thm}[de Linderbergh]
		Soit un tableau de v.a. $(X_{i,n})_{1 \leq i \leq n}$ tel que
		\begin{itemize}
			\item[\textbullet] $\forall n$, les v.a $X_{1,n},\ldots,X_{n,n}$ sont indépendantes,
			\item[\textbullet] $\forall n, \forall i \leq n, \esp(X_{i,n}) = 0$,
			\item[\textbullet] $\exists \Sigma, \lim_n \sum_{i = 1}^n \Cov(X_{i,n}) = \Sigma$
			\item[\textbullet] $\forall \varepsilon > 0, \lim_n \sum_{i = 1}^n \esp(\norme{X_{i,1}}^2 \indic_{\norme{X_{i,n}}} > \varepsilon) = 0$ (condition de Lindebergh).
		\end{itemize}
		Alors $\sum_{i = 1}^n X_{i,n} \overset{\mathcal{L}}{\longrightarrow} \normale(0,\Sigma)$.
	\end{thm}
 
	\begin{lem}
		Soit $\mathcal{U} := \{ z \in \C \mid \abs{z} \leq 1 \}$, $n \in \N^*$ et deux familles $(z_i)_{1 \leq i \leq n}$ et $(w_i)_{1 \leq i \leq n}$ de $\mathcal{U}$.
		Alors $\abs{ \prod_{i = 1}^n z_i - \prod_{i = 1}^n w_i } \leq \sum_{i = 1}^n \abs{z_i - w_i}$.
	\end{lem}
