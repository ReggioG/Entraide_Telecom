\paragraph{Rappel de règles}

	\begin{pop}
		\begin{enumerate}[(i)]
			\item $X_n \overset{\mathcal{L}}{\longrightarrow} c \iff X_n \overset{\proba}{\longrightarrow} c$
			\item $\left. \begin{array}{r}
				X_n \overset{\mathcal{L}}{\longrightarrow} X \\
				Y_n - X_n \overset{\mathcal{L}}{\longrightarrow} 0
				\end{array} \right\}
				\implies Y_n \overset{\mathcal{L}}{\longrightarrow} X$
			\item $\left. \begin{array}{r}
				X_n \overset{\mathcal{L}}{\longrightarrow} X \\
				Y_n \overset{\mathcal{L}}{\longrightarrow} c
				\end{array} \right\}
				\implies (X_n,Y_n) \overset{\mathcal{L}}{\longrightarrow} (X,c)$
			\item $\left. \begin{array}{r}
				X_n \overset{\proba}{\longrightarrow} X \\
				Y_n \overset{\proba}{\longrightarrow} c
				\end{array} \right\}
				\implies (X_n,Y_n) \overset{\proba}{\longrightarrow} (X,c)$
		\end{enumerate}
	\end{pop}


\subsection{Notation $o_P$, $O_P$}

	Soit $(X_n)$ des v.a. $\Omega \to \R^d$ et $(Y_n)$ v.a.r.
	
	\begin{defn}
		La notation $X_n = o_P(1)$ signifie $X_n \overset{\proba}{\longrightarrow} 0$.
		$X_n = o_P(Y_n)$ signifie $\exists (Z_n) \overset{\proba}{\longrightarrow} 0, X_n = Z_n Y_n$.
		$X_n = O_p(1)$ signifie que $(X_n)$ est tendue.
		On dit que $X_n$ est "bornée en probabilité" lorsqu'elle est tendue.
		$X_n = O_P(Y_n)$ signifie $\exists (Z_n) = O_p(1), X_n = Z_n Y_n$.
	\end{defn}

	\begin{pop}
		Si $X_n \overset{\mathcal{L}}{\longrightarrow} X$ alors $X_n = O_P(1)$.
	\end{pop}

	\begin{pop}
		\begin{enumerate}[(i)]
			\item $o_P(1) + O_P(1) = O_P(1)$,
			\item $o_P(1) \cdot O_P(1) = o_P(1)$,
			\item $o_P(1) + o_P(1) = o_P(1)$,
			\item $\frac{1}{1 + o_P(1)} = O_P(1)$.
		\end{enumerate}
	\end{pop}


\subsection{Lemme de Slutsky et applications}

	\begin{lem}[de Slutsky]
		Si $X_n \overset{\mathcal{L}}{\longrightarrow} X$ et $Y_n \overset{\mathcal{L}}{\longrightarrow} c$ alors $X_n + Y_n \overset{\mathcal{L}}{\longrightarrow} X + c$, $X_n Y_n \longrightarrow cX$ et $\frac{X_n}{Y_n} \longrightarrow \frac{X}{c}$ (si $c \neq 0$).
	\end{lem}


\subsection{Delta-méthode}

	Soit $g \colon \R^d \to \R^m$ dérivable en un point $\nu \in \R^d$ de matrice jacobienne $\nabla g(\nu)$.
	
	Rappel : $\lim_{h \to 0} \frac{\norme{g(\nu + g) - g(\nu) - \nabla g(\nu) \cdot h}}{\norme{h}} = 0$, $\nabla g(\nu) = \left( \frac{\partial g_i(\nu)}{\partial \nu_j} \right)_{i \in \intiff{1}{d}, j \in \intiff{1}{m}}$.

	\begin{thm}
		Soit $g \colon \R^d \to \R^m$ dérivable en $\nu$.
		Soient $T_n, T$ des v.a. sur $\R^d$ et $(r_n)$ une suite réelle telle que $r_n \longrightarrow +\infty$, $r_n(T_n - \nu) \overset{\mathcal{L}}{\longrightarrow} T$.
		Alors $r_n(g(T_n) - g(\nu)) \overset{\mathcal{L}}{\longrightarrow} \nabla g(\nu) \cdot T$.
	\end{thm}

