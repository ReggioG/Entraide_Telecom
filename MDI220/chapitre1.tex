On note $\Omega$ l'univers et $\mathcal{F}$ la tribu des événements.
On considère une variable aléatoire $X$, appelée observation, définie sur $(\Omega, \mathcal{F})$ et à valeur dans l'espace des observations $(\mathcal{X}, \mathcal{B}(\mathcal{X}))$, ou $\mathcal{B}(\mathcal{X})$ est une tribu composée de parties de $\mathcal{X}$.

\begin{defn}
	\textbf{Modèle statistique} : famille de probabilités $\mathcal{P}$ sur $\mathcal{B}(\mathcal{X})$.
	Si $\Theta$ est un ensemble quelconque tel que $\mathcal{P} = \{ P_\theta, \theta \in \Theta \}$ alors $\Theta$ est appelé \textbf{espace des paramètres} du modèle.
\end{defn}

\begin{rem}
	L'existence d'une paramétrisation est toujours acquise, quitte à prendre $\Theta = \mathcal{P}$.
\end{rem}

Si $\Theta$ peut être choisi comme sous-ensemble d'un espace euclidien, le modèle est dit \textbf{paramétrique}.
Si $\Theta \subset \Theta_1 \times \Theta_2$ où $\Theta_1$ est inclus dans un espace euclidien, le modèle est dit \textbf{semi-paramétrique}.

\begin{defn}
	Une \textbf{statistique} est une variable aléatoire s'écrivant commme une fonction mesurable des observations, de type $\varphi(X)$ où $\varphi \colon (\mathcal{X}, \mathcal{B}(\mathcal{X})) \to (\R^d, \mathcal{B}(\R^d))$ est mesurable.
\end{defn}

\begin{defn}[Identifiabilité]
	Un modèle statistique $\mathcal{P}$ décrit par un paramètre $\theta \in \Theta$ est dit \textbf{identifiable} si $\theta \mapsto P_\theta$ est injective.
	Plus généralement, une fonction $g$ de $\theta$ est dite identifiable si $\left( P_{\theta_1} = P_{\theta_2} \right) \implies \left( g(\theta_1) = g(\theta_2) \right)$.
\end{defn}

\begin{rem}
	Avec $\Theta = \mathcal{P}$ on sait qu'il existe toujours au moins une paramétrisation identifiable.
\end{rem}

\begin{defn}
	Un modèle statistique est dit \textbf{dominé} s'il existe une mesure positive $\mu$ sur $\mathcal{B}(\mathcal{X})$ telle que pour tout $\theta \in \Theta$, $P_\theta \in \mathcal{P}$ admette une densité de probabilité $p_\theta$ par rapport à $\mu$.
\end{defn}
