\documentclass[9pt,a4paper,twocolumn]{article}

\usepackage[utf8]{inputenc}
\usepackage[T1]{fontenc}
\usepackage[french]{babel}
\usepackage{lmodern}
\usepackage[left=1.6cm,right=1.6cm,top=1.3cm,bottom=1.5cm]{geometry}
\usepackage{graphicx}
\usepackage{multicol}
\usepackage{framed}
\usepackage{fancyhdr}
\usepackage{xcolor}
\definecolor{gray}{rgb}{0.5,0.5,0.5}

\usepackage{listings}
\lstset{
    language=C, 
    breaklines=true,
    basicstyle={\ttfamily},
    keywordstyle={\color{gray}},
    morekeywords={func,sem_t,pid_t}
}

\pagestyle{fancy}

%opening
\title{\vspace{-1.1cm} \textbf{Fiche d'INF104} \vspace{-1.1cm}}
\author{}
\date{}

\fancyhf{}
\renewcommand{\headrulewidth}{0pt}
\fancyfoot[C]{\tiny Che Bedara - BDE Télécom ParisTech}
\fancyfoot[RO]{\thepage}
\fancyfoot[LE]{\thepage}

%\setlength{\columnseprule}{0.3pt} 
\setlength{\columnsep}{0.5cm}
\setlength{\textwidth}{17.5cm}


\begin{document}

\maketitle

    Les machines actuelles stockent les instructions du programme et ses données dans le même espace (Von Neumann).


\section{Fork et processus}

    Processus sous {UNIX}, 3 zones mémoires:
    \begin{itemize}
        \item code exécutable (text segment)
        \item pile de données (stack segment)
        \item données statiques (data segment)
    \end{itemize}

    Lors d'un appel à \texttt{fork}, \texttt{stack}, \texttt{data} et le pointeur sur \texttt{text} sont copiés pour le nouveau processus.

    Attention à la cascade de \texttt{fork} si on ne fait pas attention au processus dans lequel on l'appelle.


\section{Ordonnancement}

    \subsection{Politiques avec file}
    Dans ces politiques, on applique l'algorithme pour l'attribution d'un seul quantum et on le remet dans la file.
    \begin{itemize}
        \item \emph{First Come First Served}: file classique
        \item \emph{Shortest Job First}: file de priorité sur la durée d'exécution 
        \item \emph{} Priorité définie par l'utilisateur (avec vieillissement pour éviter la famine).
        \item \emph{PAPS}: file de priorité élu par priorité puis par ordre d'arrivée.
    \end{itemize}

    \subsection{Politiques de type tourniquet}

    Dans ces politiques, on reprend les précédentes, mais en planifiant à l'avance les quantums pour que tous les processus aient un quantum prévu (attente bornée).
    Elles permettent d'éviter la famine.
    \begin{itemize}
        \item \emph{Round Robin}: découpe en quantum de temps et les distribue de manière uniforme aux processus.
        \item \emph{Round Robin within priority}: découpe en quantum de temps distribués en fonction des priorités.
    \end{itemize}

    \subsection{Implémentations}

    Pour {UNIX}: \emph{Round robin with priority} avec priorité dynamique (dépend du type d'interruption).


\section{Sémaphore}

    \subsection{Principe}
    Objectif: contrôler l'exécution de deux acteurs indépendants.

    C'est un objet muni d'un \emph{compteur} qui indique le nombre de jetons disponible ou en attente et d'une \emph{file d'attente} de processus. 

    Opération (métaphore du panier): 
    \begin{itemize} 
        \item \textbf{P}: prendre un jeton
        \item \textbf{V}: rendre un jeton
    \end{itemize}

    \textbf{P} bloque tant qu'il n'y a pas de jeton. 

    Attention situation d'interblocage et safety (variables partagées, etc)

    \subsubsection{Section critique}

    Sémaphore initialisé à $1$, jeton pris au début de la section critique et relâché à la fin. Un seul accès possible à la fois. 

    \subsubsection{Rendez-vous}

    \texttt{x} et \texttt{y} initialisés à $0$.

    \begin{multicols}{2}
        \texttt{PROCESSUS 1}
        \begin{lstlisting}
P (x)
V (y)
        \end{lstlisting}

        \columnbreak{}
        
        \texttt{PROCESSUS 2}
        \begin{lstlisting}
V (x)
P (y)
        \end{lstlisting}
    \end{multicols}

    \subsubsection{Producteur consommateur}

    Synchronisation entre une source et un client. Exemple d'utilisation: les pipes.
    On trouve généralement une version améliorée de ce principe avec plusieurs cases, en initialisant le sémaphore du producteur à $N$ et du consommateur à $0$.
    Note: c'est un rendez-vous avec une opération au milieu.

    \begin{multicols}{2}
        \texttt{PRODUCTEUR}
        \begin{lstlisting}
P (x) 
ecrire donnees
V (y)
        \end{lstlisting}
        \columnbreak{}
        \texttt{CONSOMMATEUR}
        \hspace{0cm}
        \begin{lstlisting}
P (y) 
read data
V (x) 
        \end{lstlisting}
    \end{multicols}


    \subsubsection{Barrière}

    \texttt{x} initialisé à $1$, \texttt{barriere} à $0$.

    \begin{lstlisting}[frame=shadowbox]
P (x)
if (nb < N-1)
    nb++
else for (i=1 to N) 
    V (barriere)
    nb--
V (x)

P (barriere)
    \end{lstlisting}

    \subsubsection{Lecteur-\'ecrivain}

    \texttt{o,x,y} initialisé à $1$.

    \texttt{LECTEUR}
    \begin{lstlisting}[frame=shadowbox]
P (o)
V (o)

P (x)
    if nb_lecteur = 0
        P (y)
    nb_lecteur++
V (x)

lecture

P (x)
    if nb_lecteur = 1
        V (y)
    nb_lecteur--
V (x)
    \end{lstlisting}


    \texttt{ECRIVAIN}
    \begin{lstlisting}[frame=shadowbox]
P (o)
P (y)

ecriture

V (o)
V (y)
    \end{lstlisting}

    \subsubsection{Dangers à vérifier}

    \begin{itemize}
        \item Famine (attente indéfinie)
        \item Interblocage
        \item Taille des sections critiques
    \end{itemize}

    \subsection{Solutions logicielles}

    Dans la suite, $i=0$ et $j=1$ désignent les deux processus.

    \subsubsection{Algorithme de Dekker}

    L'idée est d'introduire un ordre pour distinguer les deux processus.
    \begin{lstlisting}[frame=shadowbox]
etat[i] = 1

while etat[j] = 1
    if tour = j
        etat[i] = 0
        while tour = j {}
        
        etat[i] = 1;
tour = j
etat[i] = 0
    \end{lstlisting}

    \subsubsection{Algorithme de Peterson}

    On manipule une variable par processus pour l'accès.
    \begin{lstlisting}[frame=shadowbox]
etat[i] = 1
tour = j

// Faire passer j s'il veut y aller
while (etat[j] = 1) 
   && (tour == j) {} 

// section critique

etat[i] = 0
    \end{lstlisting}

    \subsection{Pipe}

    Modèle producteur-consommateur, deux utilisations en console:
    \begin{itemize}
        \item \texttt{programme-A | programme-B}, envoie la sortie de \texttt{programme-A} vers \texttt{programme-B}
        \item \texttt{programme > fichier}, écrit la sortie de programme vers le fichier \texttt{fichier}
    \end{itemize}

    Tout ce qui devait être écrit dans la sortie console ne s'affiche plus et est écrit vers la destination. \textbf{[partiel 2014, Q3, processus et fichier]}

    %TODO: utilisation de pipe P.184


\section{Signaux}

    Pour ignorer des signaux, on utilise \texttt{signal} avec le numéro du signal et l'argument \texttt{SIG\_IGN}. Sinon, \texttt{signal (sig, \&fonction)} avec \texttt{fonction} une fonction de type \texttt{void fonction (int)};

    On utilise \texttt{kill} pour envoyer des signaux:
    \begin{itemize}
        \item \texttt{SIGKILL} (9)
        \item \texttt{SIGALRM} (14)
        \item \texttt{SIGUSR1} (16)
        \item \texttt{SIGUSR2} (17)
    \end{itemize}


\section{Gestion des fichiers}

    \subsection{Généralités}
    Les informations sur les fichiers sont stockées dans un bloc nommée \emph{i-list}.
    Chaque fichier est constitué de plusieurs blocs.
    Sous système {POSIX} ({UNIX}, Linux, \ldots), même les périphériques sont exposés comme des fichiers.

    \subsection{Alignement mémoire}

    Les éléments des structures en mémoire sont alignées sur des multiples de $2$, $4$, $8$, \ldots
    
    \begin{lstlisting}[frame=shadowbox]
        struct noalign {
            char c;     // 1
            double d;   // 8
            int i;      // 4
            char c3[3]; // 3
        }
    \end{lstlisting}

    Ici, le système rajoute $7$ octets de rembourrage entre c et d, et un octet après $c3$, ce qui fait une structure de 24 octets.
    En mettant c en dessous de $c3$, on corrige le problème, et la structure n'en fait plus que $16$ octets.


\section{Partitions et mémoire}
    
    \subsection{Politiques}

    On combine généralement ces politiques avec un ramasse-miette qui défragmente les partitions.

    \begin{itemize}
        \item \emph{First fit}: on utilise la première partition libre de taille suffisante.
        \item \emph{Best fit}: on utilise la partition de taille la plus proche.
        \item \emph{Worst fit}: on utilise la partition de taille la plus grande.
    \end{itemize}

    \textbf{Knuth:} Il y a toujours un tiers de blocs libres par rapport aux blocs occupés avec ces politiques.

    \subsection{Mémoire virtuelle et pagination}

    Les adresses sont constitués de deux parties: l'\emph{adresse physique} de la page et le \emph{déplacement} dans cette page (qui doit aller jusqu'à la taille de la page).
    Pour les processus, il y a également une numérotation des \emph{pages logiques}, que le système associe ensuite à des pages physiques. Cela permet d'isoler les processus entre eux et abstraire la mémoire en virtualisant l'espace d'adressage.

    Méthode d'accès: on transforme l'adresse de logique à physique, puis on récupère la donnée dans la page physique.

    \subsection{Allocation des pages}

    Pour les exercies sur les allocations, on fait un tableau page virtuelle à gauche, temps discret en haut, et page physique dans les cases.

    \subsubsection{Stratégies}
    \begin{itemize}
        \item \emph{Least recently used}: on place les plus anciennement utilisées sur la mémoire lente.
        \item \emph{Least frequently used}: on place les moins utilisées sur la mémoire lente.
        \item \emph{First in first out}: on place les dernières créées sur la mémoire lente.
    \end{itemize}


\section{Mémo fonctions}

    \subsection{Processus}

    \begin{lstlisting}
int fork ()
    \end{lstlisting}

    Cr\'ee un nouveau processus.
    Retourne son pid dans le processus p\`ere, $0$ dans le fils, ou $-1$ en cas d'erreur.

    \begin{lstlisting}
int wait (int pid)
    \end{lstlisting}

    Attend la fin du processus de pid \texttt{pid}.
    Retourne 0 si pas d'erreur, ou le num\'ero d'erreur.

    \begin{lstlisting}
int execv (char* filename, char* argv[])
    \end{lstlisting}

    Existe sous plusieurs formes.
    Remplace le code à exécuter par le processus et la mémoire par celui du fichier, en remplissant le \texttt{argv} du \texttt{main} du programme cible.
    Ne retourne qu'en cas d'erreur \emph{avant} le lancement.

    \subsection{Sémaphores}

    \begin{lstlisting}
sem_t*  sem_open (const char* name, int options)
    \end{lstlisting}
    Déclare une nouvelle sémaphore de nom \texttt{name} (au format ``/nom''). 
    \texttt{options} vaut \texttt{O\_CREAT} ou \texttt{O\_EXCL}, seul le 1er est utile ici.


    \begin{lstlisting}
int sem_wait (sem_t* sem)
    \end{lstlisting}
    \'Equivalent de l'opération \textbf{P}, prend un jeton ou bloque.
    Retourne $0$ si aucune erreur, $-1$ sinon.


    \begin{lstlisting}
int sem_post (sem_t* sem)
    \end{lstlisting}
    \'Equivalent de l'opération \textbf{V}, rend un jeton et débloque un processus.
    Retourne $0$ si aucune erreur, $-1$ sinon.

    \subsection{Signaux}

    \begin{lstlisting}
int kill (int pid, int signal)
    \end{lstlisting}
    Envoie le signal \texttt{signal} au processus \texttt{pid}
    Retourne $0$ si aucune erreur, $-1$ sinon.

    \begin{lstlisting}
func signal (int sig, func f)
    \end{lstlisting}
    Associe \emph{pour un appel} le signal \texttt{sig} à la fonction \texttt{f}.
    La valeur de retour n'a pas d'importance, et il faut bien donner \emph{l'adresse de la fonction}.
    (\texttt{\&mafonction}).

    \begin{lstlisting}
void alarm (int seconds)
    \end{lstlisting}
    Supprime la dernière alarme, et si $\texttt{seconds} \neq 0$ rajoute une alarme qui lancera le signal \texttt{SIGALRM} dans \texttt{seconds} secondes.
    Retourne le nombre de secondes avant le déclenchement de l'alarme précédente.

    \begin{lstlisting}
int sigsetjmp (jmp_buf env, int sigmask)
    \end{lstlisting}
    Enregistre l'état du programme dans le contexte \texttt{env}. Généralement $\texttt{mask} = 0$, mais il faut réappeler \texttt{signal}.
    \texttt{env} doit être accessible, le déclarer en global (à l'extérieur des fonctions) pour l'utiliser avec des signaux.
    Attention valeur retour: retourne $0$ ou $-1$ selon si pas d'erreur, sauf si l'on revient à cette instruction avec un saut.
    Dans le dernier cas, retourne l'entier passé en paramètre de \texttt{siglongjmp}, ce qui permet de savoir si on a sauté et d'où.

    \begin{lstlisting}
int siglongjmp (jmp_buf env, int val)
    \end{lstlisting}
    Saute à l'instruction \texttt{sigsetjmp} sauvegardé dans \texttt{env} en lui faisant retourner \texttt{val}.

    \subsection{Fichiers}

    %TODO
    \begin{lstlisting}
int open (char* filename, int mode)
int fopen (char* filename, char* mode)
    \end{lstlisting}
    \texttt{open} ouvre le fichier \texttt{filename} avec le mode \texttt{mode}. \texttt{mode} peut valoir \texttt{O\_RDONLY}, \texttt{O\_WRONLY}, \texttt{O\_RDWR}, mais on peut utiliser des flags plus complexes comme \texttt{O\_CREAT}, \texttt{O\_EXCL}, \texttt{O\_APPEND}\ldots
    Pour \texttt{fopen}, \texttt{mode} prend ``r'', ``w'', ``a'' et leur version multimode avec un $+$.

    \begin{lstlisting}
int read (int fichier, char* buffer, unsigned size)
    \end{lstlisting}
    Lit \texttt{size} caractères du fichier \emph{ouvert} \texttt{fichier} et les écrit dans le tampon mémoire \texttt{buffer}.
    \texttt{buffer} \emph{doit} être initialisé à la bonne taille. 
    Renvoie une valeur strictement positive si pas d'erreur.

    \begin{lstlisting}
int write (int fichier, char* buffer, unsigned size)
    \end{lstlisting}
    Écrit \texttt{size} caractères du fichier \emph{ouvert} \texttt{fichier} en les récupérant du tampon mémoire \texttt{buffer}.
    Renvoie une valeur strictement positive si pas d'erreur.

    \begin{lstlisting}
int lseek (int fichier, int offset, int mode)
    \end{lstlisting}
    D\'eplace le curseur sur le fichier \texttt{fichier} de \texttt{offset} caractères ou bytes selon le mode.
    \texttt{mode} peut prendre les valeurs \texttt{SEEK\_CUR} (position actuelle), \texttt{SEEK\_SET} (début),
    Retourne la position du pointeur sur le fichier, ou $-1$ en cas d'erreur.

    \begin{lstlisting}
int lockf (int fichier, int mode, int size)
    \end{lstlisting}
    \texttt{mode} peut valoir \texttt{F\_LOCK} ou \texttt{F\_ULOCK}.
    La synchronisation est assurée par le système, mais pas obligatoire.

\begin{lstlisting}
int pipe (int pipefd[2])
\end{lstlisting}

Crée un pipe et remplie le tableau avec deux descripteurs de fichiers. \texttt{pipefd[0]} est réservé à la lecture, et \texttt{pipefd[1]} à l'écriture.
Retourne $0$ ou $-1$ si une erreur apparait.

\end{document}