On restreint ici le mot moral au sens de ce qui distingue le bien et le mal.

\subsection{La morale est-elle issue de relations entre apparentés ?}
    Ce serait les relations entre parents qui, en s'étendant à un cercle plus large, amènerait à la notion de morale.

    \paragraph{Nourrir son enfant : Eibl-Eibesfeldt.}
    Hypothèse : chez les animaux, les soins prodigués par les parents aux enfants constituent la matrice à partir de laquelle se sont développés des comportements affectueux et des dispositions altruistes.
    Au cours de l'évolution, ce répertoire comportemental s’est ensuite enrichi et étendu dans certaines espèces à d’autres individus (parents plus lointains ou non-apparentés).
    
    Eibl-Eibesfeldt s'appuie pour cela sur plusieurs indices et observations.
    Chez les humains on trouve notamment une certaine universalité du baiser, ou bien les réactions d'attendrissement spontanées face à un enfant (commun à d'autres espèces), comportement aussi appelé \textit{Kindchenschema}.
    Les traits du visage enfantin, pour de nombreuses espèces, se constituent surtout d’une tête arrondie et de grands yeux.
    
    \paragraph{Aider sa parentèle : Hamilton.}
    On trouve chez plusieurs espèces un comportement qui est que les individus participent à l'éducation de leurs congénères.
    L'existence de comportements de sacrifices chez les insectes sociaux, qui pousse cet investissement le plus loin, pose un problème pour la théorie de la sélection naturelle.

	Les insectes en question appartiennent à la catégorie des haplodiploïdes.
	
	Mais William Hamilton remarqua en 1964 que, si l'on prend en compte les corrélations génétiques entre les individus concernés, se sacrifier revient à œuvrer dans son propre intérêt génétique.
	La répartition génétique ayant cours chez les haplodiploïdes est telle que deux individus sœurs ont en commun 75\% de leurs gênes.
	L'individu a donc plus intérêt à se sacrifier pour sa sœur que pour ses parents afin que le gène y “gagne”.

	\paragraph{Interdire l'inceste : Westermarck.}
	L'interdiction de l'inceste, qui est présent dans la législation sur toute la planète, est “promu” par la sélection naturelle dans la mesure où l'inceste mène conduit à une plus grande probabilité de développement de certaines maladies.
	
	Pour l'anthropologue Edward Westermack, il doit exister un mécanisme naturel qui décourage l'inceste au cours de la vie.

	On trouve à Taïwan une pratique qui consiste pour une famille à adopter une petite fille qui sera destinée à devenir plus tard la femme de l'un des fils naturels.
	Des chercheurs qui se sont intéressés aux couples ainsi formés en ont conclu que ces mariages conduisent trois fois plus souvent au divorce qu'un mariage “moyen” et que leur fertilité est moindre.
	Ceci est d'autant plus fort que la jeune fille a été adoptée tôt.

	Plusieurs études indiquent effectivement que les enfants élevés ensemble développent une inhibition ou un désintérêt mutuel qui limite les chances de procréation.
	La cause de ce phénomène reste inconnue.

\subsection{La morale est-elle un système d'échanges ?}
	\paragraph{L'altruisme réciproque : Trivers}
	Certaines espèces disposent d'un système de réciprocité entre non-apparentés.

	Exemple : le don du sang chez les vampires, des chauve-souris hématophage.
	Les vampires peuvent mourir d'inanition au bout de quelques jours s'ils ne trouvent pas de proie à mordre.
	Dans le cas où un individu revient bredouille, il est d'usage qu'un autre régurgite une partie du sang qu'il a récolté pour donner au premier.
	Ces comportements augmentent (en moyenne) les chances de survie des individus.
	Les vampires gardent de plus une certaine mémoire des interactions, encourageant une réciprocité des dons, sous peine d'une plus grande chance de refus.
	
	Ceci pourrait expliquer selon Trivers qu’un contrat social élémentaire soit le résultat de l’Évolution.

	\paragraph{Peut-on également considérer les comportements humains de réciprocité comme des résultats de l'Évolution ?}
	Dans le cas d'une population de chasseur-cueilleurs on constate, de même que chez les vampires, qu'un individu a peu à perdre lorsqu'il partage le résultat de sa chasse avec d'autres, et beaucoup à gagner s'il est lui-même affamé et en demande de nourriture.
	
	Exemple : la norme du partage chez les Indiens Guayaki (voir texte).
	Mais ne serait-elle pas en réalité une pratique adaptative ?
	
	Le problème est que la réciprocité est une forme de coopération intéressée qui peut être mise en œuvre en l’absence de toute moralité.
	N'y aurait-il pas alors chez l'être humain des comportements altruistes qui ne soient pas motivés par un système d'échange ?
	On peut songer notamment à un service rendu à une personne mourante, qui a peu de chances d'être rétribué.

	\paragraph{Le force du donnant-donnant : Axelrod.}
	Des simulations informatiques menées sur le dilemme itératif du prisonnier indiquent que la stratégie « donnant-donnant » (\textit{tit-for-tat}: coopération au premier coup puis imitation systématique du comportement de l'autre joueur) bat statistiquement les autres stratégies.

	Robert Axelrod va jusqu'à faire l'hypothèse que les groupes dotés d'un système de réciprocité détiennent un avantage évolutif sur les autres.
	\begin{aquote}{David Sloan \bsc{Wilson}}
	“Selfishness beats altruism within single groups.
	Altruistic groups beat selfish groups.”
	\end{aquote}

	\paragraph{La générosité : Frans de Waal.}
	Chez les chimpanzés on trouve des comportements de partages, même entre des individus non apparentés.
	Il arrive aussi qu'un individu dominant distribue une partie de ses ressources alimentaires entre des membres de son groupe.
	Les bénéficiaires de cette distribution sont généralement des mâles assez âgés ou de niveau moyen, mais très rarement à ceux qui peuvent constituer une menace dans la hiérarchie, jeunes mâles fougueux ou bien autre dominant.
	Frans de Waal en conclue que ces distributions sont intéressées dans un sens politique.

	Faut-il voir dans ces prestations dissymétriques une parenté avec des formes humaines de générosité ou de magnanimité ?
	On peut alors penser que notre système moral a été permis par le fait que l'espèce humaine est issue d'un ancêtre commun avec les chimpanzé qui serait elle-même hiérarchisée.

\subsection{Les sentiments moraux sont-ils des produits de l'Évolution ?}

	Rousseau prétendait que l'homme était plus prédisposé à faire le bien que le mal, mettant en avant la révulsion que nous pouvons ressentir face à des actes mauvais.
	Selon lui une partie de nos sentiments moraux est fondée sur cette révulsion face à la souffrance de nos semblables.

	\paragraph{De l'oralité à la moralité.}
	Le sentiment moral d'indignation dérive-t-il du réflexe instinctif de dégoût ?
	La notion même de dégoût semble universelle et d'autres, comme la souillure, sont ancrés dans la plupart des société traditionnelles.

	D'après les travaux de Jorge Moll, il existe des similitudes entre les deux états, dans la mesure où leurs profils d'activation neurologiques se chevauchent.

	Problème général : la morale humaine est parfois soutenue par des émotions mais elle n'est pas entièrement fondée sur elles.
	En matière d'éthique, c'est même la raison que nous écoutons parfois à l'encontre de nos sentiments.

	\paragraph{De l'empathie à la compassion.}

	Cf texte sur les singes.

	On trouve chez plusieurs espèces de singes des comportements d'aide effectués envers des êtres faibles ou affaiblis (vieillards, aveugles, handicapés...).
	Ces conduites semblent manifester une capacité d'empathie.
	S'agirait-il de prémices de la compassion humaine ?

	Mais attention à l'anthropomorphisme lorsque nous analysons les comportements des grands singes.
