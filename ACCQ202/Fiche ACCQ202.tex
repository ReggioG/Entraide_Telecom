\documentclass[a4paper,11pt]{article}

\usepackage[utf8]{inputenc}
\usepackage[T1]{fontenc}
\usepackage{lmodern}
\usepackage{amsthm} %ou \usepackage{ntheorem}
\usepackage[top=1.5cm, bottom=2.2cm, left=2.4cm, right=2.4cm]{geometry}
\usepackage{fancyhdr}
\usepackage{graphicx}
\usepackage{multicol}
\usepackage{enumerate}
\usepackage{titlesec}
\usepackage{alltt}
\usepackage{kpfonts}
\usepackage[svgnames,table]{xcolor}
\usepackage[francais]{babel}
\pagestyle{fancy}



%
% Suivent les macros qui donnent la mise en forme des théorèmes, définitions, etc...
%
\newtheoremstyle{persoth}% name
{3pt}%Space above
{5pt}%Space below
{}%{\itshape}%Body font
{}%Indent amount
{\bf}%Theorem head font
{.}%Punctuation after theorem head
{.5em}%Space after theorem head 2
{}%

\newtheoremstyle{persodef}% name
{3pt}%Space above
{5pt}%Space below
{}%Body font
{}%Indent amount
{\bf}%Theorem head font
{.}%Punctuation after theorem head
{.5em}%Space after theorem head 2
{}%

\theoremstyle{persoth}% default
\newtheorem*{thm}{\noindent\textcolor{Crimson}{Théorème}}
\newtheorem*{lem}{\noindent\textcolor{MediumVioletRed}{Lemme}}
\newtheorem*{pop}{\noindent\textcolor{FireBrick}{Proposition}}
\newtheorem*{cor}{\noindent\textcolor{Brown}{Corrolaire}}

\theoremstyle{persodef}
\newtheorem*{defn}{\noindent\textcolor{magenta}{Définition}}
\newtheorem*{conj}{Conjecture}

\theoremstyle{remark}
\newtheorem*{rem}{\noindent\textcolor{Teal}{Remarque}}
\newtheorem*{ex}{\noindent\textcolor{DarkOrange}{Exemple}}
\newtheorem*{note}{\noindent\textcolor{RoyalBlue}{Notation}}
\newtheorem*{danger}{\textcolor{green}{Attention}}
\newtheorem*{voc}{\textcolor{DarkGreen}{Voc}}
\newtheorem*{hyp}{\textcolor{OrangeRed}{Hyp}}

%\newcommand\demo{\begin{proof}[\textit{Démonstration}]}


%
% Redéfinition de caractéristiques sur le format des titres de sections etc...
%
\titleformat{\section}
	[hang]% style : hang, display, runin, leftmargin, ...
	{\Large\bfseries\sffamily}% fonte numéro + titre
	{\thesection}% numéro
	{1em}% espace entre le numéro et le titre
	{}% fonte titre

\titlespacing{\section}{0pt}
	{10pt}{4pt}

\titleformat{\subsection}
	[hang]% style : hang, display, runin, leftmargin, ...
	{\large\bfseries\sffamily}% fonte numéro + titre
	{\thesubsection}% numéro
	{4pt}% espace entre le numéro et le titre
	{}% fonte titre

\titlespacing{\subsection}{0pt}
	{8pt}{2pt}


\date{}

%
% Ce qui suit permet de corriger le bug de titlesec sur la numérotation des sections
%
\usepackage{etoolbox}
\makeatletter
\patchcmd{\ttlh@hang}{\parindent\z@}{\parindent\z@\leavevmode}{}{}
\patchcmd{\ttlh@hang}{\noindent}{}{}{}
\makeatother

\usepackage[colorlinks=true,linkcolor=purple]{hyperref}

\usepackage{pgf, tikz}
\usetikzlibrary{arrows,chains,matrix,positioning,scopes,calc,intersections,through,backgrounds}
\makeatletter
\tikzset{join/.code=\tikzset{after node path={%
\ifx\tikzchainprevious\pgfutil@empty\else(\tikzchainprevious)%
edge[every join]#1(\tikzchaincurrent)\fi}}}
\makeatother
%
\tikzset{>=stealth',every on chain/.append style={join},
         every join/.style={->}}
\tikzstyle{labeled}=[execute at begin node=$\scriptstyle,
   execute at end node=$]

\fancyhf{}
\renewcommand{\headrulewidth}{0pt}
\fancyfoot[C]{\tiny Che Bedara - BDE Télécom ParisTech - Régis}
\fancyfoot[RO]{\thepage}
\fancyfoot[LE]{\thepage}

\usepackage{amsthm} %ou \usepackage{ntheorem}
\usepackage{amsmath}
\usepackage{amssymb}
\usepackage{mathrsfs}
\usepackage{amsfonts}

\definecolor{vert}{rgb}{0,0.6,0}




%
% Suivent les macros qui donnent la mise en forme des théorèmes, définitions, etc...
%
\newtheoremstyle{persoth}% name
{2pt}%Space above
{2pt}%Space below
{\itshape}%Body font
{}%Indent amount
{\bf}%Theorem head font
{.}%Punctuation after theorem head
{.5em}%Space after theorem head 2
{}%

\newtheoremstyle{persodef}% name
{2pt}%Space above
{2pt}%Space below
{}%Body font
{}%Indent amount
{\bf}%Theorem head font
{.}%Punctuation after theorem head
{.5em}%Space after theorem head 2
{}%

\theoremstyle{persoth}% default
\newtheorem*{thm}{\noindent\textcolor{Crimson}{Th}}
\newtheorem*{lem}{\noindent\textcolor{MediumVioletRed}{Lem}}
\newtheorem*{pop}{\noindent\textcolor{FireBrick}{Prop}}
\newtheorem*{cor}{\noindent\textcolor{Brown}{Cor}}

\theoremstyle{persodef}
\newtheorem*{defn}{\noindent\textcolor{magenta}{Def}}
\newtheorem*{conj}{Conjecture}

\theoremstyle{remark}
\newtheorem*{rem}{\noindent\textcolor{Teal}{Rem}}
\newtheorem*{ex}{\noindent\textcolor{DarkOrange}{Ex}}
\newtheorem*{note}{\noindent\textcolor{RoyalBlue}{Not}}
\newtheorem*{danger}{\textcolor{green}{Attention}}
\newtheorem*{voc}{\textcolor{DarkGreen}{Voc}}
\newtheorem*{hyp}{\textcolor{OrangeRed}{Hyp}}

%\newcommand\demo{\begin{proof}[\textit{Démonstration}]}





% Une macro pour obtenir de grandes fractions dans les formules en ligne.
\def\frc#1#2{\displaystyle{#1\over#2}}

% Une macro pour les vecteurs qui donne de meilleurs résultats que \overrightarrow.
\def\vect#1{%
	\vbox{\lineskip=-.04em\baselineskip=0pt
	\halign{##\cr
	\leaders\hbox{$\scriptstyle{-}$\kern-.4em}\hfil$\scriptstyle{\rightarrow}$\cr
	$#1$\cr}}}

% Majuscules d'anglaise.
\DeclareSymbolFont{rsfscript}{U}{rsfs}{m}{n}
\DeclareSymbolFontAlphabet{\mathrsfs}{rsfscript}
\newcommand\scr{\mathrsfs}


% Des macros pour les notations usuelles.
\newcommand{\ensemblenombre}[1]{\mathbf{#1}}
\newcommand{\N}{\ensemblenombre{N}}
\newcommand{\Z}{\ensemblenombre{Z}}
\newcommand{\Q}{\ensemblenombre{Q}}
\newcommand{\R}{\ensemblenombre{R}}
\newcommand{\C}{\ensemblenombre{C}}
\newcommand{\K}{\ensemblenombre{K}}
\newcommand{\U}{\ensemblenombre{U}}
\newcommand\M{\mathfrak{M}}
\newcommand\E{\mathcal{E}}
\newcommand\parties{\mathcal{P}}
\newcommand\GL{\mathcal{GL}}
\newcommand\Sym{\mathcal{S}}
\newcommand\aSym{\mathcal{A}}
\newcommand\proba{\mathbf{P}}
\newcommand\esp{\mathbf{E}}
\newcommand\Orth{\mathcal{O}}
\newcommand\cont{\mathcal{C}}
\newcommand\li{[\![}
\newcommand\ri{]\!]}
\newcommand{\diff}{\mathop{}\mathopen{}\mathrm{d}}
\newcommand{\abs}[1]{\left\lvert#1\right\rvert}
\newcommand{\norme}[1]{\left\lVert#1\right\rVert}
\newcommand{\transp}[1]{{\vphantom{#1}}^{\mathit t}{#1}}
\newcommand{\scal}[2]{\left\langle #1 \mid #2 \right\rangle}
\newcommand{\compl}[1]{{#1}^{\mathcal{C}}} % symbole du complémentaire en exposant
\newcommand\indep{\protect\mathpalette{\protect\independenT}{\perp}} % symbole d'indépendance en probas
\def\independenT#1#2{\mathrel{\rlap{$#1#2$}\mkern3mu{#1#2}}}
\newcommand\rar{\rightarrow}
\newcommand\lar{\leftarrow}

% Notation d'ensembles en algèbre
\newcommand\Hom{\mathrm{Hom}}
\newcommand\End{\mathrm{End}} % Endomorphismes
\newcommand\Isom{\mathrm{Isom}} % Isométries
\newcommand\Aut{\mathrm{Aut}} % Automorphismes
\newcommand\Int{\mathrm{Int}}


%
% Intervalles
%
% Premières définitions d'intervalles, taille non ajustable
\newcommand{\intervalle}[4]{\mathopen{#1}#2\mathclose{}\mathpunct{};#3\mathclose{#4}}
\newcommand{\inff}[2]{\intervalle{[}{#1}{#2}{]}}
\newcommand{\inof}[2]{\intervalle{]}{#1}{#2}{]}}
\newcommand{\info}[2]{\intervalle{[}{#1}{#2}{[}}
\newcommand{\inoo}[2]{\intervalle{]}{#1}{#2}{[}}
\newcommand{\iniff}[2]{\intervalle{[\![}{#1}{#2}{]\!]}}

% Secondes définitions d'intervalles, taille ajustable
\newcommand{\genericinterval}[4]
{
	\mathopen{}\mathclose{\left#1#2\mathclose{}\mathpunct{};#3\right#4}
}
\newcommand{\intff}[2]{\genericinterval{[}{#1}{#2}{]}}
\newcommand{\intoo}[2]{\genericinterval{]}{#1}{#2}{[}}
\newcommand{\intof}[2]{\genericinterval{]}{#1}{#2}{]}}
\newcommand{\intfo}[2]{\genericinterval{[}{#1}{#2}{[}}
\newcommand{\intiff}[2]{\genericinterval{[\![}{#1}{#2}{]\!]}}


%
% Opérateurs mathématiques
%
\DeclareMathOperator{\card}{Card}
\DeclareMathOperator{\tr}{Tr}
\DeclareMathOperator{\Ker}{Ker}
\DeclareMathOperator{\Vect}{Vect}
\DeclareMathOperator{\indic}{\mathbf{1}}
\DeclareMathOperator{\argth}{argth}
\DeclareMathOperator{\Id}{Id}
\DeclareMathOperator{\Gram}{Gram}
\DeclareMathOperator{\diag}{diag}
\DeclareMathOperator{\Mat}{Mat}
\DeclareMathOperator{\Sp}{Sp}
\DeclareMathOperator{\im}{Im}
\DeclareMathOperator{\Cov}{Cov}
\DeclareMathOperator{\Var}{Var}
\DeclareMathOperator{\Supp}{Supp}
\DeclareMathOperator{\sinC}{sinC}
\DeclareMathOperator{\sgn}{sgn}
\DeclareMathOperator{\argmin}{arg\, min}


% Pour des symboles « inférieur ou égal », « supérieur ou égal », « ensemble vide » et « parallèles » conformes aux usages français.
\DeclareSymbolFont{AmsA}{U}{msa}{m}{n}
\SetSymbolFont{AmsA}{bold}{U}{msa}{b}{n}
\DeclareMathSymbol\leq\mathrel{AmsA}{"36}
\DeclareMathSymbol\geq\mathrel{AmsA}{"3E}
\DeclareSymbolFont{AmsB}{U}{msb}{m}{n}
\SetSymbolFont{AmsB}{bold}{U}{msb}{b}{n}
\DeclareMathSymbol\emptyset\mathord{AmsB}{"3F}
\def\parallel{\mathrel{/\!/}}

\title{\vspace{-1.2cm} \textbf{ACCQ 202 - Information theory}}


\begin{document}

\maketitle

\vspace{-1.5cm}

\section{Source coding}

	On note $\Omega$ l'univers et $\mathcal{F}$ la tribu des événements.
On considère une variable aléatoire $X$, appelée observation, définie sur $(\Omega, \mathcal{F})$ et à valeur dans l'espace des observations $(\mathcal{X}, \mathcal{B}(\mathcal{X}))$, ou $\mathcal{B}(\mathcal{X})$ est une tribu composée de parties de $\mathcal{X}$.

\begin{defn}
	\textbf{Modèle statistique} : famille de probabilités $\mathcal{P}$ sur $\mathcal{B}(\mathcal{X})$.
	Si $\Theta$ est un ensemble quelconque tel que $\mathcal{P} = \{ P_\theta, \theta \in \Theta \}$ alors $\Theta$ est appelé \textbf{espace des paramètres} du modèle.
\end{defn}

\begin{rem}
	L'existence d'une paramétrisation est toujours acquise, quitte à prendre $\Theta = \mathcal{P}$.
\end{rem}

Si $\Theta$ peut être choisi comme sous-ensemble d'un espace euclidien, le modèle est dit \textbf{paramétrique}.
Si $\Theta \subset \Theta_1 \times \Theta_2$ où $\Theta_1$ est inclus dans un espace euclidien, le modèle est dit \textbf{semi-paramétrique}.

\begin{defn}
	Une \textbf{statistique} est une variable aléatoire s'écrivant commme une fonction mesurable des observations, de type $\varphi(X)$ où $\varphi \colon (\mathcal{X}, \mathcal{B}(\mathcal{X})) \to (\R^d, \mathcal{B}(\R^d))$ est mesurable.
\end{defn}

\begin{defn}[Identifiabilité]
	Un modèle statistique $\mathcal{P}$ décrit par un paramètre $\theta \in \Theta$ est dit \textbf{identifiable} si $\theta \mapsto P_\theta$ est injective.
	Plus généralement, une fonction $g$ de $\theta$ est dite identifiable si $\left( P_{\theta_1} = P_{\theta_2} \right) \implies \left( g(\theta_1) = g(\theta_2) \right)$.
\end{defn}

\begin{rem}
	Avec $\Theta = \mathcal{P}$ on sait qu'il existe toujours au moins une paramétrisation identifiable.
\end{rem}

\begin{defn}
	Un modèle statistique est dit \textbf{dominé} s'il existe une mesure positive $\mu$ sur $\mathcal{B}(\mathcal{X})$ telle que pour tout $\theta \in \Theta$, $P_\theta \in \mathcal{P}$ admette une densité de probabilité $p_\theta$ par rapport à $\mu$.
\end{defn}

\begin{rem}
	Tout modèle défini sur un espace fini ou dénombrable $(\mathcal{X}, \mathcal{P}(\mathcal{X}))$ est dominé par la mesure de comptage sur $\mathcal{X}$, $\mu = \sum_{x \in \mathcal{X}} \delta_x$.
\end{rem}

\begin{defn}
	L'application $\theta \to p(x ; \theta)$ s'appelle la fonction de \textbf{vraisemblance} de l'observation $x$ (avec $p(\cdot; \theta)$, ou $p_\theta(\cdot)$ la densité de la loi $P_\theta$ par rapport à une mesure dominante de référence $\mu$).
\end{defn}

\begin{note}
	Pour parler de $n$ observations on notera une loi produit $P_n = P^{\otimes n}$ lorsque les échantillons sont i.i.d, et $\mathcal{P}_n = \{ P_n, P \in \mathcal{P} \}$ le modèle associé.
\end{note}

\begin{defn}
	Le type de réponse que l'on attend d'une \emph{procédure de décision} (procédure d'estimation ou test statistique) s'appelle une \textbf{action}.
	On notera $\mathcal{A}$ l'espace des actions.
	Une \textbf{règle de décision} est alors définie comme une fonction $\delta \colon \mathcal{X} \to \mathcal{A}$.
\end{defn}

\begin{defn}
	Soit $\delta \colon \mathcal{X} \to \mathcal{A}$ une règle de décision.
	Son \textbf{risque} sous la loi $P_\theta \in \mathcal{P}$ est $R(\theta,\delta) = \esp_\theta \left[ L(\theta, \delta(X)) \right] \in \bar{\R}_+$.
\end{defn}


\section{Entropie et questionnement}

	\begin{pop}[Inégalité de Hölder]
	\begin{itemize}
	\item Si $u \in l^1$ et $v \in l^\infty$, alors $u \cdot v \in l^1$ et $\norme{u \cdot v}_1 \leq \norme{u}_1 \norme{v}_\infty$.
	\item Si $u \in l^2$ et $v \in l^2$ alors $u \cdot v \in l^1$ et $\norme{u \cdot v}_1 \leq \norme{u}_2 \norme{v}_2$ (CS).
	\end{itemize}
\end{pop}

\begin{pop}[Règles de convolution]
	$\begin{array}{|c||c|c|c|}
		\hline
		* & l^1 & l^2 & l^\infty \\ \hline \hline
		l^1 & l^1 & l^2 & l^\infty \\ \hline
		l^2 & l^2 & l^\infty & - \\ \hline
		l^\infty & l^\infty & - & - \\ \hline
	\end{array}$
	
	On a aussi, à chaque fois, $\norme{u \star v}_\gamma \leq \norme{u}_\alpha \norme{v}_\beta$.
\end{pop}

\begin{defn}
	Soit $u \in l^1$, sa transformée de Fourier à temps discret (\textbf{TFtD}) est $\mathcal{F}(u) = \hat{u} \colon \nu \mapsto \sum_{n \in \Z} u_n e^{-2i\pi \nu n}$.
	Elle est continue, que ce soit sur $\intfo{-\frac{1}{2}}{\frac{1}{2}}$ ou sur $\R$.
\end{defn}

\begin{pop}
	Soit $u,v \in l^1$, $\nu_0 \in \intfo{-\frac{1}{2}}{\frac{1}{2}}$, $\varphi$ une onde de Fourier sur $\Z$ de fréquence $\nu_0$, $m \in \Z$ et $\psi \colon x \mapsto e^{-2i\pi mx}$ une onde de Fourier sur $\intfo{-\frac{1}{2}}{\frac{1}{2}}$ de fréquence $-m$.
	\begin{itemize}
		\item La TFtD de l'impulsion en $m$ $(\delta_n^m)_n$ est une onde de Fourier de fréquence $-m$ sur $\intfo{-\frac{1}{2}}{\frac{1}{2}}$.
		\item $\mathcal{F}(u \star v) = \hat{u} \cdot \hat{v}$.
		\item $\mathcal{F}(u \cdot v) = \hat{u} \star \hat{v}$.
		\item $\forall \nu \in \intfo{-\frac{1}{2}}{\frac{1}{2}}, (\mathcal{F}(\varphi \cdot u))(\nu) = \hat{u}(\nu - \nu_0)$.
		\item Soit $u^m$ la $m$-translatée de $u$, $\mathcal{F} \left( u^m \right) = \hat{u} \cdot \varphi$, i.e. $\hat{u^m}(\nu) = \hat{u}(\nu) e^{-2i\pi m \nu}$.
		\item Si $u$ est réelle, alors $\hat{u}$ est à symétrie hermitienne : $\hat{u}(-X) = \overline{\hat{u}(\nu)}$.
		\item Si $u$ est symétrique alors $\hat{u}$ aussi.
		\item Si $u$ est symétrique et réelle alors $\hat{u}$ aussi.
	\end{itemize}
\end{pop}

\begin{pop}
	Soit un SLI $T \colon l^\infty \to l^\infty$ et $h \in l^1$ sa R.I.
	Si $u \in l^1$ et $v = T(u)$ alors : la réponse fréquentielle de $T$ est $\hat{h}$, $h \star u = v \in l^1$ et $\hat{v} = \hat{h} \hat{u}$.
\end{pop}

\begin{thm}
	On peut étendre $\mathcal{F}$ de façon unique à $l^2$ et elle forme une bijection de $l^2$ sur $L^2 \left( \intfo{-\frac{1}{2}}{\frac{1}{2}} \right)$.
	De plus, on a l'égalité de \textbf{Parseval} : $\forall u \in l^2, \norme{\hat{u}}_2 = \norme{u}_2$.
\end{thm}

\begin{thm}[\textbf{Inversion} de la TFtD]
	Si $u \in l^2$ alors on a $\forall n \in \Z, u_n = \int_{-\frac{1}{2}}^{\frac{1}{2}} \hat{u}(\nu) e^{2i\pi n\nu} \diff \nu$.
\end{thm}

\begin{thm}
	Soit $k \in \N$, on a $\left( \sum_{n \in \Z} \abs{n}^k \abs{u_n} < \infty \right) \implies \left( \hat{u} \in \cont^k \left( \intfo{-\frac{1}{2}}{\frac{1}{2}} \right) \right)$ et $\hat{u}^{(k)} = \hat{v^k}$ où $v_n^k = (-2i\pi n)^k u_n$.
\end{thm}

\begin{thm}
	Si $u \colon \Z / N\Z \to \R$.
	On note $\hat{u}$ sa transformée de Fourier discrète (\textbf{TFD}) définie sur $\Z / N\Z$ par $k \mapsto \sum_{n \in \Z / N\Z} u_n e^{-2i\pi \frac{k}{N}n}$.
\end{thm}

\section{Transmission d'information}

	\subsection{Chaîne de transmission}

	\begin{defn}
		Un \textbf{canal de transmission est la donnée} est la donnée des probabilités $\{ Q(y \mid x), x \in \mathcal{X}, y \in \mathcal{Y} \}$.
	\end{defn}

	\begin{defn}
		La capacité d'information d'un canal est $C = \max_{p_X} I(X,Y)$ (maximisation sur les distributions de $X$), où $(X,Y) \sim p_x \cdot Q_{Y \mid X}$.
	\end{defn}

	\begin{pop}
		On a $0 \leq C \leq \min \{ \log (\abs{\mathcal{X}}), \log (\abs{\mathcal{Y}}) \}$.
	\end{pop}

	On a alors la \textbf{chaîne de transmission} suivante :
	\vspace{0.5em}

	\begin{tikzpicture}
		\tikzstyle{block} = [rectangle, draw=blue, thick, fill=blue!20, text width=5.3em, text centered, rounded corners, minimum height=2em]
		\node (W) at (0,0) {$W$};
		\node [block] (f) at (2.5,0) {Codeur $f$};
		\node (X) at (5,0) {$X^n(W)$};
		\node [block] (Q) at (7.5,0) {$Q(y \mid x)$};
		\node (Y) at (10,0) {$Y^n$};
		\node [block] (g) at (12.5,0) {Décodeur $g$};
		\node (WT) at (15,0) {$\Hat{W}$};
		\draw (W) to (f);
		\draw [->] (f) to (X);
		\draw (X) to (Q);
		\draw [->] (Q) to (Y);
		\draw (Y) to (g);
		\draw [->] (g) to (WT);
	\end{tikzpicture}

	où $W$ et $\Tilde{W}$ sont des fonctions aléatoires à valeurs dans $\mathcal{M}$, l'ensemble des mots.
	$W$ est supposée suivre une loi uniforme.
	Pour un canal sans mémoire : $\proba(y^n \mid x^n(w)) = \prod_{i = 1}^n Q(y_i \mid x_i(w))$.
	Une \textbf{stratégie de transmission} désigne l'ensemble du codeur et du décodeur.

	Nos données sont $(M,n)$, avec $M = \abs{\mathcal{M}}$.

	\begin{defn}
		On note $R = \frac{\log_2 M}{n}$, en bits d'informations, le \textbf{taux de performance}.
	\end{defn}

	On veut maximiser $R$ tout en minimisant $P_e = \proba(\Tilde{W} \neq W) = \frac{1}{M} \sum_{w \in \mathcal{M}} \proba(g(Y^n) \neq w \mid X^n(w))$.

	\begin{defn}
		Un taux $R$ est dit \textbf{atteignable} s'il existe une suite de stratégies de codage $((M = 2^{nR},n))_{n \geq 1}$ telle que $P_e^{(n)} \overset{n \to +\infty}{\to} 0$.
	\end{defn}

	\begin{thm}
		Soit un canal $Q(y \mid x)$.
		On a :
		\begin{itemize}
			\item[\textbullet] $\forall R, R < C$, $R$ est atteignable,
			\item[\textbullet] $\forall R, R > C$, $R$ n'est pas atteignable.
		\end{itemize}
	\end{thm}

	\begin{note}
		Si $X,Y,Z$ forment une chaîne de Markov, on note $X - Y - Z$.
	\end{note}

	\begin{lem}
		Si $X - Y - Z$, $I(X;Y) \geq I(X;Z)$.
	\end{lem}

	\begin{lem}[\textbf{Inégalité de Fano}]
		Soit une chaîne $X - Y - \Hat{X}$.
		On a $1 + \proba(\Hat{X} \neq X) \cdot \log (\abs{\mathcal{X}}) \geq H(X \mid Y)$.
	\end{lem}

	\begin{lem}
		Soit $X^n$ et $Y^n$ l'entrée et la sortie d'un canal donné.
		Alors $I(X^n;Y^n) \leq nC$ avec $C$ la capacité du canal.
	\end{lem}


\subsection{Canal gaussien}

	Pour un canal gaussien on a $Y = X + Z$ avec $Z \sim \normale(0,\sigma^2)$ indépendant de $X$.
	
	Contrainte de puissance : le code $\{ x^n(m) \}_{m \in \mathcal{M}}$ doit satisfaire $\forall m \in \mathcal{M}, \frac{1}{n} \sum_{i = 1}^n x_i^n(w) \leq P$.
	
	\begin{thm}
		La capacité du canal gaussien $(P,\sigma^2)$ est $C = \max_{X, \esp(X^2) \leq P} I(X;Y) = \frac{1}{2} \log \left( 1 + \frac{P}{\sigma^2} \right)$ où $\frac{P}{\sigma^2}$ est le rapport signal sur bruit (SNR).
	\end{thm}


\subsection{Codage conjoint source - canal}

	Soit $S^n = S_1, S_2, \ldots, S_m$ une source i.i.d. d'alphabet $\mathcal{S}$.
	Puisque $\abs{A_\varepsilon^n} \doteq 2^{m H(S)}$ on a besoin de $m \cdot H(S)$ bits pour coder les séquences typiques, et l'on peut se restreindre à elles (compression sans erreur).
	Si l’on veut envoyer ensuite cette information à travers un canal de capacité $C$ il suffit que la longueur $n$ des mots code du codage de canal satisfasse $\frac{\log M}{n} = R \cdot H(S) < C$, où $R = \frac{m}{n}$.

	Lorsque l’on veut envoyer $S^n$ à travers un canal, on peut sans perte d’optimalié décomposer la procédure en 2 parties : comprimer $S^n$ (codage source) pour ensuite la protéger du bruit du canal en lui rajoutant de la redondance (codage de canal).


\subsection{Compression avec distorsion}

	\begin{defn}
		Une \textbf{fonction de distorsion} est de la forme
		$d \colon \begin{array}{ccc}
			\mathcal{X} \times \mathcal{X} & \to & \R_+ \\
			(x,\hat{x}) & \mapsto & d(x,\hat{x})
			\end{array}$.
	\end{defn}
	
	\begin{ex}
		\begin{itemize}
			\item[\textbullet] Distorsion de Hamming : $d(x,\hat{x})$ vaut $0$ si $\hat{x} = x$ et $1$ sinon.
			\item[\textbullet] Si $\mathcal{X} = \R$, $d(x,\hat{x}) = (x - \hat{x})^2$.
		\end{itemize}
	\end{ex}

	\begin{defn}
		La \textbf{distorsion} entre $x^n$ et $\hat{x}^n$ est donnée par $d(x^n,\hat{x}^n) = \frac{1}{n} \sum_{i = 1}^n d(x_i, \hat{x}_i)$.
	\end{defn}

	\begin{note}
		Fonction de codage : $f_n \colon \mathcal{X}^n \to \iniff{1}{2^{nR}}$.
		Fonction de décodage : $g_n \colon \iniff{1}{2^{nR}} \to \mathcal{X}^n$.
	\end{note}

	La distorsion associée à $(2^{nR},n)$ est $D = \esp \left[ d \left( X^n, g_n \circ f_n (X^n) \right) \right] = \sum_{x^n} p(x^n) d(x^n, g_n \circ f_n (x^n) )$.
	
	\begin{defn}
		$(R,D)$ est dit atteignable s'il existe $((2^{nR},n))_{n \geq 1}$ telle que $\limsup_{n \to \infty} \esp \left[ d \left( X^n, g_n \circ f_n (X^n) \right) \right] \leq D$.
		On note $R(D)$ le \textbf{taux atteignable minimal} avec distorsion $D$.
	\end{defn}

	\begin{thm}
		Pour $X$ donné, $R(D) = \min_{p_{\hat{x} \mid x}} I(X;\Hat{X})$.
	\end{thm}
	
	\begin{thm}
		Pour une source binaire i.i.d. de loi $\mathcal{B}(p)$ avec $p \leq \frac{1}{2}$ et une distorsion de Hamming, il vient\\
		$R(D) = \left\{ \begin{array}{ll}
			H_b(p) - H_b(D) & \text{si}\ 0 \leq D \leq p \\
			0 & \text{sinon}
			\end{array} \right. $.
	\end{thm}


\section{Codes polaires}

	\begin{thm}[Formule de Poisson ou le repliement spectral]
	Si $f$ est une fonction définie sur $\R$, intégrable et telle que sa transformée de Fourrier est aussi intégrable et que la suite $(f(n))_n$ est sommable, alors :
	\vspace{-0.1cm}$$
	\sum_{m\in Z} f(m)e^{-2i\pi m\nu } = \sum_{n\in Z} \hat{f}(n+\nu )
	\qquad \text{et} \qquad
	\hat{u}(\nu ) = \sum_{n\in Z} \hat{f}(n+\nu )\ .
	\vspace{-0.1cm}$$
\end{thm}

\begin{thm}[Théorème de bon échantillonnage ou théorème de Shannon]
	Si $f$ est une fonction sommable et que sa TFtC, $\hat{f}$, est nulle en dehors de $\intff{-\frac{1}{2}}{\frac{1}{2}}$, alors on a
	\vspace{-0.15cm}$$
	f(t)=\sum_{n\in \Z} f(n) \sinC (\pi (t-n))\ .
	\vspace{-0.15cm}$$
\end{thm}

\begin{thm}[Théorème de Shannon pour les énergies finies]
	Soit $f$ d'énergie finie et $u_{n} = f(n)$, alors $\norme{f}_{2} = \norme{u}_{2}$.
\end{thm}

\begin{defn} La formule
	$$\forall t \in \R, f(t) = \sum_{n\in \Z} f(n) \sinC (\pi (t-n))$$
	signifie que, si le spectre de $f$ est à support dans $\intff{-\frac{1}{2}}{\frac{1}{2}}$, alors on peut reconstruire la fonction $f$ à partir de ses échantillons.
	On parle \textbf{CNA parfait} ou \textbf{CNA idéal}.
\end{defn}


\end{document}
