On remarque que $\mathcal{L}(\mathcal{C})$ s'identifie au nombre moyen de questions à poser pour identifier une valeur $X \in \mathcal{X}$.

\begin{defn}
	[...]
\end{defn}

\begin{thm}
	On a $0 \leq H(X) \leq \log( \abs{\mathcal{X}} )$.
\end{thm}

\begin{defn}
	Soit $(X,Y) \sim p(x,y)$.
	On a $H(X,Y) = - \sum_{x,y} p(x,y) \log(p(x,y)) = - \esp_{p(x,y)} \left( \log (p(X,Y)) \right)$.
	Et pour des v.a. $X_1, \ldots, X_n$ il vient $H(X_1,\ldots,X_n) = - \esp_{p(x_1,\ldots,x_n)} \left( \log( p(X_1,\ldots,X_n)) \right)$.
\end{defn}

\begin{defn}[Entropie conditionnelle]
	$H(Y \mid X)
		= \sum_x p(x) H(Y \mid X = x)
		= - \sum_{x,y} p(x,y) \log(p(y \mid x))
		= - \esp [ \log(p(Y \mid X)) ]$
\end{defn}

\begin{thm}[\emph{Chain rule}]
	$H(X_1,\ldots,X_n) = \sum_{i = 1}^n H(X_i \mid X^{i - 1})$ où $X^i \triangleq X_1,\ldots,X_i$.
\end{thm}

\begin{pop}
	L'information mutuelle $I(X;Y) = \sum_{x,y} p(x,y) \log \left( \frac{p(x,y)}{p(x)p(y)} \right)$ vérifie
	\begin{itemize}
		\item[\textbullet] $I(X;Y) = H(X) + H(Y) - H(X,Y)$
		\item[\textbullet] $I(X;Y) = H(X) - H(X \mid Y) = H(Y) - H(Y \mid X) = I(Y;X)$
		\item[\textbullet] $I(X;X) = H(X)$
		\item[\textbullet] $I(X;Y) = D_{KL}( p_{X,Y} \| p_X \cdot p_Y )$
		\item[\textbullet] $I(X;Y) = 0 \iff X \indep Y$
		\item[\textbullet] $H(Y \mid X) \leq H(Y)$
		\item[\textbullet] $H(X^n) \leq \sum_{i = 1}^n H(X_i)$
		\item[\textbullet] $H(X)$ est concave en $p_X$
		\item[\textbullet] $H(f(X)) \leq H(x)$ pour toute fonction $f$ déterministe.
	\end{itemize}
\end{pop}

\begin{defn}
	On définit $H(X ; Y \mid Z) = \sum_{x,y,z} p(x,y,z) \log \left( \frac{p(x,y \mid z)}{p(x \mid z) p(y \mid z)} \right) = H(X \mid Z) - H(X \mid Y, Z)$.
\end{defn}

\begin{thm}
	$I(X_1,\ldots,X_n ; Y) = \sum_{i = 1}^n I(X_i ; Y \mid X^{i - 1})$.
\end{thm}

