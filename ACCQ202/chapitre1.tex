Soit $X$ une variable aléatoire discrète d'alphabet $\mathcal{X}$ et de fonction de probabilité $p$ telle que $\forall x \in \mathcal{X}, p(x) = \proba(X = x)$.
On note $p(x)$ plutôt que $p_X(x)$ par commodité, mais par $p(x)$ et $p(y)$ on fait référence à deux fonctions de probabilité distinctes.

\begin{defn}
	$X$ est une \textbf{source d'information} si $\abs{\mathcal{X}} < \infty$ et on note $\forall i \in \iniff{1}{\abs{\mathcal{X}}}, p_i = p(x_i) = \proba(X = x_i)$.
\end{defn}

\begin{defn}
	\textbf{Code} pour une source $X$ : $\mathcal{C} \colon \mathcal{X} \to \{ 0,1 \}^*$.
\end{defn}

\begin{defn}
	\textbf{Longueur moyenne} d'un code $\mathcal{C}$ : $\mathcal{L}(\mathcal{C}) = \sum_i p_i l_i$ avec $l_i$ la longueur du i\up{e} mot codé.
\end{defn}

\begin{defn}
	Un code est \textbf{non singulier} si $\forall x_i \neq x_j, \mathcal{C}(x_i) \neq \mathcal{C}(x_j)$.
\end{defn}

\begin{defn}
	L'extension d'un code $\mathcal{C}$ est $\forall n, \forall x_1,, \ldots, x_n, \mathcal{C}(x_1, \ldots, x_n) \triangleq \mathcal{C}(x_1) * \mathcal{C}(x_2) \cdots * \mathcal{C}(x_n)$.
\end{defn}

\begin{defn}
	Un code est \textbf{à décodage unique} si son extension est non singulière.
\end{defn}

\begin{defn}
	Un code est dit \textbf{instantané} si aucun mot code n'est le préfixe d'un autre.
	On dit alors qu'il s'auto-ponctue car on peut décoder en temps réel, symbole par symbole.
\end{defn}

\begin{thm}[\textbf{Inégalité de Kraft}]
	Soit $\mathcal{C}$ un code instantané avec longueurs $(l_i)$.
	Alors $\sum_i l_i \leq 1$.
	Inversement, soit $(l_i)$ une famille de longueurs.
	Si elle satisfait l'inégalité de Kraft alors il existe un code à décodage unique avec ces longueurs.
\end{thm}

\begin{thm}[de \textbf{McMillan}]
	Le théorème précédent reste valable si l'on remplace décodage instantané par décodage unique.
\end{thm}

\begin{cor}
	$\min_{\mathcal{C} \text{ à décodage unique}} \mathcal{L}(\mathcal{C}) = \min_{\mathcal{C} \text{ à décodage instantané}} \mathcal{L}(\mathcal{C})$.
\end{cor}

\begin{thm}[\textbf{Borne entropique}]
	Pour tout $\mathcal{C}$ à décodage unique, $\mathcal{L}(\mathcal{C}) \geq H(X)$, où $H(X) = - \sum_i p_i \log_2(p_i)$ est l'entropie de la source, avec égalité si et seulement si $\forall i, p_i = 2^{-l_i}$.
\end{thm}

\begin{thm}[\textbf{Inégalité de Jensen}]
	Si $f$ est convexe, alors $\esp(f(X)) \geq f(\esp(X))$.
	Si la convexité est stricte alors $\left( \esp(f(X)) \geq f(\esp(X)) \right) \iff \left( f \text{ est constante}\right)$.
\end{thm}

\begin{defn}
	La \textbf{divergence de Kullback-Leibler}, ou entropie relative, de deux probabilités $P$ et $Q$ est définie par $D_{KL}(P \| Q) = \sum_i p_i \log \left( \frac{p_i}{q_i} \right)$.
\end{defn}

C'est une mesure de dissimilarité entre les deux distributions de probabilités.

\begin{cor}
	On a $D_{KL}(P \| Q) \geq 0$ avec égalité si et seulement si $\forall i, p_i = q_i$.
\end{cor}

\subsection{Code de Shannon}

	On construit un code de Shannon en définissant les longueurs selon $l_i = \left\lceil \log \left( \frac{1}{p} \right) \right\rceil$, qui satisfait l'inégalité de Kraft et peut donc être utilisé pour produire un code instantané.
	
	\begin{pop}
		Soit $\mathcal{C}$ un code de Shannon pour $X$. Alors $H(x) \leq \mathcal{L}(\mathcal{C}) \leq  H(X) + 1$.
	\end{pop}

\subsection{Codage de Huffman}

	Pour construire un codage de Huffman on ordonne l'ensemble des $p_i$, puis l'on construit itérativement de nouveaux ensembles de probabilités en sommant à chaque étape les deux plus faibles.
	On repart ensuite à l'inverse : à partir de la dernière probabilité, égale à $1$, on va re-diviser les probabilités de sorte à construire un arbre dont les feuilles correspondront aux $p_i$.
	La profondeur de chaque feuille $i$ s'identifie alors à $l_i$.

	\begin{thm}
		Un code de Huffman minimise $\mathcal{L}(\mathcal{C})$.
	\end{thm}
