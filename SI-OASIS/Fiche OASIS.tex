\documentclass[a4paper,9pt]{article}

\usepackage[utf8]{inputenc}
\usepackage[T1]{fontenc}
\usepackage{lmodern}
\usepackage{amsthm}
\usepackage{amsmath}
\usepackage{amssymb}
\usepackage{mathrsfs}
\usepackage{amsfonts}
\usepackage[top=1.5cm, bottom=2.2cm, left=2cm, right=2cm]{geometry}
\usepackage{fancyhdr}
\usepackage{graphicx}
\usepackage{multicol}
\usepackage{enumerate}
\usepackage{alltt}
\usepackage[svgnames]{xcolor}
\usepackage[francais]{babel}
\pagestyle{fancy}

\fancyhf{}
\renewcommand{\headrulewidth}{0pt}
\fancyfoot[C]{\tiny Che Bedara - BDE Télécom ParisTech}
\fancyfoot[RO]{\thepage}
\fancyfoot[LE]{\thepage}

\usepackage{amsthm} %ou \usepackage{ntheorem}
\usepackage{amsmath}
\usepackage{amssymb}
\usepackage{mathrsfs}
\usepackage{amsfonts}

\definecolor{vert}{rgb}{0,0.6,0}




%
% Suivent les macros qui donnent la mise en forme des théorèmes, définitions, etc...
%
\newtheoremstyle{persoth}% name
{2pt}%Space above
{2pt}%Space below
{\itshape}%Body font
{}%Indent amount
{\bf}%Theorem head font
{.}%Punctuation after theorem head
{.5em}%Space after theorem head 2
{}%

\newtheoremstyle{persodef}% name
{2pt}%Space above
{2pt}%Space below
{}%Body font
{}%Indent amount
{\bf}%Theorem head font
{.}%Punctuation after theorem head
{.5em}%Space after theorem head 2
{}%

\theoremstyle{persoth}% default
\newtheorem*{thm}{\noindent\textcolor{Crimson}{Th}}
\newtheorem*{lem}{\noindent\textcolor{MediumVioletRed}{Lem}}
\newtheorem*{pop}{\noindent\textcolor{FireBrick}{Prop}}
\newtheorem*{cor}{\noindent\textcolor{Brown}{Cor}}

\theoremstyle{persodef}
\newtheorem*{defn}{\noindent\textcolor{magenta}{Def}}
\newtheorem*{conj}{Conjecture}

\theoremstyle{remark}
\newtheorem*{rem}{\noindent\textcolor{Teal}{Rem}}
\newtheorem*{ex}{\noindent\textcolor{DarkOrange}{Ex}}
\newtheorem*{note}{\noindent\textcolor{RoyalBlue}{Not}}
\newtheorem*{danger}{\textcolor{green}{Attention}}
\newtheorem*{voc}{\textcolor{DarkGreen}{Voc}}
\newtheorem*{hyp}{\textcolor{OrangeRed}{Hyp}}

%\newcommand\demo{\begin{proof}[\textit{Démonstration}]}





% Une macro pour obtenir de grandes fractions dans les formules en ligne.
\def\frc#1#2{\displaystyle{#1\over#2}}

% Une macro pour les vecteurs qui donne de meilleurs résultats que \overrightarrow.
\def\vect#1{%
	\vbox{\lineskip=-.04em\baselineskip=0pt
	\halign{##\cr
	\leaders\hbox{$\scriptstyle{-}$\kern-.4em}\hfil$\scriptstyle{\rightarrow}$\cr
	$#1$\cr}}}

% Majuscules d'anglaise.
\DeclareSymbolFont{rsfscript}{U}{rsfs}{m}{n}
\DeclareSymbolFontAlphabet{\mathrsfs}{rsfscript}
\newcommand\scr{\mathrsfs}


% Des macros pour les notations usuelles.
\newcommand{\ensemblenombre}[1]{\mathbf{#1}}
\newcommand{\N}{\ensemblenombre{N}}
\newcommand{\Z}{\ensemblenombre{Z}}
\newcommand{\Q}{\ensemblenombre{Q}}
\newcommand{\R}{\ensemblenombre{R}}
\newcommand{\C}{\ensemblenombre{C}}
\newcommand{\K}{\ensemblenombre{K}}
\newcommand{\U}{\ensemblenombre{U}}
\newcommand\M{\mathfrak{M}}
\newcommand\E{\mathcal{E}}
\newcommand\parties{\mathcal{P}}
\newcommand\GL{\mathcal{GL}}
\newcommand\Sym{\mathcal{S}}
\newcommand\aSym{\mathcal{A}}
\newcommand\proba{\mathbf{P}}
\newcommand\esp{\mathbf{E}}
\newcommand\Orth{\mathcal{O}}
\newcommand\cont{\mathcal{C}}
\newcommand\li{[\![}
\newcommand\ri{]\!]}
\newcommand{\diff}{\mathop{}\mathopen{}\mathrm{d}}
\newcommand{\abs}[1]{\left\lvert#1\right\rvert}
\newcommand{\norme}[1]{\left\lVert#1\right\rVert}
\newcommand{\transp}[1]{{\vphantom{#1}}^{\mathit t}{#1}}
\newcommand{\scal}[2]{\left\langle #1 \mid #2 \right\rangle}
\newcommand{\compl}[1]{{#1}^{\mathcal{C}}} % symbole du complémentaire en exposant
\newcommand\indep{\protect\mathpalette{\protect\independenT}{\perp}} % symbole d'indépendance en probas
\def\independenT#1#2{\mathrel{\rlap{$#1#2$}\mkern3mu{#1#2}}}
\newcommand\rar{\rightarrow}
\newcommand\lar{\leftarrow}

% Notation d'ensembles en algèbre
\newcommand\Hom{\mathrm{Hom}}
\newcommand\End{\mathrm{End}} % Endomorphismes
\newcommand\Isom{\mathrm{Isom}} % Isométries
\newcommand\Aut{\mathrm{Aut}} % Automorphismes
\newcommand\Int{\mathrm{Int}}


%
% Intervalles
%
% Premières définitions d'intervalles, taille non ajustable
\newcommand{\intervalle}[4]{\mathopen{#1}#2\mathclose{}\mathpunct{};#3\mathclose{#4}}
\newcommand{\inff}[2]{\intervalle{[}{#1}{#2}{]}}
\newcommand{\inof}[2]{\intervalle{]}{#1}{#2}{]}}
\newcommand{\info}[2]{\intervalle{[}{#1}{#2}{[}}
\newcommand{\inoo}[2]{\intervalle{]}{#1}{#2}{[}}
\newcommand{\iniff}[2]{\intervalle{[\![}{#1}{#2}{]\!]}}

% Secondes définitions d'intervalles, taille ajustable
\newcommand{\genericinterval}[4]
{
	\mathopen{}\mathclose{\left#1#2\mathclose{}\mathpunct{};#3\right#4}
}
\newcommand{\intff}[2]{\genericinterval{[}{#1}{#2}{]}}
\newcommand{\intoo}[2]{\genericinterval{]}{#1}{#2}{[}}
\newcommand{\intof}[2]{\genericinterval{]}{#1}{#2}{]}}
\newcommand{\intfo}[2]{\genericinterval{[}{#1}{#2}{[}}
\newcommand{\intiff}[2]{\genericinterval{[\![}{#1}{#2}{]\!]}}


%
% Opérateurs mathématiques
%
\DeclareMathOperator{\card}{Card}
\DeclareMathOperator{\tr}{Tr}
\DeclareMathOperator{\Ker}{Ker}
\DeclareMathOperator{\Vect}{Vect}
\DeclareMathOperator{\indic}{\mathbf{1}}
\DeclareMathOperator{\argth}{argth}
\DeclareMathOperator{\Id}{Id}
\DeclareMathOperator{\Gram}{Gram}
\DeclareMathOperator{\diag}{diag}
\DeclareMathOperator{\Mat}{Mat}
\DeclareMathOperator{\Sp}{Sp}
\DeclareMathOperator{\im}{Im}
\DeclareMathOperator{\Cov}{Cov}
\DeclareMathOperator{\Var}{Var}
\DeclareMathOperator{\Supp}{Supp}
\DeclareMathOperator{\sinC}{sinC}
\DeclareMathOperator{\sgn}{sgn}
\DeclareMathOperator{\argmin}{arg\, min}


% Pour des symboles « inférieur ou égal », « supérieur ou égal », « ensemble vide » et « parallèles » conformes aux usages français.
\DeclareSymbolFont{AmsA}{U}{msa}{m}{n}
\SetSymbolFont{AmsA}{bold}{U}{msa}{b}{n}
\DeclareMathSymbol\leq\mathrel{AmsA}{"36}
\DeclareMathSymbol\geq\mathrel{AmsA}{"3E}
\DeclareSymbolFont{AmsB}{U}{msb}{m}{n}
\SetSymbolFont{AmsB}{bold}{U}{msb}{b}{n}
\DeclareMathSymbol\emptyset\mathord{AmsB}{"3F}
\def\parallel{\mathrel{/\!/}}

\newtheoremstyle{persoth}% name
{2pt}%Space above
{2pt}%Space below
{\itshape}%Body font
{}%Indent amount
{\bf}%Theorem head font
{.}%Punctuation after theorem head
{.5em}%Space after theorem head 2
{}%

\newtheoremstyle{persodef}% name
{2pt}%Space above
{2pt}%Space below
{}%Body font
{}%Indent amount
{\bf}%Theorem head font
{.}%Punctuation after theorem head
{.5em}%Space after theorem head 2
{}%

\theoremstyle{persoth}% default
\newtheorem*{thm}{\noindent\textcolor{Crimson}{Th}}
\newtheorem*{lem}{\noindent\textcolor{MediumVioletRed}{Lem}}
\newtheorem*{pop}{\noindent\textcolor{FireBrick}{Prop}}
\newtheorem*{cor}{\noindent\textcolor{Brown}{Cor}}

\theoremstyle{persodef}
\newtheorem*{defn}{\noindent\textcolor{magenta}{Def}}
\newtheorem*{conj}{Conjecture}

\theoremstyle{remark}
\newtheorem*{rem}{\noindent\textcolor{Teal}{Rem}}
\newtheorem*{ex}{\noindent\textcolor{DarkOrange}{Ex}}
\newtheorem*{note}{\noindent\textcolor{RoyalBlue}{Not}}
\newtheorem*{danger}{\textcolor{green}{Attention}}
\newtheorem*{voc}{\normalfont{Vocabulaire}}

\renewcommand*\rmdefault{ppl}


\title{\vspace{-1.2cm} OASIS}
\date{}

\begin{document}

\maketitle

\vspace{-1.5cm}

\section{Les SLI}

	On note $\Omega$ l'univers et $\mathcal{F}$ la tribu des événements.
On considère une variable aléatoire $X$, appelée observation, définie sur $(\Omega, \mathcal{F})$ et à valeur dans l'espace des observations $(\mathcal{X}, \mathcal{B}(\mathcal{X}))$, ou $\mathcal{B}(\mathcal{X})$ est une tribu composée de parties de $\mathcal{X}$.

\begin{defn}
	\textbf{Modèle statistique} : famille de probabilités $\mathcal{P}$ sur $\mathcal{B}(\mathcal{X})$.
	Si $\Theta$ est un ensemble quelconque tel que $\mathcal{P} = \{ P_\theta, \theta \in \Theta \}$ alors $\Theta$ est appelé \textbf{espace des paramètres} du modèle.
\end{defn}

\begin{rem}
	L'existence d'une paramétrisation est toujours acquise, quitte à prendre $\Theta = \mathcal{P}$.
\end{rem}

Si $\Theta$ peut être choisi comme sous-ensemble d'un espace euclidien, le modèle est dit \textbf{paramétrique}.
Si $\Theta \subset \Theta_1 \times \Theta_2$ où $\Theta_1$ est inclus dans un espace euclidien, le modèle est dit \textbf{semi-paramétrique}.

\begin{defn}
	Une \textbf{statistique} est une variable aléatoire s'écrivant commme une fonction mesurable des observations, de type $\varphi(X)$ où $\varphi \colon (\mathcal{X}, \mathcal{B}(\mathcal{X})) \to (\R^d, \mathcal{B}(\R^d))$ est mesurable.
\end{defn}

\begin{defn}[Identifiabilité]
	Un modèle statistique $\mathcal{P}$ décrit par un paramètre $\theta \in \Theta$ est dit \textbf{identifiable} si $\theta \mapsto P_\theta$ est injective.
	Plus généralement, une fonction $g$ de $\theta$ est dite identifiable si $\left( P_{\theta_1} = P_{\theta_2} \right) \implies \left( g(\theta_1) = g(\theta_2) \right)$.
\end{defn}

\begin{rem}
	Avec $\Theta = \mathcal{P}$ on sait qu'il existe toujours au moins une paramétrisation identifiable.
\end{rem}

\begin{defn}
	Un modèle statistique est dit \textbf{dominé} s'il existe une mesure positive $\mu$ sur $\mathcal{B}(\mathcal{X})$ telle que pour tout $\theta \in \Theta$, $P_\theta \in \mathcal{P}$ admette une densité de probabilité $p_\theta$ par rapport à $\mu$.
\end{defn}

\begin{rem}
	Tout modèle défini sur un espace fini ou dénombrable $(\mathcal{X}, \mathcal{P}(\mathcal{X}))$ est dominé par la mesure de comptage sur $\mathcal{X}$, $\mu = \sum_{x \in \mathcal{X}} \delta_x$.
\end{rem}

\begin{defn}
	L'application $\theta \to p(x ; \theta)$ s'appelle la fonction de \textbf{vraisemblance} de l'observation $x$ (avec $p(\cdot; \theta)$, ou $p_\theta(\cdot)$ la densité de la loi $P_\theta$ par rapport à une mesure dominante de référence $\mu$).
\end{defn}

\begin{note}
	Pour parler de $n$ observations on notera une loi produit $P_n = P^{\otimes n}$ lorsque les échantillons sont i.i.d, et $\mathcal{P}_n = \{ P_n, P \in \mathcal{P} \}$ le modèle associé.
\end{note}

\begin{defn}
	Le type de réponse que l'on attend d'une \emph{procédure de décision} (procédure d'estimation ou test statistique) s'appelle une \textbf{action}.
	On notera $\mathcal{A}$ l'espace des actions.
	Une \textbf{règle de décision} est alors définie comme une fonction $\delta \colon \mathcal{X} \to \mathcal{A}$.
\end{defn}

\begin{defn}
	Soit $\delta \colon \mathcal{X} \to \mathcal{A}$ une règle de décision.
	Son \textbf{risque} sous la loi $P_\theta \in \mathcal{P}$ est $R(\theta,\delta) = \esp_\theta \left[ L(\theta, \delta(X)) \right] \in \bar{\R}_+$.
\end{defn}


\section{La transformation de Fourier (pour $\Z$ et $\Z / N\Z$)}

	\begin{pop}[Inégalité de Hölder]
	\begin{itemize}
	\item Si $u \in l^1$ et $v \in l^\infty$, alors $u \cdot v \in l^1$ et $\norme{u \cdot v}_1 \leq \norme{u}_1 \norme{v}_\infty$.
	\item Si $u \in l^2$ et $v \in l^2$ alors $u \cdot v \in l^1$ et $\norme{u \cdot v}_1 \leq \norme{u}_2 \norme{v}_2$ (CS).
	\end{itemize}
\end{pop}

\begin{pop}[Règles de convolution]
	$\begin{array}{|c||c|c|c|}
		\hline
		* & l^1 & l^2 & l^\infty \\ \hline \hline
		l^1 & l^1 & l^2 & l^\infty \\ \hline
		l^2 & l^2 & l^\infty & - \\ \hline
		l^\infty & l^\infty & - & - \\ \hline
	\end{array}$
	
	On a aussi, à chaque fois, $\norme{u \star v}_\gamma \leq \norme{u}_\alpha \norme{v}_\beta$.
\end{pop}

\begin{defn}
	Soit $u \in l^1$, sa transformée de Fourier à temps discret (\textbf{TFtD}) est $\mathcal{F}(u) = \hat{u} \colon \nu \mapsto \sum_{n \in \Z} u_n e^{-2i\pi \nu n}$.
	Elle est continue, que ce soit sur $\intfo{-\frac{1}{2}}{\frac{1}{2}}$ ou sur $\R$.
\end{defn}

\begin{pop}
	Soit $u,v \in l^1$, $\nu_0 \in \intfo{-\frac{1}{2}}{\frac{1}{2}}$, $\varphi$ une onde de Fourier sur $\Z$ de fréquence $\nu_0$, $m \in \Z$ et $\psi \colon x \mapsto e^{-2i\pi mx}$ une onde de Fourier sur $\intfo{-\frac{1}{2}}{\frac{1}{2}}$ de fréquence $-m$.
	\begin{itemize}
		\item La TFtD de l'impulsion en $m$ $(\delta_n^m)_n$ est une onde de Fourier de fréquence $-m$ sur $\intfo{-\frac{1}{2}}{\frac{1}{2}}$.
		\item $\mathcal{F}(u \star v) = \hat{u} \cdot \hat{v}$.
		\item $\mathcal{F}(u \cdot v) = \hat{u} \star \hat{v}$.
		\item $\forall \nu \in \intfo{-\frac{1}{2}}{\frac{1}{2}}, (\mathcal{F}(\varphi \cdot u))(\nu) = \hat{u}(\nu - \nu_0)$.
		\item Soit $u^m$ la $m$-translatée de $u$, $\mathcal{F} \left( u^m \right) = \hat{u} \cdot \varphi$, i.e. $\hat{u^m}(\nu) = \hat{u}(\nu) e^{-2i\pi m \nu}$.
		\item Si $u$ est réelle, alors $\hat{u}$ est à symétrie hermitienne : $\hat{u}(-X) = \overline{\hat{u}(\nu)}$.
		\item Si $u$ est symétrique alors $\hat{u}$ aussi.
		\item Si $u$ est symétrique et réelle alors $\hat{u}$ aussi.
	\end{itemize}
\end{pop}

\begin{pop}
	Soit un SLI $T \colon l^\infty \to l^\infty$ et $h \in l^1$ sa R.I.
	Si $u \in l^1$ et $v = T(u)$ alors : la réponse fréquentielle de $T$ est $\hat{h}$, $h \star u = v \in l^1$ et $\hat{v} = \hat{h} \hat{u}$.
\end{pop}

\begin{thm}
	On peut étendre $\mathcal{F}$ de façon unique à $l^2$ et elle forme une bijection de $l^2$ sur $L^2 \left( \intfo{-\frac{1}{2}}{\frac{1}{2}} \right)$.
	De plus, on a l'égalité de \textbf{Parseval} : $\forall u \in l^2, \norme{\hat{u}}_2 = \norme{u}_2$.
\end{thm}

\begin{thm}[\textbf{Inversion} de la TFtD]
	Si $u \in l^2$ alors on a $\forall n \in \Z, u_n = \int_{-\frac{1}{2}}^{\frac{1}{2}} \hat{u}(\nu) e^{2i\pi n\nu} \diff \nu$.
\end{thm}

\begin{thm}
	Soit $k \in \N$, on a $\left( \sum_{n \in \Z} \abs{n}^k \abs{u_n} < \infty \right) \implies \left( \hat{u} \in \cont^k \left( \intfo{-\frac{1}{2}}{\frac{1}{2}} \right) \right)$ et $\hat{u}^{(k)} = \hat{v^k}$ où $v_n^k = (-2i\pi n)^k u_n$.
\end{thm}

\begin{thm}
	Si $u \colon \Z / N\Z \to \R$.
	On note $\hat{u}$ sa transformée de Fourier discrète (\textbf{TFD}) définie sur $\Z / N\Z$ par $k \mapsto \sum_{n \in \Z / N\Z} u_n e^{-2i\pi \frac{k}{N}n}$.
\end{thm}

\section{Transformée en cosinus discret}

	\begin{defn}
	Soit $f \in L^1$.
	Sa \textbf{transformée de Fourier} est $\mathcal{F}(f) = \hat{f} := \xi \mapsto \int f(x) e^{-2i\pi \xi x} \diff x$.
\end{defn}

\begin{pop}
	Soit $f,g \in L^1, \lambda,\alpha \in \R$.
	\begin{enumerate}[(i)]
		\item $\hat{f}$ est bornée par $\norme{f}_1$, donc $\mathcal{F}$ est linéaire continue de $L^1$ dans $L^\infty$,
		\item $\hat{f}$ est continue,
		\item $\hat{f}(\xi)$ tend vers $0$ lorsque $\abs{\xi}$ tend vers $+\infty$,
		\item $\mathcal{F}(f \star g) = \hat{f} \cdot \hat{g}$,
		\item $\int \hat{f} \cdot g = \int f \cdot \hat{g}$,
		\item Si $g(x) = f(x) e^{2i\pi \alpha x}$ alors $\hat{g}(\xi) = \hat{f}(\xi - \alpha)$,
		\item Si $g(x) = f(x - \alpha)$ alors $\hat{g}(\xi) = \hat{f}(\xi) e^{-2i\pi \alpha \xi}$,
		\item Si $g(x) = \overline{f(-x)}$ alors $\hat{g}(\xi) = \overline{\hat{f}(\xi)}$,
		\item Si $g(x) = f(x / \lambda)$ avec $\lambda > 0$ alors $\hat{g}(\xi) = \lambda \hat{f}(\lambda \xi)$.
	\end{enumerate}
\end{pop}

\begin{defn}[Un couple de fonctions auxiliaires]
	Soit $n \in \N^*$.
	On a $H_n := x \mapsto e^{-\frac{\abs{x}}{n}}$ et $h_n := x \mapsto n \frac{2}{1 + 4\pi^2 (nx)^2}$.
	On remarque que $H_n(nx) = H_1(x)$ (homotéthie) et $h_n(x) = n h_1(nx)$ de sorte que $\int h_n = \int h_1$.
\end{defn}

\begin{pop}
	\begin{enumerate}[(i)]
		\item $\forall n \geq 1, \forall 1 \leq p \leq \infty, h_n \in L^p$ et $H_n \in L^p$,
		\item $\mathcal{F}(H_n) = h_n$,
		\item $\int h_n(t) \diff t = 1$,
		\item Si $f \in L^p, p < \infty$, alors $h_n \star f$ tend vers $f$ dans $L^p$,
		\item Si $f \in L^1$, alors $\forall x \in \R, (f \star h_n)(x) = \int \hat{f}(\xi) H_n(\xi) e^{2i\pi x \xi} \diff \xi$,
		\item Si $f$ est bornée et continue en $x$ alors $(f \star h_n)(x) \underset{n \to \infty}{\longrightarrow} f(x)$.
	\end{enumerate}
\end{pop}

\begin{defn}
	Si $f \in L^1$, sa \textbf{transformée de Fourier inverse} est $\bar{\mathcal{F}}(f) \colon x \mapsto \int f(t) e^{2i\pi xt} \diff t$ (continue).
\end{defn}

\begin{thm}[Théorème d'inversion]
	Si $f \in L^1$ et $\hat{f} \in L^1$ alors $\bar{\mathcal{F}}(\hat{f}) \overset{\text{p.p.}}{=} f$ (donc égalité dans $L^1$).
	En particulier, si $\hat{f} \in L^1$ alors $f$ est égale p.p. à une fonction continue car $\\bar{\mathcal{F}}$ a les mêmes propriétés que $\mathcal{F}$.
\end{thm}

\begin{cor}
	Si $f \in L^1$ et $\hat{f} = 0$ alors $f = 0$, i.e. $\mathcal{F}$ est injective.
\end{cor}

\begin{thm}[Extension à $L^2$]
	\begin{enumerate}
		\item Si $f \in L^1 \cap L^2$ alors $\hat{f} \in L^2$ et $\norme{\hat{f}}_2 = \norme{f}_2$.
		\item Il existe une unique application dans $\Orth(L^2) \cap \cont^0(L^2,L^2)$ égale à $\mathcal{F}$ sur $L^1 \cap L^2$, notée encore $\mathcal{F}$.
		\item $\im(\mathcal{F})$ est dense dans $L^2$.
		\item $\mathcal{F}$ est bijective de $L^2$ dans lui-même.
	\end{enumerate}
\end{thm}

\begin{thm}
	On étend $\bar{\mathcal{F}}$ de la même manière et il vient :
	$\quad \forall f \in L^2, \bar{\mathcal{F}}(\mathcal{F}(f)) = f,\qquad
		\forall f,g \in L^2, f \star g = \bar{\mathcal{F}}(\hat{f} \cdot \hat{g})$.
\end{thm}

\begin{defn}
	$\cont_c^\infty$ : ensemble des fonctions indéfiniment dérivables à support compact. C'est un $\C$-ev non réduit à $\{ 0 \}$.
\end{defn}

\begin{thm}
	Soit $1 \leq p < \infty$.
	\begin{enumerate}
		\item Si $g \in \cont_c^0$ et $h \in \cont_c^\infty$ alors $g \star h \in \cont_c^\infty$ et $(g \star h)^{(n)} = \left( g \star h^{(n)} \right)$.
		\item $\forall f \in L^p, f \star \rho_n \overset{L^p}{\to} f$ en notant $\rho \colon x \mapsto e^{-\frac{1}{x}} e^{-\frac{1}{1 - x}}$, $\rho_1 = \frac{\rho}{\int \rho}$ et $\forall n \geq 1, \rho_n \colon x \mapsto n \rho_1(nx)$.
		\item Les fonctions $\cont_c^\infty$ sont denses dans $L^p$.
	\end{enumerate}
\end{thm}

\begin{thm}[Échange de régularité et de décroissance à l'infini]
	\begin{enumerate}
		\item Si $f \in \cont^1 \cap L^1$ et $f' \in L^1$ alors $\mathcal{F}(f')(\xi) = 2i\pi \xi \hat{f}(\xi)$.
		\item Si $f \in L^1$ et $(x \mapsto x f(x)) \in L^1$ alors $\hat{f}$ est continûment dérivable et $\mathcal{F}(f)' = \mathcal{F}(x \mapsto -2i\pi x f(x))$.
		\item Si $f \in \cont^n \cap L^1$ et $\forall k \leq n, f^{(k)} \in L^1$ alors $\mathcal{F}\left( f^{(n)} \right) (\xi) = (2i\pi \xi)^n \hat{f}(\xi)$.
		\item Si $f \in L^1$ et $\forall k \leq n, (x \mapsto x^k f(x)) \in L^1$ alors $\hat{f}$ est $n$ fois continûment dérivable et $\mathcal{F}(f)^{(n)} = \mathcal{F}(x \mapsto (-2i\pi x)^n f(x))$.
	\end{enumerate}
\end{thm}

\begin{defn}
	On dit que $f$ est dans la \textbf{classe de Schwartz} $\mathcal{S}$ si $f \in \cont^\infty$ et $\forall n,k \in \N, f^{(n)}(x) x^k \underset{\abs{x} \to \infty}{\to} 0$.
\end{defn}

\begin{pop}
	Soit $f,g \in \mathcal{S}$ et $P \in \K[X]$. On a $f^{(n)} \in \mathcal{S}, \quad f \cdot g \in \mathcal{S}, \quad P \cdot f \in \mathcal{S}, \quad \forall 1 \leq p \leq \infty, f \in L^p\quad$ et $\cont_c^\infty \subset \mathcal{S}$.
	Donc $\mathcal{S}$ est dense dans tous les $L^p$ pour $p < \infty$.
\end{pop}

\begin{thm}
	Si $f \in \mathcal{S}$ alors $\hat{f} \in \mathcal{S}$.
\end{thm}

\begin{thm}
	La transformée de Fourier est une bijection entre $\mathcal{S}$ et lui-même et son inverse est $\mathcal{S}$.
\end{thm} 


\section{Processus aléatoires sur $\Z$}

	\subsection{Définition des processus}

	On se donne une mesure de probabilité $\proba$ sur un espace probabilisé $\Omega$.
	
	\begin{defn}
		Un \textbf{processus} $X$ est une fonction de $\Z$ vers l'ensemble des variables aléatoires (suite de v.a.).
	\end{defn}
	
	\begin{defn}
		Si $X$ est un processus tel que $\forall n, X_n \in L^1(\Omega)$.
		On dit que $X$ est \textbf{stationnaire à l'ordre 1} si $\exists m_X \in \C, \forall n \in \Z, \esp(X_n) = m_X$.
	\end{defn}
	
	\begin{defn}
		Si $X$ et $Y$ sont $L^2$ (admettent des variances) leur covariance est définie par
		$\Cov(X,Y) = \esp \left[ (X - \esp(X)) \overline{(Y - \esp(Y))} \right] = \esp(X^C \overline{Y^C})$.
		D'après l'inégalité de Cauchy-Schwartz $\abs{\Cov(X,Y)} \leq \sqrt{\Var(X) \Var(Y)}$.
	\end{defn}
	
	\begin{defn}
		On dit que le processus $X$ est \textbf{stationnaire à l'ordre 2} si $\forall n \in \Z, X_n \in L^2(\Omega)$ et
		$\forall k \in \Z, \forall n \in \Z, \Cov(X_{n + k},X_n) = \Cov(X_k,X_0)$.
		En particulier la $X_n$ ont tous la même variance.
	\end{defn}
	
	\begin{defn}
		Les processus stationnaires au sens large (SSL) sont ceux stationnaires aux ordres 1 et 2.
	\end{defn}
	
	\begin{defn}
		Soit $X$ une processus stationnaire au second ordre.
		On appelle \textbf{autocovariance} de $X$ la fonction, définie sur $\Z$, $R_k \colon k \mapsto \Cov(X_k,X_0) = \Cov(X_{n + k},X_n)$.
		On en déduit $\forall k \in \Z, R_X(-k) = \overline{R_X(k)}$ et $R_X(0) \in \R_+$ avec
		$\forall k \in \Z, \abs{R_X(k)} \leq R_X(0)$.
	\end{defn}
	
	\begin{defn}
		Soit $X$ stationnaire du second ordre.
		Si $R_X \in l^1$, on définit sa \textbf{densité spectrale de puissance} par $\forall \nu \in \intfo{-\frac{1}{2}}{\frac{1}{2}}, S_X(\nu) := \sum_{k \in \Z} R_X(k) e^{-2i\pi \nu k} = \mathcal{F}(R_X)(\nu)$.
		Elle est à valeurs réelles puisque $R_X$ est à symétrie hermitienne.
	\end{defn}
	
	\begin{defn}
	\textbf{Puissance d'un processus SSL} : norme $L^2$ au carré de $X_n$, noté $P_X$.
	Elle ne dépend pas de $n$ et est donnée par $P_X = \esp \left( \abs{X}^2 \right) = \abs{m_X}^2 + R_X(0) = m_X^2 + \int_{-\frac{1}{2}}^{\frac{1}{2}} S_X(\nu) \diff \nu$.
	\end{defn}
	
	\begin{pop}[Positivité de la DSP]
		Soit $X$ stationnaire au 2\up{nd} ordre avec $R_X$ sommable.
		Alors $\forall \nu \in \intfo{-\frac{1}{2}}{\frac{1}{2}}, S_X(\nu) \geq 0$.
	\end{pop}

\subsection{Filtrage des processus SSL}

	\begin{pop}[Filtrage par un filtre sommable]
		Soit $X$ un processus SSL avec $R_X$ sommable, et $h$ une suite sommable.
		On appelle $Y = h * X$ le processus filtré de $X$ par le noyau $h$.
		Il est défini par $\forall n \in \Z, Y_n = \sum_{l \in \Z} h_l X_{n - l}$.
		Cette somme étant prise dans $L^2(\Omega)$, on a :
		\begin{enumerate}
		\item
			Pour presque tout $\omega \in \Omega$, $\forall n \in \Z, Y_n(\omega) = \sum_{l \in \Z} h_l X_{n - l}(\omega)$.
		\item
			$Y$ est SSL. On note $\tilde{h}_n = \overline{h_{-n}}$ le signal $h$ symétrisé et conjugué.
			On a $m_Y = m_X \sum_{l \in \Z} h_l$ et $R_Y = (h * \tilde{h}) * R_X$.
		\end{enumerate}
	\end{pop}

\end{document}