\subsection{Définition des processus}

	On se donne une mesure de probabilité $\proba$ sur un espace probabilisé $\Omega$.
	
	\begin{defn}
		Un \textbf{processus} $X$ est une fonction de $\Z$ vers l'ensemble des variables aléatoires (suite de v.a.).
	\end{defn}
	
	\begin{defn}
		Si $X$ est un processus tel que $\forall n, X_n \in L^1(\Omega)$.
		On dit que $X$ est \textbf{stationnaire à l'ordre 1} si $\exists m_X \in \C, \forall n \in \Z, \esp(X_n) = m_X$.
	\end{defn}
	
	\begin{defn}
		Si $X$ et $Y$ sont $L^2$ (admettent des variances) leur covariance est définie par
		\vspace{-0.15cm}$$
		\Cov(X,Y) = \esp \left[ (X - \esp(X)) \overline{(Y - \esp(Y))} \right] = \esp(X^C \overline{Y^C})\ .
		\vspace{-0.15cm}$$
		D'après l'inégalité de Cauchy-Schwartz : $\abs{\Cov(X,Y)} \leq \sqrt{\Var(X) \Var(Y)}$.
	\end{defn}
	
	\begin{defn}
		On dit que le processus $X$ est \textbf{stationnaire à l'ordre 2} si
		$\forall n \in \Z, X_n \in L^2(\Omega)$ et
		\vspace{-0.15cm}$$
		\forall k \in \Z, \forall n \in \Z, \Cov(X_{n + k},X_n) = \Cov(X_k,X_0)\ .
		\vspace{-0.15cm}$$
		En particulier les $X_n$ ont tous la même variance.
	\end{defn}
	
	\begin{defn}
		Les processus stationnaires au sens large (\textbf{SSL}) sont ceux stationnaires aux ordres 1 et 2.
	\end{defn}
	
	\begin{defn}
		Soit $X$ une processus stationnaire au second ordre.
		On appelle \textbf{autocovariance} de $X$ la fonction, définie sur $\Z$, $R_k \colon k \mapsto \Cov(X_k,X_0) = \Cov(X_{n + k},X_n)$.
		On en déduit $\forall k \in \Z, R_X(-k) = \overline{R_X(k)}$ et $R_X(0) \in \R_+$ avec
		$\forall k \in \Z, \abs{R_X(k)} \leq R_X(0)$.
	\end{defn}
	
	\begin{defn}
		Soit $X$ stationnaire du second ordre.
		Si $R_X \in l^1$, on définit sa \textbf{densité spectrale de puissance} par
		\vspace{-0.15cm}$$
		\forall \nu \in \intfo{-\frac{1}{2}}{\frac{1}{2}}, S_X(\nu) := \sum_{k \in \Z} R_X(k) e^{-2i\pi \nu k} = \mathcal{F}(R_X)(\nu)\ .
		\vspace{-0.15cm}$$
		Elle est à valeurs réelles puisque $R_X$ est à symétrie hermitienne.
	\end{defn}
	
	\begin{defn}
	\textbf{Puissance d'un processus SSL} : norme $L^2$ au carré de $X_n$, noté $P_X$.
	Elle ne dépend pas de $n$ et est donnée par $P_X = \esp \left( \abs{X}^2 \right) = \abs{m_X}^2 + R_X(0) = m_X^2 + \int_{-\frac{1}{2}}^{\frac{1}{2}} S_X(\nu) \diff \nu$.
	\end{defn}
	
	\begin{pop}[Positivité de la DSP]
		Soit $X$ stationnaire au 2\up{nd} ordre avec $R_X$ sommable.
		Alors $\forall \nu \in \intfo{-\frac{1}{2}}{\frac{1}{2}}, S_X(\nu) \geq 0$.
	\end{pop}

\subsection{Filtrage des processus SSL}

	\begin{pop}[Filtrage par un filtre sommable]
		Soit $X$ un processus SSL avec $R_X$ sommable, et $h$ une suite sommable.
		On appelle $Y = h * X$ le processus filtré de $X$ par le noyau $h$.
		Il est défini par $\forall n \in \Z, Y_n = \sum_{l \in \Z} h_l X_{n - l}$.
		Cette somme étant prise dans $L^2(\Omega)$, on a :
		\begin{enumerate}
		\item
			Pour presque tout $\omega \in \Omega$, $\forall n \in \Z, Y_n(\omega) = \sum_{l \in \Z} h_l X_{n - l}(\omega)$.
		\item
			$Y$ est SSL. On note $\tilde{h}_n = \overline{h_{-n}}$ le signal $h$ symétrisé et conjugué.
			On a $m_Y = m_X \sum_{l \in \Z} h_l$ et $R_Y = (h * \tilde{h}) * R_X$.
			Soit ponctuellement
			$\forall k \in \Z, R_Y(k) = \sum_l (h \star \tilde{h})(l) \star R_X(k - l) = \sum_{t,m} h_t \bar{h}_m R_X(k - t + m)$
			et
			$\forall \nu \in \intfo{-\frac{1}{2}}{\frac{1}{2}}, S_Y(\nu) = \abs{\hat{h}(\nu)}^2 S_X(\nu)$ ($\hat{h}$ est la TFtD de $h$).
		\item
			Si $g$ est un autre signal sommable et $Z = g * Y$ alors $Z = (g * (h * X)) = (g * h) * X$.
		\end{enumerate}
	\end{pop}

	\begin{pop}[Filtrage récursif]
		Soit $b_0,\ldots,b_q$ et $a_0,\ldots,a_p$ des complexes tels que les polynômes $P(z) = \sum_n a_n z^n$ et $Q(z) = \sum_n b_n z^n$ n'ont pas de zéro commun et que $Q$ n'a pas de zéro sur $\U$.
		Si de plus $X$ est un processus SSL, alors il existe un unique processus SSL $Y$ tel que $\sum_i b_i Y_{n - i} = \sum_i a_i X_{n - i}$ et $Y = h * X$ où $h$ est la réponse impulsionnelle du filtre récursif stable défini par cette équation de récurrence.
	\end{pop}