\begin{thm}[Formule de Poisson ou le repliement spectral]
	Si $f$ est une fonction définie sur $\R$, intégrable et telle que sa transformée de Fourrier est aussi intégrable et que la suite $(f(n))_n$ est sommable, alors :
	\vspace{-0.1cm}$$
	\sum_{m\in Z} f(m)e^{-2i\pi m\nu } = \sum_{n\in Z} \hat{f}(n+\nu )
	\qquad \text{et} \qquad
	\hat{u}(\nu ) = \sum_{n\in Z} \hat{f}(n+\nu )\ .
	\vspace{-0.1cm}$$
\end{thm}

\begin{thm}[Théorème de bon échantillonnage ou théorème de Shannon]
	Si $f$ est une fonction sommable et que sa TFtC, $\hat{f}$, est nulle en dehors de $\intff{-\frac{1}{2}}{\frac{1}{2}}$, alors on a
	\vspace{-0.15cm}$$
	f(t)=\sum_{n\in \Z} f(n) \sinC (\pi (t-n))\ .
	\vspace{-0.15cm}$$
\end{thm}

\begin{thm}[Théorème de Shannon pour les énergies finies]
	Soit $f$ d'énergie finie et $u_{n} = f(n)$, alors $\norme{f}_{2} = \norme{u}_{2}$.
\end{thm}

\begin{defn} La formule
	$$\forall t \in \R, f(t) = \sum_{n\in \Z} f(n) \sinC (\pi (t-n))$$
	signifie que, si le spectre de $f$ est à support dans $\intff{-\frac{1}{2}}{\frac{1}{2}}$, alors on peut reconstruire la fonction $f$ à partir de ses échantillons.
	On parle \textbf{CNA parfait} ou \textbf{CNA idéal}.
\end{defn}
