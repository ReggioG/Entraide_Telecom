\begin{thm}[Formule de Poisson ou le repliement spectral]
\textit{Si f est une fonction définie sur R, intégrable et telle que sa transformée de Fourrier est aussi integrable et que la suite $f(m)$ est sommable, alors : }
$$\sum_{m\in Z} f(m)e^{-2i\pi m\nu } = \sum_{n\in Z} \hat{f}(n+\nu )$$
$$\hat{u}(\nu ) = \sum_{n\in Z} \hat{f}(n+\nu )$$
\end{thm}

\begin{thm}[Théorème de bon échantillonnage ou théorème de Shannon]
\textit{Si f est une fonction sommable et que TFtC, $\hat{f}$ est nulle en dehors de [-0.5, 0.5], alors on a : }
$$f(t)=\sum_{n\in Z}f(n)sinC(\pi (t-n))$$
\end{thm}

\begin{thm}[Théorème de Shannon pour les énergies finies]
\textit{Si f est d'énergie finie. On note $u_{n}=f(n)$, alors :}
 $$||f||_{2} = ||u||_{2}$$ 
\end{thm}

\begin{defn} La formule : 
$$\forall t\in R   f(t)=\sum_{n\in Z}f(n)sinC(\pi (t-n))$$
Signifie que si le spectre de f est à support dans [-0.5, 0.5] alors on peut reconstruire la fonction f à partir de ses échantillons. On parle \textbf{CNA parfait} ou \textbf{CNA idéal}
\end{defn}
