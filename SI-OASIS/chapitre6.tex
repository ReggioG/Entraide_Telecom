\begin{defn}[Approximation]
	Soit $x$ un vecteur et $\alpha_{n}$ une collection de M vecteurs, de plus $n_{j}\in \lbrace 1 ... M \rbrace$ on a :
	\vspace{-0.15cm}$$
	\tilde{x} = \sum_{j=0}^{m-1} a_{j} \alpha_{n_{j}}\ .
	\vspace{-0.15cm}$$
	On dit que $\tilde{x}$ est une \textbf{approximation} de $x$ dans la collection d'atomes $\alpha_{n}$ avec les coefficient $a_{j}$.
\end{defn}

\begin{defn}
	Le \textbf{taux de compression}, noté $\tau_{c}$, est défini par $\tau_{c} = \dfrac{m}{N}$.
\end{defn}

\begin{defn}
	L'\textbf{erreur relative} de compression, notée $\tau_{c}$, est définie par
	$err_{c} = \dfrac{ \norme{x-\tilde{x}} }{\norme{x}}$.
\end{defn}

\begin{defn}[Compression linéaire]
	Pour un taux de compression $\tau = \dfrac{m}{N}$, on prend pour approximation de $x$, le vecteur :
	\vspace{-0.15cm}$$
	\tilde{x} = \sum_{i=1}^{m} a_{j} \scal{x}{\alpha_{i}} \alpha_{i}\ .
	\vspace{-0.15cm}$$
	Autrement dit, on choisit les $m$ premiers vecteurs de la base $\alpha$ pour approximer $x$.
\end{defn}