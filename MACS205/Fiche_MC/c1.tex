%~ But du cours : étudier des méthodes aléatoires d'approximation d'intégrales.

Soit $(S,\mathcal{S},\mu)$ un espace mesuré où $\mu$ est une mesure positive.
Soit $\varphi \colon S \to \R$ une fonction intégrable.
On cherche à approcher $I(\varphi) = \int \varphi \diff \mu = \esp_\mu(\varphi)$.
Deux cas de figure :
\begin{itemize}
	\item[\textbullet] $\varphi$ est une fonction continue avec une expression analytique et on arrive à calculer son intégrale,
	\item[\textbullet] l'intégrale de $\varphi$ est incalculable.
		%~ Exemples : Gaussienne ou indicatrice d'ensemble $S$ où l'on ne connaît pas de forme analytique.
\end{itemize}

Les méthodes de type Monte-Carlo considérées sont de la forme suivante :
\begin{enumerate}
	\item tirer aléatoirement des points $X_1,\ldots,X_n$ sur $S$,
	\item calculer $\varphi(X_1),\ldots,\varphi(X_n)$,
	\item trouver une transformation de $(X_1, \varphi(X_1)),\ldots,(X_n, \varphi(X_n))$ qui approche $I(\varphi)$.
\end{enumerate}