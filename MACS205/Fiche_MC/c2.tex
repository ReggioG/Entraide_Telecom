$$I(\varphi) = \int \varphi \diff N = \esp_\mu(\varphi)$$

D'après la LFGN, si $X_1,\ldots,X_n$ sont i.i.d. de loi $\mu$ tel que $\esp_\mu \abs{vf} < \infty$ alors $\frac{1}{n} \sum_i \varphi(X_i) \overset{\text{p.s.}}{\longrightarrow} \esp_\mu (\varphi(X_1))$.

\begin{algorithm}[h]
\caption{\textcolor{RoyalBlue}{Monte-Carlo}}
	Générer $X_1,\ldots,X_n$ de façon indépendante sous $\mu$ \;
	Calculer $\varphi(X_1),\ldots,\varphi(X_n)$ \;
	\Sortie{$\Hat{I}_n(\varphi) = \frac{1}{n} \sum_i \varphi(X_i)$}
\end{algorithm}

\begin{pop}
	Si $\int \abs{\varphi} \diff \mu < \infty$, $\hat{I}_n(\varphi)$ est non-biaisée et consistante.
	Si $\int \abs{\varphi}^2 \diff \mu < \infty$, $\Var(\hat{I}_n(\varphi)) = \frac{1}{n} \Var(\varphi(X_1)) = \frac{1}{n} \sigma^2$ et $\sqrt{n} \left( \hat{I}_n(\varphi) - I(\varphi) \right) \overset{\mathcal{L}}{\longrightarrow} \normale(0,1)$.
\end{pop}

On estime $\sigma^2$ par $\hat{\sigma}^2 = \frac{1}{n - 1} \sum_{i = 1}^n \left( \varphi(X_i) - \hat{I}_n(\varphi) \right)^2$.

\begin{pop}
	Si $\int \abs{\varphi}^2 \diff \mu < \infty$ alors $\hat \sigma^2$ est sans biais et fortement consistant (par la LFGN).
\end{pop}

...

\begin{thm}[Inégalité de Hoeffding]
	Soit $(X_1,\ldots,X_n)$ des v.a i.i.d telles que $\forall i \in \iniff{1}{n}, a \leq X_1 \leq b$ p.s.
	Alors
	$$\proba \left( \abs{\sum_{i = 1}^n (X_i - \esp(X_i)) } > \varepsilon \right) \leq 2 e^{- \frac{3 \varepsilon^2}{n(b - a)^2}}\ .$$
\end{thm}

...


\subsection{Concentration}

	...


\subsection{Déterministe vs aléatoire}

	On se place dans le cadre de l'approximation de $\int_{\inff{0}{1}^d} \varphi(x) \diff x$ où $\varphi \colon \inff{0}{1}^d \longrightarrow \R$.

	\paragraph{Méthode déterministe des sommes de Riemann}

	On se donne $n^d$ points équidistants $\left( \frac{i_1}{n},\ldots,\frac{i_d}{n} \right)$ où $(i_1,\ldots,i_d) \in \intiff{1}{n}^d$.
	La méthode des sommes de Riemann est
	$$I_n(\varphi) = \frac{1}{n^d} \sum_{(i_1,\ldots,i_d) \in \intiff{1}{n}^d} \varphi \left( \frac{i_1}{n},\ldots,\frac{i_d}{n} \right)\ .$$

	\begin{pop}
		Si $\varphi \colon \inff{0}{1}^d \longrightarrow \R$ est $L$-lipschitzienne alors $\abs{I_n(\varphi) - I(\varphi)} \leq L \frac{\sqrt{d}}{n}$.
	\end{pop}

	...

	