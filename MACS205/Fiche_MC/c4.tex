Contexte : estimation de $I_\lambda = \int \varphi \diff \lambda$ où $\varphi \colon S \to \R$ est intégrable et $S \subseteq \R^d$.

\begin{defn}
	Si $g \colon \R^d \to \R$, on définit son support comme l'ensemble fermé $S_g = \overline{\{ x \in \R^d \mid g(x) \neq 0 \} }$.
\end{defn}

L'échantillonage d'importance se base sur la formule suivante : pour toute densité $f$ telle que $S_f \supset S_{\varphi}$,
$$I_\lambda = \int_{S_{\varphi}} \varphi(x) \diff x = \int_{S_f} \varphi(x) \diff x = \int_{S_f} \frac{\varphi(x)}{f(x)} f(x) \diff x = \esp_{X \sim f} \left[  \frac{\varphi(X)}{f(X)} \right]$$

L'échantillonage d'importance “naïf” consiste à générer $X_1,\ldots,X_n \sim f$ i.i.d puis appliquer Monte-Carlo :
$$\hat{I}_n^{(is)} = \frac{1}{n} \sum_{i = 1}^n \frac{\varphi(X_i)}{f(X_i)}\ .$$

\noindent La distribution associée à $f$ est appelée \textbf{distribution d'échantillonage}, ou bien l'\textbf{échantilloneur}.

\begin{pop}
	Si $\int \abs{\varphi} < \infty$ et $S_\varphi \subseteq S_f$, $\hat{I}_n^{(is)} \overset{\text{p.s.}}{\longrightarrow} I_\lambda$.
	Si de plus $\int \frac{\varphi^2(x)}{f(x)} \diff x < \infty$ alors :
	\begin{itemize}
		\item[\textbullet] $\sqrt{n} \left( \hat{I}_n^{(is)} - I_\lambda \right) \overset{\mathcal{L}}{\longrightarrow} \mathcal{N} \left(0,r_f^2(\varphi) \right)$ avec $r_f^2(\varphi) = \Var \left( \frac{\varphi}{f} \right) = \int \left( \frac{\varphi(x)}{f(x)} - I_\lambda \right)^2 f(x) \diff x$
		\item[\textbullet] $\frac{\sqrt{n}}{\hat{r}_n} \left( \hat{I}_n^{(is)} - I_\lambda \right) \overset{\mathcal{L}}{\longrightarrow} \mathcal{N}(0,1)$ avec $\hat{r}_n^2 = \frac{1}{n} \sum \left( \frac{\varphi(X_i)}{f(X_i)} - \hat{I}_n^{(is)} \right)^2$ l'estimateur de la variance.
	\end{itemize}
\end{pop}

Cette méthode est naïve car $f$ n'est pas choisie par rapport à $\varphi$.

%~ \begin{ex}
	%~ Soit $\varphi$ la densité d'une gaussienne centrée réduite et $f \sim \mathcal{N}(\theta,1)$.
	%~ Ici $I(\varphi) = 1$,
	%~ $$\sigma^2 + 1
	%~ = \int \frac{\varphi^2}{f^2} f \diff \lambda
	%~ = \int \frac{\varphi^2}{f} \diff \lambda
	%~ = \frac{1}{\sqrt{2\pi}} \int e^{-x^2 + (x - \theta)^2/2} \diff x
	%~ = e^{\theta^2} \underset{= 1}{\underbrace{ \frac{1}{\sqrt{2\pi}} \int e^{-(x + \theta)^2} \diff x }}$$
	%~ Les deux cas de figure extrêmes sont :
	%~ \begin{itemize}
		%~ \item[\textbullet] $\theta = 0$, d'où $\sigma^2 = 0$.
			%~ Plus généralement, si $\varphi$ n'est pas nécessairement une densité, alors le choix $f \propto \varphi$ (mais positif, e.g $f = \frac{\varphi}{\int \varphi}$) est optimal.
			%~ Ce choix dépend de la solution à notre problème de départ, donc il est impossible à réaliser.
		%~ \item[\textbullet] $\theta \gg 1$, d'où $\sigma^2 \gg 1$.
	%~ \end{itemize}
%~ \end{ex}

\begin{rem}
	Si on cherche $\esp[\varphi(Z)]$ où $Z \sim g$, alors prendre $g \cdot \varphi$ à la place de $\varphi$.
\end{rem}

Par ailleurs il existe deux méthodes de réduction de la variance où l'on s'adapte à $\varphi$ :
\begin{itemize}
	\item[\textbullet] variable de contrôle : approcher $\varphi$ dans une certaine base \textrightarrow\ pas de choix d'échantilloneur,
	\item[\textbullet] changer la mesure d'échantillonage.
\end{itemize}


\subsection{Réduction de la variance}

	On remarque que
	$r_f^2(\varphi) = 0
	%~ \iff \int \left( \frac{\varphi}{f} - I(\varphi) \right)^2 f \diff \lambda = 0
	\iff \varphi / f \overset{\text{p.p}}{=} I_\lambda$.

	\begin{thm}
		Si $\int \abs{\varphi} < \infty$, alors parmi les densités $f$ telle que $\int \frac{\varphi^2}{f} \diff \lambda < \infty$, i.e. $S_\varphi \subseteq S_f$, le minimiseur de $r_f^2(\varphi)$ est unique et donné par $f^* = \frac{\abs{\varphi}}{\int \abs{\varphi} \diff \lambda}$. La variance associée est $r_{f^*}^2(\varphi) = \left( \int \abs{\varphi} \diff \lambda \right)^2 - \left( \int \varphi \diff \lambda \right)^2$.
	\end{thm}


\subsection{Échantillonage préférentiel paramétrique}

	%~ En pratique et en dimension 1 ou 2, on peut représenter $\abs{\varphi}$ et en déduire un $f$ "proche" (visuellement) de $\abs{\varphi}$.
	On se donne une famille paramétrique $\mathcal{F} = \{ f_\theta, \theta \in \Theta \}$ de densités par rapport à la mesure de Lebesgue pour lesquelles on sait générer des v.a, avec $\Theta \subset \R^q, q \geq 1$.
	On suppose $\forall \theta \in \Theta, S_\varphi \subseteq S_{f_\theta}$.
	
	Soit $\theta^* \in \argmin_{\theta \in \Theta} r_{f_\theta}^2(\varphi) = \argmin_{\theta \in \Theta} \int \frac{\varphi^2}{f_\theta} \diff \lambda$.
	On cherche à estimer par simulation cette variance.
	On utilise l'algorithme suivant, avec en entrée $n \in \N^*$, $\mathcal{F}$ et $f_0$ l'échantillon initial.
	\begin{enumerate}[(i)]
		\item Soit $n_1 < n$ et $n_2 = n - n_1$. Générer $X_1,\ldots,X_{n_1} \sim f_0$ i.i.d.
		\item Calculer $\hat{\theta}_{n_1}^{(1)} = \argmin_{\theta \in \Theta} \hat{\psi}{n_1}^{(1)}(\theta)$ où $\hat{\psi}_{n_1}^{(1)}(\theta) := \frac{1}{n_1} \sum_{i = 1}^{n_1} \frac{\varphi(X_i)^2}{ f_\theta(X_i) f_0(X_i) }$.
		\item Générer $Z_1,\ldots,Z_{n_2} \sim f_{\hat{\theta}_{n_1}}$.
		\item Calculer $\frac{1}{n_2} \sum_{i = 1}^{n_2} \frac{\varphi(Z_i)}{f_{\hat{\theta}_{n_1}^{(1)}}(Z_i)}$
			ou $\frac{1}{n} \left( \sum_{i = 1}^{n_1} \frac{\varphi(X_i)}{f_0(X_i)} + \sum_{i = 1}^{n_2} \frac{\varphi(Z_i)}{f_{\hat{\theta}_{n_1}^{(1)}}(Z_i)} \right)$.
	\end{enumerate}
	
	\noindent On peut aussi utiliser la méthode par vraisemblance où l'on remplace $\hat{\psi}_{n_1}^{(1)}$ par $\hat{\psi}_{n_1}^{(2)} \colon \theta \mapsto \frac{1}{n_1} \sum_{i = 1}^{n_1} \ln \left( \frac{f_\theta(X_i)}{\abs{\varphi(X_i)}} \right) \frac{\abs{\varphi(X_i)}}{f_0(X_i)}$.
	On appelle $\hat{I}_{n,1}^{(is)}$ et $\hat{I}_{n,2}^{(is)}$ les deux estimateurs obtenus.
	
	\begin{lem}
		Supposons $\Theta$ compact, $\psi$ continue sur $\Theta$, qu'il existe un unique $\theta^* \in \argmin_{\theta \in \Theta} \psi(\theta)$, que $\sup_{\theta \in \Theta} \abs{\hat{\psi}(\theta) - \psi(\theta)} \underset{n}{\overset{\proba}{\longrightarrow}} 0$ et que $\hat{\theta}_{n_1}$ minimise $\hat{\psi}$.
		Alors $\abs{\hat{\theta}_{n_1} - \theta^*} \underset{n_1}{\overset{\proba}{\longrightarrow}} 0$.
	\end{lem}
	
	\begin{pop}
		$\hat{I}_{n,1}^{(is)}$ et $\hat{I}_{n,2}^{(is)}$ sont des estimateurs sans biais de $I_\lambda$.
	\end{pop}
	
	\begin{pop}
		Soit $v_0 = \Var \left( \frac{\varphi(X_1)}{f_0(X_1)} \right)$.
		Supposons $v_0 < \infty$ et $\sup_{\theta \in \Theta} \int \frac{\varphi(x)^2}{f_\theta(x)} \diff x < \infty$.
		Alors on a $\Var \left( \hat{I}_{n,1}^{(is)} \right) = \frac{1}{n_2} \left( \esp[\psi^{(1)}(\hat{\theta}_{n_1})] - I_\lambda^2 \right)$ et $\Var \left( \hat{I}_{n,2}^{(is)} \right) = \frac{1}{n^2} \left[ n_1 v_0 + n_2 \left( \esp[\psi^{(1)}(\hat{\theta}_{n_1})] - I_\lambda^2 \right) \right]$.
	\end{pop}
	
	\begin{pop}
		Supposons $\esp \left[ \psi^{(1)}(\hat{\theta}_{n_1}) - \psi^{(1)}(\theta^*) \right] = \frac{\sigma^2}{n_1} + o \left( \frac{1}{n_1} \right)$ et $v_0 > v(\theta^*) = \psi^{(1)}(\theta^*) - I_\lambda^2$.
		On a :
		$$\lim_n \sqrt{n} \inf_{1 \leq n_1 \leq n} \left( \esp \left[ n \left( \hat{I}_{n,2}^{(is)} - \int \varphi \right)^2 \right] - \esp \left[ n \left( \hat{I}_{n}^{(is*)} - \int \varphi \right)^2 \right] \right) = 2 \sigma \sqrt{v_0 - v(\theta^*)}$$
		si $v(\theta^*) > 0$,
		$$\lim_n \sqrt{n} \inf_{1 \leq n_1 \leq n} \left( \esp \left[ n \left( \hat{I}_{n,1}^{(is)} - \int \varphi \right)^2 \right] - \esp \left[ n \left( \hat{I}_{n}^{(is*)} - \int \varphi \right)^2 \right] \right) = 2 \sigma \sqrt{v(\theta^*)}$$
		et si $v(\theta^*) = 0$,
		$$\lim_n n \inf_{1 \leq n_1 \leq n} \esp \left[ n \left( \hat{I}_{n,1}^{(is)} - \int \varphi \right)^2 \right] = 4 \sigma^2$$
	\end{pop}
	
	La complexité est en $O(n + n_1^2 + n_2) = O(n)$.