\subsection{Random processes}

	We consider a probability space $(\Omega, \mathcal{F}, \proba)$, an index $T$ and a measurable space $(\mathsf{X}, \mathcal{X})$ called the observation space.

	\begin{defn}
		A \textbf{random process} defined on $(\Omega, \mathcal{F}, \proba)$, indexed on $T$ and valued in $(\mathsf{X}, \mathcal{X})$ is a collection $(X_t)_{t \in T}$ of r.v. defined on $(\Omega, \mathcal{F}, \proba)$ and taking values in $(\mathsf{X}, \mathcal{X})$.
	\end{defn}

	\begin{defn}
		For each $\omega \in \Omega$, the application $t \mapsto X_t(\omega)$ is called the \textbf{path} associated to the experiment $\omega$.
	\end{defn}

	\begin{defn}
		A \textbf{filtration} of a measurable space $(\Omega, \mathcal{F})$ is an increasing sequence $(\mathcal{F}_t)_{t \in T}$ of sub-$\sigma$-fields of $\mathcal{F}$.
		A \textbf{filtered probability space} $(\Omega, \mathcal{F}, (\mathcal{F}_t)_{t \in T}, \proba)$ is a probability space endowed with a filtration.
		A random process $(X_t)_{t \in T}$ defined on $(\Omega, \mathcal{F}, \proba)$ is said to be \textbf{adapted} to the filtration if for each $t \in T$, $X_t$ is $\mathcal{F}_t$-measurable.
		Then we note $((X_t, \mathcal{F}_t))_{t \in T}$.
	\end{defn}

	\begin{defn}
		The \textbf{natural filtration} of a process $(X_t)_{t \in T}$ is the smallest filtration with respect to which $(X_t)_{t \in T}$ is adapted, i.e. $\forall t \in T, \mathcal{F}_t^X = \sigma(X_s, s \leq t)$.
	\end{defn}

	\begin{defn}
		We call \textbf{finite dimensional distributions}, or \textbf{fidi distributions}, of the process $X$ the collection of probability measures $(\proba_I)_{I \in \mathcal{I}}$ where $\proba_I$ denotes the probability distribution of the random vector $\{ X_t, t \in I \}$.
	\end{defn}

	Let $J \subset I$ two finite subsets.
	Let us denote bu $\Pi_{I,J}$ the canonical projection of $\mathsf{X}^I$ onto $\mathsf{X}^J$ defined by $\forall x = (x_t)_{t \in I} \in \mathsf{X}^I, \Pi_{I,J}(x) = (x_t)_{t \in J}$.
	Then $\proba_I \circ \Pi_{I,J}^{-1} = \proba_J$ (compatibility condition).
	We denote $\Pi_I = \Pi_{T,I}$ and $\Pi_s = \Pi_{ \{ s \} }$ where $s \in T$.

	\begin{thm}[Kolmogorov]
		Let $\mathcal{I}$ be the set of all finite subsets of $T$.
		Suppose that, for all $I \in \mathcal{I}$, $\nu_I$ is a probability measure on $(\mathsf{X}^I,\mathcal{X}^{\otimes I})$ and that the collection $\{ \nu_I, I \in \mathcal{I} \}$ satisfies $\forall I, J \in \mathcal{I}, I \subset J, \nu_I \circ \Pi_{I,J}^{-1} = \nu_J$.
		Then there exists a unique probability measure $\proba$ on $(\mathsf{X}^T,\mathcal{X}^{\otimes T})$ such that, $\forall I \in \mathcal{I}, \nu_I = \proba \circ \Pi_I^{-1}$.
	\end{thm}

	\begin{defn}
		Let $X = (X_t)_{t \in T}$ be a random process defined on $(\Omega,\mathcal{F},\proba)$.
		The \textbf{law in the sense of fidi distribution} is the image measure $\proba^X$, that is, the unique probability measure defined on $(\mathsf{X}^T,\mathcal{X}^{\otimes T})$ that satisfies $\forall I \in \mathcal{I}, \proba^X \circ \Pi_I^{-1} = \proba_I$, i.e. $\forall (A_t)_{t \in I} \in \mathcal{X}^I, \proba^X \left( \prod_{i \in I} A_t \times \mathsf{X}^{T \setminus I} \right) = \proba (X_t \in A_t, t \in I)$.
	\end{defn}
	
	\begin{defn}
		The canonical functions defined on $(\mathsf{X}^T,\mathcal{X}^{\otimes T})$ is the collection of measurable functions $(\xi_t)_{t \in T}$ valued in $(\mathsf{X},\mathcal{X})$ as $\forall  \omega = (\omega_t)_{t \in T} \in \mathsf{X}^T, \xi_t(\omega) = \omega_t$.
		When $(\mathsf{X}^T,\mathcal{X}^{\otimes T})$ is endowed with the image measure $\proba^X$ then the \textbf{canonical process} $(\xi_t)_{t \in T}$ defined on $(\mathsf{X}^T,\mathcal{X}^{\otimes T},\proba^X)$ has the same fidi distribution as $X$.
	\end{defn}


\subsection{Gaussian processes}

	\begin{defn}
		The real valued r.v. $X$ is Gaussian if its characteristic function satisfies
		$\phi_X(u) = \esp(e^{iuX}) = \exp(i \mu u - \sigma^2 u^2 / 2)$
		where $\mu \in \R$ and $\sigma \in \R_+$.
	\end{defn}

	If $\sigma \neq 0$ then $X$ admits a probability density function $p(x) = \frac{1}{\sigma \sqrt{2 \pi}} \exp \left( - \frac{(x - \mu)^2}{2 \sigma^2} \right)$.

	\begin{defn}
		A random vector $\transp{[X_1, \ldots, X_n]}$ valued in $\R^n$ is a Gaussian vector if any linear combination of $X_1, \ldots, X_n$ is a Gaussian variable.
	\end{defn}

	Let $\mu$ denote the mean vector of $\transp{[X_1, \ldots, X_n]}$ and $\Gamma$ its covariance matrix.
	Then $\forall u \in \R^n, Y = \transp{u} X$ is Gaussian, $\esp(Y) = \transp{u} \mu$ and $\Var(Y) = \transp{u} \Gamma u$.
	Thus $\phi_X(u) = \esp \left[ \exp \left( i \transp{u} X \right) \right] = \exp \left( i \transp{u} \mu - \frac{1}{2} \transp{u} \Gamma u \right)$.
	
	\begin{pop}
		The probability distribution of an $n$-dimensional Gaussian vector $X$ is determined by its mean vector and covariance matrix $\Gamma$.
		We denote $X \sim \normale(\mu,\Gamma)$.
		Conversely, for all vector $\mu \in \R^n$ and all non-negative symmetric matrix $\Gamma$, the distribution $\normale(\mu,\Gamma)$ is well defined.
	\end{pop}

	\begin{lem}
		Let $X \sim \normale(\mu,\Gamma)$ with $\mu \in \R^n$ and $\Gamma$ a $n \times n$ non-negative symmetric matrix.
		Then $X$ has independent components if and only if $\Gamma$ is diagonal.
	\end{lem}

	\begin{pop}
		Let $X \sim \normale(\mu,\Gamma)$ with $\mu \in \R^n$ and $\Gamma$ a $n \times n$ non-negative symmetric matrix.
		If $\Gamma$ is full rank, the probability distribution of $X$ admits a density defined in $\R^n$ by
		$p(x) = \frac{1}{(2 \pi)^{n/2} \sqrt{\det(\Gamma)}} \exp \left( - \frac{1}{2} \transp{(x - \mu)} \Gamma^{-1} (x - \mu) \right)$.
	\end{pop}

	\begin{defn}
		A real-valued random process $X = (X_t)_{t \in T}$ is called a \textbf{Gaussian process} if, for all finite set of indices $I = \{ t_1, \ldots, t_n \}$, $\transp{[X_{t_1}, \ldots, X_{t_n}]}$ is a Gaussian vector.
	\end{defn}

	\begin{thm}
		Let $T$ be any set of indices, $\mu \colon T \to \R$ and $\gamma \colon T \times T \to \R$ such that all restrictions $\Gamma_I$ to the set $I \times I$ with $I \subset T$ finite are nonnegative symmetrice matrices.
		Then one can define a probability space $(\Omega, \mathcal{F}, \proba)$ and a gaussien process $(X_t)_{t \in T}$ defined on this space with mean $\mu$ and covariance function $\gamma$, that is such that, $\forall s, t \in T, \mu(t) = \esp[X_t]$ and $\gamma(s,t) = \esp[(X_s - \mu(s))(X_t - \mu(t))]$.
	\end{thm}

	\begin{defn}
		Let $T$, $\mu$ and $\gamma$ be as above.
		We denote by $\normale(\mu,\gamma)$ the law of the Gaussian process with mean $\mu$ and covariance $\gamma$ in the sense of fidi distribution.
	\end{defn}


\subsection{Strict stationarity of a random process in discrete time}

	Suppose that $T = \Z$ or $T = \N$.

	\begin{defn}
		We denote by $S \colon \mathsf{X}^T \to \mathsf{X}^T$ and call the \textbf{shift operator} the mapping defined by $\forall x = (x_t)_{t \in T}, S(x) = (x_{t + 1})_{t \in T}$.
		For all $\tau \in T$ we define $S^\tau$ by $S^\tau (x) = (x_{t + \tau})_{t \in \tau}$.
		The operator $B = S^{-1}$ is called the \textbf{backshift operator}.
	\end{defn}

	\begin{defn}
		A random process $(X_t)_{t \in T}$ is \textbf{strictly stationary} if $X$ and $S \circ X$ have the same law, i.e. $\proba^{S \circ X} = \proba^X$.
	\end{defn}

\subsection{Stationarity preserving transformations}

	In this section, we set $T = \Z$, $\mathsf{X} = \C^d$ and $\mathcal{X} = \mathcal{B}(\C^d)$ for some integer $d \geq 1$.
	
	\begin{defn}
		Let $\phi$ be a measurable function from $\left( \mathsf{X}^T, \mathcal{X}^{\otimes T} \right)$ to $\left( \mathsf{Y}^T, \mathcal{Y}^{\otimes T} \right)$ and $X = (X_t)_{t \in T}$ be a process with values in $(\mathsf{X},\mathcal{X})$.
		A $\phi$-filtering process $Y = (Y_t)_{t \in T}$ is defined as $\forall t, Y = \phi \circ X$ or, equivalently, $Y_t = \Pi_t(\phi(X))$.
		Thus $Y$ makes its values in $(\mathsf{Y}, \mathcal{Y})$.
		If $\phi$ is linear, we will say that $Y$ is obtained by linear filtering of $X$.
	\end{defn}

	\begin{defn}
		A $\phi$-filter is \textbf{shift invariant} if $\phi$ commutes with $S$.
	\end{defn}

	\begin{rem}
		A shift invariant $\phi$-filter preserves the strict stationarity and is entirely determined by its composition with the canonical projection $\Pi_0$
	\end{rem}
