\subsection{Deffinition using conditioning}

	\begin{defn}
		Let $(\Omega, \mathcal{F}, (\mathcal{F}_k)_{k \in \N}, \proba)$ be a filtered probability space and $(\mathsf{X}, \mathcal{X})$ be a measurable space.
		An adapteted stochastic process $((X_k, \mathcal{F}_k))_{k \in \N}$ on $\mathsf{X}$ is a \textbf{Markov chain} if $\forall k \in \N, \forall A \in \mathcal{X}, \proba(X_{k + 1} \in A \mid \mathcal{F}_k) = \proba(X_{k + 1} \in A \mid A_k)$.
	\end{defn}

	\begin{pop}
		Let $((X_k, \mathcal{F}_k))_{k \in \N}$ be an adapted stochastic process.
		The following propoerties are equivalent :
		\begin{enumerate}[(i)]
			\item $((X_k, \mathcal{F}_k))_{k \in \N}$ is a Markov chain,
			\item $\forall k \in \N, \forall Y \in L^1(\Omega, \sigma(X_l, l \geq k), \proba), \esp(Y \mid \mathcal{F}_k) = \esp(Y \mid X_k)$ $\proba$-a.s.
			\item $\forall k \in \N, \forall Y \in L^1(\Omega, \sigma(X_l, l \geq k), \proba), \forall Z \in L^\infty(\Omega, \mathcal{F}_k, \proba), \esp(YZ \mid X_k) = \esp(Y \mid X_k) \esp(Z \mid X_k)$ $\proba$-a.s.
		\end{enumerate}
	\end{pop}


\subsection{How to use kernels}

	Let $N$ be a kernel and $f$ be a measurable function defined on $\mathsf{Y}$.
	We denote by $N f$ the function defined on $\mathsf{X}$ by $Nf \colon x \mapsto \int_{\mathsf{Y}} N(x,\diff y) f(y)$ whenever this integral is well-defined (for instance if $f$ is non-negative).

	\begin{pop}
		Let $N$ be a kernel on $\mathsf{X} \times \mathcal{Y}$.
		Then $\forall f \in \mathbf{F}_+(\mathsf{Y}, \mathcal{Y}), N f \in \mathbf{F}_+(\mathsf{X}, \mathcal{X})$.
		Moreover, if $N$ is a probability kernel, then $\forall f \in \mathbf{F}_b(\mathsf{X}, \mathcal{X}), Nf \in \mathbf{F}_b(\mathsf{Y}, \mathcal{Y})$.
	\end{pop}
	
	\begin{lem}
		Let $M \colon \F_+(\mathsf{Y},\mathcal{Y}) \to \F_+(\mathsf{X},\mathcal{X})$ be an additive and positively homogeneous function such that $\lim_{n \to \infty} M(f_n) = M(\lim_{n \to \infty} f_n)$ for every non-decreasing sequence $(f_n)_{n \in \N}$.
		Then the function $N$ defined on $\mathsf{X} \times \mathcal{Y}$ by $N(x,A) = M(\indic_A)(x)$ is a kernel and $\forall f \in \F_+(\mathsf{Y},\mathcal{Y}), M(f)(x) = \int_{\mathsf{Y}} N(x, \diff y) f(y)$.
	\end{lem}
	
	\begin{pop}
		Let $N$ be a kernel on $\mathsf{X} \times \mathcal{Y}$ and $\mu \in \mes_+(\mathsf{X},\mathcal{X})$.
		Then $\mu N \in \mes_+(\mathsf{Y},\mathcal{Y})$.
		If $N$ is a probability kernel, then $\mu(\mathsf{X}) = \mu N(\mathsf{X})$.
	\end{pop}

	\begin{pop}
		Let $M$ and $N$ be two kernels on $\mathsf{X} \times \mathcal{Y}$ and $\mathsf{Y} \times \mathcal{Z}$.
		There exists a kernel $MN$ on $\mathsf{X} \times \mathcal{Z}$, called the \textbf{composition} or the \textbf{product} of $M$ and $N$ such that $\forall x \in \mathsf{X}, \forall A \in \mathcal{Z}, MN(x,A) = \int_{\mathsf{Y}} M(x, \diff y) N(y, A)$.
	\end{pop}

	\begin{pop}
		Let $M$ and $N$ be kernels on $\mathsf{X} \times \mathcal{Y}$ and $\mathsf{Y} \times \mathcal{Z}$.
		Then $\forall x \in \mathsf{X}, \forall f \in \F_+(\mathsf{Z},\mathcal{Z}), MNf(x) = M[Nf](x)$.
	\end{pop}
	
	\begin{pop}
		Let $M$ ba a kernel on $\mathsf{X} \times \mathcal{Y}$ and $N$ be a kernel on $\mathsf{Y} \times \mathcal{Z}$.
		Then, there exists a kernel $M \otimes N$ on $\mathsf{X} \times (\mathcal{Y} \otimes \mathcal{Z})$, called the \textbf{tensor product} of $M$ and $N$, such that
		$$\forall f \in \F_+(\mathsf{Y} \times \mathsf{Z}, \mathcal{Y} \otimes \mathcal{Z}),\
		M \otimes N f(x) = \int_{\mathsf{Y} \times \mathsf{Z}} f(v) M \otimes N(x, \diff v)
		                 = \int_{\mathsf{Y}} \left( \int_{\mathsf{Z}} f(y,z) N(y, \diff z) \right) M(x, \diff y)\ .$$
	\end{pop}
	
	\begin{lem}
		Let $M$ ba a kernel on $\mathsf{X} \times \mathcal{Y}$ and $N$ be a kernel on $\mathsf{Y} \times \mathcal{Z}$.
		\begin{enumerate}[(i)]
			\item If $M$ and $N$ are both finite kernels, then $M \otimes N$ is a finite kernel.
			\item If $M$ and $N$ are both probability kernels, then $M \otimes N$ is a probability kernel.
			\item Then tensor product of kernels is associative : $(M \otimes N) \otimes P = M \otimes (N \otimes P)$.
		\end{enumerate}
	\end{lem}


\subsection{Homogeneous Markov chains}

	\begin{defn}
		Let $(\mathsf{X},\mathcal{X})$ be a measurable space, $\nu$ a probability measure on $(\mathsf{X},\mathcal{X})$, $P$ a Markov kernel on $\mathsf{X} \times \mathcal{X}$ and $(\Omega,\mathcal{F},(\mathcal{F}_k)_{k \in \N},\proba)$ a filtered probability space.
		An adapted stochastic process $((X_k, \mathcal{F}_k))_{k \in \N}$ is called a \textbf{homogeneous Markov chain} with Markov kernel $P$ and initial distribution $\nu$ is
		$$\forall k \in \N, \forall A \in \mathcal{X},\qquad \proba(X_0 \in A) = \nu(A) \qquad \text{and} \qquad \proba(X_{k + 1} \in A \mid \mathcal{F}_k) = P(X_k,A)\ \proba\text{-a.s.}$$ 
	\end{defn}
	
	\begin{note}
		Unless specify, we consider the natural filtration and write that $(X_k)_{k \in \N}$ is a homogeneous Markov chain.
	\end{note}
	
	\begin{pop}
		An $\mathsf{X}$-valued random process $(X_k)_{k \in \N}$ is a homogeneous Markov chain with kernel $P$ and initial distribution $\nu$ if and only if, for all $k \in \N$, the distribution of $(X_1,\ldots,X_n)$ is $\nu \otimes P^{\otimes k}$.
	\end{pop}
	
	We also get that the distribution of $X_k$ is $\nu P^k$.

	\begin{thm}
		Let $\mathsf{X} = \R^d$ with $d \geq 1$ endowed with $\mathcal{X} = \mathcal{B}(\R^d)$, $P$ a Markov kernel on $(\mathsf{X},\mathcal{X})$ and $\nu$ a probability measure on $(\mathsf{X},\mathcal{X})$.
		Let $(Z_k)_{k \in \N}$ be a sequence of i.i.d. random variables uniformly distributed on $\inff{0}{1}$.
		There exists a measurable application $g$ from $(\inff{0}{1}, \mathcal{B}(\inff{0}{1}))$ to $(\mathsf{X},\mathcal{X})$ and a measurable application $f$ from $(\mathsf{X} \times \inff{0}{1}, \mathcal{X} \otimes \mathcal{B}(\inff{0}{1}))$ to $(\mathsf{X},\mathcal{X})$ such that the sequence $(X_k)_{k \in \N}$ defined by $X_0 = g(Z_0)$ and $\forall k \geq 0, X_{k + 1} = f(X_k, Z_{k + 1})$ is a homogeneous Markov chain with initial distribution $\nu$ and Markov kernel $P$.
	\end{thm}

\subsection{The canonical Chain, notation $\proba_\xi$ and $\esp_\xi$}

	\begin{thm}
		Let $P$ be a Markov kernel on $\mathsf{X} \times \mathcal{X}$ and $\nu \in \mes_1(\mathsf{X}, \mathcal{X})$.
		Then, there exists a unique probability $\proba_\nu$ on $(\mathsf{X}^\N,\mathcal{X}^{\otimes \N})$ is a Markov chain with initial distribution $\nu$ and kernel $P$.
	\end{thm}

	\begin{defn}
		The \textbf{canonical Markov chain} with kernel $P$ on $\mathsf{X} \times \mathcal{X}$ is the canonical process $(X_n)_{n \in \N}$ on the canonical filtered space $(\mathsf{X}^\N,\mathcal{X}^{\otimes \N},(\mathcal{F}_k)_{k \in \N})$ endowed with the collection of probability measures $\{ \proba_\nu, \nu \in \mes_1(\mathsf{X},\mathcal{X}) \}$ given by the previous theorem.
	\end{defn}

	The expectation associated to $\proba_\nu$ is denoted by $\esp_\nu$ and for $x \in \mathsf{X}$, $\proba_x$ and $\esp_x$ are shorthand for $\proba_{\delta_x}$ and $\esp_{\delta_x}$.
	
	\begin{pop}
		\begin{enumerate}[(i)]
			\item $\forall A \in \mathcal{X}^{\otimes}$, the function $x \mapsto \proba_x(A)$ is $\mathcal{X}$-measurable.
			\item $\forall \xi \in \mes_1(\mathsf{X},\mathcal{X}), \forall A \in \mathcal{X}^{\otimes}, \proba_\xi(A) = \int_{\mathsf{X}} \proba_x(A) \xi(\diff x)$
			\item $\forall \xi \in \mes_1(\mathsf{X}, \forall Y \in \F_+(\mathsf{X}^\N,\mathcal{X}^{\otimes \N}), \esp_\xi(Y) = \int_{\mathsf{X}} \esp_x(Y) \xi(\diff x)$
			\item $\forall \xi \in \mes_1(\mathsf{X}, \forall Y \in L^1(\mathsf{X}^\N,\mathcal{X}^{\otimes \N}, \proba_\xi), \esp(\abs{Y}) < \infty$ for $\xi$-a.e. $x$ and $\esp_\xi(Y) = \int_{\mathsf{X}} \esp_x(Y) \xi(\diff x)$
			\item $\forall \xi \in \mes_1(\mathsf{X}, \forall Y \in L^1(\mathsf{X}^\N,\mathcal{X}^{\otimes \N}, \proba_\xi), \esp_\xi(Y \mid X_0) = \esp_{X_0}(Y)$ $\proba_\xi$-a.s.
		\end{enumerate}
	\end{pop}

	\begin{defn}
		An event is $\proba_*$-a.s. if it is $\proba_\nu$-a.s. for all initial distribution $\nu$.
	\end{defn}

	\begin{lem}
		Let $A \in \mathcal{X}^{\otimes \N}$.
		We have $X \in A$ $\proba$-a.s. if and only if it is true $\proba_x$-a.s. for all $x \in \mathsf{X}$.
	\end{lem}
