\begin{defn}
	Let $\mathcal{H}$ be a complex linear space.
	An \textbf{inner-product} on $\mathcal{H}$ is a function $\scal{\cdot}{\cdot} \colon \mathcal{H} \times \mathcal{H} \to \C$ which satisfies the following properties :
	\begin{enumerate}[(i)]
		\item $\forall (x,y) \in \mathcal{H} \times \mathcal{H}, \scal{x}{y} = \overline{\scal{y}{x}}$,
		\item $\forall x, y, z \in \mathcal{H} \forall (\alpha,\beta) \in \C \times \C, \scal{\alpha x + \beta y}{z} = \alpha \scal{x}{z} + \beta \scal{y}{z}$,
		\item $\forall x \in \mathcal{H}, \left( \scal{x}{x} = 0 \right) \iff \left( x = 0 \right)$
	\end{enumerate}
	Then $\norme{\cdot} \colon x \mapsto \sqrt{\scal{x}{x}} \geq 0$ defines a norm on $\mathcal{H}$.
	Both are continuous.
\end{defn}

\begin{thm}
	For all $x, y \in \mathcal{H}$, we have :
	\begin{enumerate}[a)]
		\item Cauchy-Schwarz inequality : $\abs{\scal{x}{y}} \leq \norme{x} \cdot \norme{y}$,
		\item triangular inequality : $\abs{\norme{x} - \norme{y}} \leq \norme{x - y} \leq \norme{x} + \norme{y}$,
		\item Parallelogram inequality : $\norme{x + y}^2 + \norme{x - y}^2 = 2 \norme{x}^2 + 2 \norme{y}^2$.
	\end{enumerate}
\end{thm}

\begin{defn}
	An inner-product space $\mathcal{H}$ is called an Hilbert space if it is complete.
\end{defn}

\begin{pop}
	For all measured space $(\Omega, \mathcal{F}, \mu)$, the space $L^2(\Omega, \mathcal{F}, \mu)$ endowed with $\scal{f}{g} = \int f \bar{g} \diff \mu$ is a Hilbert space.
\end{pop}

\begin{defn}
	Two vectors $x,y \in \mathcal{H}$ are orthogonal if $\scal{x}{y} = 0$ which we denoted by $x \perp y$.
	If $\mathcal{S}$ is a subspace of $\mathcal{H}$, we write $x \perp \mathcal{S}$ if $\forall s \in \mathcal{S}, x \perp s$.
	Also we write $\mathcal{S} \perp \mathcal{T}$ if all vectors in $\mathcal{S}$ are orthogonal to $\mathcal{T}$.
\end{defn}

\begin{note}
	If $\mathcal{H} = \mathcal{A} + \mathcal{B}$ and $\mathcal{A} \perp \mathcal{B}$ we will denote $\mathcal{H} = \mathcal{A} \overset{\perp}{\oplus} \mathcal{B}$.
\end{note}

\begin{defn}
	Let $\mathcal{E}$ be a subset of an Hilbert space $\mathcal{H}$.
	The orthogonal set of $\mathcal{E}$ is defined as $\mathcal{E} = \{ x \in \mathcal{H} \mid \forall y \in \mathcal{E}, \scal{x}{y} = 0 \}$.
\end{defn}

\begin{thm}
	If $\mathcal{E}$ is a subset of an Hilbert space $\mathcal{H}$, then $\mathcal{E}^{\perp}$ is closed.
\end{thm}

\begin{defn}
	Let $E$ be a subset of $\mathcal{H}$.
	It is an orthogonal set if for all $(x,y) \in E \times E, x \neq y, x \perp y$.
	If moreover $\forall x \in E, \norme{x} = 1$, we say that $E$ is orthonormal.
\end{defn}
