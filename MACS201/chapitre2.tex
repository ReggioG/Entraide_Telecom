Let $(\Omega, \mathcal{F}, \proba)$ be a probability space.

\begin{thm}[$\pi$ - $\lambda$ theorem]
	If $\mathcal{A} \subset \mathcal{C}$ with $\mathcal{A}$ a $\pi$-system and $\mathcal{C}$ a $\lambda$-system, then $\sigma(\mathcal{A}) = \mathcal{C}$.
\end{thm}

\begin{thm}[Characterization of probability measures]
	Let $\mathcal{C}$ be a $\pi$-system on $\Omega$ and $\mathcal{F} = \sigma(\mathcal{C})$ the smallest $\sigma$-field containing $\mathcal{C}$.
	Then a probability measure $\mu$ on $(\Omega, \mathcal{F})$ is uniquely characterized by $\mu(A)$ on $A \in \mathcal{C}$.
\end{thm}

\begin{note}
	For $p > 0$, we denote by $\mathcal{L}^p(\Omega, \mathcal{F}, \proba)$ the space of random variables $X$ such that $\esp \left( \abs{X}^p \right) < \infty$ and by $L^p(\Omega, \mathcal{F}, \proba)$ the one identifying random variables that are equal $\proba$-a.s.
\end{note}

\subsection{Conditional calculus}

	\begin{lem}
		Let $X \in \mathcal{L}^1(\Omega, \mathcal{F}, \proba)$ and $\mathcal{G}$ a sub-$\sigma$-field of $\mathcal{F}$.
		Then there exists $Y \in \mathcal{L}^1(\Omega, \mathcal{G}, \proba)$ such that
		\begin{equation}\label{eq:condesp}
			\forall A \in \mathcal{G}, \esp(X \indic_A) = \esp(Y \indic_A)
		\end{equation}
		Moreover the following assertions hold.
		\begin{enumerate}[(i)]
			\item If $Y' \in \mathcal{L}^1(\Omega, \mathcal{G}, \proba)$ also satisfies \eqref{eq:condesp} then $Y' = Y$ $\proba$-a.s.
			\item If $X \in \mathcal{L}^2(\Omega, \mathcal{F}, \proba)$, then $Y = \proj(X \mid L^2(\Omega, \mathcal{G}, \proba))$.
			\item \eqref{eq:condesp} continues to hold extended as $\esp(XZ) = \esp(YZ)$ for all $\mathcal{G}$-measurable r.v. $Z$ such that $\esp(\abs{XZ}) < \infty$.
		\end{enumerate}
	\end{lem}

	\begin{defn}
		Let $X \in \mathcal{L}^1(\Omega, \mathcal{F}, \proba)$ and $\mathcal{G}$ a sub-$\sigma$-field of $\mathcal{F}$.
		The unique $Y \in L^1(\Omega, \mathcal{G}, \proba)$ defined by \eqref{eq:condesp} is called the \textbf{conditional expectation} of $X$ given $\mathcal{G}$, and denoted by $Y = \esp(x \mid \mathcal{G})$.
	\end{defn}

	\begin{pop}
		Suppose that $X, Y, Z, (X_n)_{n \geq 1} \in L^1(\Omega, \mathcal{F}, \proba)$.
		The following hold $\proba$-a.s.
		\begin{enumerate}[(i)]
			\item (linearity) $\forall a, b \in \R, \esp(aX + bY \mid \mathcal{G}) = a \esp(X \mid \mathcal{G}) + b \esp(Y \mid \mathcal{G})$
			\item If $X$ is $\mathcal{G}$-measurable, $\esp(X \mid \mathcal{G}) = X$
			\item If $\mathcal{G} = \{ \emptyset, \Omega \}$ is the trivial $\sigma$-field, then $\esp(X \mid \mathcal{G}) = \esp(X)$
			\item If $X$ is independent of $\mathcal{G}$ then $\esp(X \mid \mathcal{G}) = \esp(X)$
			\item (positivity) If $X \leq Y$ then $\esp(X \mid \mathcal{G}) \leq \esp(Y \mid \mathcal{G})$
			\item $\esp(X \mid \mathcal{G}) \vee \esp(Y \mid \mathcal{G}) \leq \esp(X \vee Y \mid \mathcal{G})$,
				$\esp(X \mid \mathcal{G})_+ \leq \esp(X_+ \mid \mathcal{G})$ and
				$\abs{\esp(X \mid \mathcal{G})} \leq \esp(\abs{X} \mid \mathcal{G})$
			\item (tower property) If $\mathcal{H}$ is a sub-$\sigma$-field of $\mathcal{F}$ such that $\mathcal{G} \subset \mathcal{H}$ then $\esp( \esp(X \mid \mathcal{H}) \mid \mathcal{G}) = \esp(X \mid \mathcal{G})$
			\item The expectation is not modified by conditional expectation : $\esp( \esp(X \mid \mathcal{G}) ) = \esp(X)$
			\item If $X$ is $\mathcal{G}$-measurable and $XY \in L^1(\Omega, \mathcal{F}, \proba)$, then $\esp(XY \mid \mathcal{G}) = X \cdot \esp(Y \mid \mathcal{G})$ 
		\end{enumerate} 
	\end{pop}

	\begin{defn}
		Let $Y$ be a r.v. and $\sigma(X)$ the sub-$\sigma$-field generated by a r.v. $X$.
		If $\esp(Y \mid \sigma(X))$ is well-defined, it is written as $\esp(Y \mid X)$ and is called the \textbf{conditional expectation} of $Y$ given $X$.
	\end{defn}
	
	\begin{defn}
		Let $\mathcal{G}$ be a sub-$\sigma$-field of $\mathcal{F}$.
		For any event $A \in \mathcal{F}$, we denote $\proba(A \mid \mathcal{G}) = \esp(\indic_A \mid \mathcal{G})$.
		The mapping $A \mapsto \proba(A \mid \mathcal{G})$ is called a \textbf{version of the conditional probability} of $A$ given $\mathcal{G}$.
	\end{defn}

	\begin{defn}
		Let $\mathcal{G}$ be a sub-$\sigma$-field of $\mathcal{F}$.
		A \textbf{regular version} of the conditional probability of $\proba$ given $\mathcal{G}$ is a function $\proba^{\mathcal{G}} \colon \Omega \times \mathcal{F} \to \intff{0}{1}$ such that
		\begin{enumerate}[(i)]
			\item For all $A \in \mathcal{F}$, $\proba^{\mathcal{G}}(A) \colon \omega \mapsto \proba^{\mathcal{G}}(\omega, A)$ is $\mathcal{G}$-measurable and is a version of the conditional probability of $A$ given $\mathcal{G}$, $\proba^{\mathcal{G}}(A) = \proba(A \mid \mathcal{G})$.
			\item For all $\omega \in \Omega$, the mapping $A \mapsto \proba^{\mathcal{G}}(\omega, A)$ is a probability on $\mathcal{F}$.
		\end{enumerate}
	\end{defn}
	
	\begin{lem}
		Let $\proba^{\mathcal{G}}$ be a regular version of the conditonal probability of $\proba$ given $\mathcal{G}$ and let $Y \in L^1(\Omega, \mathcal{F}, \proba)$.
		Then $\esp(Y \mid \mathcal{G}) = \esp^{\mathcal{G}}(Y)$ $\proba$-a.s., with $\esp^{\mathcal{G}}(Y) \colon \omega \mapsto \int Y(\omega') \proba^{\mathcal{G}}(\omega, \diff \omega')$.
	\end{lem}

	\begin{defn}
		Let $\mathcal{G}$ be a sub-$\sigma$-field of $\mathcal{F}$.
		Let $(\mathsf{Y}, \mathcal{Y})$ be a measurable space and let $Y$ be an $\mathsf{Y}$-valued random variable.
		A regular version of the conditional distribution of $Y$ given $\mathcal{G}$ is a function $\proba^{Y \mid \mathcal{G}} \colon \Omega \times \mathcal{Y} \to \intff{0}{1}$ such that
		\begin{enumerate}[(i)]
			\item For all $A \in \mathcal{Y}$, $\omega \mapsto \proba^{Y \mid \mathcal{G}}(\omega, A)$ is $\mathcal{G}$ measurable and is a version of conditional distribution of $Y$ given $\mathcal{G}$, $\proba^{Y \mid \mathcal{G}}(\cdot, A) = \proba(Y \in A \mid \mathcal{G})$ $\proba$-a.s.
			\item For every $\omega$, $A \mapsto \proba^{Y \mid \mathcal{G}}(\omega, A)$ is a probability on $\mathcal{Y}$.
		\end{enumerate}
	\end{defn}

	\begin{defn}
		Let $(\mathsf{X}, \mathcal{X})$ and $(\mathsf{Y}, \mathcal{Y})$ be two measurable spaces.
		A \textbf{kernel} is a mapping $Q \colon \mathsf{X} \times \mathcal{Y} \to \intff{0}{\infty}$ satisfying the following conditions :
		\begin{enumerate}[(i)]
			\item for every $A \in \mathcal{Y}$, the mapping $Q(\cdot, A) \colon x \mapsto Q(x,A)$ is a measurable function,
			\item for every $x \in \mathsf{X}$, the mapping $Q(x, \cdot) \colon A \mapsto Q(x,A)$ is a measure on $\mathcal{Y}$.
		\end{enumerate}
		$Q$ is said to be finite if $\forall x \in \mathsf{X}, Q(x,\mathsf{Y}) < \infty$.
		It is called a probability kernel if $\forall x \in \mathsf{X}, Q(x,\mathsf{Y}) = 1$.
		It is called a Markov kernel if it is a probability kernel on $\mathsf{X} \times \mathcal{X}$.
	\end{defn}

	\begin{defn}
		Let $X$ and $Y$ be random variables with values in the measure spaces $(\mathsf{X},\mathcal{X})$ and $(\mathsf{Y},\mathcal{Y})$ respectively.
		A \textbf{regular version of the conditional distribution of $Y$ given $X$} is a probability kernel $\proba^{X \mid Y} \colon \mathsf{X} \times \mathcal{Y} \to \intff{0}{1}$ such that $\forall A \in \mathcal{Y}, \proba^{Y \mid X}(X,A) = \proba(Y \in A \mid X)$ $\proba$-a.s.
	\end{defn}

	\begin{thm}
		Let $\mathcal{G}$ be sub-$\sigma$-field of $\mathcal{F}$.
		Let $d \geq 1$ and $Y$ be an $(\R^D, \mathcal{B}(\R^d))$-valued random variable.
		Then, there exists a regular version of the conditional distribution of $Y$ given $\mathcal{G}$, $\proba^{Y \mid \mathcal{G}}$, and this version is unique in the sense that for any other regular version $\bar\proba^{Y \mid \mathcal{G}}$ of this distribution, for $\proba$-almost every $\omega$ it holds that $\forall F \in \mathcal{F}, \proba^{Y \mid \mathcal{G}}(\omega,F) = \bar\proba^{Y \mid \mathcal{G}}(\omega,F)$.
		Moreover, if $\mathcal{G} = \sigma(X)$ for some r.v. $X$ with values in a measurable space $(\mathsf{X},\mathcal{X})$, there also exists a unique regular version (hence a probability kernel) $\proba^{Y \mid X}$.
	\end{thm}

	\begin{lem}
		Let $\proba^{Y \mid X}$ bee a regular version of the conditional expectation of $Y$ given $X$.
		Then, for any real-valued measurable function $g$ on $\mathsf{Y}$ such that $\esp(\abs{g(Y}) < \infty$, we have $\esp(g(Y) \mid X) = \int g(Y) \proba^{Y \mid X}(X, \diff y)$, $\proba$-a.s.
	\end{lem}

	\begin{pop}
		Let $\mathbf{X}$ and $\mathbf{Y}$ be two jointly Gaussian vectors, respectively valued in $\R^p$ and $\R^q$.
		Then the following holds.
		\begin{enumerate}[(i)]
			\item
				If $\Cov(\mathbf{Y})$ is invertible, then $\hat{\mathbf{X}} := \proj(\mathbf{X} \mid \Span(1,\mathbf{Y}))$ is given by
				$\hat{\mathbf{X}} = \esp(\mathbf{X}) + \Cov(\mathbf{X},\mathbf{Y}) \Cov(\mathbf{Y})^{-1} (\mathbf{Y} - \esp(\mathbf{Y}))$, and
				$\Cov(\mathbf{X} - \hat{\mathbf{X}}) = \Cov(\mathbf{X}) - \Cov(\mathbf{X}, \mathbf{Y}) \Cov(\mathbf{Y})^{-1} \Cov(\mathbf{Y},\mathbf{X})$.
			\item We have $\esp(\mathbf{X} \mid \mathbf{Y}) = \proj(\mathbf{X} \mid \Span(1,\mathbf{Y}))$.
			\item Let $\hat{\mathbf{X}} = \esp(\mathbf{X} \mid \mathbf{Y})$.
				Then $\Cov(\mathbf{X} - \hat{\mathbf{X}}) = \esp \left( \mathbf{X}\transp{(\mathbf{X} - \hat{\mathbf{X}})} \right) = \esp \left( (\mathbf{X} - \hat{\mathbf{X}}) \transp{\mathbf{X}} \right)$
				and\\ $\proba^{\mathbf{Y} \mid \mathbf{X}}(\mathbf{X},\cdot) = \normale \left( \hat{\mathbf{X}}, \Cov \left( \mathbf{X} - \hat{\mathbf{X}} \right) \right)$.
		\end{enumerate}
	\end{pop}


\subsection{Radon-Nikodym derivative}

	\begin{defn}
		If $\forall A \in \mathcal{F}, \mu(A) = \int_A \phi \diff \lambda$, we say that the $\lambda$-a.e. equivalent class of $\phi$ is the \textbf{Radon-Nikodym derivative} of $\mu$ with respect to $\lambda$, and write $\phi = \frac{\diff \mu}{\diff \lambda}$.
	\end{defn}

	\begin{defn}
		Let $\lambda$ be a measure on $(\Omega,\mathcal{F})$.
		We say that a $\sigma$-finite measure $\mu$ is \textbf{absolutely continuous} with respect to $\lambda$ or that $\lambda$ dominates $\mu$ and we write $\mu \ll \lambda$ if $\forall A \in \mathcal{F}, (\lambda(A) = 0) \implies (\mu(A) = 0)$.
	\end{defn}
	
	\begin{thm}[\textbf{Radon-Nikodym theorem}]
		Let $\lambda, \mu \in \mes_+(\Omega,\mathcal{F})$ be $\sigma$-finite measures such that $\mu \ll \lambda$.
		Then, there exists a non-negative Borel function $\phi$ such that $\forall A \in \mathcal{F}, \mu(A) = \int_A \phi \diff \lambda$.
	\end{thm}
	
	\begin{defn}
		Let $(X,Y)$ be two random elements admitting a density $f$ with respect to measure $\xi \otimes \xi'$ on $(\mathsf{X} \times \mathsf{Y}, \mathcal{X} \otimes \mathcal{Y})$.
		Then the function $(x,y) \mapsto f(y \mid x) = \frac{f(x,y)}{\int f(x,y') \diff \xi'(y')}$ is called the \textbf{conditional density} of $Y$ given $X$.
	\end{defn}

	\begin{thm}
		Let $(X,Y)$ be two random elements admitting a density $f \colon \mathsf{X} \times \mathsf{Y} \to \R_+$ with respect to $\xi \otimes \xi'$ on $(\mathsf{X} \times \mathsf{Y}, \mathcal{X} \otimes \mathcal{Y})$.
		Then, $\forall  x \in \mathsf{X}, \forall A \in \mathcal{Y}, \proba^{Y \mid X}(x,A) = \int_A f(y \mid x) \xi'(\diff y)$.
	\end{thm}
	
	\begin{lem}
		Let $P$ and $Q$ be two probabilities on the measurable space $(\Omega,\mathcal{F})$ and let $\nu \in \mes_+(\Omega,\mathcal{F})$ dominate both $P$ and $Q$ (e.g. $\nu = P + Q$).
		Let $f_P$ and $f_Q$ denote the densities of $P$ and $Q$ with respect to $\nu$.
		Then, $\KL(P \| Q) = \int \ln \left( \frac{f_P}{f_Q} \right) \diff P$ is always well defined and takes values in $\intff{0}{\infty}$.
		Moreover we have :
		\begin{enumerate}[(i)]
			\item If $Q$ does not dominate $P$ then $\KL(P \| Q) = \infty$.
			\item If $P \ll Q$ then $\KL(P \| Q) = \int \ln \left( \frac{\diff P}{\diff Q} \right) \diff P$ (may be finite or infinite).
			\item We have $\KL(P \| Q) = 0 \iff P = Q$.
		\end{enumerate}
	\end{lem}

	\begin{defn}
		The quantity $\KL(P \| Q)$ is called the \textbf{Kullback-Leibler divergence} between $P$ and $Q$.
	\end{defn}

	\begin{thm}
		Let $P$ and $Q$ be two probabilities on the measurable space $(\Omega,\mathcal{F})$ and $X$ a measurable mapping from $(\Omega,\mathcal{F})$ to $(\mathsf{X},\mathcal{X})$.
		Then we have $\KL \left( P^X \| Q^X \right) \leq \KL(P \| Q)$.
	\end{thm}

	\begin{rem}
		Recall that $\forall A \in \mathcal{X}, P^X(A) = \int_{X^{-1}(A)} \diff P$ while $\forall F \in \mathcal{F}, P(F) = \int_F \diff P$.
	\end{rem}
