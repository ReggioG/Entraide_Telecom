\begin{thm}[$\pi$ - $\lambda$ theorem]
	If $\mathcal{A} \subset \mathcal{C}$ with $\mathcal{A}$ a $\pi$-system and $\mathcal{C}$ a $\lambda$-system, then $\sigma(\mathcal{A}) = \mathcal{C}$.
\end{thm}

\begin{thm}
	Let $\mathcal{C}$ be a $\pi$-system on $\Omega$ and $\mathcal{F} = \sigma(\mathcal{C})$ the smallest $\sigma$-field containing $\mathcal{C}$.
	Then a probability measure $\mu$ on $(\Omega, \mathcal{F})$ is uniquely characterized by $\mu(A)$ on $A \in \mathcal{C}$.
\end{thm}

\begin{defn}
	Let $X \in \mathcal{L}^1(\Omega, \mathcal{F}, \proba)$ and $\mathcal{G}$ a sub-$\sigma$-field of $\mathcal{F}$.
\end{defn}
