\subsection{Définitions}

	\begin{defn}
		Soit $k$ un corps.
		Un \textbf{corps de fonctions} $K$ de dimension $n$ sur $k$ est une extension de corps de $k$ de type fini et de degré de transcendance $n$ sur $k$.
		Pour $n = 1$ on parle de \textbf{corps de fonctions de courbe} sur $k$.
	\end{defn}

	Par abus de langage on dit que $K$ est une courbe (algébrique) sur $k$.
	
	\begin{defn}
		\textbf{Droite projective} sur $k$, notée $\proba_k^1$ ou $\proba^1$ : courbe simple donnée par $k(t)$ le corps des fractions rationnelles.
	\end{defn}
	
	...

\subsection{Anneaux de valuation}

	\begin{defn}
		Soit $K$ un corps.
		Un \textbf{anneau de valuation} de $K$ est un sous-anneau $R$ de $K$ vérifiant $\forall x \in K, x \in R\ \text{ou}\ x^{-1} \in R$.
		Il est dit non-trivial si $R \neq K$.
		Lorsque $k \subset R$ est un sous-corps de $K$, on dit que $R$ est un anneau de valuation au-dessus de $k$.
	\end{defn}

	\begin{rem}
		$R$ est intègre et $K = \Frac(R)$ \textrightarrow\ on parle d'anneau de valuation dans l'absolu pour un anneau de valuation de son corps des fractions.
	\end{rem}

	\begin{defn}
		Soit $x, y \in K$. On dit que :
		\begin{itemize}
			\item[\textbullet] $x$ est \textit{plus valué} que $y$ si $\exists z \in R, x = yz$,
			\item[\textbullet] $x$ et $y$ ont la  \textit{même valuation} si $\exists z \in R^\times, x = yz$.
		\end{itemize}
		Ceci définit une relation d'équivalence dont les classes sont appelées \textbf{valuations} et notées $v_R(x)$ ou $v(x)$.
		On note $v(0) = \infty$ mais cette classe est mise à part et on ne considère généralement pas qu'il s'agisse d'une valuation.
	\end{defn}

	\begin{rem}
		On a défini une relation d'ordre total sur les valuations (plus $\infty$ qui est le plus grand élément).
	\end{rem}

	\begin{note}
		On définit $v(x) + v(y) = v(xy)$ et $\forall c \in R^\times, v(c) = v(1) = 0$.
	\end{note}

	\begin{defn}
		Soit $\Gamma := K^\times / R^\times$ l'ensemble des valuations.
		Le groupe abélien $(\Gamma, +)$ est appelé \textbf{groupe des valuations} (ou des \textbf{valeurs}) de $R$.
	\end{defn}

	\begin{defn}
		Si $\Gamma = \Z$, i.e. est engendré par un unique élément, on dira que $R$ est un anneau de valuation \textbf{discrète}.
	\end{defn}

	\begin{pop}
		Soit $R$ un anneau de valuation de $K$. On a :
		\begin{enumerate}[(i)]
			\item $v(x) = \infty \iff x = 0$
			\item $v(xy) = v(x) + v(y)$
			\item $v(x + y) \geq \min \{ v(x), v(y) \}$
			\item $v(x + y) = \min \{ v(x), v(y) \}$ si $v(x) = v(y)$
		\end{enumerate}
		De plus $v(K^\times) = \Gamma$ et $R = \{ x \in K \mid v(x) \geq 0 \}$.
	\end{pop}

	\begin{ex}
		Soit $K = k(t)$ et $h$ un polynôme unitaire irréductible sur $k$.
		On pose, pour $f \in k[t]$, $v_h(f)$ est l'exposant de la plus grande puissance de $h$ qui divise $f$.
		Si $g \in k[t] \setminus \{ 0 \}$, $v_h \left( \frac{f}{g} \right) = v_h(f) - v_h(g)$.
		Alors $v_h$ vérifie les conditions ci-dessus et atteint $1$ en $h$.
		De plus $R$ est l'ensemble des fractions rationnelles sans $h$ facteur du dénominateur.
	\end{ex}

	\begin{ex}
		Soit $p$ premier et $K = \Q$.
		Pour $m \in \Z$, on pose $v_p(m)$ la valuation $p$-adique de $m$, i.e. l'exposant de la plus grande puissance de $q$ qui divise $m$.
		Si $\frac{n}{m} \in \Q$, $v_p \left( \frac{n}{m} \right) = v_p(n) - v_p(m)$.
		Alors $v_p$ vérifie les conditions ci-dessus et atteint $1$ en $p$.
		De plus $R$ est l'ensemble des rationnels dont le dénominateur réduit n'est pas multiple de $p$.
	\end{ex}

	\begin{rem}
		Si $A$ est un anneau intègre et $v \colon A \to \Z \cup \{ \infty \}$ vérifie \textit{(i)}, \textit{(ii)} et \textit{(iii)} alors il existe une unique fonction $v \colon \Frac(A) \to \Z \cup \{ \infty \}$ qui prolonge le $v$ donné sous les mêmes conditions, à savoir $v \colon \frac{x}{y} \mapsto v(x) - v(y)$ où $y \neq 0$.
		Si, de plus, $v$ est positive sur $A$ alors $A \subset R$ ou $R$ est l'anneau de la valuation.
	\end{rem}

	\begin{defn}
		Un anneau $R$ est dit \textbf{local} s'il vérifie l'une des propriétés équivalentes suivantes :
		\begin{enumerate}[(i)]
			\item $R$ a un unique idéal maximal,
			\item le complémentaire de $R^\times$ dans $R$ est un idéal (forcément maximal),
			\item pour tout $x \in R$, soit $x$ est inversible, soit $1 - cx$ est inversible pour tout $c \in R$.
		\end{enumerate}
	\end{defn}
	
	\begin{pop}
		Un anneau de valuation est un anneau local.
		Son idéal maximal est $\mathfrak{m}_v = \{ x \in R \mid v(x) > 0 \}$.
	\end{pop}
	
	\begin{defn}
		On note parfois $\mathcal{O}_v$ l'anneau de valuation associé à la valuation $v$.
		Le corps $\varkappa_v = \mathcal{O}_v / \mathfrak{m}_v$ s'appelle \textbf{corps résiduel} de la valuation $v$.
	\end{defn}

	\begin{pop}
		Si $v$ est une valuation au-dessus de $k$ (corps de base) alors $\varkappa_v$ est une extension de $k$.
		Son degré (s'il est fini) s'appellera degré sur $k$ de la valuation $v$.
	\end{pop}

	\begin{defn}
		Soit $K$ un corps de fonction de courbe sur $k$.
		Une valuation non triviale au-dessus de $k$ sur un corps $K$ de fonctions de $k$ s'appelle une \textbf{place} de $K$ (ou de la courbe $C$ telle que $K = k(C)$).
		On note $\mathscr{V}_{K/k}$ ou $\mathscr{V}_C$ l'ensemble de ces places.
	\end{defn}

	\begin{pop}
		Soit $K$ un corps, $A \subset K$ un sous-anneau et $\mathfrak{p}$ un idéal premier de $A$.
		Alors il existe un anneau de valuation $R$ de $K$ tel que $A \subset R \subset K$ et $\mathfrak{m} \cap A = \mathfrak{p}$ où $\mathfrak{m}$ est l'idéal maximal de $R$.
	\end{pop}

	Cette proposition sert à construire des valuations “centrées” sur un idéal premier $\mathfrak{p}$ qu'on s'est donné.
	
	\begin{pop}
		Soit $K$ un corps et $A \subset K$ un sous-anneau.
		Alors $B := \bigcap_{A \subset R \subset K} R$ (avec $R$ anneau de valuation de $K$) est exactement l'ensemble des $x \in K$ entiers (algébriques) sur $A$ au sens où il existe $f \in A[t]$ unitaire, non constant, à coefficients dans $A$ tels que $f(x) = 0$.
		$B$ est donc un sous-anneau de $K$ et s'appelle \textbf{fermeture intégrale} de $A$ dans $K$, ou \textbf{clôture intégrale} lorsque $K = \Frac(A)$.
		En particulier, si $k$ est un sous-corps de $K$ alors $B$ est la fermeture algébrique de $k$ dans $K$.
	\end{pop}

	\begin{pop}
		Soit $\mathcal{O}_v$ un anneau de valuation discrète de valuation $v$.
		Un élément $t \in \mathcal{O}_v$ engendre $\mathfrak{m}$ en tant qu'idéal si et seulement si $v(t) = 1$.
		Il est appelé \textbf{uniformisante} de $\mathcal{O}_v$ et pour un tel $t$ fixé (il en existe) :
		\begin{itemize}
			\item[\textbullet] tout $x \neq 0$ de $K$ a une représentation unique sous la forme $x = u t^r$ avec $u \in \mathcal{O}_v^\times$ et $r \in \Z$, avec $r = v(x)$,
			\item[\textbullet] tout idéal $I \neq \{ 0 \}$ de $\mathcal{O}_v$ est l'idéal $\mathfrak{m}^r = \{ x \in \mathcal{O}_v \mid v(x) \geq r \}$ engendré par $t^r$ pour un certain $r \in \N$.
		\end{itemize}
	\end{pop}


\subsection{Places des courbes}

	\begin{lem}
		Soit $K$ un corps de fonctions de courbes sur $k$ et $v$ une valuation de $K$ au-dessus de $k$.
		Alors :
		\begin{enumerate}[(i)]
			\item Si $x$ vérifie $v(x) \neq 0$ et $v(x) < \infty$ alors $x$ est transcendant sur $k$ et le corps $K$ est fini sur $k(x)$.
			\item Si $x_1,\ldots,x_n$ vérifient $0 < v(x_1) < v(x_2) < \cdots < v(x_n) < \infty$, alors $x_1,\ldots,x_n$ sont linéairement indépendants sur $k(x_n)$, et en particulier le degré $[K : k(x_n)]$ (fini) est supérieur ou égal à $n$.
			\item Si $x$ vérifie $0 < v(x) < \infty$ alors $[\varkappa_v : k] \leq [K : k(x)]$.
		\end{enumerate}
	\end{lem}

	\begin{pop}
		Soit $K$ un corps de fonctions de courbe sur $k$.
		Alors toutes les places de $K$ sont discrètes.
	\end{pop}

	Dans ce cas $\varkappa_v$ est une extension finie, donc algébrique, de $k$.
	Le degré $[\varkappa_v : k]$ s'appelle aussi degré de la place $v$.
	S'il vaut $1$, i.e. $\varkappa_v = k$, $v$ est dite rationnelle.
	C'est notamment le cas si $v$ est algébriquemet clos.
	
	\begin{defn}
		Soit $K$ un corps de fonctions de courbe sur $k$.
		Si $f \in K$ et $v \in \mathscr{V}_K$ on peut définir l'\textbf{évaluation} de $f$ en $v$
		$$f(v) \in \varkappa_v, \quad f(v) = \left\{ \begin{array}{l}
			\text{la classe de $f \in \mathcal{O}_v$ modulo $\mathfrak{m}_v$ lorsque $v(f) \geq 0$} \\
			\text{le symbole spécial $\infty$ (pas celui de $v(0)$) lorsque $v(f) = 0$}
			\end{array} \right.$$
		Dans le cas $f(v) = \infty$ on dit que $f$ a un \textbf{pôle} en $v$.
		On a trois possibilités exclusives:
		\begin{align*}
			v(f) > 0 & \iff f(v) = 0 \iff f \in \mathfrak{m}_v \qquad \text{$f$ a un \textbf{zéro} en $v$} \\
			v(f) < 0 & \iff f(v) = \infty \iff f \not\in \mathcal{O}_v \qquad \text{$f$ a un \textbf{pôle} en $v$} \\
			v(f) = 0 & \iff f(v) \in \varkappa_v^\times \iff f \in \mathcal{O}_v^\times
		\end{align*}
		$v(f)$ est appelé multiplicité du zéro de $f$ en $v$, et $-v(f)$ multiplicité du pôle.
		Si $v(f) = 1$, $f$ est appélé \textbf{paramètre local} pour $K$ en $v$ (comme uniformisante).
	\end{defn}

	\begin{pop}
		La fermeture algébrique $\tilde{k}$ de $k$ dans $K$ peut s'appeler \textbf{corps des constantes} et coïncide avec $\{ f \in K \mid \forall v \in \mathscr{V}_K, v(f) = 0 \}$, ces fonctions $f$ étant dites \textbf{constantes}.
		On a alors l'équivalence suivante :
		\begin{align*}
		\text{$f$ n'est pas constante} & \iff \quad \text{$f$ est transcendante} \quad \iff \quad \text{$\exists v \in \mathscr{V}_K$ où $f$ ait un pôle} \\
		                               & \iff \quad \text{$f$ n'est pas nulle et $\exists v \in \mathscr{V}_K$ où $f$ ait un zéro}
		\end{align*}
	\end{pop}
	
	\begin{rem}
		Tous les corps résiduels $\varkappa_v$ sont des extensions de $\tilde{k}$.
		Notamment $[\tilde{k} : k]$ divise tous les $\deg(v) = [\varkappa_v : k]$ et, en particulier, s'il existe une place \textbf{rationnelle}, i.e. telle que $\deg(v) = 1$, ou simplement deux places de degrés premiers entre eux, on a $\tilde{k} = k$.
	\end{rem}

\subsection{Les places de la droite projective}

	Soit $h \in k[t]$ unitaire et irréductible, $v_h(f)$ pour $f \in k[t]$ l'exposant de $h$ dans la décomposition de $f$ en polynômes irréductibles et $\forall f,g \in k[t], v_h \left( \frac{f}{g} \right) = v_h(f) - v_h(g)$.

	\begin{pop}
		Le corps résiduel $\varkappa_h$ de la place $v_h$ est le corps de rupture $k[t] / (h)$ de $h$ sur $k$.
	\end{pop}

	\begin{rem}
		La valeur de $f$ en la place $v_\xi$, définie comme $v_h$ où $h = t - \xi, \xi \in k^{\alg{}}$, peut s'identifier à la valeur $f(\xi)$ dans le corps $k(\xi) = k[t] / (h)$.
	\end{rem}
	
	\begin{rem}
		Une autre valuation non-triviale de $k(t)$ au-dessus de $k$ est $v_\infty \colon \frac{f}{g} \mapsto \deg(g) - \deg(f)$.
	\end{rem}
	
	\begin{pop}
		Soit $k$ un corps.
		Alors les places du corps $k(t)$ sont exactement $v_\infty$ et les places $v_h$.
	\end{pop}
	
	\begin{rem}
		Lorsque $k$ est algébriquement clos, les places de $\proba_k^1$ s'identifient donc aux éléments de $k$ ($x \in k$ est identifié à $f \in k(t) \mapsto v_x(f)$) plus l'élément $\infty$ (correspondant à la valuation $v_\infty$).
	\end{rem}

\subsection{L'indépendance des valuations}

	


\subsection{L'identité du degré}

	


\subsection{Diviseurs sur les courbes}

	


\subsection{Espaces de Riemann-Roch}

	


\subsection{Différentielles de Kähler}

	\begin{defn}
		Soit $k$ un anneau et $A$ une $k$-algèbre.
		On appelle espace des \textbf{différentielles de Kähler} de $A$ sur $k$, noté $\Omega^1_{A/k}$, le $A$-module engendré par les symboles $\diff x$ pour $x \in A$, soumis aux relations suivantes :
		\begin{enumerate}[(i)]
			\item $\diff \colon A \to \Omega^1_{A/k}$ est linéaire, i.e. $\forall x, x' \in A, \diff(x + x') = \diff x + \diff x'$ et $\forall c \in k, \forall x \in A, \diff(cx) = c \diff x$,
			\item $\forall x, y \in A, \diff(xy) = x \diff y + y \diff x$.
		\end{enumerate}
		Donc $\Omega^1_{A/k}$ est le quotient du $A$-module libre de base $\{ \diff x, x \in A \}$ par le sous-module engendré par les relations ci-dessus.
	\end{defn}
	
	\begin{pop}
		Soit $K$ une extension de corps de $k$ de type fini.
		Les propriétés suivantes sont équivalentes :
		\begin{itemize}
			\item[\textbullet] si la caractéristique est $p > 0$ alors, dans $K$, les corps $K^p$ et $k$ sont linéairement disjoints sur $k^p$,
			\item[\textbullet] il existe une base de transcendance $(t_1,\ldots,t_n)$ pour laquelle $K$ est (algébrique) séparable sur $k(t_1,\ldots,t_n)$.
		\end{itemize}
		Lorsque ces conditions sont vérifiées on dit que $K$ est \textbf{séparable}.
		On dit aussi que $(t_1,\ldots,t_n)$ est une base de transcendance \textbf{séparante}.
	\end{pop}
	
	\begin{rem}
		Toute extension de corps en caractéristique $0$ est séparable.
	\end{rem}
	
	\begin{pop}
		Si $K = k(C)$ est le corps des fractions d'une courbe sur un corps $k$ et qu'au moins une des hypothèses suivantes est satisfaite :
		\begin{itemize}
			\item[\textbullet] le corps de base $k$ est parfait,
			\item[\textbullet] la courbe $C$ est irréductible.
		\end{itemize}
		alors l'extension $K$ est séparable.
	\end{pop}

	\begin{pop}
		Soit $K$ une extension de corps de $k$ de type fini et séparable.
		Soit $(t_1,\ldots,t_n)$ une base de transcendance séparante.
		Alors $\Omega_{K/k}^1$ est un $K$-espace vectoriel de base $\diff t_1,\ldots,\diff t_n$.
		Réciproquement, si $t_1,\ldots,t_n \in K$ sont tels que $\diff t_1,\ldots,\diff t_n$ sont linéairement indépendants sur $K$, alors ils sont une base de transcendance séparante.
	\end{pop}
	
	...

\subsection{Théorème de Riemann-Roch}

	


\subsection{Points et places}

	


\subsection{Revêtements de courbes}

	