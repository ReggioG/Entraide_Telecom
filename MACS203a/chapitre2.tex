\subsection{Variables aléatoires}

	\begin{defn}
		Soit $\T$ un ensemble et $\{ X_\tau, \tau \in \T \}$ une famille quelconque de v.a.
		La $\sigma$-algèbre $\mathcal{X}$ engendrée par cette famille est la plus petite $\sigma$-algèbre sur $\Omega$ telle que $X_\tau$ est $\mathcal{X}$-mesurable pour tout $\tau \in \T$, i.e.
		$$\mathcal{X} = \sigma(X_\tau, \tau \in \T) = \sigma(\{ X_\tau^{-1}(A) \mid \tau \in \T, A \in \mathcal{B}(\R) \})\ .$$
	\end{defn}

	\begin{lem}
		Soit $X$ et $Y$ deux v.a. sur $(\Omega,\mathcal{A},\proba)$ à valeurs respectivement dans $\R$ et $\R^n$.
		Alors $X$ est $\sigma(Y)$-mesurable ssi $\exists f \colon \R^n \to \R, X = f(Y)$.
	\end{lem}

	% ...

\subsection{Espérance de variables aléatoires}

	%\begin{defn}
		%On dit que $X$ a une densité de probabilité $f_X$ si $\proba^X$ est absolument continue par rapport à la mesure de Lebesgue sur $\R$ et $\forall B \in \mathcal{B}_{\R}, \proba(X \in B) = \int_B f_X(x) \diff x$.
	%\end{defn}

	\begin{thm}[\textbf{Inégalité de Jensen}]
		Soit $X \in \mathcal{L}^1(\mathcal{A},\proba)$ et $g \colon \R^d \to \Bar{\R}$ une fonction convexe telle que $\esp(\abs{g(X)}) < \infty$.
		Alors $\esp(g(X)) \geq g(\esp(X))$.
	\end{thm}

	\begin{defn}
		Soit $X$ une v.a. à valeurs dans $\R^d$.
		Sa fonction caractéristique est $\Phi_X \colon \begin{array}{ccc}
			\R^d & \to & \C \\
			u & \mapsto & \esp \left[ e^{i \scal{u}{X}} \right]
		\end{array}$.
	\end{defn}

	\begin{lem}
		$\Phi_X(0) = 1$ et $\Phi_X$ est continue bornée (par 1) sur $\R^d$.
	\end{lem}

	\begin{pop}
		Soit $X \sim \mathcal{N}(b,V)$.
		On a $\Phi_X(u) = e^{\scal{u}{b} - \frac{1}{2} \scal{u}{Vu}}$.
	\end{pop}

	\begin{pop}
		Soit $X$ réelle avec $\esp(\abs{X}^p) < \infty$ pour un certain $p \in \N^*$.
		Alors $\Phi_X$ est $p$ fois dérivable et $\forall k \in \iniff{1}{p}, \Phi_X^{(k)}(0) = i^k \esp(X^k)$.
	\end{pop}

	% ...


\subsection{Espaces $\mathcal{L}^p$ et convergences fonctionnelles des v.a.}

	La corrélation entre deux v.a. $X$ et $Y$ est $\Cor(X,Y) = \frac{\Cov(X,Y)}{\norme{X}_2 \norme{Y}_2}$.
	
	Le théorème de Pythagore s'écrit
	$$\esp(XY) = 0 \implies \esp[(X + Y)^2] = \esp \left[ X^2 \right] + \esp \left[ Y^2 \right]
		\qquad \text{ou} \qquad
		\Cov(X,Y) = 0 \implies \Var(X + Y) = \Var(X) + \Var(Y)\ .$$
	et la loi du parallélogramme s'écrit $\norme{X + Y}_2^2 + \norme{X - Y}_2^2 = 2 \norme{X}_2^2 + 2 \norme{Y}_2^2$.

	% ...
	
	%\begin{defn}
		%Soit $(X_n)_n$ et $X$ dans $\mathcal{L}^0$.
		%On dit que $(X_n)_n$ \textbf{converge en probabilité} vers $X$ si $\forall \varepsilon > 0, \lim_{n \to \infty} \proba[\abs{X_n - X} \geq 0] = 0$.
	%\end{defn}

	\begin{lem}
		La convergence p.s. ou la  convergence en norme dans $L^p$ impliquent la convergence en probabilité.
	\end{lem}

	\begin{lem}
		La convergence en probabilité est équivalente à la convergence au sens de la distance $D \colon (X,Y) \mapsto \esp(\abs{X - Y} \wedge 1)$
	\end{lem}
	
	\begin{thm}
		$(L^0,D)$ est un espace métrique complet.
	\end{thm}

	%\begin{thm}
		%Soit $(X_n)_n$ et $X$ des v.a. dans $\mathcal{L}^0$.
		%\begin{enumerate}[(i)]
			%\item $X_n \longrightarrow X$ p.s. ssi $\sup_{m \geq n} \abs{X_m - X} \longrightarrow 0$ en probabilité.
			%\item $X_n \overset{\proba}{\longrightarrow} X$ ssi de toute suite croissantes $(n_k)_k \subset \N$, on peut extraire une sous-suite $(n_{k_j})_j$ telle que $X_{n_{k_j}} \overset{\text{p.s.}}{\longrightarrow} X$.
		%\end{enumerate}
	%\end{thm}

	%\begin{cor}[Slutsky]
		%Soit $\phi$ continue.
		%Si $X_n \longrightarrow X$ en probabilités, alors $\phi(X_n) \longrightarrow \phi(X)$ en probabilité.
	%\end{cor}

	\begin{defn}
		Une famille $C$ de v.a. est dite \textbf{uniformément intégrable} (U.I.) si $\lim_{c \to \infty} \sup_{X \in C} \esp \left[ \abs{X} \indic_{\abs{X} \geq c} \right] = 0$.
	\end{defn}

	\begin{thm}
		Soit $(X_n)_n$ et $X$ des v.a. dans $\mathcal{L}^1$.
		Alors $X_n \longrightarrow  X$ dans $L^1$ si et seulement si $X_n \longrightarrow X$ en probabilité et $(X_n)_n$ est U.I.
	\end{thm}


\subsection{Convergence en loi}

	
