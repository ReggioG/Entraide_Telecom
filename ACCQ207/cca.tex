\subsection{Définitions}

	\begin{defn}
		Une \textbf{courbe elliptique} sur un corps $K$ est 
		\begin{itemize}
			\item[\textbullet] soit la donnée d'une courbe algébrique $E$ projective lisse de genre 1 sur $K$ et d'un point $O_E \in E(K)$,
			\item[\textbullet] soit la donnée d'une équation “de Weierstrass” de la forme $y^2 + a_1 xy + a_3 y = x^3 + a_2 x^2 + a_4 x + a_6$
				qui définit une courbe plane où les coefficients $a_1,a_2,a_3,a_4,a_6 \in K$ sont choisis pour que $E$ soit lisse.
				La courbe admet alors un unique point à l'infini, noté $O_E$.
		\end{itemize}
	\end{defn}
	
	\begin{rem}
		Lorsque $K$ est de caractéristique différente de 2 ou 3, on se ramène par changement de variable à une équation de la forme $y^2 = x^3 + ax + b$.
		La lissité équuivaut donc à ce que $x^3 + ax + b$ soit sans racine double, i.e $\Delta = 4 a^3 + 27 b^2 \neq 0$ dans $K$.
	\end{rem}
	
	Plan projectif : $\mathbf{P}^2 = \mathbf{A}^2 \cup \mathbf{P}^1$ où $\mathbf{P}^2$ correspond à $(X : Y : Z)$, $\mathbf{A}^2$ est le plan affine $(X : Y : 1)$ et $\mathbf{P}^1$ est la droite à l'infini $Z = 0$, $(X : Y : 0)$.


\subsection{Loi de groupe}

	\begin{lem}
		Soit $D \in \Div^0(E)$, alors $\exists ! P \in E(K), D \sim (P) - (O_E)$.
	\end{lem}
	
	On a donc une bijection $\begin{array}{rcl} E(K) & \overset{\sim}{\to} & Cl^0 (E)_K \\ P & \mapsto & (P) - (O_E) \end{array}$ avec $Cl^0 (E)_K$ le groupe des classes d'équivalence linéaire de diviseurs de degré 0 définis sur $K$.
	
	\begin{defn}
		On munit $E(K)$ d'une loi de groupe $+$ en transportant la loi de $Cl^0 (E)_K$ par cette bijection.
	\end{defn}
	
	\begin{pop}
		\begin{enumerate}[(i)]
			\item L'élément neutre de $E(K)$ est $O_E$.
			\item $\forall P,Q \in E(K)$, $P + Q$ dans $E(K)$ est l'unique point tel que $(P) - (O_E) + (Q) - (O_E) \sim (P + Q) - (O_E)$ dans $\Div^0(E)$, i.e. tel que $\exists f \in E(K), \divg(f) = (P) + (Q) - (P + Q) - (O_E)$.
			\item Soit $D = \sum_{P \in E(K)} n_P \cdot (P)$ un diviseur sur $E$.
				Alors $D$ est principal si et seulement si $\deg(D) = \sum_P n_P = 0$ et $\sum_p n_P P = O_E$ dans $E(K)$.
			\item En particulier $P + Q + R = O_E \iff (P) - (O_E) + (Q) - (O_E) + (R) - (O_E) \sim 0$ dans $\Div^0(E)$.
			\item $\forall P \in E(K)$, $-P \in E(K)$ est l'unique point tel que $\exists f, \divg(f) = (P) + (-P) - 2 (O_E)$.
		\end{enumerate}
	\end{pop}
	
	\begin{rem}
		On a $P + Q + R = O_E$ si et seulement si $P,Q,R$ sont les trois points d'intersection de $E$ et d'une droite.
	\end{rem}
	
	\begin{rem}
		Si $P$ est un point de coordonnées affines $(x_P,y_P)$ alors $-P$ a pour coordonnées $(x_P,-y_P)$.
	\end{rem}
	
	Formule explicite : $x_{P + Q} = \lambda^2 - x_P - x_Q$ et $y_{P + Q} = -y_P + \lambda (x_P - x_{P + Q})$, avec $\lambda = \frac{y_Q - y_P}{x_Q - x_P} = \frac{3 x_P^3 + a}{2 y_P}$.
	
	\begin{defn}
		Soit $E$ une courbe elliptique sur $K$ et $n \in \N^*$.
		On note $E[n] = \{ P \in E(\bar{K}) \mid n \cdot P = O_E \}$ où $n \cdot P = P + P + \ldots + P$ pour la loi de $E$.
		On dit que $P$ est de $n$-torsion s'il vérifie $n \cdot P = O_E$;
	\end{defn}
	
	\begin{rem}
		La loi est commutative donc $E[n]$ est un sous-groupe de $E(\bar K)$.
		De même $E[n] \cap E(K)$ est un sous-groupe de $E(K)$.
	\end{rem}
	
	\begin{ex}
		Les points de $2$-torsion sont les $P$ tels que $P + P = O_E$.
		Donc soit $P = O_E$, soit $P = (x,0)$ avec $x^3 + ax + b = 0$.
		Ce polynôme a trois racines dans $\bar K$, donc $E[2] \simeq \Z/2\Z \times \Z/2\Z$.
	\end{ex}
	
	...\\
	...\\
	...%TODO
	
	\begin{pop}
		Il existe deux fractions rationnelles $F_m(x)$ et $G_m(x)$ telles que $\forall P = \begin{bmatrix} x \\ y \end{bmatrix} \in E, m \cdot P = \begin{bmatrix} F_m(x) \\ y G_m(x) \end{bmatrix}$ et le dénominateur de $F_m(x)$ et $G_m(x)$ est une puissance d'un certain polynôme $\psi_m$, appelé polynôme de $m$-division, de degré $\deg \psi_m = \left\{ \begin{array}{l}
		\frac{m^2 - 1}{2}\ \text{si $m$ est impair} \\
		3 + \frac{m^2 - 4}{2}\ \text{si $m$ est pair} \end{array}\right.$.
	\end{pop}
	
	\begin{proof}[Principe de la preuve]
		On prouve la propriété par récurrence avec les formules pour $(m + 1)P = mP + P$.
	\end{proof}
	
	\begin{rem}
		On a $-P = \begin{bmatrix} x \\ -y \end{bmatrix}$ et $m \cdot P = \begin{bmatrix} F_m(x) \\ -y G_m(x) \end{bmatrix} = m \cdot (-P)$.
		De plus $P$ est de $m$-torsion ssi $mP$ est le point $O_E$ à l'infini, ssi $x$ est un pôle de $F_m$ (et de $G_m$), ssi $x$ est racine de $\psi_m$.
	\end{rem}
	
	\begin{ex}
		...
	\end{ex}
	
	\begin{cor}
		On a $\abs{E[m]} \leq m^2$.
	\end{cor}
	
	\begin{proof}
		$E[m] = \{ (x,y) \in E \mid \psi_m(x) = 0 \} \cup \{ O_E \}$.
		Si $m$ est impair, $\psi_m$ a au plus $\frac{m^2 - 1}{2}$ racines $x$ et pour chaque tel $x$, au plus deux valeurs de ...\\
		...%TODO
	\end{proof}
	
	\begin{rem}
		En réalité on sait dire mieux dans le cas $m = p^e$.
		Si $p \neq \card(K)$ alors $\abs{E[m]} = m^2$ et si $p = \card(K)$ alors $\abs{E[m]} = 1$ ou $m$.
	\end{rem}
	
	\begin{cor}
		Soit $K = \F_q$ un corps fini.
		Alors $E(\F_q)$ est le produit d'au plus deux groupes cycliques, c'est-à-dire que ou bien $E(\F_q) \simeq \Z / N\Z$ est cyclique, ou bien $E(\F_q) \simeq \Z / m\Z \times \Z / n\Z$, ...%TODO
	\end{cor}
	
	\begin{proof}
		On a $E(\F_q) \simeq Z / n_1\Z \times \cdots \times Z / n_r\Z$ où $n_{i + 1} | n_i$.
		Soit $l$ premier tel que $l | n_r$, donc $\forall i, l | n_i$.
		Alors
		\begin{align*}
		E(\F_q)[l] & \subset E[l]\\
		\quad \simeq & 
		\end{align*}
	\end{proof}
	
	\begin{thm}[de Hasse]
		Soit $K = \F_q$ un corps fini et $E : y^2 = x^3 + ax + b$.
		Alors $\abs{E(\F_q)} \in \intff{q + 1 - 2 \sqrt{q}}{q + 1 + 2 \sqrt{q}}$, ou encore $\abs{E(\F_q)} = q + 1 - t$ pour un entier $t \in \Z, \abs{t} \leq 2 \sqrt{q}$ ($t$ la “trace”).
	\end{thm}
	
	\begin{rem}
		$E(\F_q) = \{ O_E \} \cup \{ (x,y) \in \F_q \times \F_q \mid y^2 = x^3 + ax + b \}$ et ce deuxième ensemble a pour cardinal $q$, le nombre de valeur possibles pour $x$.
		En effet, à $x$ fixé il y a 0, 1 ou 2 valeurs possibles pour $y$.
		0 si $x^3 + ax + b$ n'est pas carré dans $\F_q$, 1 si $x^3 + ax + b = 0$, donc $y = 0$ et 2 si $x^3 + ax + b \neq  0$ est un carré dans $\F_q$.
		
		Il y a autant de carrés que de non carrés dans $\F_q$.
		On s'attend donc à ce que $x^3 + ax + b$ soit à peu près aussi souvent carré que non carré.
		Le théorèmme de Hasse dit que l'excès de l'un ou de l'autre est inférieur à $2 \sqrt{q}$.
	\end{rem}
	
	\begin{ex}
		Prenons $K = \F_5, \abs{E[\F_5]} \in \intff{2}{10}, E : y^2 = x^3 - x$, donc $a = -1$ et $b = 0$.
		Alors $\Delta = 4a^3 + 27 b^2 = 1 \neq 0$ et $E(\F_5)$ contient les éléments suivants :
		$$0_E,
		\begin{bmatrix} 0 \\ 0 \end{bmatrix},
		\begin{bmatrix} 1 \\ 0 \end{bmatrix},
		\begin{bmatrix} 2 \\ 1 \end{bmatrix},
		\begin{bmatrix} 2 \\ 4 \end{bmatrix},
		\begin{bmatrix} 3 \\ 2 \end{bmatrix},
		\begin{bmatrix} 3 \\ 3 \end{bmatrix},
		\begin{bmatrix} 4 \\ 0 \end{bmatrix}$$
		d'où $\abs{E(\F_(5)} = 8$.
		On veut l'identifier à un groupe abélien de cardinal 8 qui soit produit d'au plus deux groupes cycliques.
		Or il existe deux tels groupes : $\Z / 4\Z \times \Z / 2\Z$ et $\Z / 8\Z$.
		
		Remarquons alors que $E(\F_5)[2] = \left\{ O_E, \begin{bmatrix} 0 \\ 0 \end{bmatrix}, \begin{bmatrix} 1 \\ 0 \end{bmatrix}, \begin{bmatrix} 4 \\ 0 \end{bmatrix} \right\}$ et $\Z / 8\Z[2] = 4\Z / 8\Z = \{ \bar 0, \bar 4 \}$.
		Donc l'isomorphisme se fait nécessairement avec $\Z / 4\Z \times \Z / 2\Z$.\\
	\end{ex}
	
	Rappel : expliciter un isomorphisme entre en groupe abélien donné et un produit de cycliques revient à trouver une “base” de ce groupe.
	Soit $G$ donné.
	On veut un isomorphisme explicite $G \simeq \Z / n_1\Z \times \cdots \times \Z / n_r\Z$.
	\begin{enumerate}[1.]
		\item nécessairement $n_1 = \omega(G)$ et on cherche $x_1 \in G$ d'ordre maximal $\omega(G)$,
		\item alors $G / \langle x_1 \rangle \simeq \Z / n_2\Z \times \cdots \times \Z / n_r\Z$ et l'on trouve récursivement $z_2,\ldots,z_r$ une “base” de ce quotient,
		\item relever chaque $z_i \in G / \langle x_1 \rangle$ en un $x_i \in G$ de sorte que $\langle x_1 \rangle \cap \langle x_2,\ldots,x_r \rangle = \{ 0 \}$.\\
	\end{enumerate}
	
	Dans ce cas :
	\begin{enumerate}[(1)]
		\item on cherche $P \in E(\F_5)$ d'ordre 4.
			Les éléments de $E(\F_5) \simeq \Z/4\Z \times Z/2\Z$ sont d'ordre 1, 2 ou 4.
			Tout élément qui n'est pas de 2-torsion est d'ordre 4.
			$P = \begin{bmatrix} 2 \\ 1 \end{bmatrix}$ convient.
		\item Trouver $Q$ d'ordre 2 tel que
			\begin{align*}
				\langle P \rangle \cap \langle Q \rangle = \{ O_E \} & \iff Q\ \text{d'ordre 2}, Q \not\in \langle P \rangle = \{ O_E, P, 2P, 3P \} \\
				                                                     & \iff Q\ \text{d'ordre 2}, Q \neq  2P
			\end{align*}
			Calcul de $2P$ : $\lambda = \frac{3 \cdot 2^2 - 1}{2 \cdot 1} = -\frac{4}{2} = 3$ donc $2P = \begin{bmatrix} 3^2 - 2 - 2 \\ 3(2 - 0) - 1 \end{bmatrix} = \begin{bmatrix} 0 \\ 0 \end{bmatrix}$.
			On peut prendre $Q = \begin{bmatrix} 1 \\ 0 \end{bmatrix}$.
	\end{enumerate}
	
	\begin{ex}
		$E : y^2 = x^3 + x$, $\Delta = 4 \neq 0$ et $E(\F_5) = \left\{ O_E, \begin{bmatrix} 0 \\ 0 \end{bmatrix}, \begin{bmatrix} 2 \\ 0 \end{bmatrix}, \begin{bmatrix} 3 \\ 0 \end{bmatrix} \right\}$, d'où $\abs{E(\F_5)} = 4$ et $E(\F_5) \simeq \Z / 2\Z \times \Z / 2\Z$.
	\end{ex}
	
	\begin{ex}
		$E : y^2 = x^3 + x + 2$, $E(\F_5) = \left\{ O_E, ... \right\}$
		...%TODO
	\end{ex}


\subsection{Isogénies}

	\begin{defn}
		Soit $E$ et $E'$ deux courbes elliptiques.
		Une isogénie $E \to E'$ est un morphisme de courbes algébriques $E \to E'$ qui envoie $O_E$ sur $O_{E'}$.
	\end{defn}
	
	Concrètement, si $E : y^2 = x^3 + ax + b$ et $E' : y^2 = x^3 + a' x + b'$, un morphisme de $E$ dans $E'$ est donné par des fonctions rationnelles $f(x,y)$ et $g(x,y)$ telles que
	$$\forall x, y^2 = x^3 + ax + b \implies g(x,y)^2 = f(x,y)^3 + a'  f(x,y) + b'$$
	et le morphisme envoie $\begin{bmatrix} x \\ y \end{bmatrix} \in E$ sur $\begin{bmatrix} f(x,y) \\ g(x,y) \end{bmatrix} \in E'$.
	
	\begin{ex}
		Soit $u \in K^\times$.
		$\begin{pmatrix} x \\ y \end{pmatrix} \mapsto \begin{pmatrix} u^2 x \\ u^3 y \end{pmatrix}$ définit une isogénie de $y^2 = x^3 + ax + b$ dans $y^2 = x^3 + u^4 ax + u^6 b$.
		
		Cette isogénie est inversible (isomorphisme) d'inverse $\begin{pmatrix} x \\ y \end{pmatrix} \mapsto \begin{pmatrix} u^{-2} x \\ u^{-3} y \end{pmatrix}$.
	\end{ex}
	
	On peut montrer que tout isomorphisme entre deux courbes elliptiques est de cette forme.
	Par ailleurs, remplacer la courbe $y^2 = x^3 + a x + b$ par $y^2 = x^3 + u^4 a x = u^6 b$ laisse invariant $j = \frac{4 a^3}{4 a^3 + 27 b^2}$.
	
	\begin{thm}
		Deux courbes sont isomorphes (sur $\bar K$) ssi elles ont le même $j$-invariant.
		De plus pour tout choix de $j \in K$, il existe $a, b \in K$ tels que la courbe $y^2 = x^3 + ax + b$ est une courbe elliptique de $j$-invariant égal à $j$.
	\end{thm}
	
	\begin{proof}
		Prouvons la deuxième partie.
		On cherche $a, b$ tels que $A = 4a^3$ et $B = 27 b^2$.
		De la sorte $j = \frac{A}{A + B} = \frac{1}{1 + B/A}$.
		
		Dans le cas $j \neq 0$, $1 + \frac{B}{A} = \frac{1}{j}$,...
		
		...%TODO
	\end{proof}
	
	\begin{ex}
		Supposons que $K$ contient un élément $i$ tel que $i^2 = -1$, comme $i = 2$ dans $K = \F_5$.
		Alors $\begin{pmatrix} x \\ y \end{pmatrix} \mapsto \begin{pmatrix} -x \\ iy \end{pmatrix}$ définit une isogénie, et même un isomorphisme de le courbe $y^2 = x^3 - x$ dans elle-même.
	\end{ex}
	
	\begin{ex}
		Soit $E$ une courbe elliptique et $m$ un entier.
		Alors $\begin{array}{lcr} E & \to & E \\ P & \mapsto & mP \end{array}$ est une isogénie de $E$.
	\end{ex}
	
	\begin{ex}
		Soit $K = \F_p$, $E : y^2 = x^3 + ax + b$.
		Le Frobenius $\varphi(x,y) = (x^p, y^p)$ définit une isogénie de $E$.
		En effet, $\forall x,y \in \bar \F_p$,
		$$y^2 = x^3 + ax + b \implies (y^p)^2 = (y^2)^p = (x^3 + ax + b)^p = (x^p)^3 + a x^p + b\ .$$
	\end{ex}
	
	\begin{thm}
		Si $E \to E'$ est une isogénie, alors elle définit un morphisme de groupes $E(K) \to E'(K)$.
	\end{thm}
	
	\begin{proof}
		Les lois de $E(K)$ et $E'(K)$ sont celles de leurs groupes de classes de diviseurs.
	\end{proof}
	
	\begin{ex}
		Cas $K = \F_5$, $E : y^2 = x^3 - x$ et les points de $E(\F_5)$ sont
		$$O_E, \underset{2P}{\begin{pmatrix} 0 \\ 0 \end{pmatrix}},
		\underset{Q}{\begin{pmatrix} 1 \\ 0 \end{pmatrix}},
		\underset{P}{\begin{pmatrix} 2 \\ 1 \end{pmatrix}},
		\underset{3P}{\begin{pmatrix} 2 \\ 4 \end{pmatrix}},
		\underset{3P + Q}{\begin{pmatrix} 3 \\ 2 \end{pmatrix}},
		\underset{P + Q}{\begin{pmatrix} 3 \\ 3 \end{pmatrix}},
		\underset{2P + Q}{\begin{pmatrix} 4 \\ 0 \end{pmatrix}}$$
		On le munit de l'isogénie $\begin{pmatrix} x \\ y \end{pmatrix} \mapsto \begin{pmatrix} -x \\ 2y \end{pmatrix}$.
		Elle respecte la structure de groupe et induit un isomorphisme de groupes de $E(\F_5) \simeq \Z / 4\Z \times \Z / 2\Z$ dans lui-même.
	\end{ex}
	
	...%TODO
	
	\begin{rem}
		...%TODO
	\end{rem}
	
	\begin{exc}
		On prend $y^2 = x^3 - x$.
		Ses zéros sont $P_1 = \begin{pmatrix} -1 \\ 0 \end{pmatrix}$, $P_2 = \begin{pmatrix} 0 \\ 0 \end{pmatrix}$ et $P_3 = \begin{pmatrix} 1 \\ 0 \end{pmatrix}$.
		
		On veut montrer $\divg(x) = 2 (P_2) - 2 (O_E)$ et $\divg(y) = (P_1) + (P_2) + (P_3) - 3(O_E)$.
		
		On a $K[x,y] / (y^2 - (x^3 - x), x) = K[y]/(y^2)$.
		Si on veut formaliser : il s'agit d'une courbe plane lisse donnée par une équation $F(x,y) = 0$.
		On se donne une fonction $h(x,y)$ qui n'est pas multiple de $F$ (sinon $h$ est identiquement nulle sur la courbe) et de dimension finie.
		On veut trouver l'ordre d'annulation de $h$ en un point $P$ de la courbe, i.e l'anneau quotient $K[x,y]/(F,h)$.
		
		On a $K[x,y]/(F,h) \simeq A_1 \times \cdots \times  A_r$ produit fini d'anneaux de dimensions finies.
		$r$ est le nombre de points d'intersection de $F = 0$ et $h = 0$ et chaque $A_i$ correspond à un $P_i$ point d'intersection.
		De plus $\dim A_i$ est l'ordre de $h$ en $P_i$.
		
		Pour garder uniquement le $A_i$ qui correspond à $P$ il faut en plus trouver une autre fonction $\psi$ telle que $\psi$ s'annule en $P$ avec une grande multiplicité (convient si supérieure à $\deg(F) \cdot \deg(h)$) et $\psi \neq 0$ sur les autres $P_i$.
		Alors $K[x,y] / (F,h,\psi) = A_{i_0}$ où $i_0$ est l'indice de $P$.
		
		Souvent on aura pas besoin de tout ça et on pourra conclure plus rapidement par un simple argument de degré.
		
		On obtient alors $\divg \left( \frac{x}{y} \right) = \divg(x) - \divg(y) = (P_2) + (O_E) - (P_1) - (P_3)$.
		Si l'on veut dérouler l'algo jusqu'au bout, par exemple pour trouver $v_{P_2}(y)$, on cherche une fonction $\psi$ qui s'annule en $P_2$ à l'ordre 3 et non nul en $P_1$ et $P_3$.
		Par exemple $\psi = x^3$ convient.
		On obtient $K[x,y] / (y^2 - (x^3 - x), y, x^3) \simeq K[x]/(x)$ de dimension 1.
	\end{exc}
	
	\begin{lem}
		Soient $P$ et $Q$ deux points de $E$ et $l_{P,Q} : \lambda x + \mu y + \nu$ l'équation affine de la droite $(PQ)$ (ou bien de la tangente à la courbe en $P$ si $P = Q$).
		Alors $\divg(l_{P,Q}) = (P) + (Q) + (-P-Q) - 3(O_E)$.
	\end{lem}
	
	On se donne $D = \sum_P n_P(P)$ tel que $\sum n_P = 0$ et $\sum n_P P = O_E$.
	
	Problème : trouver $f$ tel que $D = \divg(f)$.
	
	Quitte à réécrire les points autant de fois que leur multiplicité $D = (P_1) + \ldots + (P_n) - (Q_1) - \ldots - (Q_n)$ avec $P_1 + \ldots + P_n = Q_1 + \ldots + Q_n$ dans $E$.
	On procède par récurrence sur $n$.
	
	Pour $n = 0$ et $D = 0$, $f = 1$ convient.
	Pour $n = 1$, $D = (P_1) - (P_1) = 0$ et on retombe sur $n = 0$.
	
	On suppose $n \geq 2$ et qu'on sait faire jusqu'à $n - 1$.
	
	...
	
	...
	%TODO


\subsection{Les courbes algébriques en crypto}

	Les courbes algérbriques ont été introduites pour la factorisation \textrightarrow\ généralisation de l'algo $p - 1$ de Pollard qui permet de trouver facilement les diviseurs premiers $p$ d'un entier $N$ tel que $p - 1$ soit friable.
	On doit donc éviter ce type de facteur.
	
	Cela repose sur $N = pq \implies (\Z / N\Z)^\times \simeq (\Z / p\Z)^\times \times (\Z / q\Z)^\times$.
	L'algorithme $p - 1$ de Pollard fonctionne si $\F_p^\times$ est friable.
	
	Lenstra généralise cela avec des courbes elliptiques.
	Soit $a, b \in \Z / N\Z$ et $E : y^2 = x^3 + ax + b$.
	On a $E(\Z / N\Z) = E(\F_p) \times E(\F_q)$ (avec des bizarreries pour $O_E$).
	Quand $a$ et $b$ varient, $\abs{E(\F_q)}$ varie et on arrive toujours à tomber sur un friable.
	
	Plus tard les courbes algébriques ont été utilisées pour fabriquer des cryptosystèmes en remplaçant $\F_p^\times$ par $E(\F_p)$ où le log discret est plus difficile.


\subsection{Couplages}

	Les couplages ont été introduits pour la cryptanalyse pour ramener certaines instances du log discret elliptique au log discret classique.
	On fabrique ensuite des cryptosystèmes avec des propriétés spéciales.
	
	\begin{defn}
		Soient $A, B, G$ trois groupes abéliens, avec une loi notée additivement pour $A$ et $B$, et multiplicativement pour $G$.
		Un \textbf{couplage} $e \colon A \times B \to G$ est une application bilinéaire, i.e telle que $e(a + a',b) = e(a,b) e(a',b)$ et $e(a,b + b') = e(a,b) e(a,b')$.
		
		Dans le cas particulier $A = B$ :
		\begin{itemize}
			\item[\textbullet] le couplage est dit \textbf{symétrique} si $e(a,b) = e(b,a)$,
			\item[\textbullet] le couplage est dit \textbf{alterné} si $\forall a, e(a,a) = 1$, ce qui implique l'antisymétrie $e(b,a) = e(a,b)^{-1}$.
		\end{itemize}
	\end{defn}
	
	\paragraph{Construction d'un couplage alterné} sur les groupes de $m$-torsion d'une courbe elliptique.
	On travaille sur $K = \F_q$ où $q = p^n$ avec $p$ premier et on note $\bar K$ la cloture algébrique de $K$.
	Soit $m \in \N$ tel que $p \not\mid m$.
	Soit $\mu_m \subset \bar K^\times$ le groupe des racines $m$-ièmes de l'unité.
	$\mu_m$ forme un groupe cyclique d'ordre $m$, donc $\mu_m \simeq \Z / m\Z$.
	
	\begin{rem}
		$X^m - 1$ est bien à racines simples dans $\bar K$ car sa dérivée est $m X^{m - 1}$ et $\pgcd(X^m - 1, m X^{m - 1}) = 1$.
	\end{rem}
	
	Soit $E$ une courbe elliptique sur $K$.
	$E[m] \subset E(\bar K)$ est le produit de deux groupes cycliques d'ordre $m$.
	
	On va construire le couplage de Weil $e_m \colon E[m] \times E[m] \to \mu_m$.
	Soient $P,Q \in E[m]$ et $D_{P,Q} = (P) + (P + Q) + (P + 2Q) + \ldots + (P + (m-1)Q) - (O_E) - (Q) - (2Q) - \ldots - ((m-1)Q)$.
	Ce diviseur est principal (puisque $mP = O_E$ par définition de $P \in E[m]$)..
	
	On construit effectivemet une fonction $f_{PQ}$ telle que $\divg(f_{PQ}) = D_{PQ}$.\\
	
	
	Soit maintenant  $\tau_Q \colon \begin{array}{lcr} E & \to & E \\ R & \mapsto & Q + R\end{array}$ l'application de translation par $Q$.
	C'est un morphisme algébrique de $E$ dans lui-même.
	
	Si $f$ est une fonction de $E$, $\tau_Q^* f$ est la fonction telle que $\tau_Q^* f(R) = f(Q + R)$.
	Donc $\divg(\tau_Q^* f)$ est translatée de $\divg(f)$ par $-Q$.
	
	En particulier, pour $f = f_{P,Q}$, $\div(\tau_Q^* f_{P,Q}) = \divg(f_{P,Q}) = D_{PQ}$ invariant par translation par $Q$.
	Donc $\tau_Q^* f_{P,Q} = c \cdot f_{P,Q}$ pour une certaine constante $c \in \bar K^*$.
	
	$$\tau_{2Q}^* f_{P,Q} = \tau_Q^* \tau_Q^* f_{P,Q} = \tau_Q^* (c \cdot f_{P,Q}) = c \cdot \tau_Q^* f_{P,Q} = c^2 f_{P,Q}$$
	
	Par suite il vient $\tau_{mQ}^* f_{P,Q} = c^m f_{P,Q}$.
	Or $mQ = O_E$ car $Q \in E[m]$, donc $\tau_{mQ}^* f_{P,Q} = f_{P,Q}$, d'où $c \in \mu_m$.
	
	\begin{defn}
		$e_m(P,Q) = \frac{\tau_Q^* f_{P,Q}}{f_{P,Q}}$.
	\end{defn}
	
	\begin{thm}
		\begin{enumerate}[(i)]
			\item $e_m(P + P', Q) = e_m(P,Q) e_m(P',Q)$,
			\item $e_m(P, Q + Q') = e_m(P,Q) = e_m(P,Q')$,
			\item $e_m(P,P) = 1$,
			\item si $P,Q \in E[m](K)$ alors $e_m(P,Q) \in K$,
			\item $e_m \colon E[m] \times E[m] \to \mu_m$ est surjective.
		\end{enumerate}
	\end{thm}
	
	\begin{proof}
		\textit{(ii)} et \textit{(v)} sont admises (difficiles).
		
		Pour \textit{(i)} : on peut choisir $f_{P + P',Q} = f_{P,Q} \cdot \tau_{-P}^*  f_{P',Q}$ et alors
		$$\frac{\tau^*_Q f_{P + P',Q}}{f_{P + P',Q}}
			= \frac{\tau_Q^* f_{P,Q}}{f_{P,Q}}
			\cdot \frac{\tau_Q^* \tau_{-P}^* f_{P',Q}}{\tau_{-P}^* f_{P',Q}}
			= \frac{\tau_Q^* f_{P,Q}}{f_{P,Q}}
			\cdot\tau_{-P} \left( \frac{\tau_Q^* f_{P',Q}}{f_{P',Q}} \right)$$
		
		Pour \textit{(iii)} : $D_{P,P} = 0$, $f_{P,P} = 1$.
		
		Pour \textit{(iv)} : $f_{P,Q}$ diviseur de $D_{P,Q}$ est construite par récurrence.
		% TODO
	\end{proof}